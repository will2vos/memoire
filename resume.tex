\chapter*{Résumé}               % ne pas numéroter
\label{chap-resume}             % étiquette pour renvois
\phantomsection\addcontentsline{toc}{chapter}{\nameref{chap-resume}} % inclure dans TdM

Les environnements forestiers jouent un rôle majeur à l’échelle planétaire à travers leurs fonctions écosystémiques et leur apport économique. 
Toutefois, la croissance démographique humaine et l’augmentation des besoins en produits ligneux entraînent une intensification de l'exploitation des forêts. 
Ces changements peuvent amener  une perte d'habitats essentiels à la biodiversité et perturber les dynamiques écologiques entre les espèces qui peuplent ces écosystèmes. 
Cette étude vise à quantifier les effets des traitements sylvicoles tels que les coupes totales et les coupes partielles sur la dynamique de la faune du sol en forêt mixte tempérée. 
Les objectifs étaient premièrement de mesurer l’impact des coupes forestières sur des variables environnementales influençant l’utilisation de l’habitat par la faune du sol 
et deuxièmement d’évaluer les effets directs et indirects de ces traitements sur les amphibiens et les arthropodes du sol à travers le réseau trophique. 
Nous avons émis une hypothèse que les variables environnementales associées à l’habitat de la petite faune (volume de débris ligneux, ouverture de la canopée et profondeur de litière) varient en fonction de l’intensité des coupes forestières. 
Ainsi, les traitements de coupes constitueraient une variable englobant les changements de conditions environnementales. 
Selon notre deuxième hypothèse, les traitements de coupe forestière entraîneraient une modification de l'utilisation de l'habitat 
par les salamandres cendrées (\textit{Plethodon cinereus}), les carabes (\textit{Carabidae}), et les collemboles (\textit{Collembola}), de manière directe ou via des interactions trophiques, 
allant des prédateurs vers les proies. 
Pour tester ces hypothèses, nous avons développé un modèle d’équations structurelles (SEM) intégrant des modèles d’occupation et des modèles linéaires mixtes. 
Cette approche nous a permis de mesurer simultanément, sous forme de réseau, l’effet des coupes forestières sur les variables environnementales, d’une part, et sur la cooccurrence entre les différents taxons, d’autre part. 
Nos résultats montrent que les traitements de coupes forestières influencent significativement les variables environnementales. 
Les coupes totales ont entraîné une ouverture plus grande de la canopée, une diminution de la profondeur de la litière et une réduction du volume de débris ligneux, 
tandis que les coupes partielles permettent une meilleure rétention de ces attributs environnementaux. 
Cependant, les traitements sylvicoles n’ont, dans l’ensemble, pas eu d’effet direct significatif sur la probabilité d’occupation des salamandres et des carabes, ni sur la biomasse des collemboles. 
Nous avons toutefois mesuré une probabilité d’occupation légèrement plus faible pour les salamandres dans les traitements de coupe totale par rapport aux coupes partielles, bien que cet effet soit marginal (IC à 90\%). 
De plus, aucun changement de cooccurrence entre ces groupes d’espèces n’a été observé, ce qui suggère que les effets des coupes forestières ne se propagent pas à travers le réseau trophique des prédateurs vers les proies du sol forestier. 
Notre étude met en évidence la pertinence des modèles d’équations structurelles pour analyser les réseaux trophiques complexes des écosystèmes forestiers. 
Cette recherche a permis d'approfondir la compréhension des relations existantes entre les pratiques sylvicoles et la dynamique de la faune du sol forestier. 
L'apport de ces connaissances permettra de fournir de meilleurs outils pour orienter les plans de gestion et adapter les pratiques sylvicoles. 


\begin{otherlanguage*}{french}
\textbf{Mots-clés:} \textit{Plethodon cinereus}, carabe, collembole, modèle d'équations structurelles, cooccurrence, coupe forestière.
\end{otherlanguage*}
