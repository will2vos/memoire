\documentclass[12pt, letterpaper]{article}
\usepackage{graphicx}

\usepackage[T1]{fontenc}
\usepackage{titling}

\setlength{\droptitle}{-10em}

\title{Rapport de cheminement}
\author{William Devos}
\date{Février 2024}

\begin{document}

\maketitle

Au cours des six derniers mois, nous avons achevé les mesures sur les échantillons en quantifiant 
la biomasse sèche des collemboles. Tous les collemboles ont été regroupés par unité d'échantillonnage, 
puis séchés par lyophilisation et pesés à l'aide d'une microbalance. Les données de biomasse ont été ajoutés 
aux données de présence/absence que nous avions déjà pour les salamandres et les carabes. Les carabes ont été 
divisés en deux groupes pour distinguer ceux représentant des proies pour la salamandres cendrée de ceux représentant des 
compétiteurs (e.g. débris ligneux, litière, alimentation). 
Le seuil utilisé pour séparer les espèces de carabes est basé sur la capacité d'une salamandre cendrée à ingérer un carabe.
Le seuil utilisé dans notre étude est défini par l'ouverture maximale de la mâchoire chez une salamandre cendrée. \\

Nous avons développé un modèle d'équation structurelle (SEM) pour comprendre de manière intégrative l'impact des 
traitements de coupe forestière (coupe totale, coupe partielle) sur la cooccurrence de la faune du sol, 
ainsi que sur les variables environnementales jouant un rôle important dans les probabilités d'occupation 
des espèces étudiées (Figure \ref*{fig:SEM}). Le SEM intègre des modèles linéaires mixtes 
pour estimer les paramètres des collemboles et des variables environnementales (volume de débris ligneux (CWD), 
profondeur de litière, ouverture du couvert), ainsi que des modèles d'occupation pour estimer les paramètres des 
salamandres et des deux groupes de carabes. Les modèles d'occupation tiennent compte de la détection imparfaite en 
estimant la probabilité de détection à partir du volume de débris ligneux et des précipitations. 
Les analyses ont été réalisé avec une approche Bayésienne et nous avons estimé les paramètres en utilisant la méthode de Monte-Carlo par chaînes de Markov (MCMC).


\pagebreak

\begin{figure}[ht!]
	\centering
	\includegraphics[scale=0.60]{fig_sem3.png}
	\caption{Modèle théorique illustrant les relations directes et indirectes entre les traitements de coupe forestière, les variables environnementales et la petite faune. 
    Chaque flèche indique l'effet potentiel d'une variable explicative sur une variable de réponse.}
	\label{fig:SEM}
	\end{figure}  

Les résultats actuels montrent un effet marginal des traitements de coupes forestières sur la probabilité d'occupation des salamandres.
Les salamandres auraient avec probabilité plus faible d'occuper les sites de coupes totale comparativement aux sites de coupe partielles. 
Nos résultats ne montrent pas d'effet significatifs des traitements de coupes sur la probabilité d'occupapation des carabes ou la biomasse des collemboles.
En revanche, nous avons observé un effet positif des précipitations sur la probabilité de détection des deux groupes de carabes. 
En ce qui concerne les variables environnementales, les traitements de coupe forestière augmentent de manière significative l'ouverture de la canope.  
La profondeur de la litière semble diminuée avec l'intensité du coupe forestière cependant les effets ne sont pas significatif.
Le CWD ne semble pas être affecté par les traitements de coupe forestière selon nos résultats.\\

\pagebreak

Je suis actuellement dans la rédaction de mon mémoire. 
Voici un échéancier des prochains mois, dans le but de faire le dépôt initial de mon mémoire d’ici le 5 mai 2024 :



    \begin{center}
        \begin{tabular}{|c c c c|} 
            \hline
            Activité & Durée & Début & Fin \\ [0.5ex] 
            \hline\hline
            Rédaction méthodes & 1 sem & 15 janvier & 21 janvier \\ 
            \hline
            Rédaction résultats & 2 sem & 5 février & 18 février \\
            \hline
            Rédaction discussion + conclusion & 3 sem & 19 février & 10 mars \\
            \hline
            Rédaction introduction générale & 3 sem & 11 mars & 31 mars \\
            \hline
            Rédaction conclusion générale + résumé & 2 sem & 1 avril & 14 avril \\ 
            \hline
            Corrections finales & 3 sem & 15 avril & 5 mai \\
            \hline
            Dépôt initial & 1 j & avant le 5 mai 2024 & . \\
            \hline
            Dépôt final & .& mai 2024 & . \\ 
            \hline
        \end{tabular}
    \end{center}
    

\end{document}
