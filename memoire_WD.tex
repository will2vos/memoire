
\documentclass[MSc,bibsection,english,french]{ulthese}

\usepackage[utf8]{inputenc}
%\ifxetex\else \usepackage[utf8]{inputenc} \fi

\setlength{\beforechapskip}{20pt} % reduce space before chapter

\usepackage{amsmath}              % recommandé pour les mathématiques
\usepackage{icomma}               % gestion de la virgule dans les nombres
\usepackage{gensymb}
\usepackage{caption}
\usepackage{graphicx}
\usepackage{chngcntr}           % numerotation figure en annexe par section
\usepackage{array}
\usepackage{pdflscape}
%\usepackage{mathpazo}            % texte et mathématiques en Palatino
\usepackage{listings}            % insérer language R dans texte
\usepackage{xcolor}        % colorer le texte pour corrections
\definecolor{myYellow}{RGB}{248.7,225.0,63.1}
\usepackage{soul}          % colorer le texte plus long pour wrap up
\sethlcolor{myYellow}

\lstset{language=R,
    basicstyle=\small\ttfamily,
    stringstyle=\color{purple},
    breaklines=true,
    keywordstyle=\color{blue},
    commentstyle=\color{purple},
}


\usepackage{chngcntr}           % numerotation figure en annexe par section

\usepackage{rotating}
\usepackage[T1]{fontenc}
\usepackage{lmodern}
\usepackage{amssymb}
\usepackage{mathrsfs}
\usepackage{url}
%modify interline spacing
\usepackage{setspace}
\usepackage{lscape}
\usepackage{fixltx2e}   % permet de mettre des mots en indice dans tableau
\usepackage{multirow}   % permet de fusionner de colonne dans tableau
\usepackage{listings}   % insérer language R dans texte

% \usepackage{fontsize}
%   \changefontsize[12pt]{10pt}

% Define typographic struts, as suggested by Claudio Beccari
%   in an article in TeX and TUG News, Vol. 2, 1993.
\newcommand\Tstrut{\rule{0pt}{2.6ex}}         % = `top' strut
\newcommand\Bstrut{\rule[-0.9ex]{0pt}{0pt}}   % = `bottom' strut

\DeclareCaptionLabelFormat{tableformat}{#1 #2}
\captionsetup[table]{labelformat=tableformat, labelsep=period}

  % \DeclareCaptionLabelFormat{Sformat}{#1 S#2}  
  % \DeclareCaptionLabelSeparator{none}{}    %<---
  % \captionsetup[table]{labelformat=Sformat}
  
\usepackage[protrusion=true, expansion=true, shrink=55, stretch=55, 
tracking=true, kerning=true, spacing=true, 
final]{microtype}
\usepackage{natbib}
\setcitestyle{round, sort}

%\usepackage{bibentry}
\frenchbsetup{%
  StandardItemizeEnv=true,       % format standard des listes
  ThinSpaceInFrenchNumbers=true, % espace fine dans les nombres
  og=«, fg=»                     % caractères « et » sont les guillemets
}

\addto\extrasfrench{
  \renewcommand{\bibsection}{\section*{\bibname}\prebibhook}}


%\usepackage{multibib} %citer plusieurs bibliographies FR/ENG
%\newcites{Eng}{My English Biblio}

%default style in French (defined in babel)
%\bibliographystyle{ecologyNewFR}
%second style in English for second bibliography
%\bibliographystyleEng{ecologyNewEN}
%%%%%%%%%%%%%%%%

% pour tableau noir et blanc
\usepackage{longtable}%
\AtBeginEnvironment{longtable}{%
  \addfontfeature{RawFeature=+tnum;-onum}%  <--- requires LuaTeX
}


%change name of English references section
%\addto{\captionsenglish}{\renewcommand{\bibname}{References}}

  \titre{Impact des traitements sylvicoles en forêt mixte tempérée sur la cooccurrence entre amphibiens et arthropodes}

  \auteur{William Devos}
  \direction{Marc J. Mazerolle, directeur de recherche}
  \codirection{Mathieu Bouchard, codirecteur de recherche}

  \annee{202}

\begin{document}

\frontmatter                    % pages liminaires

\frontispice                    % production de la page frontispice

\chapter*{Résumé}               % ne pas numéroter
\label{chap-resume}             % étiquette pour renvois
\phantomsection\addcontentsline{toc}{chapter}{\nameref{chap-resume}} % inclure dans TdM

Les environnements forestiers jouent un rôle majeur à l’échelle planétaire à travers leurs fonctions écosystémiques et leur apport économique. 
Toutefois, la croissance démographique humaine et l’augmentation des besoins en produits ligneux entraînent une intensification de l'exploitation des forêts. 
Ces changements peuvent amener  une perte d'habitats essentiels à la biodiversité et perturber les dynamiques écologiques entre les espèces qui peuplent ces écosystèmes. 
Cette étude vise à quantifier les effets des traitements sylvicoles tels que les coupes totales et les coupes partielles sur la dynamique de la faune du sol en forêt mixte tempérée. 
Les objectifs étaient premièrement de mesurer l’impact des coupes forestières sur des variables environnementales influençant l’utilisation de l’habitat par la faune du sol 
et deuxièmement d’évaluer les effets directs et indirects de ces traitements sur les amphibiens et les arthropodes du sol à travers le réseau trophique. 
Nous avons émis une hypothèse que les variables environnementales associées à l’habitat de la petite faune (volume de débris ligneux, ouverture de la canopée et profondeur de litière) varient en fonction de l’intensité des coupes forestières. 
Ainsi, les traitements de coupes constitueraient une variable englobant les changements de conditions environnementales. 
Selon notre deuxième hypothèse, les traitements de coupe forestière entraîneraient une modification de l'utilisation de l'habitat 
par les salamandres cendrées (\textit{Plethodon cinereus}), les carabes (\textit{Carabidae}), et les collemboles (\textit{Collembola}), de manière directe ou via des interactions trophiques, 
allant des prédateurs vers les proies. 
Pour tester ces hypothèses, nous avons développé un modèle d’équations structurelles (SEM) intégrant des modèles d’occupation et des modèles linéaires mixtes. 
Cette approche nous a permis de mesurer simultanément, sous forme de réseau, l’effet des coupes forestières sur les variables environnementales, d’une part, et sur la cooccurrence entre les différents taxons, d’autre part. 
Nos résultats montrent que les traitements de coupes forestières influencent significativement les variables environnementales. 
Les coupes totales ont entraîné une ouverture plus grande de la canopée, une diminution de la profondeur de la litière et une réduction du volume de débris ligneux, 
tandis que les coupes partielles permettent une meilleure rétention de ces attributs environnementaux. 
Cependant, les traitements sylvicoles n’ont, dans l’ensemble, pas eu d’effet direct significatif sur la probabilité d’occupation des salamandres et des carabes, ni sur la biomasse des collemboles. 
Nous avons toutefois mesuré une probabilité d’occupation légèrement plus faible pour les salamandres dans les traitements de coupe totale par rapport aux coupes partielles, bien que cet effet soit marginal (IC à 90\%). 
De plus, aucun changement de cooccurrence entre ces groupes d’espèces n’a été observé, ce qui suggère que les effets des coupes forestières ne se propagent pas à travers le réseau trophique des prédateurs vers les proies du sol forestier. 
Notre étude met en évidence la pertinence des modèles d’équations structurelles pour analyser les réseaux trophiques complexes des écosystèmes forestiers. 
Cette recherche a permis d'approfondir la compréhension des relations existantes entre les pratiques sylvicoles et la dynamique de la faune du sol forestier. 
L'apport de ces connaissances permettra de fournir de meilleurs outils pour orienter les plans de gestion et adapter les pratiques sylvicoles. 


\begin{otherlanguage*}{french}
\textbf{Mots-clés:} \textit{Plethodon cinereus}, carabe, collembole, modèle d'équations structurelles, cooccurrence, coupe forestière.
\end{otherlanguage*}
                % résumé français
\chapter*{Abstract}             % ne pas numéroter
\label{chap-abstract}           % étiquette pour renvois
\phantomsection\addcontentsline{toc}{chapter}{\nameref{chap-abstract}} % inclure dans TdM

Forest environments play a major role on a global scale through their ecosystem functions and the economic benefits they provide. 
However, human population growth and increasing demand for wood products are leading to increased logging. 
These changes can lead to habitat loss essential for biodiversity and disrupt ecological dynamics between species inhabiting these ecosystems. 
This study aims to quantify the effects of silvicultural treatments, such as clearcutting and partial cutting, on soil fauna dynamics in temperate mixed forests. 
The objectives were, first, to assess the impact of logging on environmental variables influencing soil fauna habitat use and, second, 
to evaluate the direct and indirect effects of these treatments on soil amphibians and arthropods through trophic networks. 
We hypothesized that environmental variables associated with small fauna habitats (volume of coarse woody debris, canopy openness, and litter depth) vary depending on the intensity of logging. 
Thus, harvesting treatments would act as an overarching variable encompassing changes in environmental conditions. 
According to our second hypothesis, forest harvesting treatments would modify habitat use by Eastern red-backed salamanders (\textit{Plethodon cinereus}), ground beetles (\textit{Carabidae}), and springtails (\textit{Collembola}), 
either directly or through trophic interactions from predators to prey. 
To test these hypotheses, we developed a structural equation model (SEM) integrating occupancy models and linear mixed models. 
This approach allowed us to simultaneously measure, in a network framework, the effects of forest logging on environmental variables and the cooccurrence between different taxa. 
Our results show that harvesting treatments significantly influence environmental variables. 
Clearcutting led to greater canopy openness, reduced litter depth, and decreased coarse woody debris volume,  
whereas partial cutting allowed for better retention of these environmental attributes. 
Overall, silvicultural treatments had no significant direct effects on the occupancy probabilities of salamanders and ground beetles or the biomass of springtails. 
However, we measured a slightly lower occupancy probability for salamanders in clearcuts compared to partial cuts, although this effect was marginal (90\% CI). 
Moreover, no changes in cooccurrence between these groups were observed, suggesting that the effects of logging do not propagate through trophic networks from predators to prey. 
Our study highlights the relevance of structural equation models for analyzing complex trophic networks of forest ecosystems. 
This research deepens our understanding of the relationships between silvicultural practices and the dynamics of forest soil fauna. 
\hl{The environmental variables influencing the habitat of the studied species (coarse woody debris volume, litter depth, and canopy openness) vary depending on the intensity of logging. 
We therefore recommend promoting harvesting methods that minimize habitat disturbance, such as partial cuts. 
These practices help preserve key environmental attributes, benefiting soil fauna.} 
The knowledge gained from this study provides valuable tools for guiding management plans and adapting silvicultural practices. 

\begin{otherlanguage*}{english}
  \textbf{Keywords:} \textit{Plethodon cinereus}, ground beetle, springtail, structural equation modelling, cooccurrence, logging.
  
\end{otherlanguage*}
              % résumé anglais
\cleardoublepage

\setcounter{tocdepth}{2}
\setcounter{secnumdepth}{2}

\tableofcontents                % production de la TdM
\cleardoublepage

\listoftables                   % production de la liste des tableaux
\cleardoublepage

\listoffigures                  % production de la liste des figures
\cleardoublepage

\chapter*{Remerciements}        % ne pas numéroter
\label{chap-remerciements}      % étiquette pour renvois
\phantomsection\addcontentsline{toc}{chapter}{\nameref{chap-remerciements}} % inclure dans TdM


 Ce travail n'aurait pas pris sa forme actuelle sans le soutien que j'ai reçu tout au long de mon parcours de maîtrise.

 Mes remerciements vont d'abord à mon directeur, Marc J. Mazerolle, pour m'avoir acceuilli au sein de son laboratoire de recherche et pour son soutien constant tout au long de ce projet. 
 Sa confiance, ses précieux conseils en écologie, en statistique et en gestion de projet, ainsi que sa disponibilité ont été des atouts inestimables.
 Je souhaite également exprimer ma gratitude à mon co-directeur, Mathieu Bouchard, dont le soutien et les conseils dans un domaine que je maîtrisais peu, la foresterie, ont été d'une grande aide. 
 Sa vision du projet et ses recommandations m'ont permis de prendre des décisions plus judicieuses et éclairées.

 Je suis également reconnaissant envers Patricia Raymond et Émilie Champagne du Ministère des Ressources Naturelles et de Forêts, Direction de la recherche forestière, pour leur aide et 
 pour m'avoir permis de participer au projet enrichissant qu'est DREAM-Qc.
 Un grand merci également aux techniciennes Karine Thériault et Marie-Claude Martin pour leur soutien logistique et leurs conseils avisés.
 Je suis reconnaissant pour l'aide que j'ai reçu de la part de mon auxiliaire de terrain, Rebecca Dubé Messier. Sa motivation et son travail assidu ont largement contribué à la qualité de ce mémoire.

 Le parcours d'un projet de maîtrise et toujour plus facile quand on est bien entouré. 
 Pour cela je remercie les membres de mon laboratoire : 
 Aurore Fayard, Laura Millard, Mariano Feldman, Anais Baillet, Lucas Voirin, Félicia Beaulieu, Jeanne Dudemaine, Laurianne Plante, Naomie Herpin Saunier. 
 Votre soutien, nos activités extérieures ainsi que nos discussions ont été des moments précieux pour moi.
 Un merci spécial à Joëlle Spooner, qui a été une amie, une colocataire et une collègue de laboratoire exceptionnelle tout au long de ce parcours. 
 Sa présence a été un véritable soutien.

 Je souhaite également remercier mes parents, Isabelle et Patrick, mes beaux-parents, Michel et Anne-Chantal, ainsi que mes frères, Pierre et Guillaume, pour leur soutien depuis 
 mon plus jeune age. Votre support dans ma passion pour la biologie est inestimable.
 Je conclus ces remerciements en exprimant ma reconnaissance envers la personne qui m'accompagne au quotidien, Léonie St-Onge. 
 Merci pour ton écoute, ta patience, tes conseils et pour être simplement toi.

         % remerciements
\chapter*{Avant-propos}         % ne pas numéroter
\label{chap-avantpropos}        % étiquette pour renvois
\phantomsection\addcontentsline{toc}{chapter}{\nameref{chap-avantpropos}} % inclure dans TdM

%\usepackage[french]{babel}

Ce mémoire « par article » en Sciences forestières présenté à l'Université Laval est divisé en plusieurs parties : 
l'introduction, écrite en français, aborde en premier lieu la problématique liée à l'impact potentiel des traitements sylvicoles sur 
l'utilisation de l'habitat par la petite faune du sol dans le cadre d'un projet de migration assistée des arbres.
Le corps de ce document comprend le chapitre principal, intitulé 
"Direct and indirect effects on soil fauna of silvicultural treatments in the context of forest assisted migration", rédigé en anglais sous la forme d'un article scientifique. 
Suite au dépôt de ce mémoire, ce chapitre sera soumis à la revue scientifique \textit{Forest Ecology and Management}. 
La conclusion, rédigée en français, constitue la dernière partie de ce mémoire.
William Devos est l'auteur principal de cet article, suivi de Mathieu Bouchard, co-directeur de recherche, et Marc J. Mazerolle, directeur de recherche et dernier auteur. 
Les trois auteurs ont participé à l'élaboration du projet et à la rédaction. 
William Devos a pris en charge la planification du terrain, ainsi que la collecte et le traitement des données.
Marc J. Mazerolle et William Devos ont élaboré l'intégralité des modèles statistiques.
Marc J. Mazerolle et Mathieu Bouchard ont supervisé l'ensemble du projet de recherche.
           % avant-propos

\mainmatter                     % corps du document

\chapter*{Introduction}         % ne pas numéroter
\label{chap-introduction}       % étiquette pour renvois
\phantomsection\addcontentsline{toc}{chapter}{\nameref{chap-introduction}} % inclure dans TdM

% \usepackage[french]{babel}

\section*{Mise en contexte}
\label{sec:contexte}
\phantomsection\addcontentsline{toc}{section}{\nameref{sec:contexte}}

Les environnements forestiers jouent un rôle majeur à l’échelle planétaire à travers leurs fonctions écosystémiques et l'apport \hl{économique qu’ils représentent} en fournissant de nombreux biens et services \citep{Balvanera2006Quantifyingevidence}. 
Ces milieux permettent de réguler les flux de nutriments et d’énergie en participant notamment à la séquestration du carbone, à la stabilisation du climat, à la rétention de l’eau ainsi qu’à la conservation de la biodiversité \citep{Balvanera2006Quantifyingevidence,Diaz2006BiodiversityLoss,Canadell2008Managingforests,Pawson2013Plantationforests}. 
Toutefois, la croissance démographique et l’augmentation des besoins en produits ligneux amènent une hausse de l’exploitation forestière depuis maintenant plusieurs décennies \citep{Foley2005GlobalConsequences}. 
Par conséquent, le changement d’affectation des sols, causé par la récolte de bois, l’agriculture et l’urbanisation, occasionne une diminution globale du territoire forestier, entraînant une perte de biodiversité \citep{Bengtsson2000Biodiversitydisturbances,Sala2000Globalbiodiversity,Naeem2012functionsbiological,Bichet2016Maintaininganimal}. 
La biodiversité est cependant essentielle pour le bon fonctionnement des écosystèmes terrestres, et plusieurs engagements internationaux ont souligné l'urgence de freiner ce déclin et encouragent une gestion durable des forêts \citep{Newbold2015Globaleffects}. 

De nombreuses recherches ont démontré l'impact de la gestion forestière sur différents groupes fauniques, tels que les oiseaux, les mammifères, les reptiles, les amphibiens et les insectes \citep{Drapeau2000LandscapeScaleDisturbances,Suffice2015Shorttermeffects,bouchardBeetleCommunityResponse2016b,Hughes2019Impactnatural,Mazerolle2021Woodlandsalamander}. 
Les pratiques sylvicoles peuvent provoquer une hausse de mortalité chez les espèces, perturber ou restreindre leurs déplacements, intensifier les interactions biotiques, modifier leurs cycles de vie et altérer leur morphologie ainsi que leur physiologie \citep{Sergio2018Animalresponses}. 
De plus, l'exploitation des peuplements entraîne des pertes d'habitat et la fragmentation des milieux naturels, diminuant ainsi l'accès à la nourriture, aux refuges et aux sites de reproduction, tout en réduisant la taille et la diversité génétique des populations \citep{Bouderbala2023Longtermeffect}.
Ces pratiques causent également des problèmes de connectivité entre les habitats, les groupes d'espèces et les processus écologiques \citep{Lindenmayer2006Generalmanagement}. 
Une baisse de la connectivité peut compromettre la capacité des populations à se rétablir et à se maintenir après une perturbation, ce qui affecte négativement la conservation de la faune \citep{Lamberson1994ReserveDesign}. 
Cela restreint de la même façon le déplacement des individus au sein de leur habitat et le long des corridors écologiques, limitant ainsi le flux génétique entre populations et augmentant le risque d'extinction locale \citep{Saccheri1998Inbreedingextinction}. 

De plus, à l’échelle du paysage, l'exploitation forestière peut amener une homogénéisation du paysage en ramenant les peuplements à des stades précoces de succession, ce qui favorise une sous-représentation des forêts des stades de succession plus avancés \citep{Cyr2009Forestmanagement,Boucher2017Cumulativepatterns}. 
Cette pratique peut réduire la complexité structurelle des forêts, affectant parfois des aspects tels que la composition des espèces d'arbres, la stratification verticale, la structure d'âges, la dynamique de succession et la fréquence des perturbations \citep{Bergeron2000Speciesstand,Commarmot2005Structurevirgin,Varga2005Treesizediversity}. 
Ces modifications sont particulièrement préoccupantes pour la biodiversité, notamment pour les espèces qui vivent en forêts mixtes et conifériennes matures \citep{Tremblay2018Harvestinginteracts,Cadieux2020Projectedeffects}.
Le maintien de l'hétérogénéité structurelle joue un rôle important, car cela amène une variété d'habitats et permet de renforcer la connectivité en améliorant la dispersion de certaines espèces. 
C’est pourquoi l’aménagement forestier peut représenter un risque particulièrement important pour les espèces dépendantes des attributs de forêts complexes ainsi que pour celles ayant une faible capacité de dispersion  \citep{Norden2001Conceptualproblems,Martin2021indicatorspecies}. 
De plus, un peuplement avec une structure plus hétérogène est plus résilient lors d'une perturbation, car il facilite la régénération des espèces d'arbres présentes et le retour de celles qui ont disparu \citep{Kuuluvainen2009Forestmanagement}. 

Les modifications des caractéristiques forestières peuvent participer à la perte d'une partie de la diversité spécifique et fonctionnelle associée au milieu forestier. 
Certaines pratiques de gestion peuvent ainsi réduire la résilience des forêts à l'échelle locale et entraîner une diminution des services écosystémiques \citep{Hooper2012globalsynthesis,Edwards2014Maintainingecosystem}. 


\section*{Traitements sylvicoles}
\label{sec:sylvicole}
\phantomsection\addcontentsline{toc}{section}{\nameref{sec:sylvicole}}

Au cours des dernières décennies, les pratiques sylvicoles qui favorisent la sauvegarde de la biodiversité ont été encouragées afin de réduire les effets néfastes de la gestion intensive \citep{Gustafsson2012Retentionforestry}. 
\hl{L'aménagement écosystémique} a ainsi été suggérée comme une stratégie pour une gestion forestière durable \citep{Perry1998scientificbasis,Kuuluvainen2002Naturalvariabilitya}. 
Cette approche a pour but d'imiter les perturbations naturelles, les structures des peuplements et les dynamiques de succession qui en découlent. 
Elle s'inspire des processus naturels de la forêt, comme les ouvertures créées par les chablis, la mortalité partielle ou les perturbations à plus grande échelle comme les feux de forêt ou les épisodes cycliques de ravageurs \citep{Bergeron1999Forestmanagementa,Bergeron2007Usingknowledge}.
Elle permet également de préserver la biodiversité et d'entretenir la résilience des écosystèmes tout en assurant une variété de services écosystémiques \citep{Szaro1998emergenceecosystem,MacDicken2015Globalprogress}. 
\hl{Plus précisément, l'aménagement écosystémique utilise les traitements sylvicoles à l'échelle des peuplements, comme la coupe totale et la coupe partielle, pour reproduire les impacts écologiques associés aux perturbations naturelles. }
Toutefois, le niveau de perturbation varie selon le type de traitement appliqué, allant de perturbations plus légères à des perturbations plus intensives dans le milieu \citep{Thiffault2006Harvestingintensity}. 

Historiquement, les coupes totales sont parmi les pratiques sylvicoles les plus répandues dans les forêts tempérées et boréales \citep{Fedrowitz2014Canretention,Chaudhary2016Impactforest}. 
Elles font partie d’une gestion forestière permettant d’accroître la rentabilité des opérations forestières à relativement court terme, afin de répondre aux besoins des industries de transformation du bois \citep{Irland2011TimberProductivitya}. 
Sur le plan technique, elles sont relativement simples à réaliser, car l'intégralité du couvert forestier est retirée en une seule récolte, occasionnant une déforestation temporaire de la parcelle. 
Les coupes totales entraînent généralement une homogénéisation de la structure des peuplements \citep{Brashears2004AssessmentCanopy,Martin2020Forestmanagement}. 
\hl{Cette pratique peut ainsi être intégrée dans une approche de gestion écosystémique visant à imiter des perturbations naturelles de grande ampleur, telles que les feux de forêt, et à favoriser le maintien des stades précoces de la succession forestière} \citep{Bergeron1999Forestmanagementa}. 
\hl{De plus, certaines espèces d’arbres se régénèrent plus efficacement en coupe totale, en fonction de leur tolérance à l’ombre et de leur dépendance aux perturbations du sol} \citep{Bergeron1999Forestmanagementa}. 
Toutefois la simplification engendré par ce type de traitement peu parfois bouleverser des processus écologiques et évolutifs en amenant un changement drastique au niveau de la diversité biologique \citep{Holling2001UnderstandingComplexity}. 
Ainsi, l’utilisation de méthodes qui modifient radicalement la structure des milieux par rapport à leur état d’origine peut accroître le risque de déclin de certaines communautés biologiques \citep{Hanski2000Extinctiondebt}. 

Les coupes partielles sont un autre exemple d'\hl{aménagement écosystémique} visant à maintenir la composition, la structure et les fonctions de l'écosystème dans les limites de leur variabilité naturelle \citep{Bergeron1999Forestmanagementa,Raymond2009irregularshelterwood}.
Ce type de coupe implique une récolte sélective d'arbres, souvent orientée par des critères écologiques ou économiques spécifiques. 
Cela permet de conserver une couverture forestière partielle qui protège les semis et favorise une régénération naturelle \citep{Raymond2009irregularshelterwood}.
Les coupes partielles favorisent une structure forestière irrégulière en maintenant plusieurs classes d'âge ou de tailles d'arbres. 
Cette hétérogénéité peut inclure une diversité verticale et horizontale à l'intérieur des peuplements forestiers \citep{Raymond2009irregularshelterwood}.
\hl{Ainsi, les coupes partielles peuvent être appliquées afin de maintenir les stades de succession plus près de la dynamique naturelle des écosystèmes en forêt mixte} \citep{Bergeron1999Forestmanagementa}. 
Contrairement aux coupes totales, les coupes partielles ont une période prolongée de régénération, 
permettant une gestion à long terme et l'intégration d'objectifs de conservation \citep{Raymond2009irregularshelterwood}. 
Les coupes partielles \hl{peuvent permettre} une meilleure conservation des espèces associées au milieu forestier en maintenant des attributs importants comme les débris ligneux, 
plusieurs strates de canopée, différentes tailles d'arbres et un microclimat favorable, 
ce qui est bénéfique pour les espèces sensibles aux variations climatiques ou à la perte d'habitat \citep{Hansen1991Conservingbiodiversity}. 


\section*{Faune du sol}
\label{sec:soilfauna}
\phantomsection\addcontentsline{toc}{section}{\nameref{sec:soilfauna}}

Au sein de la biodiversité forestière, la faune du sol joue un rôle écologique prédominant en contribuant à la circulation de la matière et de l'énergie à travers la chaine alimentaire et en participant au recyclage des nutriments \citep{Seibold2021contributioninsects,Kudrin2023metaanalysiseffects}. 
Toutefois, ces organismes ont souvent une capacité de dispersion limitée en raison de leur petite taille, ce qui les rend vulnérables face aux pertes locales d’habitats \citep{Kudrin2023metaanalysiseffects}. 
Cette communauté est ainsi identifiée comme l'une des plus touchées par les perturbations environnementales \citep{Marshall2000Impactsforest,Coyle2017Soilfauna}. 

Les coupes forestières amènent généralement des changements brusques et importants dans les habitats de la faune du sol \citep{Lindo2003Microbialbiomass,Paillet2010Biodiversitydifferences,Fedrowitz2014Canretention,Chaudhary2016Impactforest}. 
Par exemple, le retrait de la canopée augmente l'exposition de la surface du sol au rayonnement solaire, entraînant une hausse des températures ainsi qu'une modification de l'humidité du sol \citep{Lindo2003Microbialbiomass,Brook2008Synergiesextinction,Zhang2022Intensiveforest}. 
Ces effets sont amplifiés par une augmentation de la vitesse du vent et une intensification des précipitations atteignant le sol \citep{Keenan1993ecologicaleffects,Heithecker2007Edgerelatedgradients}. 
Les modifications de la structure du sol, telles qu'une compaction accrue due au passage des machines forestières, réduisent également la porosité du sol et affectent négativement la faune qui y vit \citep{Battigelli2004Shorttermimpact}. 
De plus, les opérations forestières perturbent la disponibilité des nutriments en modifiant la qualité et la quantité de la litière, en altérant les sécrétions racinaires, en favorisant le lessivage et en modifiant les propriétés chimiques du sol \citep{Covington1981Changesforest,Marshall2000Impactsforest,Lindo2003Microbialbiomass,Battigelli2004Shorttermimpact}. 
Ces bouleversements des conditions microclimatiques et physico-chimiques ont des répercussions directes sur la biodiversité, 
notamment sur les organismes dépendant de microhabitats tels que le bois mort, les cavités des arbres matures ou les plaques racinaires \citep{Spies1999Dynamicforest,Christensen2005Deadwood,Brassard2008EffectsForest}. 
Ces éléments structurels, souvent réduits ou éliminés par les opérations de coupe, sont généralement remplacés par des pistes de débardage et des chemins d'exploitation \citep{Hansen1991Conservingbiodiversity}. 

La présence de microhabitats joue pourtant un rôle essentiel pour une grande partie de la faune du sol, notamment les arthropodes, les amphibiens et les microorganismes \citep{Paillet2010Biodiversitydifferences,Fedrowitz2014Canretention,Chaudhary2016Impactforest}. 
La disparition de ces structures augmente la vulnérabilité de nombreuses espèces, tout en créant des conditions environnementales défavorables à leur survie. 
Par conséquent, les espèces forestières qui dépendent de la fraîcheur et de l'humidité caractéristiques des sols non perturbés peuvent voir leurs populations décliner, voire disparaître localement \citep{Kudrin2023metaanalysiseffects}. 
À l'inverse, les espèces adaptées aux milieux plus ouverts et secs peuvent coloniser les parcelles récemment coupées, entraînant ainsi une modification de la composition spécifique. 

Plusieurs études se sont intéressées aux changements des caractéristiques forestières comme le débris ligneux grossier, la profondeur de la litière, l'ouverture de la canopée, 
ainsi que sur leur impact sur la faune du sol après les opérations de récolte, dans le but d'orienter la gestion forestière \citep{Semlitsch2002CriticalElements,McKenny2006Effectsstructural}. 
Cependant, la faune du sol englobe une grande diversité de taxons, présentant des différences biologiques et écologiques significatives \citep{Kudrin2023metaanalysiseffects}. 
En conséquence, leur réponse à l'exploitation forestière varie selon le type de traitement appliqué et le groupe d'espèces étudiées \citep{Malmstrom2009Dynamicssoil,Paillet2010Biodiversitydifferences}. 

Les amphibiens et les arthropodes jouent un rôle important dans l'écologie forestière, principalement au niveau de la chaine trophique et dans la circulation du carbone au niveau des sols \citep{Burton1975Energyflow,loveiEcologyBehaviorGround1996,Handa2014Consequencesbiodiversity}. 
Malgré le fait qu'ils soient abondants dans les forêts nord-américaines, ces groupes ne sont pas souvent pris en compte dans les plans de gestion forestière \citep{deMaynadier1995relationshipforest}. 
Pourtant, les amphibiens et les arthropodes ont connu un fort déclin au cours des dernières décennies, principalement en raison des pratiques d'aménagement forestier \citep{Houlahan2000Quantitativeevidence,Stuart2004Statustrends,Wagner2021Insectdecline}. 
Leur sensibilité aux changements environnementaux et leur capacité de dispersion limitée en font des modèles d'étude pertinents pour mieux comprendre les effets des pratiques forestières 
sur la biodiversité et l'intégrité écologique des forêts \citep{pongeVerticalDistributionCollembola2000,Ojala2001Dispersalmicroarthropods,birdChangesSoilLitter2004,Maleque2009Arthropodsbioindicators}. 
Parmi les groupes d'espèces fréquemment étudiés chez ces taxons, on retrouve la salamandre cendrée de l'Est (\textit{Plethodon cinereus} (Green, 1818)), les carabes (Carabidae) et les collemboles (Collembola). 

\subsection*{Salamandre cendrée}

La salamandre cendrée, de la famille des Plethodontidae, représente l’une des plus grandes biomasses de vertébrés dans de nombreuses forêts de l’est de l’Amérique du Nord \citep{Burton1975Salamanderpopulations,Petranka1993Effectstimber,semlitschAbundanceBiomassProduction2014a}. 
En tant qu'espèce exclusivement terrestre et dépourvue de poumons, elle dépend de la respiration cutanée et donc des conditions d'humidité pour assurer ses échanges gazeux \citep{Heatwole1961Relationsubstrate}. 
Ce mode de respiration la contraint à occuper des microhabitats spécifiques, en surface lorsque la température et l'humidité sont favorables, ou en profondeur dans le sol durant des périodes moins propices \citep{Grizzell1949HibernationSite,FraserEmpiricalEvaluation1976,Jaeger1980MicrohabitatsTerrestrial}. 
Le retrait des refuges à la surface, tels que les débris ligneux, après la récolte peut réduire la qualité de l'habitat pour les salamandres et diminuer le temps qu'elles passent à la surface du sol forestier \citep{Achat2015Quantifyingconsequences,Peele2017Effectswoody}. 
Étant donné que les salamandres cendrées se nourrissent et se reproduisent à la surface du sol forestier, des conditions de surface défavorables, un compactage des sols et de faibles niveaux de débris ligneux peuvent affecter la dynamique des populations \citep{Peterman2014Spatialvariation}. 
Prédateur généraliste, la salamandre cendrée contribue de manière significative à la régulation des invertébrés détritivores, influençant directement les processus de décomposition, la circulation des nutriments dans les sols et la dynamique du carbone \citep{Burton1975Energyflow,Wyman1998Experimentalassessment,Walton2013Topdownregulation,Hickerson2017Easternredbacked}. 
Par ailleurs, elle constitue une proie de haute valeur nutritive pour divers prédateurs tels que les oiseaux, les mammifères et les reptiles, renforçant ainsi son rôle dans les dynamiques trophiques forestières \citep{Burton1975Energyflow,Pough1987abundancesalamanders}. 
Compte tenu de sa sensibilité aux perturbations environnementales, notamment en raison de sa respiration cutanée, la salamandre cendrée est souvent utilisée comme indicateur de la qualité des sols forestiers \citep{Welsh2001caseusing,Fisher-Reid2024Easternredbacked}. 
Son abondance et sa présence sont des paramètres couramment évalués pour mesurer l'impact des pratiques sylvicoles sur la biodiversité \citep{Harpole1999Effectsseven,Grialou2000effectsforest,Homyack2009Longtermeffects,Hocking2013Effectsexperimental,Mazerolle2021Woodlandsalamander}. 

\subsection*{Carabes}

Avec environ 40 000 espèces répertoriées, les carabes constituent l'une des familles les plus diversifiées parmi les coléoptères et représentent l'une des plus fortes abondances d'arthropodes du sol \citep{Erwin1985taxonpulse,loveiEcologyBehaviorGround1996,Rochefort2006GroundBeetle}. 
Ils sont largement distribués et abondants dans presque tous les écosystèmes terrestres, bien que les espèces varient dans leur sélection d'habitats \citep{loveiEcologyBehaviorGround1996,kotzeFortyYearsCarabid2011a,Larochelle2003naturalhistory}. 
Les changements environnementaux et les modifications d'habitat favorisent certaines espèces au détriment d'autres, ce qui rend leur diversité et leur sensibilité particulièrement intéressantes pour étudier l'impact des perturbations environnementales \citep{Rainio2003Groundbeetles}. 
Ces insectes ont une courte durée de vie, occupent une position élevée dans la chaine alimentaire et réagissent rapidement et de manière complexe aux modifications de leur environnement \citep{loveiEcologyBehaviorGround1996}.
Les carabes jouent un rôle écologique dans la régulation des populations d'invertébrés, se nourrissant principalement d'aphides, de collemboles et de gastéropodes, tout en étant à leur tour des proies pour divers amphibiens, reptiles, oiseaux et mammifères \citep{loveiEcologyBehaviorGround1996}. 
Les carabes sont particulièrement sensibles à la complexité structurelle des peuplements forestiers, et ce, à différentes échelles temporelles et spatiales \citep{Butterfield1995Carabidbeetle,loveiEcologyBehaviorGround1996,Niemela2007effectsforestry}.

\subsection*{Collemboles}

Parmi les taxons de la mésofaune du sol, les collemboles représentent un des groupes les plus abondants et les plus diversifiés de petits arthropodes \citep{rusekBiodiversityCollembolaTheir1998}. 
Différentes communautés de collemboles occupent un ensemble de niches écologiques allant de la litière aux différents horizons du sol \citep{pongeVerticalDistributionCollembola2000}.
La répartition verticale de ces communautés dépend essentiellement des conditions abiotiques du sol telles que la luminosité, le taux d’humidité ou encore la porosité.
Ces invertébrés exercent une influence sur l'écologie forestière à travers leur influence sur les micro-organismes, sur la décomposition et le cycle des nutriments.
Étant principalement fongivores et détritivores, les différentes communautés de collemboles influencent les taux de décomposition du bois, affectent la structure physique et les taux de minéralisation de la litière, 
influencent l'absorption des nutriments et régulent la communauté microbienne, en plus de participer à la formation de microstructure du sol \citep{Petersen1982comparativeanalysis,Neher2012Linkinginvertebrate,Maass2015Functionalrole,Potapov2016Connectingtaxonomy}. 
Les collemboles jouent également un rôle dans chaine trophique en constituant une ressource alimentaire abondante pour plusieurs organismes comme les amphibiens, les coléoptères, les arachnides, les oiseaux et les reptiles \citep{Burton1975Energyflow,Bauer1982Predationcarabid,rusekBiodiversityCollembolaTheir1998}.

\section*{Objectifs et hypothèses}
\label{sec:objectifs}
\phantomsection\addcontentsline{toc}{section}{\nameref{sec:objectifs}}

Grâce à leurs relations trophiques, leur sensibilité aux variations des conditions environnementales et leur dépendance aux caractéristiques forestières, ces trois groupes d'espèces constituent des modèles pertinents pour étudier l'impact des pratiques forestières sur la faune du sol. 
La plupart des travaux qui se sont intéressés aux impacts des traitements sylvicoles sur la faune discutent généralement des effets directs des perturbations sur un ou plusieurs groupes d'espèces, 
sans tenir compte des relations existantes entre les variables environnementales et les différents groupes d'espèces, 
négligeant ainsi les effets indirects des coupes sur la faune du sol \citep{josephIntegratingOccupancyModels2016,Pollierer2021Diversityfunctional,Kudrin2023metaanalysiseffects}. 
\hl{Mon projet a pour but de combler cette lacune en essayant de comprendre comment les effets à court terme de différents types de récolte forestière se propagent à l’intérieur du réseau écologique forestier et influencent la dynamique de la faune du sol. }
Ultimement, ces nouvelles connaissances permettront de mieux informer la gestion durable des forêts.
Les objectifs liés à cette étude sont :

\begin{enumerate}
    \item De quantifier l'effet des traitements de coupes forestières sur les variables environnementales qui influencent l'utilisation de l'habitat par la faune du sol. 
    \item De mesurer l'impact des coupes forestières sur l'utilisation de l'habitat par la faune du sol.
\end{enumerate}

L’hypothèse 1.1 attachée à notre premier objectif soutient que les variables environnementales (volume de débris ligneux, ouverture de la canopée, profondeur de litière) favorables à l’utilisation de l’habitat par les espèces varient en fonction de l’intensité des coupes forestières. 
En ce sens, l’intensité du traitement de coupe est utilisée comme indicateur global représentant les changements dans les conditions environnementales.

L'hypothèse 2.1 liée à notre deuxième objectif stipule que les traitements de coupe forestière entraînent une modification de l'utilisation de l'habitat 
par la faune du sol et que ces effets se propagent à travers le réseau trophique, allant des prédateurs vers les proies. 
Spécifiquement, les coupes affectent l'utilisation de l'habitat par les salamandres et les grands carabes (compétiteurs de salamandres), 
ce qui modifie ensuite la sélection d'habitats des petits carabes (proies de salamandres), puis enfin des collemboles (proies de grands carabes et salamandres). 
Cette hypothèse repose sur le fait que la salamandre cendrée, en tant que prédateur sensible aux perturbations, est reconnue pour son rôle important dans la régulation des communautés d'invertébrés du sol forestier \citep{Wyman1998Experimentalassessment,MichaelWalton2005Salamandersforestfloor,Walton2006Salamandersforestfloor,Walton2013Topdownregulation,Hickerson2017Easternredbacked}. 
Parallèlement, les carabes ont une relation complexe avec les salamandres, se positionnant à la fois comme compétiteurs et proies de celles-ci \citep{Jaeger1980MicrohabitatsTerrestrial,loveiEcologyBehaviorGround1996,Gall2003BehavioralInteractions}. 


\cleardoublepage

\bibliography{References.bib}
\bibliographystyle{ecologyNewFR.bst}
          % introduction
\chapter{Direct and indirect effects on soil fauna of silvicultural treatments in the context of forest assisted migration}     % numéroté
\label{chapitre1-articles}    

William Devos$^1$, Mathieu Bouchard$^1$, Marc Mazerolle$^1$

%href{mailto:william.devos.1@ulaval.ca}
$^1$ Centre d'étude de la forêt, Département des sciences du bois \\ 
et de la forêt, Université Laval, Québec, QC G1V 0A6, Canada. \\ 

\clearpage

\section*{Résumé}
\label{sec:resume1}
\phantomsection\addcontentsline{toc}{section}{\nameref{sec:resume1}}

\begin{otherlanguage*}{french}
  <Résumé de l'article en français. Obligatoire.>

  \textbf{Mots-clés} : <ajouter des mots clés>
\end{otherlanguage*}

\clearpage

\section*{Abstract}
\label{sec:abstract1}
\phantomsection\addcontentsline{toc}{section}{\nameref{sec:abstract1}}

\begin{otherlanguage*}{english}
  <English abstract of the paper. Optional, but recommended.>

\textbf{Keywords}: <add some keywords> 
\end{otherlanguage*}

\cleardoublepage

\section*{Introduction}
\label{sec:intro1}
\phantomsection\addcontentsline{toc}{section}{\nameref{sec:intro1}}

%\defcitealias{keylist}{alias}

Due to their ecosystem services and economic values, forests play a predominant role on a global scale \citep{Balvanera2006Quantifyingevidence}. 
Within terrestrial ecosystems, they maintain significant biodiversity and act as regulators of biogeochemical factors \citep{Pawson2013Plantationforests}. 
However, the reality of climate change poses urgent challenges for the sustainability of current forests \citep{McKenney2009Climatechange,Messier2022Warningnatural,Seidl2017Forestdisturbances,Trumbore2015Foresthealth}. 
Despite international commitments to reduce greenhouse gas emissions, climate projections predict a global temperature exceeding 1.5 \up{o}C above pre-industrial temperatures \citep{Matthews2022Currentglobal}. 
Canada is particularly vulnerable to this warming, due to its northern latitudes \citep{Alo2008Potentialfuture,Bush2019Canadachanging} and forests in eastern North America will be particularly affected by this disruption \citep{Park2014Canboreal,Mahony2017closerlook,Messier2022Warningnatural,Sittaro2017Treerange}. 
Various studies have predicted lengthening and intensification of drought periods, an increase in wildfires and a higher presence of biotic disturbances \citep{Gatti2021Amazoniacarbon,Heidari2021Effectsclimate,Joyce2013Climatechange,Parmesan2007Influencesspecies}. 
Additionally, shifts in phenology and plant distribution are expected \citep{Aitken2008Adaptationmigration,Chuine2010Whydoes,Gray2013Trackingsuitable,Zhu2012Failuremigrate}. 
However, climate changes are occurring more rapidly than trees ability to adapt or migrate \citep{Aitken2008Adaptationmigration,Harrison2020Plantcommunity,Loarie2009velocityclimate,Messier2022Warningnatural,Williams2013Preparingclimate,Vitt2010Assistedmigration}, 
consequently threatening the growth and survival of these species \citep{Sittaro2017Treerange,Woodall2018Decadalchanges,Zhu2012Failuremigrate}.
This would ultimately result in a shift in forest composition, impacting forest management and conservation efforts \citep{Chmura2011Forestresponses,Lo2011Linkingclimate,McKenney2009Climatechange}.

Several calls for adaptation in forest management have been made to preserve forest ecosystems and their benefits \citep{Messier2021sakeresilience,Nagel2017Adaptivesilviculture}. 
Among the proposed solutions, assisted tree migration is suggested as a mitigation measure involving the movement of individuals or genetic material from their original climatic range to a more suitable area for species survival and growth in the future \citep{Dumroese2015Considerationsrestoring,Palik2022Operationalizingforestassisted,Park2023Provenancetrials,Park2018Informationunderload,Pedlar2011implementationassisted,Vitt2010Assistedmigration,Williams2013Preparingclimate}. 
This movement would rapidly change stand composition, meeting conservation needs, maintaining ecosystem services and preserving economic value \citep{Pedlar2011implementationassisted,Ste-Marie2011Assistedmigration,Winder2011Ecologicalimplications}. 
However, there remains a lack of knowledge and uncertainty surrounding assisted tree migration \citep{Park2018Informationunderload,Klenk2015assistedmigration}, particularly regarding the trade-offs between preserving one species and the risks to the ecosystem of the host territory \citep{Hewitt2011Takingstock,McLachlan2007frameworkdebate,Vitt2010Assistedmigration}.

To address this knowledge gap and reduce uncertainty, various forestry scenarios are currently under examination to mitigate risks \citep{royoDesiredREgenerationAssisted2023}. 
Therefore, the study of different silvicultural interventions such as overstory treatments is considered since they influence stand growth, health and composition \citep{Ameray2021Forestcarbon,Chaudhary2016Impactforest,Man2008Elevenyearresponses,MontoroGirona2018ConiferRegeneration,PamerleauCouture2015Effectthree}. 
For example, clear-cut treatments involve the removal of all trees within a designated area and is commonly used in intensive forest management plans focused on increasing wood productivity and quality over a short period to meet industry needs \citep{Ameray2021Forestcarbon}. 
On the other hand, partial-cut treatments entail selective tree removal, maintaining a portion of the stand and are usually applied in extensive management plans that favor natural regeneration, mimic natural disturbances and preserve the ecological value habitats \citep{Ameray2021Forestcarbon,Barg1999Influencepartial,Irland2011Timberproductivity,Tong2020Forestmanagement}. 
However, forest harvesting leads to changes in environmental conditions, such as soil compaction, increased solar exposure, higher winds and increased precipitation reaching the forest floor \citep{Keenan1993ecologicaleffects,Lindo2003Microbialbiomass,Heithecker2007Edgerelatedgradients}. 
Ultimately, these changes can affect nutrient availability and impact soil biodiversity \citep{Battigelli2004Shorttermimpact,Chaudhary2016Impactforest,Covington1981Changesforest,Fedrowitz2014Canretention,Kudrin2023metaanalysiseffects,Lindo2003Microbialbiomass,Paillet2010Biodiversitydifferences,rousseauLongtermEffectsBiomass2018}.
Soil fauna plays a crucial role in forest ecosystems by contributing to the circulation of matter and energy \citep{Kudrin2023metaanalysiseffects,Seibold2021contributioninsects}. 
Amphibians and arthropods are among the groups most often affected by environmental disturbances such as forestry practices \citep{Hartshorn2021reviewforest,Semlitsch2009Effectstimber,Stuart2004Statustrends} or climate change \citep{Alford1999Globalamphibian,Houlahan2000Quantitativeevidence,Milanovich2010Projectedloss,Parmesan2006EcologicalEvolutionary,Pounds2006Widespreadamphibian,Warren2018projectedeffect}.
Consequently, to better understand the impact of  treatments on soil fauna, we chose to study the Eastern Red-backed Salamander (\textit{Plethodon cinereus} (Green, 1818)), ground beetles (Carabidae) and springtails (Collembola).

The Eastern Red-backed Salamander is one of the most significant biomass among vertebrates in North American forests \citep{Burton1975Salamanderpopulations,Petranka1993Effectstimber,semlitschAbundanceBiomassProduction2014a}. 
Like other Plethodontidae, this salamander is strictly terrestrial and relies on skin respiration due to the absence of lungs \citep{Heatwole1961Relationsubstrate}. 
It occupies forest soils when temperature and humidity levels are optimal for cutaneous respiration. 
Outside these periods, it will move vertically in the soil to maintain favorable conditions to its survival \citep{Grizzell1949HibernationSite}. 
This species has a small home range and typically shows philopatric behavior \citep{Yurewicz2004ResourceAvailability}. 
Its role is significant in forest ecosystems as it acts as a generalist predator and regulates detritivore invertebrate populations \citep{Burton1975Energyflow,Hickerson2017Easternredbacked,Walton2013Topdownregulation}. 
It also serves as prey in trophic networks and constitutes a rich energy food source \citep{Burton1975Energyflow,Pough1987abundancesalamanders}. 
Due to its cutaneous respiration, this salamander is highly sensitive to environmental disturbances \citep{Welsh2001caseusing} and is commonly used as a bioindicator \citep{Baecher2018Environmentalgradients,gibbsDistributionWoodlandAmphibians1998,Heatwole1962EnvironmentalFactors,Harpole1999Effectsseven,Hocking2013Effectsexperimental,Mazerolle2021Woodlandsalamander}.

On their part, ground beetles gather the highest specific diversity among beetles with 40,000 identified species and represent one of the most abundant groups among soil arthropods \citep{Erwin1985taxonpulse,loveiEcologyBehaviorGround1996,Rochefort2006GroundBeetle}. 
As voracious carnivores and polyphagous predators, they act as regulators of invertebrate populations \citep{loveiEcologyBehaviorGround1996}. 
Ground beetles are also prey for several species of amphibians, reptiles, birds and mammals \citep{loveiEcologyBehaviorGround1996}. 
While widely distributed in terrestrial ecosystems, habitat selection varies among species \citep{Larochelle2003naturalhistory}. 
Ground beetles are often classified into three species communities : mature and closed forest species, open habitat species and generalist species \citep{Niemela2007effectsforestry}. 
This variation in habitat selection makes ground beetles an interesting taxon to study during environmental disturbances \citep{bouchardBeetleCommunityResponse2016b,Halme1993Carabidbeetles,Heliola2001Distributioncarabid,koivulaBorealCarabidbeetleColeoptera2002a,Luff1992Classificationprediction,Niemela2001Carabidbeetles,Rainio2003Groundbeetles,Work2008Evaluationcarabid}.

As for springtails, they are polyphyletic species of arthropods belonging to the mesofauna established in forest soils. 
These invertebrates have a high species richness and represent a significant abundance \citep{rusekBiodiversityCollembolaTheir1998}. 
Different springtail communities occupy a range of ecological niches from litter to various soil horizons \citep{pongeVerticalDistributionCollembola2000}.
The vertical distribution of these communities depends mainly on abiotic conditions such as light, humidity, or porosity. 
Therefore, springtails can be used to characterize a substrate based on the community found there \citep{rusekBiodiversityCollembolaTheir1998}. 
Primarily fungivore and detritivore, these organisms play a predominant ecological role by feeding largely on fungi, bacteria, actinomycetes and algae. 
They contribute to the decomposition of organic matter, nutrient transformation and energy transfer in terrestrial ecosystems \citep{Cuchta2019importantrole,Hattenschwiler2005Biodiversitylitter,Marsden2020Howagroforestry,Petersen2000Collembolapopulations,rusekBiodiversityCollembolaTheir1998,Wolters1991SoilInvertebrates}. 
Springtails also represent a food source for several species of arachnids, beetles, amphibians, reptiles and birds. 
This group of species is commonly used in studies focusing on the effects of environmental changes on forest soils and mesofauna \citep{farskaManagementIntensityAffects2014,rousseauWoodyBiomassRemoval2019,Salmon2008Relationshipssoil}.

These three taxa are relevant to study the impact of silvicultural treatments as their sensitivity to environmental changes and trophic relationships enable the analysis of disturbance effects on soil fauna dynamics.
Several studies have already studied the effects of those treatments on soil fauna.
However, most of them have only focused on the direct impacts of disturbances for one or more species group, neglecting the existing relationships between environmental variables and species groups \citep{josephIntegratingOccupancyModels2016,Kudrin2023metaanalysiseffects,Pollierer2021Diversityfunctional}. 
It is essential to have a better comprehension of those treatments effects, their propagation within the ecological network and their repercussions on soil organism communities.
Ultimately, this knowledge acquisition will provide useful tools to facilitate sustainable forest management.

Our study aims to understand how silvicultural practices, conducted in an assisted tree migration context, affect the dynamics of forest soil ecosystems. 
The objectives associated with this aim are firstly to measure the impact of overstory treatments on habitat use by soil fauna, and secondly to compare overstory 
treatments and environmental variables that influence habitat use by soil fauna.
Regarding our first objective, we hypothesized that overstory treatments result in a modification of habitat use by soil fauna and propagate 
through the trophic network. Specifically, the harvests initially affect habitat use by salamanders and large ground beetles, 
followed by modifications in habitat selection for small ground beetles, and ultimately affecting springtail biomass.
The hypothesis associated with our second objective suggested that environmental variables, favorable to habitat use by taxa, fluctuate according to 
the intensity of forest harvests. Therefore, overstory treatments can be used as a variable encompassing changes in environmental conditions.
As a result, more intense treatments would decrease litter depth and CWD, as a consequence of reduced accumulation of leaves and woody debris on the forest floor, 
and increase canopy openness. 


\section*{Material and methods}
\label{sec:matmet1}
\phantomsection\addcontentsline{toc}{section}{\nameref{sec:matmet1}}

\subsection*{Study area}
\label{subsec:area}
\phantomsection\addcontentsline{toc}{subsection}{\nameref{subsec:area}}

\begin{otherlanguage*}{english}
  Our study was conducted within the Portneuf Wildlife Reserve in the Captiale-Nationale administrative region near Lac des Amanites and Rivière-à-Pierre (47°07'N, 72°24'W, Figure \ref{fig:area}). 
  This area is located within the balsam fir (\textit{Abies balsamea})-yellow birch (\textit{Betula alleghaniensis}) bioclimatic domain \citep{saucierChapitreEcologieForestiere2009}.
  Other tree species include sugar maple (\textit{Acer saccharum}), red maple (\textit{Acer rubrum}), white spruce (\textit{Picea glauca}), black spruce (\textit{Picea mariana}), red spruce (\textit{Picea rubens}), white birch (\textit{Betula papyrifera}) and quaking aspen (\textit{Populus tremuloides}) \citep{olaBelowgroundCarbonStocks2024}. 
  Study sites rest on a deep glacial till as surface deposit with a moderately well-drained sandy loams soil \citep{CanadianSystemSoil1998}.
  The mean daily temperature is 4\up{o}C based on the 1981-2010 period at the nearest weather station (Lac aux sables, \citealp{environmentcanadaCanadianClimateNormals2019}). 
  Based on the same report, the mean annual precipitation and snowfall are 1133.2 mm and 230.3 cm, respectively.
  We used the assisted migration experimental system established in 2018 by the Ministère des Ressources naturelles et des Forêts to collect our data (\citealp{royoDesiredREgenerationAssisted2023}).
  This system is a factorial experimental design with a split-plots replicated in four blocks. 
  Each whole block (200 m x 140 m) is split in two overstory treatment : clear-cut and partial-cut. 
  See \cite{royoDesiredREgenerationAssisted2023} for more details about the assisted migration experiment.

\end{otherlanguage*}

\begin{figure}[ht!]
	\centering
	\includegraphics[scale=0.60]{fig_area4.png}
	\caption[Localization of the Captiale-Nationale administrative region in Quebec, Canada and position of the study area near Lac des Amanites in Portneuf Wildlife Reserve, Quebec, Canada.]
  {Localization of the Captial-Nationale administrative region in Quebec, Canada (A) and position of the study area near Lac des Amanites in Portneuf Wildlife Reserve, Quebec, Canada (B) where the assisted migration experimental system was implemented in 2018 (47°07'N, 72°24'W).}
	\label{fig:area}
	\end{figure}  



\subsection*{Sampling design}
\label{subsec:sampling}
\phantomsection\addcontentsline{toc}{subsection}{\nameref{subsec:sampling}}


We selected a total of 60 sampling units measuring 10 meters by 7.5 meters each to collect our data: we used four blocks serving as replicates, 
with each block containing six sampling units for both the clear-cut and partial-cut overstory treatments, 
while three additional sampling units were positioned outside the block, serving as controls (Figure \ref*{fig:blockSU}).
Sampling units serving as controls were separated from blocks by at least 10 meters to remove treatments effects.
In each sampling unit, we used three sampling methods to collect species data: artificial coverboards, pitfall traps, and soil cores. 

Artificial coverboards is commonly employed to count salamanders \citep{hesedUncoveringSalamanderEcology2012,Mazerolle2021Woodlandsalamander,mooreComparisonPopulationEastern2009c}. 
This method helps to standardize the number and size of sampled ground objects under which animals will hide, thereby reducing variability \citep{hydeSamplingPlethodontidSalamanders2001}. 
Coverboards were made of untreated spruce wood, measured 25 cm x 30 cm x 5 cm and were placed directly on the ground without litter underneath \citep{Mazerolle2021Woodlandsalamander}. 
Six coverboards were set per sample unit and spaced by at least 2.5 m, resulting in a total of 360 coverboards.
Two rows of three boards each were aligned along the length of the sampling units and centrally positioned (Figure \ref{fig:blockSU}).
All coverboards were placed outdoors in March 2022 to allow for natural aging, increasing the likelihood of salamander usage \citep{hedrickEffectsCoverboardAge2021,smithEffectsCoverBoard2014a}.
Throughout the surveys, coverboards were inspected once on the same day, and salamanders were counted without any manipulation.

Pitfall traps were used to capture ground beetles.
This method is commonly employed to assess the abundance and species richness of soil invertebrates, such as ground beetles \citep{baarsCatchesPitfallTraps1979,knappEffectPitfallTrap2012,kotzeFortyYearsCarabid2011a,loveiEcologyBehaviorGround1996,spenceSamplingCarabidAssemblages1994a}. 
Pitfall traps were Multipher\up{\textregistered{}} traps and included a container with a diameter of 12.5 cm, a depth of 25 cm and a cover raised 4.5 cm above the trap 
to prevent debris and rain from filling the container \citep{bouchardBeetleCommunityResponse2016b,Jobin1988MultiPherinsect,mooreEffectsTwoSilvicultural2004}.
They were equipped with a protective grid with a mesh size of 15 mm, limiting trap access to carabid-sized individuals and reducing the chances of predation by small mammals. 
Typically, a preserving liquid (such as propylene glycol or alcohol) is placed in the bottom of the container to preserve captured individuals. 
In this study, to avoid killing salamanders that are small enough to pass through the grid, we decided to use dry pitfall traps without any preserving liquid \citep{luffFeaturesInfluencingEfficiency1975}. 
We added wet sponges to the bottom of each container to maintain a suitable level of humidity for salamanders.
We centered one pitfall in each sampling and control units, resulting in a total of 60 pitfall traps (Figure \ref{fig:blockSU}). 
Traps were inserted in the soil at a depth allowing the container's opening to be juxtaposed with the soil surface. 
During periods when the traps were not visited daily, all pitfall traps were closed with adhesive tape around the opening to prevent individuals capture. 
On the first day of each survey, the traps were opened and captured ground beetles were collected for each remaining day for a five day period. 
Individuals were preserved in 70\% alcohol and identified at the species level afterwards.
Identification was conducted with a ZEISS SteREO Discovery.V12 bionocular microscope using \cite{larochelleManuelIdentificationCarabidae1976} identification keys.
Ground beetles were categorized in two groups as salamander prey or competitors based on the salamander gape size (Table \ref{tab:carabid}, \citealp{jaegerFoodLimitedResource1972,magliaModulationPreycaptureBehavior1995,magliaOntogenyFeedingEcology1996}).

Soil cores was used to sample springtails and are commonly employed to assess mesofauna in litter and different soil horizons \citep{chauvatChangesSoilFaunal2011a,farskaManagementIntensityAffects2014,pongeVerticalDistributionCollembola2000,salamonEffectsPlantDiversity2004,wuCompositionSpatiotemporalVariation2014}. 
Two soil cores with litter collection were harvested per sampling unit per survey using a soil sampling pedological probe. 
Cores had a diameter of 5 cm, a depth of 5 cm and a 15 cm x 15 cm litter quadrat was collected above each soil sample \citep{raymond-leonardSpringtailCommunityStructure2018a,rousseauForestFloorMesofauna2018}.
Both substrates were used to target mesofauna and obtain springtail communities directly related to the ecology of salamanders and ground beetles \citep{chauvatChangesSoilFaunal2011a,edwardsAssessmentPopulationsSoilinhabiting1991,raymond-leonardSpringtailCommunityStructure2018a,rousseauForestFloorMesofauna2018}.
Soil and litter from the same unit were pooled in Ziploc\up{\texttrademark{}} bags and stored in a cooler at $\pm$ 4 °C \citep{chauvatChangesSoilFaunal2011a,rousseauForestFloorMesofauna2018}, providing 60 samples per survey.
Each sample was placed in an individual Tullgren dry-funnel for springtail extraction within 48 h after they were collected \citep{rousseauForestFloorMesofauna2018,rusekBiodiversityCollembolaTheir1998,wuCompositionSpatiotemporalVariation2014}. 
The extraction process lasted six days with a gradual temperature increase (25 °C to 50 °C) \citep{raymond-leonardSpringtailCommunityStructure2018a}.
Springtails were preserved in 75\% alcohol \citep{wuCompositionSpatiotemporalVariation2014} prior to isolation from surrounding organisms, with subsequent identification at the family level.
Identification was done with a ZEISS SteREO Discovery.V12 binocular microscope and a Leitz orthoplan phase-contrast fluorescent trinocular microscope using \cite{bellingerChecklistCollembolaWorld1996} identification keys.
Springtail dry biomass of each sampled was measured with micro balance (Sartorius Cubis\up{\texttrademark{}} MSA3.6P-000-DM, city, state, country) after being lyophilized (Labconco FreeZone Bulk tray dryer 78060 series, city, state, country).

We conducted four surveys of five consecutive days each, namely in mid-May, mid-June, mid-July, and mid-August, during the summer 2022.
Blocks were visited in a random sequence to reduce effects of time of day and observer fatigue.

\pagebreak

\begin{figure}[ht]
	\centering
	\includegraphics[scale=0.50]{fig_blockSU2.png}
	\caption[Design of one block and one sampling unit with three sampling methods.]{
  Design of a block (left) and a sampling unit (right). 
  The block contains two overstory treatments : clear-cut (grey background), partial-cut (white background). 
  Fifteen sampling units were used per block : six per overstory treatment and three controls (\textbf{c}) outside each block.
  Each sampling unit contained six artificial coverboards (rhombus) and one pitfall trap (triangles). Two soil cores (circles) were collected per survey.
  }
	\label{fig:blockSU}
	\end{figure}  

\vspace{0.5cm}


\subsection*{Environmental variables}
\label{subsec:EnvVar}
\phantomsection\addcontentsline{toc}{subsection}{\nameref{subsec:EnvVar}}

In each sampling unit, we measured several environmental variables that could affect occupancy probability.
CWD and litter depth play a crucial role in habitat use for salamanders, ground beetles and springtails as
they serve for feeding and protection  \citep{birdChangesSoilLitter2004,groverInfluenceCoverMoisture1998a,harmonEcologyCoarseWoody1986,koivula.LeafLitterSmallscale1999,mckennyEffectsStructuralComplexity2006,patrickEffectsExperimentalForestry2006a}. \\
Salamanders also utilize CWD as shelter to maintain suitable temperature and moisture levels during dry periods \citep{groverInfluenceCoverMoisture1998a,Jaeger1980MicrohabitatsTerrestrial,patrickEffectsExperimentalForestry2006a}
We used 400 m\up2 plots centered inside every sampling unit to estimate CWD (20 m $\times$  20 m)(\citealp{methotGuideInventaireEchantillonnage2014}). 
We only selected CWD with a basal diameter greater than or equal to 9 cm and a length greater than or equal to 1 m.
Subsequently, we measured the basal diameter, the apical diameter, and length of CWD with a tree caliper.
Segments of CWD outside the plot boundaries were not considered.
We employed the conic–paraboloid formula to estimate the volume of each CWD \citep{fraverRefiningVolumeEstimates2007} :

\begin{equation}
  \text{Volume} = L/12 \times (5A_b + 5A_u + 2\sqrt{A_b \times A_u})
\end{equation}

\vspace{0.5cm}

Where $L$ is the length of log (cm), $A_b$ the basal area (cm\up{2}), and $A_u$ the apical area (cm\up{2}).
We measured litter depth next to each coverboard and estimated the mean depth per sampling unit \citep{Mazerolle2021Woodlandsalamander}. \\
We also assessed canopy openness, as it may influence species occupancy \citep{henneronForestPlantCommunity2017,koivulaBorealCarabidbeetleColeoptera2002a,kotzeFortyYearsCarabid2011a,messereForestFloorDistribution1998,tilghmanMetaanalysisEffectsCanopy2012}.
Measurements were conducted at the center of all sampling units using a spherical densiometer \citep{lemmonSphericalDensiometerEstimating1956} and were taken 130 cm above the ground. 
We took four measurements per sampling unit, oriented toward each of the four cardinal points, and computed the mean as an estimate of canopy openness in each sampling unit.

We collected data for air temperature, air humidity and precipitation levels during the summer 2022.
Two compact weather stations (Em50 Digital Decagon Data Logger, Part \#40800, Meter Group Inc., USA), were used inside both overstory treatments.
Each weather station was equipped with a probe measuring temperature, air humidity, and atmospheric pressure, 1.30 m above the ground (VP-4 Sensor (Temp/RH/Barometer), Part \#40023). 
Rain gauges were installed in the clear-cut treatments to monitor precipitation levels.
The temperature and humidity sensors were programmed for record data every 15 minutes. 
We used the means of both weather stations to get daily average measurements.
These variables fluctuate on a daily basis, affecting species activity and, consequently, the probability of detecting individuals. 
\citep{butterfieldCarabidLifeCycle1996,kotzeFortyYearsCarabid2011a,loveiEcologyBehaviorGround1996,odonnellPredictingVariationMicrohabitat2014a,spotilaRoleTemperatureWater1972}.

\subsection*{Statistical analyses}
\label{subsec:analyses}
\phantomsection\addcontentsline{toc}{subsection}{\nameref{subsec:analyses}} 

% Expliquer pourquoi d'un coté on regarde les variables emvironnementales et pourquoi de l'utre on regarde les espèces
% POurquoi ne pas intégré les varable du sol avec les espèces
% Expliquer comment les les modèles d'occupation et les GLM sont inclus dans le SEM comment il sont intgrer
% expliquer dans les methodes les analyses de comparaison entre coupe forestière et variables environnementales
% préciser que l'enseble de test sont fait avec le SEM

\subsubsection{Structural equations models} 

To assess the effects of overstory treatments on the habitat use (hypothesis 1.1) and the relationship between overstory treatments and environmental variables (hypothesis 2.1), 
we employed a structural equation models (SEM) combining occupancy models and linear mixed models (LMM) \citep{graceSpecificationStructuralEquation2010,josephIntegratingOccupancyModels2016,mackenzieOccupancyEstimationModeling2006a}.
This approach enables us to test both hypotheses within the same analysis using the same data and method, thereby reducing estimation variations.
One part of the SEM was designed to evaluate the direct and indirect effects of overstory treatments on taxa (Figure \ref*{fig:SEM}), 
while another section focused on studying the variations in environmental variables across different overstory treatments. 
SEM are employed to analyze complex relationships among observed and latent variables (unobserved but inferred from observed data) to investigate multicausal ecological processes \citep{graceStructuralEquationModeling2008}.

We combined occupancy models with SEM to measure the impact of overstory treatments on the occupancy probabilities of salamanders and ground beetles. 
Occupancy models allow accounting for imperfect detection since salamanders and ground beetles are cryptic species \citep{baileyEstimatingSiteOccupancy2004,spiersEstimatingSpeciesMisclassification2022}.
This method use repeated surveys to estimate the presence probabilities of species and the probability of detecting them if they are present, 
leading to an estimation of the species true occupancy \citep{mackenzieEstimatingSiteOccupancy2002,mazerolleMakingGreatLeaps2007}.

By combining SEM and occupancy models, the occupancy probability estimated from the occupancy models becomes a latent variable in the SEM. 
The SEM then explores how the explanatory variables directly and indirectly affect this latent occupancy variable.
This approach enables us to investigate complex relationships influencing the presence of salamanders and ground beetles 
while addressing the issue of imperfect detection.

LMM were incorporated into the SEM to estimate parameters of springtail biomass and environmental variables.
LMM allows us to account for random variations of blocks, enhancing model accuracy. 
Those models can handle data with hierarchical structures and can estimate the amount of variation due to fixed effects while also accounting for 
random effects.




\begin{figure}[ht]
	\centering
	\includegraphics[scale=0.55]{fig_sem.png}
	\caption[Theoretical model illustrating the anticipated relationships between overstory treatments, environmental variables and species groups.]
  {Theoretical model illustrating the anticipated relationships between overstory treatments, coarse woody debris volume, canopy openness, litter depth,
   salamander occupancy, ground beetle occupancy, and springtail biomass in the Portneuf Wildlife Reserve, Quebec, Canada. 
   Each arrow indicates the direction of a potential effect, from an explanatory variable to a response variable.}
	\label{fig:SEM}
\end{figure}  

\subsubsection{Occupancy models} 


To estimate the occupancy probabilities of salamanders and ground beetles, we used the observed detection at site $i$ during survey $j$, 
represented by $\text{Salamander}_{ij}$, $\text{Carabid.comp}_{ij}$ and $\text{Carabid.prey}_{ij}$. Observed detection followed a Bernoulli distribution with $z_{i} \times p_{ij}$ as parameter, 
where $z_{i}$ represents the latent occupancy state of species groups at a site $i$ and $p_{ij}$ represents the probability of detecting species groups at a site $i$ during a survey $j$ :


\begin{align}
  \text{Salamander}_{ij} &\sim \text{Bernoulli}(z_{\text{Salamander}_i} \times p_{\text{Salamander}_{ij}}) \nonumber \\
  \text{Carabid.comp}_{ij} &\sim \text{Bernoulli}(z_{\text{Carabid.comp}_i} \times p_{\text{Carabid.comp}_{ij}})  \\
  \text{Carabid.prey}_{ij} &\sim \text{Bernoulli}(z_{\text{Carabid.prey}_i} \times p_{\text{Carabid.prey}_{ij}}) \nonumber
\end{align}


$z_{i}$ follows a Bernoulli distribution with $\psi$ as parameter, 
which represent the occupancy probability of each species group. 
The latent occupancy state $z$ indicates whether a site is occupied by a species group ($z = 1$) or not ($z = 0$) :


\begin{align}
  z_{\text{Salamander}_i} &\sim \text{Bernoulli}(\psi_{\text{Salamander}_{\text{Group}_i}}) \nonumber \\
  z_{\text{Carabid.comp}_i} &\sim \text{Bernoulli}(\psi_{\text{Carabid.comp}_{\text{Group}_i}}) \\
  z_{\text{Carabid.prey}_i} &\sim \text{Bernoulli}(\psi_{\text{Carabid.prey}_{i}}) \nonumber
\end{align}


Salamanders and large ground beetle occupancy probabilities ($\psi_{\text{Salamander}_{\text{Group}_i}}$, $\psi_{\text{Carabid.comp}_{\text{Group}_i}}$ ) are drawn from uniform distribution $\text{U}(0, 1)$ for each overstory group $i$, 
whereas small ground beetle occupancy probabilities ($\psi_{\text{Carabid.prey}_{i}}$) were estimated from a linear predictor with salamander latent occupancy state ($z_{\text{Salamander}_i}$) and overstory treatments 
($\text{Cutpartial}_i$, $\text{Cutclear}_i$) as explanatory variables on a logit scale. 
$\beta$s represent the coefficients associated with each parameter affecting occupancy in the linear predictor. 
We assumed vague normal priors, $\text{N}(0, \sigma = 10)$ for those $\beta$ parameters :


\begin{align}
  \text{logit}(\psi_{\text{Carabid.prey}_i}) &= \beta_{0[\text{Carabid.prey}]} + \beta_{z_{\text{Salamander}}[\text{Carabid.prey}]} \times z_{\text{Salamander}_i} + \nonumber \\
  &\beta_{\text{Cutpartial}[\text{Carabid.prey}]} \times \text{Cutpartial}_i + \\
  &\beta_{\text{Cutclear}[\text{Carabid.prey}]} \times \text{Cutclear}_i \nonumber
\end{align}

We applied a logit scale on parameters influencing the probability of detecting species groups ($\text{logit}(p_{ij})$), allowing us to use 
the coarse woody debris ($\text{CWD}_i$), the precipitation levels ($\text{Precipitation}_{ij}$) as explanatory variables with a block random effect ($\alpha_{Block}$). 
$\alpha$s represent the coefficients associated with each explanatory variable affecting detection in the linear predictor. 
We assumed vague normal priors for CWD, precipitation levels $\text{N}(0, \sigma = 10)$ and block random effects $\text{N}(0, \sigma_{Block})$ 
where $\sigma_{Block}$ is drawn form uniform distribution $\text{U}(1, 10)$. 
Air temperature and relative humidity were excluded from detection variables due to a high correlation with the precipitation level.


\begin{align}
  \text{logit}(p_{\text{Salamander}_{ij}}) &= \alpha_{0[\text{Salamander}]} + \alpha_{\text{CWD}[\text{Salamander}]} \times \text{CWD}_i + \nonumber \\
  &\alpha_{\text{Precipitation}[\text{Salamander}]} \times \text{Precipitation}_{ij} + \alpha_{\text{Block}[\text{Salamander}]_{\text{Block}_i}} \nonumber
\end{align}

\begin{align}
  \text{logit}(p_{\text{Carabid.comp}_{ij}}) &= \alpha_{0[\text{Carabid.comp}]} + \alpha_{\text{CWD}[\text{Carabid.comp}]} \times \text{CWD}_i + \\
  &\alpha_{\text{Precipitation}[\text{Carabid.comp}]} \times \text{Precipitation}_{ij} + \alpha_{\text{Block}[\text{Carabid.comp}]_{\text{Block}_i}} \nonumber 
\end{align}

\begin{align}
  \text{logit}(p_{\text{Carabid.prey}_{ij}}) &= \alpha_{0[\text{Carabid.prey}]} + \alpha_{\text{CWD}[\text{Carabid.prey}]} \times \text{CWD}_i + \nonumber \\
  &\alpha_{\text{Precipitation}[\text{Carabid.prey}]} \times \text{Precipitation}_{ij} + \alpha_{\text{Block}[\text{Carabid.prey}]_{\text{Block}_i}} \nonumber 
\end{align}

%%%%%%%%%%%%%%%%%%%
%%%%%%%%%%%%%%%%%%% fin


% \begin{align}
%   z_{\text{Salamander}_i} &\sim \text{Bernoulli}(\psi_{\text{Salamander}_{\text{Group}_i}}) \nonumber \\
%   \text{logit}(p_{\text{Salamander}_{ij}}) &= 
%   \alpha_{0[\text{Salamander}]} + \alpha_{\text{CWD}[\text{Salamander}]} \times \text{CWD}_i + \\
%   &\alpha_{\text{Precipitation}[\text{Salamander}]} \times \text{Precipitation}_{ij} + \alpha_{\text{Block}[\text{Salamander}]_{\text{Block}_i}} \nonumber \\
%   \text{Salamander}_{ij} &\sim \text{Bernoulli}(z_{\text{Salamander}_i} \times p_{\text{Salamander}_{ij}}) \nonumber
% \end{align} 

% \begin{align}
% z_{\text{Salamander}_i} &\sim 
% \text{Bernoulli}(\psi_{\text{Salamander}_{\text{Group}_i}}) \nonumber \\
% \text{logit}(p_{\text{Salamander}_{ij}}) &= 
% \alpha_{0[\text{Salamander}]} + \alpha_{\text{CWD}[\text{Salamander}]} \times \text{CWD}_i + \\
% &\alpha_{\text{Precipitation}[\text{Salamander}]} \times \text{Precipitation}_{ij} + \alpha_{\text{Block}[\text{Salamander}]_{\text{Block}_i}} \nonumber \\
% \text{Salamander}_{ij} &\sim 
% \text{Bernoulli}(z_{\text{Salamander}_i} \times p_{\text{Salamander}_{ij}}) \nonumber
% \end{align} 

% \begin{align}
%   z_{\text{Carabid.comp}_i} &\sim 
%   \text{Bernoulli}(\psi_{\text{Carabid.comp}_{\text{Group}_i}}) \nonumber \\
%   \text{logit}(p_{\text{Carabid.comp}_{ij}}) &= 
%   \alpha_{0[\text{Carabid.comp}]} + \alpha_{\text{CWD}[\text{Carabid.comp}]} \times \text{CWD}_i + \\
%   &\alpha_{\text{Precipitation}[\text{Carabid.comp}]} \times \text{Precipitation}_{ij} + \alpha_{\text{Block}[\text{Carabid.comp}]_{\text{Block}_i}} \nonumber \\
%   \text{Carabid.comp}_{ij} &\sim \text{Bernoulli}(z_{\text{Carabid.comp}_i} \times p_{\text{Carabid.comp}_{ij}}) \nonumber
%   \end{align}

% \begin{align}
%   \text{logit}(\psi_{\text{Carabid.prey}_i}) &= 
%   \beta_{0[\text{Carabid.prey}]} + \beta_{z_{\text{Salamander}}[\text{Carabid.prey}]} \times z_{\text{Salamander}_i} + \nonumber \\
%   &\beta_{\text{Cutpartial}[\text{Carabid.prey}]} \times \text{Cutpartial}_i + \beta_{\text{Cutclear}[\text{Carabid.prey}]} \times \text{Cutclear}_i \nonumber\\
%   z_{\text{Carabid.prey}_i} &\sim 
%   \text{Bernoulli}(\psi_{\text{Carabid.prey}_i}) \nonumber \\
%   \text{logit}(p_{\text{Carabid.prey}_{ij}}) &= \alpha_{0[\text{Carabid.prey}]} + \alpha_{\text{CWD}[\text{Carabid.prey}]} \times \text{CWD}_i +  \\
%   &\alpha_{\text{Precipitation}[\text{Carabid.prey}]} \times \text{Precipitation}_{ij} + \alpha_{\text{Block}[\text{Carabid.prey}]_{\text{Block}_i}} \nonumber \\
%   \text{Carabid.prey}_{ij} &\sim \text{Bernoulli}(z_{\text{Carabid.prey}_i} \times p_{\text{Carabid.prey}_{ij}}) \nonumber
% \end{align}

\subsubsection{Linear mixed models} 

We used linear mixed models to assess how overstory treatments affect springtail biomass ($\text{Springtail}_{i}$) and 
environmental variables ($\text{CWD}_{i}$, $\text{Canopy}_{i}$, $\text{Litter}_{i}$) at site $i$ :

\begin{align}
  \text{Springtail}_{i} &\sim \text{N} (\mu_{\text{Springtail}_i}, \sigma_{\text{Springtail}_{\text{Group}_i}}) \nonumber \\
  \text{CWD}_{i} &\sim \text{N} (\mu_{\text{CWD}_i}, \sigma_{\text{CWD}_{\text{Group}_i}}) \\
  \text{Canopy}_{i} &\sim \text{N} (\mu_{\text{Canopy}_i}, \sigma_{\text{Canopy}_{\text{Group}_i}}) \nonumber \\
  \text{Litter}_{i} &\sim \text{N} (\mu_{\text{Litter}_i}, \sigma_{\text{Litter}_{i}}) \nonumber 
\end{align}

Springtail biomass and environmental variables estimations are drawn from Normal distributions where means estimations at site $i$ ($\mu_{i}$) included overstory treatments ($\text{Cutpartial}_i$, $\text{Cutclear}_i$) and block random effect ($\alpha_{\text{Block}}$) as linear predictors. 
Estimations of $\mu_{\text{Springtail}_i}$ also accounted for latent occupancy state of salamanders and both ground beetles groups ($z_{Salamander}$, $z_{SCarabid.comp}$, $z_{Carabid.prey}$ ). 
We assumed vague normal priors for overstory treatments, latent occupancy states, $\text{N}(0, \sigma = 10)$ and block random effects $\text{N}(0, \sigma_{Block})$ 
where $\sigma_{Block}$ is drawn form uniform distribution $\text{U}(0, 50)$. 
The $\beta$ coefficients represent the weights assigned to each explanatory in the linear predictor.
Due to heteroscedasticity with $\text{Springtail}$, $\text{CWD}$ and $\text{Canopy}$, we allowed each overstory groups $j$ to have their own variances ($\sigma_j \sim \text{U}(0,150)$) :

\begin{align}
  \mu_{\text{Springtail}_i} &= \beta_{0[\text{Springtail}]} + \beta_{\text{Cutpartial}[\text{Springtail}]} \times \text{Cutpartial}_i + \nonumber\\
  &\beta_{\text{Cutclear}[Springtail]} \times \text{Cutclear}_i + \beta_{z_{\text{Salamander}}[\text{Springtail}]} \times z_{Salamander} +  \nonumber\\
  &\beta_{z_{\text{Carabid.prey}}[\text{Springtail}]} \times z_{Carabid.prey} + \beta_{z_{\text{Carabid.comp}}[\text{Springtail}]} \times z_{Carabid.comp} + \nonumber\\
  &\alpha_{\text{Block}[\text{Springtail}]_{\text{Block}_i}} \nonumber
\end{align}

\begin{align}
  \mu_{\text{CWD}_i} &= \beta_{0[\text{CWD}]} + \beta_{\text{Cutpartial}[\text{CWD}]} \times \text{Cutpartial}_{i} + \nonumber\\
  & \beta_{\text{Cutclear}[\text{CWD}]} \times \text{Cutclear}_{i} + \alpha_{\text{Block}[\text{CWD}]_{\text{block}_i}} 
\end{align}


\begin{align}
  \mu_{\text{Canopy}_i} &= \beta_{0[\text{Canopy}]} + \beta_{\text{Cutpartial}[\text{Canopy}]} \times \text{Cutpartial}_{i} + \nonumber \\
  & \beta_{\text{Cutclear}[\text{Canopy}]} \times \text{Cutclear}_{i} + \alpha_{\text{Block}[\text{Canopy}]_{\text{block}_i}} \nonumber
\end{align}

\begin{align}
  \mu_{\text{Litter}_i} &= \beta_{0[\text{Litter}]} + \beta_{\text{Cutpartial}[\text{Litter}]} \times \text{Cutpartial}_{i} + \nonumber\\
  & \beta_{\text{Cutclear}[\text{Litter}]} \times \text{Cutclear}_{i} + \alpha_{\text{Block}[\text{Litter}]_{\text{block}_i}} \nonumber
\end{align}


Analysis was performed with a Bayesian approach and parameters were estimated using Markov chain Monte Carlo (MCMC) with JAGS 4.3.0 include in the jagsUI package in R 4.3.1 \citep{lunnBUGSProjectEvolution2009,kellnerJagsUIWrapperRjags2024,rcoreteamLanguageEnvironmentStatistical2020}. 
We ran the model with five chains, 200,000 iterations each \citep{gelmanUnderstandingPredictiveInformation2014}. A burn-in period of 75,000 iterations was used and a thinning rate of 5 was applied. 
We verified the convergence of MCMC chains by examining trace plots, posterior density plots, and applying the Brooks-Gelman-Rubin statistic. 
The JAGS model code is available in Table

\clearpage

\section*{Results}
\label{sec:results1}
\phantomsection\addcontentsline{toc}{section}{\nameref{sec:results1}}


\subsection*{Soil fauna}
\label{subsec:taxa}
\phantomsection\addcontentsline{toc}{subsection}{\nameref{subsec:taxa}} 

Model diagnostics indicated that the chain lengths were sufficient, as the Brooks-Gelman-Rubin statistic was below 1.1. 
Trace plot analysis revealed that all chains had converged towards similar values, and none of the ratios of MCMC error to posterior standard deviation exceeded 5\%.

\vspace{10pt}

\begin{figure}[ht]
	\centering
	\includegraphics[scale=0.60]{fig_sem_res3.png}
	\caption[Results from structural equation modeling analysis revealing effects of overstory treatments on coarse woody debris volume,
  canopy openness, litter depth, salamander occupancy, carabid occupancy, and springtail biomass.]
  {Results from SEM analysis showing effects of overstory treatments on CWD, 
  canopy openness, litter depth, salamander occupancy, carabid occupancy, and springtail biomass in the Portneuf Wildlife Reserve, 
  Quebec, Canada. Bold arrows represent significant effects, while gray arrows indicate no discernible effects. 
  Values above bold arrows represent average differences between posterior distributions of two overstory groups: 
  partial-cut (PC), clear-cut (CC), and control (C). Estimates marked with one asterisk (*) 
  indicate a 90\% credible interval (CI) excluding 0, while estimates marked with two asterisks (**) indicate a 95\% CI excluding 0.}
	\label{fig:SEMres}
\end{figure}  

\vspace{10pt}

The average salamander observation per survey was 8 individuals with a range of 17 salamanders, during the summer 2022.
We captured 189 ground beetles belonging to 30 species, with 49 ground beetles found in the partial-cut treatments, 105 in the clear-cuts, and 35 in the control sites (Table \ref{tab:carabid}). 
We collected 468 springtails representing 12 families, with 219 springtails collected from partial-cut treatments, 131 from clear-cuts, and 118 from the control areas (Table \ref{tab:springtail}). 
The average springtail biomass collected per overstory treatments was 24.3 $\mu$g (SD = 18.2 $\mu$g), 56.8 $\mu$g (SD = 78.0 $\mu$g), and 31.1 $\mu$g (SD = 52.8 $\mu$g) in the partial-cut treatments, clear-cuts and control sites, respectively.

The effects of overstory treatments on habitat selection by species groups were generally not significant. 
Salamander occupancy probability was marginally lower in sites subjected to clear-cutting compared to those with partial-cutting (90\% CI : [-0.74, -0.07], Figure \ref{fig:pcin}, Table \ref{tab:overstorysp}). 
However, these two groups did not differ from the control sites. 
Occupancy probability for both carabid groups and the springtail biomass, did not vary significantly between the overstory treatments and control sites (Table \ref{tab:overstorysp}). 
Overall, the presence of salamanders had no significant impact on the occupancy probabilities of ground beetles representing prey for salamanders, and we did not observe significant effects of either salamanders or ground beetles on springtail biomass (Table \ref{tab:overstorysp}).

\vspace{10pt}

\begin{table}[ht]
  \centering
  \caption[Differences between overstory treatments on the forest soil fauna.]
  {Differences between overstory treatments on salamander occupancy, both ground beetle groups occupancy and springtail biomass. 
  This table also show estimated effect of salamanders presence on ground beetles form the salamanders prey group occupancy 
  and the effects of salamanders and both ground beetle groups presence on springtail biomass, during the summer 2022 in the Portneuf Wildlife Reserve, Quebec, Canada.}
  \label{tab:overstorysp}
  \begin{tabular}{lllll} 
      \hline
      &&&&95\% Bayesian \\
      Variable&Unit& Comparison & Estimate &  credible interval \\ [0.5ex] 
      \hline     
      Salamander           &probability& Partial vs control & \hspace{1mm}0.07 & [-0.29, 0.45] \\ 
      occupancy       && Clear vs control  & -0.38 & [-0.75, 0.11] \\ 
                          && Clear vs partial  & -0.45 & [-0.74, -0.07]$^{a}$ \\       
      Carabid$_{competitor}$ &probability& Partial vs control & -0.12 & [-0.35, 0.15] \\
      occupancy       && Clear vs control  & -0.06 & [-0.29, 0.20] \\ 
                          && Clear vs partial  & \hspace{1mm}0.06 & [-0.19, 0.30] \\ 
      Carabid$_{prey}$    &logit& Partial vs control & \hspace{1mm}3.31 & [-10.12, 17.72] \\
      occupancy             && Clear vs control  & \hspace{1mm}10.19 & [-4.15, 24.45] \\ 
                          && Clear vs partial  & \hspace{1mm}6.88 & [-12.81, 23.42] \\  
                          && Salamander        & -2.20 & [-17.15, 16.59] \\  
      Springtail          &$\mu$g& Partial vs control & \hspace{1mm}8.11 & [-9.38, 25.40] \\
      biomass             && Clear vs control  & \hspace{1mm}2.11 & [-13.98, 18.11] \\ 
                          && Clear vs partial  & -6.00 & [-29.09, 17.26] \\  
                          && Salamander        & \hspace{1mm}6.80 & [-10.43, 23.26] \\ 
                          && Carabid$_{competitor}$      & \hspace{1mm}0.56 & [-16.75, 17.66] \\ 
                          && Carabid$_{prey}$      & \hspace{1mm}7.62 & [-8.93, 24.09] \\ 
      \hline
      \multicolumn{5}{l}{\textbf{Note:} Estimates from Bayesian SEM are presented in terms of mean number with 95\%} \\
      \multicolumn{5}{l}{credible intervals, where an interval excluding 0 indicates a difference between groups.} \\
      \multicolumn{5}{l}{$^{a}$Marginal difference based on 90\% Bayesian credible interval excluding 0}
  \end{tabular}
\end{table}


\clearpage

\begin{figure}[ht]
  \centering
  \includegraphics[scale=0.55]{fig_pcin.png}
  \caption[Occupancy probability of salamanders under overstory treatments]
  {Occupancy probability of salamanders within two overstory treatments and controls during the summer 2022 in the Portneuf Wildlife Reserve device, Quebec, Canada. 
  Error bars denote 95\% credible intervals around estimates.}
  \label{fig:pcin}
\end{figure}

\vspace{10pt}

We did not observe significant impacts of CWD and precipitation level on salamander detection probabilities. 
However, the precipitation level had a positive effect on detection probability for both small ground beetles (95\% CI : [0.59, 1.77]) and large carabid (95\% CI : [0.70, 3.23]) (Table \ref{tab:detection}). 
CWD had no significant impact on ground beetle detection probabilities.


\begin{table}[ht]
  \centering
  \caption[Estimated effects of coarse woody debris and precipitation level on detection probabilities of salamanders and both ground beetle groups.]
  {Estimated effects of coarse woody debris and precipitation level on detection probabilities of salamanders and both ground beetle groups, during the summer 2022 in the Portneuf Wildlife Reserve,  Quebec, Canada.}
  \label{tab:detection}
  \begin{tabular}{lllll} 
      \hline
      &&&95\% Bayesian \\
      Variable & Taxa & Estimate &  credible interval \\ [0.5ex] 
      \hline      
      Precipitation       & Salamander              & \hspace{1mm}0.11 & [-0.83, 1.06] \\ 
                          & Carabid$_{competitor}$  & \hspace{1mm}1.17 & [0.59, 1.77] \\ 
                          & Carabid$_{prey}$        & \hspace{1mm}1.87 & [0.70, 3.23] \\  
      \hline      
      Coarse woody debris & Salamander              & -0.59 & [-1.39, 0.12] \\ 
                          & Carabid$_{competitor}$  & \hspace{1mm}0.06 & [-0.26, 0.38] \\ 
                          & Carabid$_{prey}$        & \hspace{1mm}0.27 & [-0.74, 1.37] \\   

      \hline
      \multicolumn{4}{l}{\textbf{Note:} Estimates from Bayesian SEM are presented in terms of mean} \\
      \multicolumn{4}{l}{number with 95\% credible intervals, where an interval excluding 0 indicates} \\
      \multicolumn{4}{l}{a difference between groups.} \\
  \end{tabular}
\end{table}


\subsection*{Environmental variables}
\label{subsec:ResEnv}
\phantomsection\addcontentsline{toc}{subsection}{\nameref{subsec:ResEnv}} 

Environmental variables usually differed between forest cutting treatments and control conditions. 
We found that clear-cutting treatments had significantly less CWD compared to partial-cutting (95\% CI : [-1.15, -0.43]) (Figure \ref{fig:envar} A, Table \ref{tab:overstoryenvar}). 
However, these both treatments did not differ from the control sites. 
Canopy openness was significantly higher in both the partial-cut (95\% CI : [1.97, 11.02]) and clear-cut treatments (95\% CI : [51.39, 77.06]) when compared 
to the control sites, with the clear-cuts having a greater percentage of openness than the partial-cuts (95\% CI : [44.61, 70.76], Figure \ref{fig:envar} B, Table \ref{tab:overstoryenvar}). 
Conversely, litter depth was lower in both the partial-cut (95\% CI : [-2.44, -0.65]) and clear-cut treatments (95\% CI : [-4.28, -2.50]) compared to the controls, 
with the depth being lower in the clear-cuts than in the partial-cuts (95\% CI : [-2.57, -1.12], Figure \ref{fig:envar} C, Table \ref{tab:overstoryenvar}).

\vspace{10pt}

\begin{figure}[ht]
  \centering
  \includegraphics[scale=0.23]{fig_envar2.png}
  \caption[Environmental variables estimations with a potential effect soil species within two different overstory treatments and control.]
  {Environmental variables with a potential effect soil species estimations within two different overstory treatments and control 
  during the summer 2022 in the Portneuf Wildlife Reserve, Quebec, Canada. Error bars denote 95\% credible intervals around estimates.}
  \label{fig:envar}
\end{figure}

\begin{table}[ht]
  \centering
  \caption[Differences between overstory treatments on environmental variables that could affect habitat selection of fauna on the forest soil.]
  {Differences between overstory treatments on environmental variables that could affect habitat selection of fauna on the forest soil during the summer 2022 in the Portneuf Wildlife Reserve,
  Quebec, Canada.}
  \label{tab:overstoryenvar}
  \begin{tabular}{lllll} 
      \hline
      &&&&95\% Bayesian \\
      Variable&Unit& Comparison & Estimate &  credible interval \\ [0.5ex] 
      \hline
      Coarse woody debris &m\up{3}& Partial vs control & \hspace{1mm}0.02 & [-1.01, 1.06] \\ 
                 && Clear vs control  & -0.77 & [-1.79, 0.23] \\ 
                          && Clear vs partial  & -0.79 & [-1.15, -0.43] \\
      Canopy openness     &\%& Partial vs control & \hspace{1mm}6.49 & [1.97, 11.02] \\ 
                      && Clear vs control  & \hspace{1mm}64.19 & [51.39, 77.06] \\ 
                          && Clear vs partial  & \hspace{1mm}57.69 & [44.61, 70.76] \\ 
      Litter depth        &cm& Partial vs control & -1.54 & [-2.44, -0.65] \\ 
                      && Clear vs control  & -3.39 & [-4.28, -2.50] \\ 
                          && Clear vs partial  & -1.85 & [-2.57, -1.12] \\       
      \hline
      \multicolumn{5}{l}{\textbf{Note:} Estimates from Bayesian SEM are presented in terms of mean number with 95\%} \\
      \multicolumn{5}{l}{credible intervals, where an interval excluding 0 indicates a difference between groups.} \\
  \end{tabular}
\end{table}




\clearpage

\section*{Discussion}
\label{sec:discu1}
\phantomsection\addcontentsline{toc}{section}{\nameref{sec:discu1}}

\section*{Conclusion}
\label{sec:conclu1}
\phantomsection\addcontentsline{toc}{section}{\nameref{sec:conclu1}}

\section*{Acknowledgements}
\label{sec:acknowl1}
\phantomsection\addcontentsline{toc}{section}{\nameref{sec:acknowl1}}

\section*{Conflict of interest}
\label{sec:conflict1}
\phantomsection\addcontentsline{toc}{section}{\nameref{sec:conflict1}}

None declared
\section*{Author contributions}
\label{sec:author1}
\phantomsection\addcontentsline{toc}{section}{\nameref{sec:author1}}

\cleardoublepage

\begin{otherlanguage}{english}
\bibliographystyle{ecologyNewEN} % Style de citation en français
\bibliography{References}
\addcontentsline{toc}{section}{References}
\end{otherlanguage}
    % chapitre 1
\chapter*{Conclusion générale}           % ne pas numéroter
\label{chap-conclusion}         % étiquette pour renvois
\phantomsection\addcontentsline{toc}{chapter}{\nameref{chap-conclusion}} % inclure dans TDM

Le but de cette étude était de comprendre comment les traitements sylvicoles, effectués dans un contexte de migration assistée, affectent la dynamique des écosystèmes du sol forestier.
Les objectifs qui s’y rattachaient étaient, d'une part, de quantifier l'effet des traitements de coupes forestières sur les variables environnementales qui influencent 
l'utilisation de l'habitat par la faune du sol et, d'autre part, de mesurer l'impact des coupes forestières sur l'occupation de cet habitat par la faune du sol.
Pour répondre à ces deux objectifs, nous avons développé un modèle d'équations structurelles pour mesurer l'effet des coupes sur le volume de débris ligneux, 
la profondeur de litière et l'ouverture de la canopée, ainsi que sur la probabilité d'occupation des salamandres cendrées, des carabes et sur la biomasse de collemboles.
Notre première hypothèse stipulait que les variables environnementales favorables à l'utilisation de l'habitat par les espèces fluctuent 
en fonction de l'intensité des coupes forestières et qu'ainsi, les traitements de coupes forestières constituent une variable englobant les changements de conditions environnementales. 
Notre seconde hypothèse soutenait que les traitements de coupe forestière entraînent une modification de l'utilisation de l'habitat par la faune du sol et se propagent à travers 
le réseau trophique des salamandres vers les collemboles, en passant par les carabes. 
Les principales conclusions tirées de cette étude sont :

\begin{enumerate}
  \item Les coupes forestières ont un effet significatif sur les variables environnementales influençant l'utilisation de l'habitat par la faune du sol. Ces changements suivent de manière générale le niveau d'intensité du traitement appliqué, avec des perturbations plus importantes dans les coupes totales, suivis des coupes partielles.
  \item La probabilité d'occupation des salamandres et des carabes ne semble globalement pas affectée par les différents traitements de coupes forestières comparativement aux sites témoins. Toutefois, un faible effet a été mesuré chez les salamandres où la probabilité d'occupation de celles-ci était marginalement plus faible dans les coupes totales par rapport aux coupes partielles.
  \item La biomasse des collemboles ne variait pas de façon significative entre les différents traitements sylvicoles.
  \item La relation de cooccurence entre les trois taxons ne semble pas subir de changement en fonction du type de coupe. Ainsi, la biomasse des collemboles n'est pas affectée par le changement de probabilité d'occupation des salamandres et des carabes, et la présence des salamandres ne modifie pas la probabilité d'occupation des carabes de petite taille.
\end{enumerate}


\subsection{Résultat, limitations et ouvertures}

Notre étude a révélé que les coupes forestières peuvent avoir un impact négatif sur certaines variables environnementales importantes pour la faune du sol. 
Ces effets sont particulièrement prononcés dans les traitements de coupe totale, où l'on observe une réduction du volume de débris ligneux, une augmentation significative de l'ouverture de la canopée et une diminution de la profondeur de la litière. 
L'étude a également permis d'observer que les traitements de coupe partielle permettent une meilleure rétention des attributs environnementaux, puisque les effets pour les trois variables mesurées étaient moindres dans ce type de traitement comparativement aux témoins.
Malgré les changements environnementaux observés, notre étude ne permet pas de conclure que les traitements de coupe forestière influencent directement ou indirectement l'occupation de l'habitat par la salamandre cendrée et les carabes ou la biomasse des collemboles. 
Notre hypothèse concernant la relation entre les taxons et le modèle d'équation structurelle qui en découle suggérait que l'effet des perturbations causées par les coupes forestières se propage à travers le réseau trophique des prédateurs vers les proies. 
Toutefois, la récolte forestière ne semble pas affecter la cooccurrence entre ces trois groupes d'espèces, selon ce type de relation. 

Une autre approche serait d'envisager que les perturbations environnementales affectent en premier lieu le bas de la chaîne alimentaire, avant de se propager à travers le réseau trophique jusqu'aux prédateurs \citep{Laigle2021Directindirect}. 
En effet, les espèces les plus sensibles aux perturbations sont généralement celles ayant une grande taille corporelle et un niveau trophique élevé \citep{Seibold2015Associationextinction,Nolte2019Habitatspecialization,Hagge2021Whatdoes}. 
En tant que consommateurs au sommet de la chaîne alimentaire au niveau du sol, les salamandres seraient ainsi particulièrement vulnérables aux changements dans les niveaux trophiques inférieurs. 
Un modèle alternatif d'équations structurelles aurait pu être élaboré pour examiner si les traitements sylvicoles peuvent influencer les collemboles, avec des effets se propageant ensuite vers les carabes et les salamandres \citep{Laigle2021Directindirect}. 
Nous ne pensons cependant pas que ce type de dynamique puisse avoir lieu dans notre cas, étant donné que nous n'avons mesuré aucun effet direct des traitements de coupes forestières sur la biomasse des collemboles. 

Les SEM (modèle d’équations structurelles) sont des outils puissants pour étudier la dynamique des milieux naturels, analyser les relations entre diverses variables sous forme de réseau et observer la propagation des effets dans des systèmes complexes, 
tels que les écosystèmes forestiers \citep{graceSpecificationStructuralEquation2010}.  
Plusieurs études ont souligné l’importance d’inclure les interactions biotiques, notamment entre espèces appartenant à différents niveaux trophiques, pour mieux comprendre les impacts des perturbations 
sur le fonctionnement des écosystèmes \citep{Thebault2003Foodwebconstraints,Seibold2018necessitymultitrophic,Laigle2021Directindirect}.  
Dans cette étude, l’intégration de modèles d’occupation et de modèles linéaires mixtes au SEM a ainsi permis d’adopter une approche multi-trophique sous forme de réseau, 
afin d’analyser les interactions entre différents taxons en réponse aux traitements de coupe forestière \citep{josephIntegratingOccupancyModels2016}.  
Cette méthodologie nous a permis de quantifier les effets directs et indirects des coupes forestières sur la faune du sol, tout en évaluant les modifications de plusieurs variables environnementales. 
Cependant, la complexité des SEM dépend fortement de la quantité de données disponibles, car l’utilisation d’un modèle complexe avec un jeu de données limité peut entraîner des problèmes d’estimation lors des analyses.  
Dans cette étude, les données recueillies se limitent à une seule année, ce qui restreint le niveau de complexité possible dans l'élaboration du SEM. 
Afin de refléter les potentielles différences annuelles, il serait essentiel de reproduire cette recherche pour intégrer la dimension temporelle dans l’analyse.

Nous avions l’intention d’utiliser des modèles \textit{N}-mélange pour quantifier l'impact des traitements sylvicoles sur l'abondance des différents taxons \citep{Royle2004Nmixturemodels,Mazerolle2021Woodlandsalamander}. 
Intégré au SEM, cette approche aurait permis d'acquérir une compréhension plus fine de l'impact des traitements sylvicoles sur nos espèces. 
Malheureusement, le nombre de salamandres observées n'a pas été suffisant pour utiliser ce type d'analyse. 
Il est possible que le manque de vieillissement des planches servant de refuge artificiel explique en partie le faible descompte des salamandres. 
En effet, la salamandre cendrée favorise habituellement les débris ligneux ayant un stade de décomposition avancé pour s'abriter \citep{Otto2011ComparingCover,hedrickEffectsCoverboardAge2021}. 
Dans cette étude, les planches ont été installées en milieu naturel seulement trois mois avant le début de l'échantillonnage, ce qui pourrait limiter leur utilisation par les salamandres. 
Concernant les carabes et les collemboles, les étapes d’identification et de pesée requièrent la collecte des individus. 
Le retrait d’individus ne respecte pas la condition de fermeture démographique de la population requise par les modèles \textit{N}-mélange  \citep{Royle2004Nmixturemodels}. 
Il n’était donc pas possible d’utiliser ces modèles d’abondance avec les données de carabes et de collemboles

Le dispositif développé dans le cadre du projet DREAM (Desired Regeneration through Assisted Migration) et utilisé dans notre étude a été conçu pour examiner la migration assistée en tant que solution pour préserver les forêts face aux changements climatiques \citep{royoDesiredREgenerationAssisted2023}. 
Cette méthode consiste à déplacer des individus ou du matériel génétique d’une région climatique originelle vers une zone mieux adaptée aux conditions futures \citep{Vitt2010Assistedmigration}. 
Cette approche permettrait de modifier rapidement la composition des peuplements pour les adapter aux climats futurs, répondant ainsi à des besoins de conservation \citep{Dumroese2015Considerationsrestoring,Park2018Informationunderload,Park2023Provenancetrials}. 
Cependant, un manque de connaissances et un degré d’incertitude subsistent autour de cette mesure d’adaptation \citep{Klenk2015assistedmigration,Park2018Informationunderload}. 
Il serait donc important d’évaluer si la migration assistée peut influencer la faune présente dans le milieu hôte.
Lors de notre étude, les semis issus de diverses origines géographiques étaient encore trop jeunes pour potentiellement modifier les conditions environnementales et influencer les groupes d’espèces étudiés. 
Il serait néanmoins pertinent de répéter cette expérience lorsque les arbres auront atteint une maturité suffisante pour modifier les conditions au sol. 
De plus, le dispositif que nous avons utiliser dans le cadre de cette étude se trouve en forêt mixte tempérée.
Toutefois, il est possible que la réponse des espèces aux perturbations puisse varier selon le type d'habitat dans lequel elles se trouvent \citep{Kudrin2023metaanalysiseffects}. 
Il serait intéressant de reproduire ce type d'expérience dans d'autres environnements, tels que les forêts feuillues ou conifériennes.

Enfin, une meilleure compréhension des relations entre les perturbations, les variables environnementales et la biodiversité est cruciale pour préserver la pérennité des écosystèmes forestiers. 
L’ensemble de cette recherche avait pour but d'approfondir la compréhension des relations existantes entre les pratiques sylvicoles et la dynamique des communautés du sol forestier. 
L'amélioration des connaissances permettra, à long terme, de fournir de meilleurs outils pour orienter les plans de gestion et adapter les pratiques sylvicoles.


\cleardoublepage

\bibliography{References.bib}
\bibliographystyle{ecologyNewFR.bst}            % conclusion

\cleardoublepage

\renewcommand{\bibname}{Bibliographie générale}
\addcontentsline{toc}{chapter}{Bibliographie générale}
\bibliographystyle{ecologyNewEN}
\bibliography{References}


\appendix                       % annexes le cas échéant
\counterwithin{figure}{section}
\counterwithin{table}{section}

\chapter{Supplementary material for the article}     % numérotée
\label{chap:supp}                   % étiquette pour renvois (à compléter!)

\pagebreak
\section{Carabids and collembola identifications}

\begin{table}[h]
    \centering
    \caption[List of carabid species captured during summer 2021 in Portneuf Wildlife Reserve and classification between salamanders competitor and prey groups.]
    {List of carabids captured during summer 2021 in Portneuf Wildlife Reserve}
    \label{tab:carabid}
    \begin{tabular}{lll} 
        \hline
        Group & Specie & Count \\ [0.5ex] 
        \hline      
        Competitor          & \textit{Chlaenius sericeus}               & 2 \\  
                            & \textit{Harpalus erythropus}              & 1 \\
                            & \textit{Harpalus faunus}                  & 1 \\
                            & \textit{Harpalus rufipes}                 & 1 \\
                            & \textit{Patrobus longicornis}             & 1 \\
                            & \textit{Platynus decentis}                & 2 \\
                            & \textit{Poecilus lucublandus}             & 3 \\
                            & \textit{Pterostichus adstrictus}          & 27 \\
                            & \textit{Pterostichus coracinus}           & 32 \\
                            & \textit{Pterostichus diligendus}          & 3 \\
                            & \textit{Pterostichus mutus}               & 10 \\
                            & \textit{Pterostichus pensylvanicus}       & 18 \\
                            & \textit{Pterostichus punctatissimus}      & 1 \\
                            & \textit{Pterostichus tristis}             & 25 \\
                            & \textit{Sphaeroderus canadensis}          & 6 \\
                            & \textit{Sphaeroderus nitidicollis}        & 1 \\
                            \hline 
        Prey                & \textit{Agonum affine}                    & 8 \\ 
                            & \textit{Agonum palustre}                  & 3 \\
                            & \textit{Agonum punctiforme}               & 2 \\ 
                            & \textit{Agonum retractum}                 & 3 \\ 
                            & \textit{Bradycellus lugubris}             & 1 \\
                            & \textit{Bradycellus semipubescens}        & 1 \\
                            & \textit{Calathus gregarius}               & 1 \\
                            & \textit{Harpalus providens}               & 1 \\
                            & \textit{Loricera pilicornis}              & 1 \\
                            & \textit{Notiobia terminata}               & 5 \\
                            & \textit{Pseudamara arenaria}              & 7 \\
                            & \textit{Pterostichus commutabilis}        & 1 \\
                            & \textit{Synuchus impunctatus}             & 20 \\
                            & \textit{Trechus apicalis}                 & 1 \\
                            \hline 
        \textbf{Total}      &                                           & 189 \\
        \hline
    \end{tabular}
  \end{table}

  \begin{table}[ht]
    \centering
    \caption[List of sringtails order and families]
    {List of sringtails order and families captured during summer 2021 in Portneuf Wildlife Reserve.}
    \label{tab:springtail}
    \begin{tabular}{lll} 
        \hline
        Order & Family & Count \\ [0.5ex] 
        \hline      
        Entomobryomorpha    & Entomobryidae     & 62 \\  
                            & Isotomidae        & 182 \\
                            & Tomoceridae       & 8 \\
        Poduromorpha        & Hypogastruridae   & 92 \\
                            & Neanuridae        & 11 \\
                            & Onychiuridae      & 33 \\
                            & Tullbergidae      & 24 \\
        Symphypleona        & Dicyrtomidae      & 6 \\
                            & Katiannidae       & 6 \\
                            & Neelidae          & 23 \\
                            & Sminthuridae      & 10 \\
                            & Sminthurididae    & 11 \\
                            \hline 
        \textbf{Total}      &                   & 468 \\
        \hline
    \end{tabular}
  \end{table}

  \clearpage

\section{Structural equation models}


\begin{table}[ht!]
\caption[Equations used to estimate impact of overstory treatments on Red-backed salamanders (\textit{Plethodon cinereus}) and ground beetle occupancy, springtail biomass and environmental variables that could effect soil fauna habitat selection.]
{JAGS code used to estimate impact of overstory treatments on Red-backed salamanders (\textit{Plethodon cinereus}) and ground beetle occupancy, springtail biomass and environmental variables that could effect soil fauna habitat selection in Portneuf Wildlife Reserve, Québec, Canada.}
\label{ann:SEM_equation}
\end{table}

\subsubsection{Occupancy model component} 

The occupancy models of salamanders and ground beetles used the observed detections and non-detections at site $i$ during survey $j$, 
represented by $\text{Salamander}_{ij}$, $\text{Large.carabids}_{ij}$ and $\text{Small.carabids}_{ij}$. These detections and non-detections followed a Bernoulli distribution with $z_{i} \times p_{ij}$ as the success parameter, 
where $z_{i}$ represents the latent occupancy state of a given species group at site $i$ and $p_{ij}$ represents the probability of detecting the same species group at site $i$ during survey $j$ :


\begin{align}
  \text{Salamander}_{ij} &\sim \text{Bernoulli}(z_{\text{Salamander}_i} \times p_{\text{Salamander}_{ij}}) \nonumber \\
  \text{Large.carabids}_{ij} &\sim \text{Bernoulli}(z_{\text{Large.carabids}_i} \times p_{\text{Large.carabids}_{ij}})  \\
  \text{Small.carabids}_{ij} &\sim \text{Bernoulli}(z_{\text{Small.carabids}_i} \times p_{\text{Small.carabids}_{ij}}) \nonumber
\end{align}


The latent occupancy state of salamanders and large ground beetles at a site ($z_{i}$) followed a Bernoulli distribution 
with probability of occupancy ($\psi$) of the given species for a given harvest treatment (control, partial cut, clearcut):


\begin{align}
  z_{\text{Salamander}_i} &\sim \text{Bernoulli}(\psi_{\text{Salamander}_{\text{Treat}_i}}) \nonumber \\
  z_{\text{Large.carabids}_i} &\sim \text{Bernoulli}(\psi_{\text{Large.carabids}_{\text{Treat}_i}})
\end{align}


We used uniform distribution priors ($\text{U}(0, 1)$) for the occupancy probabilities of salamanders and large ground beetles in a given treatment. 

For small ground beetles, the latent occupancy state was drawn from a Bernoulli distribution, where the occupancy probability ($\psi_{Small.carabids_{i}}$) 
depended on salamander latent occupancy ($z_{\text{Salamander}_i}$) and overstory treatments (control as reference):


\begin{align}
  \text{logit}(\psi_{\text{Small.carabids}_i}) &= \beta_{0[\text{Small.carabids}]} + \beta_{z_{\text{Salamander}}[\text{Small.carabids}]} \times z_{\text{Salamander}_i} + \nonumber \\
  &\beta_{\text{Cutpartial}[\text{Small.carabids}]} \times \text{Cutpartial}_i + \\
  &\beta_{\text{Cutclear}[\text{Small.carabids}]} \times \text{Cutclear}_i \nonumber
\end{align}

We assumed vague normal priors for the $\beta$ parameters, $\text{N}(0, \sigma = \sqrt{10})$. 
We allowed the detection probability of a given species groups ($\text{logit}(p_{ij})$) with the volume of coarse woody debris ($\text{CWD}_i$) and the precipitation levels 
($\text{Precipitation}_{ij}$) as explanatory variables, and a block random effect ($\alpha_{Block}$) to reflect the experimental design:


\begin{align}
  \text{logit}(p_{\text{Salamander}_{ij}}) &= \alpha_{0[\text{Salamander}]} + \alpha_{\text{CWD}[\text{Salamander}]} \times \text{CWD}_i + \nonumber \\
  &\alpha_{\text{Precipitation}[\text{Salamander}]} \times \text{Precipitation}_{ij} + \alpha_{\text{Block}[\text{Salamander}]_{\text{Block}_i}} \nonumber
\end{align}

\begin{align}
  \text{logit}(p_{\text{Large.carabids}_{ij}}) &= \alpha_{0[\text{Large.carabids}]} + \alpha_{\text{CWD}[\text{Large.carabids}]} \times \text{CWD}_i + \\
  &\alpha_{\text{Precipitation}[\text{Large.carabids}]} \times \text{Precipitation}_{ij} + \alpha_{\text{Block}[\text{Large.carabids}]_{\text{Block}_i}} \nonumber 
\end{align}

\begin{align}
  \text{logit}(p_{\text{Small.carabids}_{ij}}) &= \alpha_{0[\text{Small.carabids}]} + \alpha_{\text{CWD}[\text{Small.carabids}]} \times \text{CWD}_i + \nonumber \\
  &\alpha_{\text{Precipitation}[\text{Small.carabids}]} \times \text{Precipitation}_{ij} + \alpha_{\text{Block}[\text{Small.carabids}]_{\text{Block}_i}} \nonumber 
\end{align}


We used vague normal priors for CWD and precipitation levels $\text{N}(0, \sigma = \sqrt{10})$. 
The priors for block random effects were $\text{N}(0, \alpha_{Block})$, where $\alpha_{Block}$ was drawn from a uniform distribution $\text{U}(0, 10)$. 
Precipitation was highly correlated with air temperature and relative humidity. 
For this reason, we did not include these two variables in our models.


\subsubsection{Linear mixed model component} 

We used linear mixed models to assess how overstory treatments affect springtail biomass ($\text{Springtail}_{i}$) and 
environmental variables ($\text{CWD}_{i}$, $\text{Canopy}_{i}$, $\text{Litter}_{i}$) at site $i$ :

\begin{align}
  \text{Springtail}_{i} &\sim \text{N} (\mu_{\text{Springtail}_i}, \sigma_{\text{Springtail}_{\text{Group}_i}}) \nonumber \\
  \text{CWD}_{i} &\sim \text{N} (\mu_{\text{CWD}_i}, \sigma_{\text{CWD}_{\text{Group}_i}}) \\
  \text{Canopy}_{i} &\sim \text{N} (\mu_{\text{Canopy}_i}, \sigma_{\text{Canopy}_{\text{Group}_i}}) \nonumber \\
  \text{Litter}_{i} &\sim \text{N} (\mu_{\text{Litter}_i}, \sigma_{\text{Litter}_{i}}) \nonumber 
\end{align}

Due to heteroscedasticity in the Springtail biomass, we allowed each treatment group $j$ to have their own variances ($\sigma_j \sim \text{U}(0,150)$). 
Springtail biomass was drawn from a normal distribution $\text{N} (\mu_{\text{Springtail}_i}, \sigma_{\text{Springtail}_{\text{Group}_i}})$, where $\sigma_{\text{Springtail}_{\text{Group}_i}}$ denotes the 
residual variance of a given treatment group and ($\mu_{i}$) corresponds to the linear predictor including overstory treatments ($\text{Cutpartial}_i$, $\text{Cutclear}_i$), 
a block random effect ($\alpha_{\text{Block}}$), as well as the latent occupancy state of salamanders and of each ground beetle group 
($z_{Salamander}$, $z_{Large.carabids}$, $z_{Small.carabids}$):


\begin{align}
  \mu_{\text{Springtail}_i} &= \beta_{0[\text{Springtail}]} + \beta_{\text{Cutpartial}[\text{Springtail}]} \times \text{Cutpartial}_i + \nonumber\\
  &\beta_{\text{Cutclear}[Springtail]} \times \text{Cutclear}_i + \beta_{z_{\text{Salamander}}[\text{Springtail}]} \times z_{Salamander} +  \nonumber\\
  &\beta_{z_{\text{Small.carabids}}[\text{Springtail}]} \times z_{Small.carabids} + \beta_{z_{\text{Large.carabids}}[\text{Springtail}]} \times z_{Large.carabids} + \nonumber\\
  &\alpha_{\text{Block}[\text{Springtail}]_{\text{Block}_i}} \nonumber
\end{align}


Again, we assumed vague normal priors for the coefficients $\text{N}(0, \sigma = \sqrt{10})$. 
We used $\text{N}(0, \alpha_{Block})$ priors for the block random effects, where $\alpha_{Block})$ is drawn from a uniform distribution $\text{U}(0, 50)$. 


\begin{align}
  \mu_{\text{CWD}_i} &= \beta_{0[\text{CWD}]} + \beta_{\text{Cutpartial}[\text{CWD}]} \times \text{Cutpartial}_{i} + \nonumber\\
  & \beta_{\text{Cutclear}[\text{CWD}]} \times \text{Cutclear}_{i} + \alpha_{\text{Block}[\text{CWD}]_{\text{block}_i}} 
\end{align}

\begin{align}
  \mu_{\text{Canopy}_i} &= \beta_{0[\text{Canopy}]} + \beta_{\text{Cutpartial}[\text{Canopy}]} \times \text{Cutpartial}_{i} + \nonumber \\
  & \beta_{\text{Cutclear}[\text{Canopy}]} \times \text{Cutclear}_{i} + \alpha_{\text{Block}[\text{Canopy}]_{\text{block}_i}} \nonumber
\end{align}

\begin{align}
  \mu_{\text{Litter}_i} &= \beta_{0[\text{Litter}]} + \beta_{\text{Cutpartial}[\text{Litter}]} \times \text{Cutpartial}_{i} + \nonumber\\
  & \beta_{\text{Cutclear}[\text{Litter}]} \times \text{Cutclear}_{i} + \alpha_{\text{Block}[\text{Litter}]_{\text{block}_i}} \nonumber
\end{align}


\clearpage

\begin{table}[ht]
\caption[JAGS code used to estimate impact of overstory treatments on Red-backed salamanders (\textit{Plethodon cinereus}) and ground beetle occupancy, springtail biomass and environmental variables that could effect soil fauna habitat selection.]
    {JAGS code used to estimate impact of overstory treatments on Red-backed salamanders (\textit{Plethodon cinereus}) and ground beetle occupancy, springtail biomass and environmental variables that could effect soil fauna habitat selection in Portneuf Wildlife Reserve, Québec, Canada.}
    \label{ann:SEM_script}
\end{table}

\begin{lstlisting}
##SEM
model {

##CWD
##priors
beta0.cwd ~ dnorm(0, 0.01)
beta.Cutpartial.cwd ~ dnorm(0, 0.01)
beta.Cutclear.cwd ~ dnorm(0, 0.01)

##block random effect
for(m in 1:nblocks) {
   alpha.block.cwd[m] ~ dnorm(0, tau.block.cwd)
}

##variance of block
tau.block.cwd <- pow(sigma.block.cwd, -2)
sigma.block.cwd ~ dunif(0, 50)

##allow each group to have different variance
for(j in 1:3) {
   tau.cwd[j] <- pow(sigma.cwd[j], -2)
   sigma.cwd[j] ~ dunif(0, 150)
}

##iterate over each observation
for (i in 1:nsites) {

    ##linear predictor  
    mu.cwd[i] <- beta0.cwd + beta.Cutpartial.cwd * Cutpartial[i] + 
       beta.Cutclear.cwd * Cutclear[i] + alpha.block.cwd[Block[i]]

    ##response
    CWD_tot[i] ~ dnorm(mu.cwd[i], tau.cwd[Group[i]])
}

##derived parameters
for(i in 1:nsites) {
    pred.cwd[i] <- mu.cwd[i]
    res.cwd[i] <- CWD[i] - mu.cwd[i]
    res.pearson.cwd[i] <- res.cwd[i]/sigma.cwd[Group[i]]
}

##Canopy openness
##priors
beta0.can ~ dnorm(0, 0.01)
beta.Cutpartial.can ~ dnorm(0, 0.01)
beta.Cutclear.can ~ dnorm(0, 0.01)

##block random effect
for(m in 1:nblocks) {
   alpha.block.can[m] ~ dnorm(0, tau.block.can)
}

##variance of block
tau.block.can <- pow(sigma.block.can, -2)
sigma.block.can ~ dunif(0, 50)

##allow each group to have different variance
for(j in 1:3) {
   tau.can[j] <- pow(sigma.can[j], -2)
   sigma.can[j] ~ dunif(0, 150)
}

##iterate over each observation
for (i in 1:nsites) {

    ##linear predictor  
    mu.can[i] <- beta0.can + beta.Cutpartial.can * Cutpartial[i] + 
       beta.Cutclear.can * Cutclear[i] + alpha.block.can[Block[i]]

    ##response
    Canopy[i] ~ dnorm(mu.can[i], tau.can[Group[i]])
}

##derived parameters
for(i in 1:nsites) {
    pred.can[i] <- mu.can[i]
    res.can[i] <- Canopy[i] - mu.can[i]
    res.pearson.can[i] <- res.can[i]/sigma.can[Group[i]]
}

##Litter depth
##priors
beta0.lit ~ dnorm(0, 0.01)
beta.Cutpartial.lit ~ dnorm(0, 0.01)
beta.Cutclear.lit ~ dnorm(0, 0.01)

##block random effect
for(m in 1:nblocks) {
   alpha.block.lit[m] ~ dnorm(0, tau.block.lit)
}

##variance of block
tau.block.lit <- pow(sigma.block.lit, -2)
sigma.block.lit ~ dunif(0, 50)

##allow each group to have different variance
tau.lit <- pow(sigma.lit, -2)
sigma.lit ~ dunif(0, 150)

##iterate over each observation
for (i in 1:nsites) {

    ##linear predictor  
    mu.lit[i] <- beta0.lit + beta.Cutpartial.lit * Cutpartial[i] + 
       beta.Cutclear.lit * Cutclear[i] + alpha.block.lit[Block[i]]

       ##response
       ##Litter[i] ~ dnorm(mu.lit[i], tau.lit[Group[i]])
       Litter[i] ~ dnorm(mu.lit[i], tau.lit)
}

##derived parameters
for(i in 1:nsites) {
    pred.lit[i] <- mu.lit[i]
    res.lit[i] <- Litter[i] - mu.lit[i]
    res.pearson.lit[i] <- res.lit[i]/sigma.lit
}

##salamander component
##priors for psi
for(k in 1:ngroups) {
   psi.sal[k] ~ dunif(0, 1)
}

##priors for p
alpha0.sal ~ dnorm(0, 0.01)
alpha.precip.sal ~ dnorm(0, 0.01)
alpha.cwd.sal ~ dnorm(0, 0.01)

##block random effect
for(m in 1:nblocks) {
   alpha.block.sal[m] ~ dnorm(0, tau.block.sal)
}

tau.block.sal <- pow(sigma.block.sal, -2)
sigma.block.sal ~ dunif(0, 10)

##salamander model
for(i in 1:nsites) {
   ##occupancy
   z.sal[i] ~ dbern(psi.sal[Group[i]])
   for(j in 1:nvisits) {
      ##p
      logit.p.sal[i, j] <- alpha0.sal + alpha.cwd.sal * CWD[i] + alpha.precip.sal * Precip[i, j] + alpha.block.sal[Block[i]]
      p.sal[i, j] <- exp(logit.p.sal[i, j])/(1 + exp(logit.p.sal[i, j]))

      eff.p.sal[i, j] <- z.sal[i] * p.sal[i, j]
      y.sal[i, j] ~ dbern(eff.p.sal[i, j])
   }
}

finiteOcc.sal <- sum(z.sal[])

####carab beetles - prey
##psi priors
psi.beta0.car.prey ~ dnorm(0, 0.01)
psi.beta.z.sal.car.prey ~ dnorm(0, 0.01)
psi.beta.Cutpartial.car.prey ~ dnorm(0, 0.01)
psi.beta.Cutclear.car.prey ~ dnorm(0, 0.01)

##p priors
alpha0.car.prey ~ dnorm(0, 0.01)
alpha.precip.car.prey ~ dnorm(0, 0.01)
alpha.cwd.car.prey ~ dnorm(0, 0.01)

##block random effects
for(m in 1:nblocks) {
   alpha.block.car.prey[m] ~ dnorm(0, tau.block.car.prey)
}

tau.block.car.prey <- pow(sigma.block.car.prey, -2)
sigma.block.car.prey ~ dunif(0, 10)

for(i in 1:nsites) {
   ##occupancy
   logit(psi.car.prey[i]) <- psi.beta0.car.prey + psi.beta.z.sal.car.prey * z.sal[i] +
    	psi.beta.Cutpartial.car.prey*Cutpartial[i] + psi.beta.Cutclear.car.prey*Cutclear[i]
    	
   z.car.prey[i] ~ dbern(psi.car.prey[i])

   for(j in 1:nvisits) {
      ##p
      logit.p.car.prey[i, j] <- alpha0.car.prey + alpha.cwd.car.prey * CWD[i] + alpha.precip.car.prey * Precip[i, j]
      + alpha.block.car.prey[Block[i]]
      p.car.prey[i, j] <- exp(logit.p.car.prey[i, j])/(1 + exp(logit.p.car.prey[i, j]))

      eff.p.car.prey[i, j] <- z.car.prey[i] * p.car.prey[i, j]
      y.car.prey[i, j] ~ dbern(eff.p.car.prey[i, j])
   }
}

finiteOcc.car.prey <- sum(z.car.prey[])

####carab beetles - comp
##psi priors
for(k in 1:ngroups) {
   psi.car.comp[k] ~ dunif(0, 1)
}

##p priors
alpha0.car.comp ~ dnorm(0, 0.01)
alpha.precip.car.comp ~ dnorm(0, 0.01)
alpha.cwd.car.comp ~ dnorm(0, 0.01)

##block random effects
for(m in 1:nblocks) {
   alpha.block.car.comp[m] ~ dnorm(0, tau.block.car.comp)
}

tau.block.car.comp <- pow(sigma.block.car.comp, -2)
sigma.block.car.comp ~ dunif(0, 10)

for(i in 1:nsites) {
      ##occupancy
      z.car.comp[i] ~ dbern(psi.car.comp[Group[i]])

   	for(j in 1:nvisits) {
      	   ##detection
      	   logit.p.car.comp[i, j] <- alpha0.car.comp + alpha.cwd.car.comp * CWD[i] + alpha.precip.car.comp * Precip[i, j]
           + alpha.block.car.comp[Block[i]]

      	   ##p
      	   p.car.comp[i, j] <- exp(logit.p.car.comp[i, j])/(1 + exp(logit.p.car.comp[i, j]))
      	   eff.p.car.comp[i, j] <- z.car.comp[i] * p.car.comp[i, j]
      	   y.car.comp[i, j] ~ dbern(eff.p.car.comp[i, j])
      	}
}

finiteOcc.car.comp <- sum(z.car.comp[])

##collembola data
##priors
beta0.coll ~ dnorm(0, 0.01)
beta.Cutpartial.coll ~ dnorm(0, 0.01)
beta.Cutclear.coll ~ dnorm(0, 0.01)
beta.z.sal.coll ~ dnorm(0, 0.01)
beta.z.car.prey.coll ~ dnorm(0, 0.01)
beta.z.car.comp.coll ~ dnorm(0, 0.01)

##block random effect
for(m in 1:nblocks) {
   alpha.block.coll[m] ~ dnorm(0, tau.block.coll)
}

tau.block.coll <- pow(sigma.block.coll, -2)
sigma.block.coll ~ dunif(0, 50)

##allow each group to have different variance
for(j in 1:3) {
   tau.coll[j] <- pow(sigma.coll[j], -2)
   sigma.coll[j] ~ dunif(0, 150)
}

##iterate over each observation
for (i in 1:nsites) {

    ##linear predictor  
    mu.coll[i] <- beta0.coll +
    beta.Cutpartial.coll*Cutpartial[i] +
    beta.Cutclear.coll*Cutclear[i] + beta.z.sal.coll*z.sal[i] +
    beta.z.car.prey.coll*z.car.prey[i] +
    beta.z.car.comp.coll*z.car.comp[i] + alpha.block.coll[Block[i]]

    ##response
    y.coll[i] ~ dnorm(mu.coll[i], tau.coll[Group[i]])
}

##derived parameters
for(i in 1:nsites) {
    pred.coll[i] <- mu.coll[i]
    res.coll[i] <- y.coll[i] - mu.coll[i]
    res.pearson[i] <- res.coll[i]/sigma.coll[Group[i]]
}
}
\end{lstlisting}

\clearpage

\chapter*{Bibliographie}         
\label{chap:biblio}         
\phantomsection\addcontentsline{toc}{chapter}{\nameref{chap:biblio}} 

\nocite{*}
\renewcommand{\bibsection}{}
\begin{otherlanguage}{english}
\bibliography{References}
\bibliographystyle{ecologyNewEN}
\end{otherlanguage}
                % annexes


\end{document}

