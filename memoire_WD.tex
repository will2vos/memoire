
\documentclass[MSc,bibsection,english,french]{ulthese}

\usepackage[utf8]{inputenc}
%\ifxetex\else \usepackage[utf8]{inputenc} \fi

%\setlength{\beforechapskip}{20pt} % reduce space before chapter

\usepackage{amsmath}              % recommandé pour les mathématiques
\usepackage{icomma}               % gestion de la virgule dans les nombres
\usepackage{gensymb}
\usepackage{caption}
\usepackage{graphicx}
\usepackage{chngcntr}           % numerotation figure en annexe par section
\usepackage{array}
\usepackage{pdflscape}
%\usepackage{mathpazo}            % texte et mathématiques en Palatino
\usepackage{listings}            % insérer language R dans texte
\usepackage{xcolor}        % colorer le texte pour corrections
\definecolor{myYellow}{RGB}{248.7,225.0,63.1}
\usepackage{soul}          % colorer le texte plus long pour wrap up
\sethlcolor{myYellow}

\lstset{language=R,
    basicstyle=\small\ttfamily,
    stringstyle=\color{purple},
    breaklines=true,
    keywordstyle=\color{blue},
    commentstyle=\color{purple},
}


\usepackage{chngcntr}           % numerotation figure en annexe par section

\usepackage{rotating}
\usepackage[T1]{fontenc}
\usepackage{lmodern}
\usepackage{amssymb}
\usepackage{mathrsfs}
\usepackage{url}
%modify interline spacing
\usepackage{setspace}
\usepackage{lscape}
\usepackage{fixltx2e}   % permet de mettre des mots en indice dans tableau
\usepackage{multirow}   % permet de fusionner de colonne dans tableau


% Define typographic struts, as suggested by Claudio Beccari
%   in an article in TeX and TUG News, Vol. 2, 1993.
\newcommand\Tstrut{\rule{0pt}{2.6ex}}         % = `top' strut
\newcommand\Bstrut{\rule[-0.9ex]{0pt}{0pt}}   % = `bottom' strut

\DeclareCaptionLabelFormat{tableformat}{#1 #2}
\captionsetup[table]{labelformat=tableformat, labelsep=period}

  % \DeclareCaptionLabelFormat{Sformat}{#1 S#2}  
  % \DeclareCaptionLabelSeparator{none}{}    %<---
  % \captionsetup[table]{labelformat=Sformat}
  
\usepackage[protrusion=true, expansion=true, shrink=55, stretch=55, 
tracking=true, kerning=true, spacing=true, 
final]{microtype}
\usepackage{natbib}
\setcitestyle{round, sort}

%\usepackage{bibentry}
\frenchbsetup{%
  StandardItemizeEnv=true,       % format standard des listes
  ThinSpaceInFrenchNumbers=true, % espace fine dans les nombres
  og=«, fg=»                     % caractères « et » sont les guillemets
}

\addto\extrasfrench{
  \renewcommand{\bibsection}{\section*{\bibname}\prebibhook}}


%\usepackage{multibib} %citer plusieurs bibliographies FR/ENG
%\newcites{Eng}{My English Biblio}

%default style in French (defined in babel)
%\bibliographystyle{ecologyNewFR}
%second style in English for second bibliography
%\bibliographystyleEng{ecologyNewEN}
%%%%%%%%%%%%%%%%

% pour tableau noir et blanc
\usepackage{longtable}%
\AtBeginEnvironment{longtable}{%
  \addfontfeature{RawFeature=+tnum;-onum}%  <--- requires LuaTeX
}


%change name of English references section
%\addto{\captionsenglish}{\renewcommand{\bibname}{References}}

  \titre{Impactes des traitements sylvicoles dans un contexte de migration assistée d'arbres sur la cooccurence entre amphibiens et arthropodes}

  \auteur{William Devos}
  \direction{Marc J. Mazerolle, directeur de recherche}
  \codirection{Mathieu Bouchard, codirecteur de recherche}

  \annee{2024}

\begin{document}

\frontmatter                    % pages liminaires

\frontispice                    % production de la page frontispice

\chapter*{Résumé}               % ne pas numéroter
\label{chap-resume}             % étiquette pour renvois
\phantomsection\addcontentsline{toc}{chapter}{\nameref{chap-resume}} % inclure dans TdM

Les environnements forestiers jouent un rôle majeur à l’échelle planétaire à travers leurs fonctions écosystémiques et leur apport économique. 
Toutefois, la croissance démographique humaine et l’augmentation des besoins en produits ligneux entraînent une intensification de l'exploitation des forêts. 
Ces changements peuvent amener  une perte d'habitats essentiels à la biodiversité et perturber les dynamiques écologiques entre les espèces qui peuplent ces écosystèmes. 
Cette étude vise à quantifier les effets des traitements sylvicoles tels que les coupes totales et les coupes partielles sur la dynamique de la faune du sol en forêt mixte tempérée. 
Les objectifs étaient premièrement de mesurer l’impact des coupes forestières sur des variables environnementales influençant l’utilisation de l’habitat par la faune du sol 
et deuxièmement d’évaluer les effets directs et indirects de ces traitements sur les amphibiens et les arthropodes du sol à travers le réseau trophique. 
Nous avons émis une hypothèse que les variables environnementales associées à l’habitat de la petite faune (volume de débris ligneux, ouverture de la canopée et profondeur de litière) varient en fonction de l’intensité des coupes forestières. 
Ainsi, les traitements de coupes constitueraient une variable englobant les changements de conditions environnementales. 
Selon notre deuxième hypothèse, les traitements de coupe forestière entraîneraient une modification de l'utilisation de l'habitat 
par les salamandres cendrées (\textit{Plethodon cinereus}), les carabes (\textit{Carabidae}), et les collemboles (\textit{Collembola}), de manière directe ou via des interactions trophiques, 
allant des prédateurs vers les proies. 
Pour tester ces hypothèses, nous avons développé un modèle d’équations structurelles (SEM) intégrant des modèles d’occupation et des modèles linéaires mixtes. 
Cette approche nous a permis de mesurer simultanément, sous forme de réseau, l’effet des coupes forestières sur les variables environnementales, d’une part, et sur la cooccurrence entre les différents taxons, d’autre part. 
Nos résultats montrent que les traitements de coupes forestières influencent significativement les variables environnementales. 
Les coupes totales ont entraîné une ouverture plus grande de la canopée, une diminution de la profondeur de la litière et une réduction du volume de débris ligneux, 
tandis que les coupes partielles permettent une meilleure rétention de ces attributs environnementaux. 
Cependant, les traitements sylvicoles n’ont, dans l’ensemble, pas eu d’effet direct significatif sur la probabilité d’occupation des salamandres et des carabes, ni sur la biomasse des collemboles. 
Nous avons toutefois mesuré une probabilité d’occupation légèrement plus faible pour les salamandres dans les traitements de coupe totale par rapport aux coupes partielles, bien que cet effet soit marginal (IC à 90\%). 
De plus, aucun changement de cooccurrence entre ces groupes d’espèces n’a été observé, ce qui suggère que les effets des coupes forestières ne se propagent pas à travers le réseau trophique des prédateurs vers les proies du sol forestier. 
Notre étude met en évidence la pertinence des modèles d’équations structurelles pour analyser les réseaux trophiques complexes des écosystèmes forestiers. 
Cette recherche a permis d'approfondir la compréhension des relations existantes entre les pratiques sylvicoles et la dynamique de la faune du sol forestier. 
L'apport de ces connaissances permettra de fournir de meilleurs outils pour orienter les plans de gestion et adapter les pratiques sylvicoles. 


\begin{otherlanguage*}{french}
\textbf{Mots-clés:} \textit{Plethodon cinereus}, carabe, collembole, modèle d'équations structurelles, cooccurrence, coupe forestière.
\end{otherlanguage*}
                % résumé français
\chapter*{Abstract}             % ne pas numéroter
\label{chap-abstract}           % étiquette pour renvois
\phantomsection\addcontentsline{toc}{chapter}{\nameref{chap-abstract}} % inclure dans TdM

Forest environments play a major role on a global scale through their ecosystem functions and the economic benefits they provide. 
However, human population growth and increasing demand for wood products are leading to increased logging. 
These changes can lead to habitat loss essential for biodiversity and disrupt ecological dynamics between species inhabiting these ecosystems. 
This study aims to quantify the effects of silvicultural treatments, such as clearcutting and partial cutting, on soil fauna dynamics in temperate mixed forests. 
The objectives were, first, to assess the impact of logging on environmental variables influencing soil fauna habitat use and, second, 
to evaluate the direct and indirect effects of these treatments on soil amphibians and arthropods through trophic networks. 
We hypothesized that environmental variables associated with small fauna habitats (volume of coarse woody debris, canopy openness, and litter depth) vary depending on the intensity of logging. 
Thus, harvesting treatments would act as an overarching variable encompassing changes in environmental conditions. 
According to our second hypothesis, forest harvesting treatments would modify habitat use by Eastern red-backed salamanders (\textit{Plethodon cinereus}), ground beetles (\textit{Carabidae}), and springtails (\textit{Collembola}), 
either directly or through trophic interactions from predators to prey. 
To test these hypotheses, we developed a structural equation model (SEM) integrating occupancy models and linear mixed models. 
This approach allowed us to simultaneously measure, in a network framework, the effects of forest logging on environmental variables and the cooccurrence between different taxa. 
Our results show that harvesting treatments significantly influence environmental variables. 
Clearcutting led to greater canopy openness, reduced litter depth, and decreased coarse woody debris volume,  
whereas partial cutting allowed for better retention of these environmental attributes. 
Overall, silvicultural treatments had no significant direct effects on the occupancy probabilities of salamanders and ground beetles or the biomass of springtails. 
However, we measured a slightly lower occupancy probability for salamanders in clearcuts compared to partial cuts, although this effect was marginal (90\% CI). 
Moreover, no changes in cooccurrence between these groups were observed, suggesting that the effects of logging do not propagate through trophic networks from predators to prey. 
Our study highlights the relevance of structural equation models for analyzing complex trophic networks of forest ecosystems. 
This research deepens our understanding of the relationships between silvicultural practices and the dynamics of forest soil fauna. 
The environmental variables influencing the habitat of the studied species (coarse woody debris volume, litter depth, and canopy openness) vary depending on the intensity of logging. 
We therefore recommend promoting harvesting methods that minimize habitat disturbance, such as partial cuts. 
These practices help preserve key environmental attributes, benefiting soil fauna. 
The knowledge gained from this study provides valuable tools for guiding management plans and adapting silvicultural practices. 

\begin{otherlanguage*}{english}
  \textbf{Keywords:} \textit{Plethodon cinereus}, ground beetle, springtail, structural equation modelling, cooccurrence, logging.
  
\end{otherlanguage*}
              % résumé anglais
\cleardoublepage

\setcounter{tocdepth}{2}
\setcounter{secnumdepth}{2}

\tableofcontents                % production de la TdM
\cleardoublepage

\listoftables                   % production de la liste des tableaux
\cleardoublepage

\listoffigures                  % production de la liste des figures
\cleardoublepage

\chapter*{Remerciements}        % ne pas numéroter
\label{chap-remerciements}      % étiquette pour renvois
\phantomsection\addcontentsline{toc}{chapter}{\nameref{chap-remerciements}} % inclure dans TdM

%\usepackage[french]{babel}

 Ce travail n'aurait pas pris sa forme actuelle sans le soutien que j'ai reçu tout au long de mon parcours de maîtrise.

 Mes remerciements vont d'abord à mon directeur, Marc J. Mazerolle, pour m'avoir accueilli au sein de son laboratoire de recherche et pour son soutien constant tout au long de ce projet. 
 Sa confiance, ses précieux conseils en écologie, en statistique et en gestion de projet, ainsi que sa disponibilité ont été des atouts inestimables.
 Je souhaite également exprimer ma gratitude à mon co-directeur, Mathieu Bouchard, dont le soutien et les conseils dans un domaine que je maîtrisais peu, la foresterie, ont été d'une grande aide. 
 Sa vision du projet et ses recommandations m'ont permis de prendre des décisions plus judicieuses et éclairées.

 Je suis également reconnaissant envers Patricia Raymond et Émilie Champagne du Ministère des Ressources Naturelles et de Forêts, Direction de la recherche forestière, pour leur aide et 
 pour m'avoir permis de participer au projet enrichissant qu'est DREAM-Qc.
 Un grand merci également aux techniciennes Karine Thériault et Marie-Claude Martin pour leur soutien logistique et leurs conseils avisés.
 Je suis reconnaissant pour l'aide que j'ai reçu de la part de mon auxiliaire de terrain, Rebecca Dubé Messier. Sa motivation et son travail assidu ont largement contribué à la qualité de ce mémoire.

 Le parcours d'un projet de maîtrise et toujours plus facile quand on est bien entouré. 
 Pour cela je remercie les membres de mon laboratoire : 
 Aurore Fayard, Laura Millard, Mariano Feldman, Anais Baillet, Lucas Voirin, Félicia Beaulieu, Jeanne Dudemaine, Laurianne Plante et Naomie Herpin Saunier. 
 Votre soutien, nos activités extérieures ainsi que nos discussions ont été des moments précieux pour moi.
 Un merci spécial à Joëlle Spooner, qui a été une amie, une colocataire et une collègue de laboratoire exceptionnelle tout au long de ce parcours. 
 Sa présence a été un véritable soutien.

 Je souhaite également remercier mes parents, Isabelle et Patrick, mes beaux-parents, Michel et Anne-Chantal, ainsi que mes frères, Pierre et Guillaume, pour leur soutien depuis 
 mon plus jeune âge. Votre support dans ma passion pour la biologie est inestimable.
 Je conclus ces remerciements en exprimant ma reconnaissance envers la personne qui m'accompagne au quotidien, Léonie St-Onge. 
 Merci pour ton écoute, ta patience, tes conseils et pour être simplement toi.

         % remerciements
\chapter*{Avant-propos}         % ne pas numéroter
\label{chap-avantpropos}        % étiquette pour renvois
\phantomsection\addcontentsline{toc}{chapter}{\nameref{chap-avantpropos}} % inclure dans TdM

%\usepackage[french]{babel}

Ce mémoire «par article» en Sciences forestières présenté à l'Université Laval est divisé en plusieurs parties : 
l'introduction générale, écrite en français, aborde en premier lieu la problématique liée à l'impact potentiel des traitements sylvicoles sur 
l'utilisation de l'habitat par la petite faune du sol dans le cadre d'un projet de migration assistée des arbres.
Le corps de ce document comprend le chapitre principal, intitulé 
"Direct and indirect effects on soil fauna of silvicultural treatments in the context of forest assisted migration", rédigé en anglais sous la forme d'un article scientifique. 
Suite au dépôt de ce mémoire, ce chapitre sera soumis à la revue scientifique \textit{Forest Ecology and Management}. 
William Devos est l'auteur principal de cet article, suivi de Mathieu Bouchard, co-directeur de recherche et Marc J. Mazerolle, directeur de recherche et dernier auteur.
La conclusion générale rédigée en français constitue la dernière partie de ce mémoire.
           % avant-propos

\mainmatter                     % corps du document

\chapter*{Introduction générale}         % ne pas numéroter
\label{chap-introduction}       % étiquette pour renvois
\phantomsection\addcontentsline{toc}{chapter}{\nameref{chap-introduction}} % inclure dans TdM

% \usepackage[french]{babel}

\section*{Mise en contexte}
\label{sec:contexte}
\phantomsection\addcontentsline{toc}{section}{\nameref{sec:contexte}}

Les écosystème forestiers jouent un rôle essentiel dans la biosphère à travers leur rôle économique et leur valeur écosystèmiques en fournissant une multitude de produit et services \citep{Balvanera2006Quantifyingevidence}. 
Leur présence permet de réguler les flux de nutriments et d'énergie, notamment à travers la séquestration du carbone, la régulation du climat, la rétention de l'eau et la conservation de la biodiversité \citep{Balvanera2006Quantifyingevidence,Diaz2006BiodiversityLoss,Canadell2008Managingforests,Pawson2013Plantationforests}. 

Cependant la croissance démographique mondiale, l'augmentation des besoins en produits forestiers et autres services ont conduit à une intensification des pratiques d'exploitation forestière au cours des dernières décennies \citep{Foley2005GlobalConsequences}. 
Ces changements d'affectations des terres, comme la récoltes de bois, l'agriculture et l'urbanisation, entrainent ainsi la perte d'écosystèmes forestiers et de biodiversité \citep{Bengtsson2000Biodiversitydisturbances,Sala2000Globalbiodiversity,Naeem2012functionsbiological,Bichet2016Maintaininganimal}. 

En ramenant les peuplements forestiers à des stades précoces de succession, l'exploitation forestière tend à homogénéiser le paysage, favorisant ainsi une surreprésentation des forêts en début de succession 
au détriment des stades plus avancés \citep{Cyr2009Forestmanagement,Boucher2017Cumulativepatterns}. 
Cette pratique réduit la complexité structurelle des forêts, affectant des aspects tels que la composition des espèces d'arbres, la stratification verticale, la structure d'âges, 
la dynamique de succession et la fréquence des perturbations \citep{Commarmot2005Structurevirgin}. 
Ces modifications sont particulièrement préoccupantes pour la biodiversité, notamment pour les espèces associées aux forêts mixtes et conifériennes matures \citep{Tremblay2018Harvestinginteracts,Cadieux2020Projectedeffects}.
L'hétérogénéité structurelle des forêts joue un rôle significatif dans la conservation de la biodiversité, car elle offre une variété d'habitats et renforce la connectivité en favorisant la dispersion de certaines espèces. 
Par conséquent, l'exploitation forestière peut représenter une menace et un risque d'extinction pour les espèces dépendant des caractéristiques des forêts anciennes et pour celles ayant une faible capacité de dispersion \citep{Norden2001Conceptualproblems,Martin2021indicatorspecies}. 
De plus, un peuplement avec une structure plus hétérogène est plus résilient lors d'une perturbation, car il facilite la régénération des espèces arbres présentes et le retour de celles qui ont disparu. 

Cette biodiversité est reconnue comme essentielle pour le bon fonctionnement des écosystèmes terrestres et les engagements internationaux ont souligné l'urgence de freiner cette perte tout en promouvant une gestion durable des forêts \citep{Scherer-Lorenzen2005ForestDiversity,Parviainen2007Maintenanceconservation,Newbold2015Globaleffects}. 
De nombreuses recherches ont mis en évidence l'impact de la gestion forestière sur divers groupes fauniques, tels que les oiseaux, chauves-souris, papillons, tortues, petits mammifères et insectes \citep{Summerville2011Managingforest, Currylow2012ShortTermForest, Kaminski2013EffectsForest, Kellner2013Shorttermresponses, Caldwell2019ComparisonBat}. 
Les pratiques sylvicoles peuvent provoquer la mortalité des animaux, perturber ou restreindre leurs déplacements, intensifier les interactions biotiques, modifier leurs cycles de vie, altérer leur morphologie et leur physiologie, voire affecter leurs formes polymorphiques \citep{Sergio2018Animalresponses}. 
De plus, l'exploitation forestière entraîne des pertes d'habitat et la fragmentation des milieux naturels, réduisant ainsi l'accès à la nourriture, aux refuges et aux sites de reproduction, tout en diminuant la taille et la diversité génétique des populations \citep{Coelho2020Effectsanthropogenic}.
Les coupes forestières peuvent également réduire la connectivité entre les habitats, les communautés et les processus écologiques ce qui peut avoir des conséquences grave pour la conservation de la faune \citep{Lindenmayer2006Generalmanagement}. 
Une diminution de la connectivité peut compromettre la capacité des populations à se rétablir et à se maintenir après une perturbation \citep{Lamberson1994ReserveDesign}. 
Cette diminution restreint aussi le déplacement des individus dans leur habitat et le long des corridors écologiques, réduisant ainsi le flux génétique entre populations et augmentant le risque d'extinction locale \citep{Saccheri1998Inbreedingextinction}.

De manière générale, les modifications des attributs forestiers contribuent à une perte de la diversité spécifique et fonctionnelle. 
À long terme, cela peut réduire la résilience des forêts à l'échelle locale et entraîner une diminution des services écosystémiques \citep{Hooper2012globalsynthesis,Edwards2014Maintainingecosystem}. 

%

Parmi les interventions sylvicoles ayant un impact significatif sur les milieux naturels, les coupes forestières occupent une place importante. 
Toutefois, le degré de perturbation causé par ces coupes varie en fonction du type de traitement utilisé, allant de perturbations proches du naturel ou semi-naturel dans le cadre de la gestion forestière extensive, à des perturbations plus artificielles dans le cas de la gestion forestière intensive \citep{Ameray2021Forestcarbon}. 
Le choix des traitements appliqués dépendra principalement des objectifs économiques et écologiques fixés par le plan d'aménagement.


Historiquement, les coupes totales font partie des pratiques sylvicoles les plus courantes dans les forêts tempérés et boréaux \citep{Fedrowitz2014Canretention,Chaudhary2016Impactforest}. 
Elles font partie d'une gestion forestière intensive largement utilisée pour accroître la productivité et la qualité du bois à court terme, afin de répondre aux besoins croissants de l'industrie et d'augmenter la rentabilité \citep{Irland2011TimberProductivitya}.
Ces coupes implique l'abattage de tous les arbres dans une zone définis.
Elles sont techniquement facile à exécuter, car l'ensemble du couvert supérieur de la parcelle est retiré en une seule récolte, entraînant une déforestation temporaire d'une zone auparavant boisée. 
Les coupes totale utilise une structure équienne avec une seule espèce, ce qui simplifie la structure forestière et réduit la diversité biologique entrainant une homogénéisation des peuplements \citep{Rosenvald2008whatwhen}. 
Cette simplification du milieu perturbe des processus écologiques et évolutifs importants et entraine un déclin de la résilience des forêts \citep{Holling2001UnderstandingComplexity}. 
De plus, la période de rotation est plus courte pour ce type de traitement ce qui amène un fréquence des perturbation plus élevé. 
Le maintient de pratiques de gestion forestière transformant radicalement les structures des écosystèmes par rapport à celles observées naturellement augmente de façon importante les probabilités d'observer un déclin progressif de la biodiversité ainsi que l'extinction locale d'espèces \citep{Hanski2000Extinctiondebt}.  
Cependant, certains auteurs souligne que les coupes totales peuvent être utiliser pour imitation des perturbations naturelles de grande ampleur \citep{Greenberg1995comparisonbird}. 

Ces dernières décennies, des pratiques sylvicoles combinant la récolte de bois et la préservation de la biodiversité ont été promues pour atténuer les impacts des coupes totales \citep{Gustafsson2012Retentionforestry}.
Une gestion écosystemique caractérisé par l'émulation des perturbations naturelles a été proposée comme une stratégie prometteuse pour une gestion durable des forêts \citep{Perry1998scientificbasis,Kuuluvainen2002Naturalvariabilitya}. 
Selon cette stratégie, les actions de gestion sont planifiées de manière à émuler les perturbations naturelles et leurs résultats, y compris les structures des peuplements et les successions. 
La gestion écosytémique a pour but de préserver la biodiversité et de maintenir la résilience des peuplements, tout en garantissant la disponibilité d'une grande variété de services écosystémiques \citep{Szaro1998emergenceecosystem,MacDicken2015Globalprogress}.

Les coupes partielles font partie d'une gestion écosystémique visant à préserver la composition et la structure forestières \citep{Bergeron1999Forestmanagementa}.
Elles sont habituellement utilisées dans le cadre d'un plan d'aménagement extensif qui favorise la régénération naturelle et reproduit les perturbations naturelles \citep{Irland2011Timberproductivity}. 
Les coupes partielles consistuent une suppression sélective d'arbres tout en laissant une partie du peuplement intacte \citep{Ameray2021Forestcarbon}. 
La rétention d'arbres dans les coupes partielles offre des structures de succession tardive, favorisant ainsi la préservation d'une plus grande biodiversité \citep{Ameray2021Forestcarbon}.
Elles sont souvent employées pour stimuler la croissance des arbres les plus vigoureux, encourager la diversité des espèces ou préserver une canopée ouverte \citep{Irland2011Timberproductivity}.
Ce type de traitement repose sur une structure multi-âge, avec ou sans mélanges d'espèces, et est caractérisée par des rotations plus longues \citep{Kuuluvainen2009Forestmanagement}. 
La conservation d'arbres en coupe partielle et l'allongement de la période de rotation offrent d'autres avantages sur le plan économique et écosystémique \citep{Ameray2021Forestcarbon}. 
Cela favorise la séquestration du carbone, préserve l'apport de matière organique ainsi que le cycle de nutriment, tout en maintenant un structure hétérogène et différentes niches écologiques pour la faune \citep{Dahlgren1994effectswholetree,Barg1999Influencepartial,Tong2020Forestmanagement,Ameray2021Forestcarbon}.

Comme mentionné précédemment, l'impact de la gestion forestière sur la biodiversité dépend du type de traitement sylvicole utilisé. Cependant, la réaction de la faune à un traitement peut varier selon le type de forêt et les groupes d'espèces concernés \citep{Paillet2010Biodiversitydifferences,Kudrin2023metaanalysiseffects}.

Le défis de préserver la biodiversité et les écosystèmes forestiers tout en répondant aux exigences économiques est d'autant plus grand dans le contexte actuelle des changements climatiques. 
La hausse globale de température représente une menace de plus pour la pérennité de la faune et de la flore en modifiant de façon importantes les conditions environnementales \citep{McKenney2009Climatechange,Trumbore2015Foresthealth,Seidl2017Forestdisturbances,Messier2022Warningnatural}. 
Parmi ces modifications, on prévoit un allongement et une intensification des périodes de sècheresse, une augmentation du nombre de feux de forêt, une altération des régimes de précipitation et une hausse des perturbations biotiques \citep{Parmesan2007Influencesspecies,Joyce2013Climatechange,Gatti2021Amazoniacarbon,Heidari2021Effectsclimate}. 
À cela s'ajoutent des changements dans la phénologie ainsi que dans la distribution des arbres due au manque d'adaptation des végétaux aux nouvelles conditions de leur milieu \citep{Aitken2008Adaptationmigration,Chuine2010Whydoes,Zhu2012Failuremigrate,Gray2013Trackingsuitable}.
De plus, les stresseurs climatiques agissent de façon additive ou synergique avec les activités forestières et l'interactions entre ces facteurs exerce un impact beaucoup plus grand sur la biodiversité et l'environnement \citep{Brook2008Synergiesextinctiona,Tremblay2018Harvestinginteracts,Ochs2022Responseterrestrial,Bouderbala2023Longtermeffect}. 
La composition des forêts pourrait ainsi être altérée, entraînant des ajustements dans les pratiques d'aménagement forestier et les stratégies de conservation \citep{McKenney2009Climatechange,Chmura2011Forestresponses,Lo2011Linkingclimate}.
Différentes mesures d'atténuation ont été proposées pour prévenir la perte des forêts et améliorer la résilience de celle-ci comme l'augmentation de la diversité fonctionnelle et l'amélioration de la connectivité à travers le paysage forestier \citep{Messier2019functionalcomplex}.
D'autres solutions tel que la migration assistée d'arbres ont été proposée comme une mesure d'atténuation permettant le déplacement d'individus ou de matériel génétique depuis un territoire climatique originel vers une zone de climat futur plus propice à la croissance des arbres \citep{Vitt2010Assistedmigration,Dumroese2015Considerationsrestoring,Park2018Informationunderload,Park2023Provenancetrials}. 
La migration assistée d'arbre permettrait de modifier rapidement la composition des peuplements pour mieux convenir au climat futur de celui-ci répondant ainsi aux besoins de conservation, maintenant les services écosystémiques et préservant la valeur économique \citep{Pedlar2011implementationassisted,Ste-Marie2011Assistedmigration,Winder2011Ecologicalimplications}.
Cependant un manque de connaissances et un degré d'incertitude subsistent toutefois autour des nouvelles mesures d'adaptation \citep{Klenk2015assistedmigration,Park2018Informationunderload}. 
Il est donc essentiel de comprendre l'impact des traitements sylvicoles sur la biodiversité, dans un contexte où l'aménagement forestier doit adapter sa pratique aux changements climatiques et aux déclin de la biodiversité, tout en répondant aux besoin économiques.


\section*{Espèces à l'étude}
\label{sec:species}
\phantomsection\addcontentsline{toc}{section}{\nameref{sec:species}}

Au sein de la biodiversité forestières, la faune du sol possède une importance primordiale dans les écosystèmes forestiers, en contribuant entre autres à la circulation de la matière et de l'énergie à travers la chaîne alimentaire, ainsi qu'au recyclage des nutriments \citep{Seibold2021contributioninsects,Kudrin2023metaanalysiseffects}.
Cependant cette communauté fait partie des groupes d'espèces les plus affecté par les pertubations environnementales au sein de la biodiveristé forestières \citep{Marshall2000Impactsforest,Coyle2017Soilfauna}. 
En raison de leur petite taille, la faune du sol possède une capacité de dispersion plus restreinte, ce qui les rend plus vulnérables face aux perturbations de leur environnement \citep{Kudrin2023metaanalysiseffects}.

Les pratiques sylvicoles comme les coupes forestières peuvent induisent des modifications brusque et drastique dans les propriétés de l'habitat forestiers. 
Ces altérations peuvent affecter négativement la biodiversité, particulièrement la faune vivant à la hauteur du sol \citep{Lindo2003Microbialbiomass,Paillet2010Biodiversitydifferences,Fedrowitz2014Canretention,Chaudhary2016Impactforest}. 
Le retrait de la canopée augmente l'exposition de la surface du sol au rayonnement solaire, ce qui entraîne une élévation de la température et une modification de l'humidité du sol \citep{Lindo2003Microbialbiomass,Brook2008Synergiesextinction,Zhang2022Intensiveforest}. 
Ces changements sont exacerbés par une augmentation de la vitesse du vent et une intensification des précipitations atteignant le sol \citep{Keenan1993ecologicaleffects,Heithecker2007Edgerelatedgradients}. 

La structure du sol subit également des altérations, notamment une compaction accrue due au passage des machines forestières, ce qui peut affecter la porosité du sol et, par conséquent, la faune qui y vit \citep{Battigelli2004Shorttermimpact,Mazerolle2021Woodlandsalamander}. 
De plus, les opérations forestières perturbent la disponibilité des nutriments en modifiant la qualité et la quantité de litière, en altérant les sécrétions racinaires, en provoquant un lessivage, et en changeant les propriétés chimiques du sol \citep{Covington1981Changesforest,Marshall2000Impactsforest,Lindo2003Microbialbiomass,Battigelli2004Shorttermimpact}. 
Ce bouleversement des conditions microclimatiques et physico-chimiques du sol peut entraîner des répercussions directes sur la biodiversité, 
en particulier sur les organismes qui dépendent des microhabitats comme le bois mort, les cavités dans les arbres matures ou les plaques racinaires \citep{Berg1994ThreatenedPlant,Spies1999Dynamicforest,Bouget2005Shorttermeffect,Christensen2005Deadwood,Brassard2008EffectsForest}.
Ces éléments structurels, souvent éliminés ou réduits lors des opérations de coupe sont généralement remplacés par des pistes de débardage et les chemins d'exploitation \citep{Hansen1991ConservingBiodiversity}. 

Les microhabitats jouent pourtant un rôle crucial pour une grande partie de la faune du sol, comme les arthropodes, les amphibiens et les microorganismes \citep{Paillet2010Biodiversitydifferences,Fedrowitz2014Canretention,Chaudhary2016Impactforest}. 
La disparition de ces refuges accroît la vulnérabilité de nombreuses espèces, tout en favorisant l’émergence de conditions environnementales défavorables à leur survie. 
Ainsi, les espèces forestières qui dépendent des conditions de fraîcheur et d’humidité typiques des sols forestiers non perturbés peuvent voir leur population diminuer, voire disparaître localement \citep{Kudrin2023metaanalysiseffects}. 
En revanche, des espèces associées aux zones ouvertes et plus sèches peuvent coloniser les parcelles récemment coupées, modifiant ainsi la composition spécifique des communautés fauniques.

Plusieurs recherches se sont intéressés aux changements des caractéristiques forestières comme le débris ligneux grossiers, la profondeur de la litière, l'ouverture de la canopée 
et leur influence sur la faune du sol après la récolte forestière afin de guider la gestion forestière \citep{Semlitsch2002CriticalElements,McKenny2006Effectsstructural}. 

La faune du sol regroupe cependant une grande diversité de taxons, caractérisés par des différences significatives sur le plan biologique et écologique \citep{Kudrin2023metaanalysiseffects}. 
Par conséquent, leur réponse à l'exploitation forestière varier selon le type de traitement appliqué et le groupe d'espèce étudié \citep{Malmstrom2009Dynamicssoil,Paillet2010Biodiversitydifferences}.

Malgré leur abondance dans les écosystèmes forestiers de l'est de l'Amérique du Nord, les amphibiens et les arthropodes ont longtemps été négligés dans les stratégies de gestion forestière \citep{deMaynadier1995relationshipforest}. 
Or, des recherches ont souligné leur importance écologique, tant pour le fonctionnement des réseaux trophiques que pour la dynamique du carbone au sein des sols forestiers. 
Leur sensibilité aux changements environnementaux et leurs capacité de dispersion restrainte en font des modèles d’étude pertinent pour mieux comprendre les effets des pratiques forestières sur la biodiversité et l'intégrité écologique des forêts \citep{pongeVerticalDistributionCollembola2000,Ojala2001Dispersalmicroarthropods,birdChangesSoilLitter2004,Maleque2009Arthropodsbioindicators}.
De plus, les amphibiens et les arthropodes ont subi un déclin majeur au cours des dernières décennies, principalement dû aux pratiques d'aménagement forestier et aux changements climatiques \citep{Houlahan2000Quantitativeevidence,Stuart2004Statustrends,Warren2018projectedeffect,Wagner2021Insectdecline}. 

Parmi les groupes d'espèces fréquemment étudiés chez ces taxons, on retrouve la salamandre cendrée de l'Est (\textit{Plethodon cinereus} (Green, 1818)), les carabes (Carabidae) et les collemboles (Collembola).

\subsection*{Salamandre cendrée}

La salamandre cendrée, un membre des Plethodontidae, représente l'une des biomasses les plus importantes chez les vertébrés des forêts nord-américaines \citep{Burton1975Salamanderpopulations,Petranka1993Effectstimber,semlitschAbundanceBiomassProduction2014a}. 
En tant qu'espèce exclusivement terrestre et dépourvue de poumons, elle dépend de la respiration cutanée et donc des conditions d'humidité pour assurer ses échanges gazeux \citep{Heatwole1961Relationsubstrate}. 
Ce mode de respiration la contraint à occuper des microhabitats spécifiques, en surface lorsque la température et l'humidité sont favorables, ou en profondeur dans le sol durant des périodes moins propices \citep{Grizzell1949HibernationSite,FraserEmpiricalEvaluation1976,Jaeger1980MicrohabitatsTerrestrial}. 
Le retrait des refuges à la surface, tels que les débris ligneux, après la récolte peut réduire la qualité de l'habitat pour les salamandres et diminuer le temps qu'elles passent à la surface du sol forestier \citep{Achat2015Quantifyingconsequences,Peele2017EffectsWoody}. 
Étant donné que les salamandres cendrées se nourrissent et se reproduisent à la surface du sol forestier, des conditions de surface défavorables, un compactage des sols et de faibles niveaux de CWD peuvent affecter la dynamique des populations \citep{Peterman2014Spatialvariation}. 
Prédateur généraliste, la salamandre cendrée contribue de manière significative à la régulation des invertébrés détritivores, influençant directement les processus de décomposition, la circulation des nutriments dans les sols et la dynamique du carbone \citep{Burton1975Energyflow,Wyman1998Experimentalassessment,Walton2013Topdownregulation,Hickerson2017Easternredbacked}. 
Par ailleurs, elle constitue une proie de haute valeur nutritive pour divers prédateurs tels que les oiseaux, mammifères et reptiles, renforçant ainsi son rôle dans les dynamiques trophiques forestières \citep{Burton1975Energyflow,Pough1987abundancesalamanders,Petranka1998SalamandersUnited}. 
Compte tenu de sa sensibilité aux perturbations environnementales, notamment en raison de sa respiration cutanée, la salamandre cendrée est souvent utilisée comme indicateur de la qualité des sols forestiers \citep{Welsh2001caseusing}. 
Son abondance et sa présence sont ainsi des paramètres couramment évalués pour mesurer l'impact des pratiques sylvicoles sur la biodiversité \citep{Harpole1999Effectsseven,Grialou2000effectsforest,Homyack2009Longtermeffects,Hocking2013Effectsexperimental,Mazerolle2021Woodlandsalamander}. 

\subsection*{Carabes}

Nous utilisons les carabes comme deuxième organismes modèles en raison de leur documentation détaillé sur les plans taxonomique et écologique \citep{loveiEcologyBehaviorGround1996}. 
Ces insectes ont une courte durée de vie, occupent une position élevée dans la chaîne alimentaire et réagissent rapidement et de manière complexe aux modifications de leur environnement \citep{loveiEcologyBehaviorGround1996}.
Les carabes jouent un rôle écologique dans la régulation des populations d'invertébrés, se nourrissant principalement d'aphides, de collemboles et d'escargots, tout en étant à leur tour des proies pour divers amphibiens, reptiles, oiseaux et mammifères \citep{loveiEcologyBehaviorGround1996}. 
Ils sont particulièrement sensibles à la complexité structurelle des peuplements forestiers, et ce, à différentes échelles temporelles et spatiales \citep{Butterfield1995Carabidbeetle,loveiEcologyBehaviorGround1996,Niemela2007effectsforestry}.
Avec environ 40 000 espèces répertoriées, les carabes constituent l'une des familles les plus diversifiées parmi les coléoptères et représentent l'une des plus fortes abundances d'arthropodes du sol \citep{Erwin1985taxonpulse,loveiEcologyBehaviorGround1996,Rochefort2006GroundBeetle}. 
Ils sont largement distribués et présents en nombre significatif dans presque tous les écosystèmes terrestres, bien que les espèces varient dans leur sélection d'habitats \citep{loveiEcologyBehaviorGround1996,kotzeFortyYearsCarabid2011a,Larochelle2003naturalhistory}. 
Les changements environnementaux et les modifications d'habitat favorisent certaines espèces au détriment d'autres, ce qui rend leur diversité et leur sensibilité particulièrement intéressantes pour étudier l'impact des perturbations environnementales \citep{Rainio2003Groundbeetles}.

\subsection*{Collemboles}

Parmi les taxons de la mésofaune du sol, les collemboles représentent un des groupes les plus abondants et les plus diversifiés de petits arthropodes \citep{rusekBiodiversityCollembolaTheir1998}. 
Différentes communautés de collemboles occupent un ensemble de niches écologiques allant de la litière aux différents horizons du sol \citep{pongeVerticalDistributionCollembola2000}.
La répartition verticale de ces communautés dépend essentiellement des conditions abiotiques du sol telles que la luminosité, le taux d’humidité ou encore la porosité.
Ces invertébrés exercent une influence significative sur l'écologie forestière à travers leur influence sur les micro-organismes, sur la décomposition et le cycle des nutriments.
Étant principalement fongivore et détritivore, les différentes communautés de collemboles influencent les taux de décomposition du bois, affectent la structure physique et les taux de minéralisation de la litière, 
influencent l'absorption des nutriments et régulent la communauté microbienne, en plus de participer à la formation de microstructure du sol \citep{Petersen1982ComparativeAnalysisa,Neher2012Linkinginvertebrate,Maass2015Functionalrole,Potapov2016Connectingtaxonomy}. 
Les collemboles participent également à la chaine trophique en constituant une ressource alimentaire abondantes pour plusieurs organismes comme les amphibiens, les coléoptères, les arachnides, les oiseaux et les reptiles.

Grâce aux relations trophiques qui les lient, leur sensibilité aux variations des conditions environnementales, et leur dépendance aux caractéristiques forestières comme les débris ligneux et la litière, 
ces trois groupes d'espèces constituent des modèles pertinents pour étudier l'impact des pratiques forestières sur la faune du sol. 

La plupart des travaux qui se sont intéressés aux impacts des traitements sylvicoles sur la faune discutent généralement des effets directs des perturbations sur un ou plusieurs groupes d'espèces, 
sans tenir compte des relations existantes entre les variables environnementales et les différents groupes d'espèces, 
négligeant ainsi les effets indirects des coupes sur la faune du sol \citep{josephIntegratingOccupancyModels2016,Pollierer2021Diversityfunctional,Kudrin2023metaanalysiseffects}. 
Mon projet comblait justement cette lacune en essayant de comprendre comment les effets des traitements sylvicoles se propagent à l’intérieur du réseau écologique forestier et influence la dynamique de la faune du sol.  
Ultimement, ce gain de connaissances fournira des outils précieux pour faciliter la gestion durable des forêts.


\section*{Objectifs et hypothèses}
\label{sec:objectifs}
\phantomsection\addcontentsline{toc}{section}{\nameref{sec:objectifs}}

Le but de mon étude était de comprendre comment les traitements sylvicoles, effectués dans un contexte de migration assistée, 
affecte la dynamique des écosystèmes du sol forestier. Les objectifs qui s’y rattachaient étaient :

\begin{enumerate}
    \item De quantifier l'effet des traitements de coupes forestières sur les variables environnementales qui influencent l'utilisation de l'habitat par la faune du sol.
    \item De mesurer l'impact des coupes forestières sur l'utilisation de l'habitat par la faune du sol.
\end{enumerate}

L'hypothèse 1.1 liée à notre deuxième objectif soutenait que les variables environnementales favorables à l'utilisation de l'habitat par les espèces fluctuent 
en fonction de l'intensité des coupes forestières. Ainsi, les traitements de coupes forestières constituent une variable englobant 
les changements de conditions environnementales.

L'hypothèse 2.1 attachée à notre premier objectif stipulait que les traitements de coupe forestière entraînent une modification de l'utilisation de l'habitat 
par la faune du sol et se propage à travers le réseau trophique. Spécifiquement, les coupes affectent l'utilisation de l'habitat par les salamandres et 
les grands carabes (compétiteurs de salamandres), ce qui modifie ensuite la sélection d'habitat des petits carabes (proies de salamandres), puis enfin des collemboles (proies de grands carabes et salamandres).



\cleardoublepage

\bibliography{References.bib}
\bibliographystyle{ecologyNewFR.bst}
          % introduction
\chapter{Soil fauna occupancy responses to cutting treatments in assisted tree migration context}     % numéroté
\label{chap:SEM}    

William Devos$^1$, Mathieu Bouchard$^1$, Marc Mazerolle$^1$

%href{mailto:william.devos.1@ulaval.ca}
$^1$ Centre d'étude de la forêt, Département des sciences du bois \\ 
et de la forêt, Université Laval, Québec, QC G1V 0A6, Canada. \\ 

\clearpage



\section*{Résumé}
\label{sec:resume1}
\phantomsection\addcontentsline{toc}{section}{\nameref{sec:resume1}}

\begin{otherlanguage*}{french}
  <Résumé de l'article en français. Obligatoire.>

  \textbf{Mots-clés} : <ajouter des mots clés>
\end{otherlanguage*}

\clearpage

\section*{Abstract}
\label{sec:abstract1}
\phantomsection\addcontentsline{toc}{section}{\nameref{sec:abstract1}}

\begin{otherlanguage*}{english}
  <English abstract of the paper. Optional, but recommended.>

\textbf{Keywords}: <add some keywords> 
\end{otherlanguage*}

\cleardoublepage

\section*{Introduction}
\label{sec:intro1}
\phantomsection\addcontentsline{toc}{section}{\nameref{sec:intro1}}

%\defcitealias{keylist}{alias}

\section*{Material and methods}
\label{sec:matmet1}
\phantomsection\addcontentsline{toc}{section}{\nameref{sec:matmet1}}

\subsection*{Study area}
\label{subsec:area}
\phantomsection\addcontentsline{toc}{subsection}{\nameref{subsec:area}}

\begin{otherlanguage*}{english}
  Our study was conduct within the Portneuf Wildlife Reserve in the Captial-Nationale administrative region (72°24'W, 47°07'N) near Rivière-à-Pierre and Lac Amanites. 
  This area is located within the balsam fir-yellow birch bioclimatic domain (Saucier et al., 2009) and lie on a deep glacial till as surface deposit (Gosselin, 1998).
  The mean daily temperature is 4 ◦C for the 1981-2010 period of the nearest weather station (Lac aux sables)(Environment Canada, 2019). 
  Based on the same report, the mean annual precipitation and snowfall are respectively 1133.2 mm and 230.3 cm.
  We used the assisted migration experimental system establish in 2018 by the Minister of Natural Resources and Forests to collect our data (MNRF)(Royo, 2023).
  This system is a factorial experimental design with a split-split-split-plots using 4 replications (complete random blocks). 
  Each whole block (200 m x 140 m) is occupied by an overstory treatment (clearcut and 50\% shelterwood cut). 
  A cervid exclusion treatments (excluded and non-excluded) is used as a split plot and a competing vegetation treatments as a split-split plot. 
  Climatic analogs associated with three climate projections (current climate, mid-century 2050, and end of century 2080) 
  are used as a split-split-split plots with the seedlings of 9 species in mixed planting: black cherry (\textit{Prunus serotina}), northern red oak (\textit{Quercus rubra}), 
  northern white-cedar (\textit{Thuja occidentalis}), shagbark hickory, (\textit{Carya ovata}), sugar maple (\textit{Acer saccharum}), red pine (\textit{Pinus resinosa}), 
  red spruce (\textit{Picea rubens}), white spruce (\textit{Picea glauca}), and white pine (\textit{Pinus strobus}).

\end{otherlanguage*}


\subsection*{Sampling design}
\label{subsec:sampling}
\phantomsection\addcontentsline{toc}{subsection}{\nameref{subsec:sampling}}

We selected six samples units (10 m x 7,5 m each) in both overstory treatment in four blocks and added three controls units outside every blocks for a total of 60 units in our system. 
Controls were separated from the blocks per at least 10 meters to remove treatments effects.
Each sample unit contained three sampling methods to collect our data. 
Artificial cover boards were used to count red-backed salamanders (Hesed, 2012; Mazerolle et al., 2021; Moore, 2009), 
pitfall traps for the carabid monitoring (sources), and soil cores for springtails assessment (sources) (figure).

Artificial cover boards is commonly employed for salamander sampling and yield similar or superior results compared to other methods such as active searching (Hyde, Simons, 2001; Moore, 2009). 
The use of coverboards helps reduce variability in the number and size of sampled ground objects (Hyde and Simons 2001). 
Cover boards were made of spruce wood, measured 25 cm x 30 cm x 5 cm and were placed directly on the ground without litter underneath 
to maintain higher humidity level under the boards (Mazerolle et al., 2021). 
Six boards were centered in each sample and control units and separate between each other by a minimum distance of 1.5 m (figure), resulting in a total of 360 cover boards. 
All 360 boards were placed outside duing march 2022 to provide aging of the boards under natural conditions, thereby increasing their likelihood of being used by salamanders (article).  

Pitfall trapping is a passive sampling method used to assess the species richness of ground-dwelling invertebrates (Knapp, Růžička, 2012; Kotze et al., 2011; Lövei,  Sunderland, 1996). 
This capture method works particularly well for active arthropods moving on the ground, such as ground beetles (Baars, 1979; Lövei, Sunderland, 1996). 
One pitfall was positioned at the center of each sample and control units (figure), resulting in a total of 60 pitfall traps.

Pitfall traps included a container with a diameter of 12.5 cm, a depth of 25 cm and a cover with an opening below the cover (figure). 
Each pitfall traps were placed in the soil at a depth allowing the container's opening to be juxtaposed with 
the soil surface (Figure 4). These traps are produced by Bio.Contrôle services and are equipped with a cover raised 4.5 cm above the trap to prevent debris and rain from filling 
the container, as well as a protective grid with a mesh size of 15 mm, limiting trap access to carabid-sized individuals and reducing the chances of predation by small mammals (Figure 5). 
Typically, a preserving liquid (such as propylene glycol) is placed in the bottom of the container to preserve captured individuals. 
For our sampling, we will recommend using traps without preserving liquid (Luff, 1975). 
This choice was made considering the probabilities of salamander capture. 
Salamanders can have a width similar to that of some ground beetles, making it difficult to limit access to one without influencing the sampling of the other. 
For ethical reasons, we prefer not to use liquid and will add a wet sponge to the bottom of each container to maintain a suitable level of humidity for salamanders (Figure 4).
During the trapping period, the pitfall traps will be visited daily to collect their contents, release accidentally captured salamanders, and limit predation in the traps (Luff, 1975). 
We will conduct four trapping periods of five consecutive days each, namely in mid-May, mid-June, mid-July, and mid-August. 
Outside the sampling periods, all traps will be closed with their cover and adhesive tape around the mesh of the grid to prevent capturing individuals. 
During site visits, arthropods will be collected using a net and then transferred to an individual and labeled container containing 75\% alcohol. 
The containers will be brought back to the laboratory at the Forest Research Directorate for sorting and identification of ground beetles.

Soil cores

limitation des capture (detection vs abondances du au retrait des individues)
However, it would be possible to account for imperfect and variable species detection in our models to enable comparison (Mazerolle et al., 2007).


The site will be visited four times during the summer of 2022, namely in mid-May, mid-June, mid-July, and mid-August. 

This number of repetitions will allow for assessing the temporal effect during analyses. The visit to all sites will last for one week. 
The order in which blocks will be visited will be determined randomly so that the time of day or the sampler's fatigue level does not impact the results. 
During each visit, all boards will be checked to count the number of salamanders present per sampling unit. 
No individual manipulation is deemed necessary in our study. Each lifted board will be replaced, minimizing disturbance and environmental impact as much as possible.






 

\subsection*{Sites habitat characteristics}
\label{subsec:habitat}
\phantomsection\addcontentsline{toc}{subsection}{\nameref{subsec:habitat}}

\subsection*{Statistical analyses}
\label{subsec:analyses}
\phantomsection\addcontentsline{toc}{subsection}{\nameref{subsec:analyses}}

\clearpage

\section*{Results}
\label{sec:results1}
\phantomsection\addcontentsline{toc}{section}{\nameref{sec:results1}}

\clearpage

\section*{Discussion}
\label{sec:discu1}
\phantomsection\addcontentsline{toc}{section}{\nameref{sec:discu1}}

\section*{Conclusion}
\label{sec:conclu1}
\phantomsection\addcontentsline{toc}{section}{\nameref{sec:conclu1}}

\section*{Acknowledgements}
\label{sec:acknowl1}
\phantomsection\addcontentsline{toc}{section}{\nameref{sec:acknowl1}}

\section*{Conflict of interest}
\label{sec:conflict1}
\phantomsection\addcontentsline{toc}{section}{\nameref{sec:conflict1}}

None declared

\section*{Author contributions}
\label{sec:author1}
\phantomsection\addcontentsline{toc}{section}{\nameref{sec:author1}}

\cleardoublepage


\begin{otherlanguage}{english}
\bibliography{references.bib}
\bibliographystyle{ecologyNewEN.bst}
\addcontentsline{toc}{section}{References}
\end{otherlanguage}
    % chapitre 1
\chapter*{Conclusion}           % ne pas numéroter
\label{chap-conclusion}         % étiquette pour renvois
\phantomsection\addcontentsline{toc}{chapter}{\nameref{chap-conclusion}} % inclure dans TDM

Le but de cette étude était de comprendre comment les traitements sylvicoles, effectués dans un contexte de migration assistée, affectent la dynamique des écosystèmes du sol forestier.
Les objectifs qui s’y rattachaient étaient, d'une part, de quantifier l'effet des traitements de coupes forestières sur les variables environnementales qui influencent 
l'utilisation de l'habitat par la faune du sol et, d'autre part, de mesurer l'impact des coupes forestières sur l'utilisation de cet habitat par la faune du sol.
Pour répondre à ces deux objectifs, nous avons développé un modèle d'équations structurelles pour tester et mesurer l'effet des coupes sur le volume de débris ligneux, 
la profondeur de litière et l'ouverture de la canopée, ainsi que sur la probabilité d'occupation des salamandres cendrées, des carabes et sur la biomasse de collemboles.
Notre première hypothèse stipulait que les variables environnementales favorables à l'utilisation de l'habitat par les espèces fluctuent 
en fonction de l'intensité des coupes forestières et qu'ainsi, les traitements de coupes forestières constituent une variable englobant les changements de conditions environnementales. 
Notre seconde hypothèse soutenait que les traitements de coupe forestière entraînent une modification de l'utilisation de l'habitat par la faune du sol et se propagent à travers 
le réseau trophique des salamandres vers les collemboles, en passant par les carabes. 
Les principales conclusions tirées de cette étude sont :

\begin{enumerate}
  \item Les coupes forestières ont un effet significatif sur les variables environnementales influençant l'utilisation de l'habitat par la faune du sol. Ces changements suivent de manière générale le niveau d'intensité du traitement appliqué, avec des perturbations plus importantes dans les coupes totales, suivi des coupes partielles.
  \item La probabilité d'occupation des salamandres et des carabes ne semble globalement pas affectée par les différents traitements de coupes forestières comparativement aux sites témoins. Toutefois, un effet marginal a été mesuré chez les salamandres où la probabilité d'occupation de celles-ci était plus faible dans les coupes totales par rapport aux coupes partielles.
  \item La biomasse des collemboles ne variait pas de façon significative entre les différents traitements sylvicoles.
  \item La relation entre les trois groupes d'espèces ne semble pas être variée en fonction des coupes. Ainsi, la biomasse des collemboles n'était pas affectée par le changement de probabilité d'occupation des salamandres et des carabes, et la présence des salamandres ne modifie pas la probabilité d'occupation des carabes de petite taille.
\end{enumerate}



Section résultat et perspective de recherche :
présentation des résultatsinterprétations de résultats
inclure des amélioration dans la recherche, des chose a améliorer ou a investiguer

importance de l'étude pour combler le manque de connaissances
avantage de L,étude
SEM
projet dreamQC étude a long terme sur l'impacte de la migration assistée sur petite faune 

% complexité du SEM

% trait

kudrin The various reactions of soil invertebrates to disturbances may be attributed to their functional traits [34,35]. 
Recent works suggest that focusing on functional traits can provide greater insights into the mechanisms driving ecosystem change and recovery [36–38].

% top -down vs bottom-up

Contrairement à notre approche top-down, certaine étude soutienne que l'effet des perturbations dans le réseau trophique se propage de façon bottom-up et impacterais premièrement le bas de la chaine alimentaire ce qui impacterais par la suite les prédateur.
Notre étude n'ayant pas mesuré d'effet directe des coupe forestières sur la biomasse des collemboles, il est peu probable que nous ayons observé ce genre d'effets. 

% type de foret

kudrin :
 l'abondance des colembole diminue dans les foret coniférienne mais ne change pas dans les foret mixte ou feuillu.
 la richesse spécifique des collemboles  ne change pas dans les foret conniférienne mais augmente dans les foret feuillu.
 l'abondance des coléoptère diminue dans les foret feuillu et mixte mais pa en foret connifère.
 le typede foret n'as pas d'efet sur la richesse spécifique des coléoptère.

 % Comparer avec d'autres types de foret (spatiale)
  répéter le type d'étude dans d'autre type de foret, conifère, feuillu. (kudrin)
  Our analysis revealed a significantly weaker effect of harvesting on the abundance of Collembola and Oribatida in deciduous and mixed forests compared to coniferous forests (Figure 5). The negative response of soil fauna to harvesting may be due to significant changes in abiotic conditions [12]. Several studies have documented significant alterations in temperature, soil moisture, soil compaction, and the quality and thickness of forest litter resulting from harvesting coniferous forests [59–61]. kudrin.
 
 
 % Quels sont les defaults ou amelioration du projet, ainsi que ses limitations.
  % une plus longue prériode d'échantillonnage
  limited number of sampling periods
  limitation dans la complexité du des analyses SEM

  otto 2011:
  Natural cover object searches provided greater power for detecting a similar change in occupancy, largely because of high sampling size

  % limitation dans le nombre d'individus pour les salamandres
    intervalle de confiance
  % essaie d,autre methodes
  reproduire l'étude avec de l'abondance fournira une compréhention approfondit mais ne pouvez pas etre utilisé ici en raison de la recolte des individus
  contrainte limitant de population fermé
  abondance au lieu d'occupation nécéssite ( carabes et collemboles), CMR pour les salamandres
  s'interresser au trait fonctionnel 

  % choix de classification des groupes
  trait fonctionnel a la place de taxonomique

  % degrée de décomposition des planche , voir otto pour info sur importance des planches dégradée

  % ouverture

Laigle2021Directindirect
Numerous studies highlighted the need to incorporate biotic interactions, in particular between organisms of different trophic levels (Lavorel 2013; Brousseau and others 2018a; Seibold and others 2018), to better understand the effects of disturbance on ecosystem functioning (Werner and Peacor 2003; The bault and Loreau 2003; Moretti and others 2013)

Studies of biodiversity responses to disturbance have traditionally focused on direct effects on a single taxon and trophic level at a given point in time, neglecting potential interactions between species of other trophic levels (Werner and Peacor 2003; Barnes and others 2016; Brose and Hillebrand 2016).
Forest disturbance intensity was found to have a bottom-up effect on species community composition, from lower trophic levels (for example, detritivorous springtails) up to soil fauna top predators (for example, running spiders).


  % Faire de étude à plus court terme mais surtout a plus long terme (temporel)
  we have limited knowledge of how quickly soil animals react to harvesting and what the rate of their recovery is.
  Despite the relatively large datasets on the abundance of Collembola and Coleoptera, no temporal dependence of the effect size over twenty years was found. kudrin


  Several mechanisms may explain the effect of management on forest biodiversity: changes in tree age structure, vertical stratification, and composition of tree species, which affect light, temperature, moisture, litter, and topsoil conditions (Sebastia et al. 2005; Standovar et al. 2006); presence of microhabitats (e.g., dead wood, veteran trees, cavities, root plates) specific to unmanaged (Berg et al. 1994; Bouget 2005a; Christensen et al. 2005; Gibb et al. 2005) or managed forests (e.g., skid trails and haul roads) (Hansen et al. 1991; Gosselin 2004); and forest cover continuity and features resulting from extensive management in the past (Hjalten et al. 2007). The pattern of response may therefore depend on which of the above mechanisms, or which combinations of them, have the strongest effects on different taxonomic or functional groups. paillet 

  kudrin :
Les forêts de conifères et les forêts de feuillus, par exemple, sont très différentes en termes de conditions pédologiques et microclimatiques, ce qui se reflète dans la dissimilarité de la composition et de la structure de la faune du sol [14,19,20].

% Force de l'utilisation des SEM

% Possibilité de mesuré l'abondance  et limitation            % conclusion

\appendix                       % annexes le cas échéant
\counterwithin{figure}{section}
\counterwithin{table}{section}
%\include{Appendix_Fayard_Aurore_111283522}

\bibliographystyle{ecologyNewFR} % Style de citation en anglais
\bibliography{references}

\end{document}


% pdflatex memoire_WD.tex
% bibtex chapitre1-articles
% bibtex memoire_WD
% pdflatex memoire_WD.tex
% pdflatex memoire_WD.tex
