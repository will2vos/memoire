\chapter*{Introduction générale}         % ne pas numéroter
\label{chap-introduction}       % étiquette pour renvois
\phantomsection\addcontentsline{toc}{chapter}{\nameref{chap-introduction}} % inclure dans TdM

\section*{Mise en contexte}
\label{sec:contexte}
\phantomsection\addcontentsline{toc}{section}{\nameref{sec:contexte}}

\section*{Objectifs et hypothèses}
\label{sec:objectifs}
\phantomsection\addcontentsline{toc}{section}{\nameref{sec:objectifs}}

Notre étude vise à comprendre de façon intégrative l'impacte de la préparation de sites, en vue d'une migration assistée d'arbres en forêt mixte, sur la dynamique des écosystème du sol forestier.
Notre objectif est de dévelloper un modèle d'équations structurelles pour quantifier les impactes directes et indirectes des traitements de coupes forestières (coupe totale, coupe partielle) sur : 

\begin{enumerate}
    \item Les attributs forestiers propices à la sélection d'habitats par la faune du sol tel que le volume de débris ligneux, la profondeur de litière ainsi que l'ouverture de la canopée. 
    \item Le relations de cooccurences entre la salamandres cendrée de l'Est (\textit{Plethodon cinereus}), les carabes (Carabidae) et les collemboles (Collembola).
\end{enumerate}

L'hypothèse 1.1 associé à notre objectif soutient que les traitements de coupes forestières influence directement les attributs environnementaux des forêts en 
réduisant la profondeur de litière ainsi qu'en diminuant le volume de débris ligneux, suite à une baisse de recrutement de feuilles et de bois au sol. 
À l'inverse, nous pensons qu'une coupe forestière plus intensive amène une hausse dans l'ouverture de la canopée.

L'hypothèse 1.2 liée à notre objectif atteste que les traitements de coupes forestières entraine, sous forme d'effet cascade, un changement dans l'utilisation de l'habitat par la faune du sol. 
En premier lieu, l'influences des coupes forestières modifient la sélection d'habitat du haut de la chaîne trophique (salamandres, carabes de grande taille)
puis influence ensuite les niveaux trophiques inférieurs tels que les petits carabes, 
et enfin, les collemboles.

\cleardoublepage

\bibliography{References.bib}
\bibliographystyle{ecologyNewFR.bst}
