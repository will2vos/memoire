\chapter*{Introduction générale}         % ne pas numéroter
\label{chap-introduction}       % étiquette pour renvois
\phantomsection\addcontentsline{toc}{chapter}{\nameref{chap-introduction}} % inclure dans TdM

\section*{Mise en contexte}
\label{sec:contexte}
\phantomsection\addcontentsline{toc}{section}{\nameref{sec:contexte}}
À travers les services écosystémiques qu'elles fournissent et leurs valeurs économiques, les forêts jouent un rôle prédominant à l'echelle planétaire \citep{Balvanera2006Quantifyingevidence}.
L'environnement qu'elles apportent dans les écosystème terrestre permet le maintien d’une biodiversité importante ainsi qu'une régulation de facteurs biogéochimiques \citep{Pawson2013Plantationforests}. 
Cependant, les changements climatiques représentent aujourd'hui un des défis les plus préoccupant pour le maintient des forêts tel qu'on les connait \citep{McKenney2009Climatechange,Messier2022Warningnatural,Seidl2017Forestdisturbances,Trumbore2015Foresthealth}.  
Malgré les engagements internationaux pris au courant des dernière décennie pour diminuer la hausse de température lié aux activités anthropiques, 
les projections climatiques actuelles nous indiques une hausse future de température globale dépassant 1.5\up{o}C par rapport à l'air préindustrielle \citep{Matthews2022Currentglobal}.
En raison de ses latitudes nordiques, le Canada est particulièrement vulnérable à l'augmentation des températures et aux perturbations environnementales qui en découlent \citep{Alo2008Potentialfuture,Bush2019Canadachanging}. 

Les forêts de l'Est de l'Amérique du Nord sont et seront donc ainsi particulièrement touchées par cette hausse de température \citep{Park2014Canboreal,Mahony2017closerlook,Messier2022Warningnatural,Sittaro2017Treerange}.
Parmis les conséquences attendu, on prévoit un allongement et une intensifications des périodes de sècheresse, une augmentation du nombre de feu de forêt et une hausse des perturbations biotiques \citep{Gatti2021Amazoniacarbon,Heidari2021Effectsclimate,Joyce2013Climatechange,Parmesan2007Influencesspecies}. 
À cela s'ajoute des changements dans la phenologie \citep{Chuine2010Whydoes} ainsi que dans la distribution des arbres \citep{Gray2013Trackingsuitable,Zhu2012Failuremigrate} due au manque d'adaptation des végétaux aux nouvelles conditions de leur milieu \citep{Aitken2008Adaptationmigration}.
Les changements climatiques se produisent toutefois plus rapidement que la capacité d'adaptation ou de déplacement des arbres \citep{Aitken2008Adaptationmigration,Harrison2020Plantcommunity,Loarie2009velocityclimate,Messier2022Warningnatural,Williams2013Preparingclimate,Vitt2010Assistedmigration}, 
menaçant ainsi leur capacité à se maintenir dans des milieux favorables à leur survie et leur croissance \citep{Sittaro2017Treerange,Woodall2018Decadalchanges,Zhu2012Failuremigrate}.
Des changements pourraient ainsi être observés dans la composition des forêts menant à des modifications dans la pratique de l'amenagement forestier et de la conservation \citep{Chmura2011Forestresponses,Lo2011Linkingclimate,McKenney2009Climatechange}.


\section*{Migration assitée et préparation de sites}
\label{sec:fam}
\phantomsection\addcontentsline{toc}{section}{\nameref{sec:fam}}

Plusieurs appel à adapter l'amenagement ont été proposer afin de préserver les milieux forestiers et leurs bienfaits \citep{Messier2021sakeresilience,Nagel2017Adaptivesilviculture}.
Différentes mesures d'atténuation ont été proposées pour prévenir la perte des forêts et améliorer la résilience de celle-ci comme l'augmentation de la diversité fonctionnel et l'amélioration de la connectivité à travers le paysage forestier \citep{Messier2019functionalcomplex}.
Parmi toutes les solutions envisagées, la migration assisté d'arbres est proposé comme une mesure d'atténuation permettant le déplacement d'individus ou de matériel génétique depuis un territoire climatique originel vers une zone de climat futur plus propices à la croissances des arbres \citep{Dumroese2015Considerationsrestoring,Palik2022Operationalizingforestassisted,Park2023Provenancetrials,Park2018Informationunderload,Pedlar2011implementationassisted,Vitt2010Assistedmigration,Williams2013Preparingclimate}. 
La migration assistée d'arbre permettrait de modifier rapidement la composition des peuplements pour mieux convenir au climat future de celui-ci \citep{Pedlar2011implementationassisted} 
répondant ainsi aux besoins de la préservation des espèces, au maintient de la valeur économique et à la préservation de la biodiversité et des services écosystémiques \citep{Ste-Marie2011Assistedmigration,Winder2011Ecologicalimplications}.

Un manque de connaissances et un degré d'incertitude subsiste toutefois autour de la migration assistée \citep{Park2018Informationunderload,Klenk2015assistedmigration}, 
notamment en ce qui concerne le dilemme entre les avantages de la préservation d'une espèce et les risques potentiels pour les espèces du territoire d'accueil \citep{Hewitt2011Takingstock,McLachlan2007frameworkdebate,Vitt2010Assistedmigration}.
Afin de reduire cette incertittude, divers scénarios sylvicoles sont actuellement étudiés dans le but d'améliorer nos connaissancesde et de limiter les risques associés à la migration assistée \citep{royoDesiredREgenerationAssisted2023}.
Les traitements sylvicoles sont généralement utilisées pour influencer la croissance, la santé et la composition des peuplements forestiers.
Parmis ces traitements, les coupes totales et les coupes partielles font partie des opérations couramment utilisés en foresterie \citep{Ameray2021Forestcarbon,Chaudhary2016Impactforest,Man2008Elevenyearresponses,MontoroGirona2018ConiferRegeneration,PamerleauCouture2015Effectthree}. 
La coupes totale implique l'abattage de tous les arbres dans une zone déterminée.
Elles sont généralement employées dans le cadre d'un plan d'aménagement intensif visant à accroître la productivité et la qualité du bois 
sur une courte période de temps dans afin de répondre aux besoins de l'industrie et d'optimiser les bénéfices \citep{Ameray2021Forestcarbon}.
Les coupes partielle se définisse par la suppression sélective d'arbres tout en laissant une partie du peuplement intacte. 
Ce type de coupe est habituellement utilisé dans le cadre d'un plan d'aménagement extensif qui favorise la régénération natuelle et imite les perturbation naturelle.
Cette méthode est souvent utilisée pour favoriser la croissance des arbres les plus vigoureux, pour encourager la diversité spécifique ou pour maintenir une canopée ouverte \citep{Ameray2021Forestcarbon,Irland2011Timberproductivity}.
La conservation d'arbres  en coupe partielle et l'allongement de la période de rotation offrent d'autres avantages sur le plan économique et écosystèmique. 
Cela favorise la séquéstration du carbone, préserve l'apport de matière organique ainsi que le cycle de nutriment, 
tout en maintenant différentes niches écologiques pour la faune \citep{Ameray2021Forestcarbon,Barg1999Influencepartial,Tong2020Forestmanagement}.

Toutefois, l'application de coupes forestières peux avoir des répercutions sur les conditions environnementales des forêts, 
telles que l'augmentation de l'exposition au rayonnement solaire , la hausse de vitesse des vents et la réception accrue de précipitations au sol suite à l'élimination de la canopée, 
menant à une augmentation de l'ensoleillement, de la température et de l'humidité du sol \citep{Keenan1993ecologicaleffects,Lindo2003Microbialbiomass,Heithecker2007Edgerelatedgradients}.
De plus, la disponibilité de nutriment au niveau sol ainsi que la compaction de celui-ci peuvent être affecté par l'application de coupe forestières \citep{Battigelli2004Shorttermimpact,Covington1981Changesforest,Lindo2003Microbialbiomass,rousseauLongtermEffectsBiomass2018}. 
Ultimement, les changements environnementaux résultant d'une coupe forestière auront un impact sur la biodiversité forestière \citep{Chaudhary2016Impactforest,Fedrowitz2014Canretention,Paillet2010Biodiversitydifferences}, 
tout particulièrement la faune vivant au niveau des sols \citep{Chaudhary2016Impactforest,Lindo2003Microbialbiomass,Kudrin2023metaanalysiseffects}.
Cette faune possède une importance primordiale dans les écosystèmes forestiers, en contribuant notamment à la circulation de la matière et de l'énergie à travers la chaîne alimentaire, ainsi qu'au recyclage des nutriments \citep{Kudrin2023metaanalysiseffects,Seibold2021contributioninsects}.
Au sein de ces espèces, les amphibiens et les arthropodes sont parmi les groupes les plus affectés par les perturbations environnementales, 
telles que les traitements sylvicoles \citep{Hartshorn2021reviewforest,Semlitsch2009Effectstimber,Stuart2004Statustrends} et les changements climatiques \citep{Alford1999Globalamphibian,Houlahan2000Quantitativeevidence,Milanovich2010Projectedloss,Parmesan2006EcologicalEvolutionary,Pounds2006Widespreadamphibian,Warren2018projectedeffect}. 
Ces taxons sont ainsi couramment étudiés comme indicateurs de l'état des forêts et du degré de perturbation de celles-ci \citep{birdChangesSoilLitter2004,Maleque2009Arthropodsbioindicators,pongeVerticalDistributionCollembola2000}.


\section*{Espèces à l'étude}
\label{sec:species}
\phantomsection\addcontentsline{toc}{section}{\nameref{sec:species}}

Afin de mieux comprendre l'effet des traitement sylvicoles sur la dynamique de la faune du sol, nous avons choisi d'étudier trois groupes d'espèces : la salamandre cendrée de l'Est (\textit{Plethodon cinereus} (Green, 1818)), 
les carabes (Carabidae) et les collemboles (Collembola).

\subsection*{Salamandre cendrée}

La salamandre cendrée constitue une des plus importantes biomasses chez les vertébrés des forêts nord-américaines \citep{Burton1975Salamanderpopulations,Petranka1993Effectstimber,semlitschAbundanceBiomassProduction2014a}.
Comme le reste des Plethodontidae, cette salamandre est exclusivement terrestre. Dépourvue de poumons, sa respiration se fait de façon cutanée. 
Elle dépend ainsi de l'humidité présente dans son environnement pour effectuer des échanges gaseux avec sn environnment \citep{Heatwole1961RelationSubstratea}. 
Cette salamandre vit au niveau du sol forestier lorsque la température et le niveau d'humidité sont optimal pour effectuer sa respiration cutanée. 
En dehors de ces périodes, elle se déplacera verticalement dans le sol pour maintenir des conditions propices à sa survie \citep{Grizzell1949HibernationSite}.  
Le déplacement vertical des salamandres cendrée suit un cycle saisonnier ou l'abondance des individue à la surface sera plus grande durant le printemps et l'automne \citep{FraserEmpiricalEvaluation1976,Jaeger1980MicrohabitatsTerrestrial}.
Cette espèce possède un petit domaine vital et adopte un comportement philopatrique \citep{Yurewicz2004ResourceAvailability}.
Son rôle dans les écosystèmes forestier est prédominant en tant que prédateur généraliste participant à la régulation des invertébrés détritivores \citep{Burton1975Energyflow,Hickerson2017Easternredbacked,Walton2013Topdownregulation}, 
ce qui influe directement sur les processus de décomposition des matières organiques et la mobilisation des nutriments en forêt \citep{Burton1975Energyflow,Wyman1998Experimentalassessment}. 
La salamandre cendrée jouent également le rôle de proie au sein des réseaux trophiques, constituant une source de nourriture très énergétique et riche en protéines pour de nombreuses espèces \citep{Burton1975Energyflow,Pough1987abundancesalamanders}.
La sensibilité de cette salamandre aux perturbations environnementales en raison de sa respiration cutanée en fait un bon indicateur de la qualité environnementale des sols \citep{Welsh2001caseusing}.
Elle est ainsi couramment utilisé en tant que bioindicateur pour évaluer les perturbations environnementales associées aux traitements sylvicoles.
Plusieurs études se sont ainsi interréssé à l'utilisation de l'habitat \citep{Baecher2018Environmentalgradients,gibbsDistributionWoodlandAmphibians1998,Heatwole1962EnvironmentalFactors,Mossman2019Twosalamander}, 
à l'abondance \citep{Harpole1999Effectsseven,Hocking2013Effectsexperimental,Homyack2009Longtermeffects,Grialou2000effectsforest,Mazerolle2021Woodlandsalamander} 
ou encore à la richesse spécifique \citep{Petranka1993Effectstimber,Welsh2001caseusing}.


\subsection*{Carabe}

La famille des carabidé rassemblant la plus fortes diversités spécifiques parmi les coléoptères avec 40,000 espèces identifiés citep{Erwin1985taxonpulse}.
Elle représente une des plus importantes abondances au sein des arthropodes vivant au sol \citep{loveiEcologyBehaviorGround1996,Rochefort2006GroundBeetle}.
Ces insectes terrestre sont majoritairement nocturne et vivent la principalement au niveau du sol \citep{loveiEcologyBehaviorGround1996,Rochefort2006GroundBeetle}.
Ce groupe taxonomique est largement répandu dans les écosysteme terrestres et se retrouve dans une grand variété de milieu naturels \citep{Larochelle2003naturalhistory}. 
Toutefois la composition des communautés de carabes varie en fonction du type d'habitat.
Les espèces sont ainsi courrament divisé en trois groupes, celle favorisant le milieux forestiers mature et fermé, celle associé au milieu ouvert et les espèces généralistes \citep{Niemela2007effectsforestry}.
Cette variation dans la séléction d'habitat font des carabes un des taxon les plus utilisé dans l'étude de perturbations environnementales en milieu forestier.
Plusieurs travaux se sont notamment intérréssé aux effet des traitements sylvicoles sur ce groupe d'espèce tel que : les coupe totale \citep{Heliola2001Distributioncarabid,koivulaBorealCarabidbeetleColeoptera2002a,Niemela1993Effectsclearcut}, 
les coupe partielle et l'élagage \citep{Lemieux2004Groundbeetle,Peck2004Longertermeffects,mooreEffectsTwoSilvicultural2004}
ainsi la une préparation de site \citep{Duchesne*1999EffectsClearCutting}.
D'autres études se sont intérréssées à la fragmentation, à la pollution environnementale, à la classification d'habitats et à la catégorisation des valeurs nutritives des sols forestiers \citep{bouchardBeetleCommunityResponse2016b,Halme1993Carabidbeetles,Luff1992Classificationprediction,Niemela2001Carabidbeetles,Rainio2003Groundbeetles,Work2008Evaluationcarabid}.
La majorité des carabes sont carnivores et polyphages en plus d’être des prédateurs voraces \citep{loveiEcologyBehaviorGround1996}. 
Ils agissent ainsi comme des régulateur des populations d’invertébrés dans la chaine trophique et consomme principalement des aphides, des collemboles et des escargots \citep{loveiEcologyBehaviorGround1996}. 
Les carabes représente aussi des proies pour plusieurs espèces d’amphibiens, de reptiles, d’oiseau et de mammifères \citep{loveiEcologyBehaviorGround1996}. 


\subsection*{Collembole}

Les collemboles sont un regroupement polyphyletique d'arthropodes faisant partie de la mesofaune occupant les sols forestiers.
Ces invertébrés comportent une grande richesse spécifique et représentent une abondance importante en milieu forestier \citep{rusekBiodiversityCollembolaTheir1998}, 
malgré leurs capacité de dispertion limité due à leur petites tailles et leur absence d’ailes et leur milieu de vie \citep{Ojala2001Dispersalmicroarthropods}.
Différente communauté de collemboles occupent un ensemble de niches écologiques allant de la litière aux différents horizon du sol \citep{pongeVerticalDistributionCollembola2000}.
La répartition verticale de ces communautés dépend principalement des conditions abiotiques tel que la luminosité,le taux d’humidité ou encore la porosité.
Les collemboles peuvent ainsi servir à charactériser un substrat en fonction de la communauté qu’on y retrouve \citep{rusekBiodiversityCollembolaTheir1998}.
Le mode de vie édaphique des collemboles contribue de plus à la formation de microstructures dans les sols.
Ce taxon joue un rôle prédominant dans de nombreux processus écologiques. 
Principalement fongivores et détritivores, ces organismes se nourrissent en grande partie de champignons, de bactéries, d'actinomycètes et d'algues. 
Chaque espèce est cependant spécialisée dans un type spécifique de ressources alimentaires \citep{Chen1995Foodpreference,rusekBiodiversityCollembolaTheir1998}.
Les collemboles contribuent ainsi de façon importante à la décomposition de la matière organique, la transformation de nutriments et 
au transfert d’énergie dans les écosystèmes terrestres \citep{Cuchta2019importantrole,Hattenschwiler2005Biodiversitylitter,Marsden2020Howagroforestry,Petersen2000Collembolapopulations,rusekBiodiversityCollembolaTheir1998,Wolters1991SoilInvertebrates}
Avec les acariens, les collemboles accélérent de 10 à 20\% la décomposition de la matière organique des sol \citep{Hattenschwiler2005Biodiversitylitter}.
Les collemboles participent également à la chaîne trophique en tant que proie pour plusieurs espèces d’arachnides, de coléoptères, d’amphibiens, 
de reptiles et d’oiseaux. De plus, ils servent d’hôtes parasitaires à certains nématodes, trématodes, protozoaires, bactéries et champignons \citep{rusekBiodiversityCollembolaTheir1998}.
L'étude des collemboles est pertinente pour ce projet en raison du lien trophique proie-prédateur existant entre ce taxon et les salamandres et les carabes. 
De plus, ce groupe d'espèces est couramment utilisé dans le cadre d'études portant sur les impacts que peuvent avoir les traitements sylvicoles sur la mésofaune \citep{farskaManagementIntensityAffects2014,rousseauWoodyBiomassRemoval2019,Salmon2008Relationshipssoil}.

En résumé, le choix de ces trois groupes d'espèces est pertinent pour étudier l'impact des traitements sylvicoles. 
Leur sensibilité aux perturbations environnementales et leurs relations trophiques permettent d'analyser l'effet des changements environnementaux sur la dynamique de la faune du sol.
Plusieurs recherches se sont déja interréssé aux effets des traitements sylvicoles sur la faune du sol dans le passé. 
Cependant, la plupart discutent des effets directs des perturbations sur un ou plusieurs groupes d'espèces, 
sans tenir compte des relations existantes entre les variables environnementales et les différents groupes d'espèces, 
négligeant ainsi les effets indirectes des coupes sur la faune du sol \citep{josephIntegratingOccupancyModels2016,Kudrin2023metaanalysiseffects,Pollierer2021Diversityfunctional}. 
Il est cependant indispensable de comprendre quels sont les effets des traitements sylvicoles sur la faune du sol, 
comment ces effets se propages à l'intérieur du réseau écologique forestier et connaitre l'ensemble des répercussions sur la structure 
et la dynamique des communautés d'organismes du sol. 
Ultimement, cette acquisition de connaissances fournira des outils précieux pour faciliter la gestion durable des forêts.


\section*{Objectifs et hypothèses}
\label{sec:objectifs}
\phantomsection\addcontentsline{toc}{section}{\nameref{sec:objectifs}}

Le but de notre étude est de comprendre l'impact des traitements sylvicoles, effectuée en vue d'une migration assistée d'arbres en forêt mixte, sur la dynamique des écosystèmes du sol forestier.
L'objectif est de quantifier les impacts des traitements de coupes forestières sur la faune du sol et son habitat. 
Le moyen d'atteindre cet objectif est de construire un modèle d'équations structurelles pour mesurer les effets directs et indirects des traitements de coupes coupe totale et de coupe partielle sur : 

\begin{enumerate}
    \item Le relations de cooccurences entre la salamandres cendrée de l'Est (\textit{Plethodon cinereus}), les carabes (Carabidae) et les collemboles (Collembola).
    \item Les attributs forestiers propices à la sélection d'habitats par la faune du sol tel que le volume de débris ligneux, la profondeur de litière ainsi que l'ouverture de la canopée. 
\end{enumerate}

% Normalement avant de formuler des hypothèses tu fais une genre de revue de littérature (courte) qui fait un peu le tour de la question 
% et qui identifie les besoins de connaissances. J’imagine que ca va venir. (Mathieu)

L'hypothèse 1.1 liée à notre premier sous-objectif atteste que les traitements de coupes forestières entraînent un changement dans l'utilisation de l'habitat par la faune du sol en se propageant via le réseau trophique. 
Spécifiquement, les coupes affectent l'utilisation d'habitat des salamandres et des carabes de grandes tailles et modifie par la suite la sélection d'habitat de carabes de petite taille puis enfin des collemboles. 

L'hypothèse 1.2 associé à notre deuxième sous-objectif soutient que les traitements de coupes forestières influencent directement les attributs environnementaux des forêts important pour la faune du sol, 
tel qu'un apport en débris de coupe issus des opérations forestières ou un changement de luminosité associé à l'ouverture du couvert.
% hypothèse sur les changements


\cleardoublepage

\bibliography{References.bib}
\bibliographystyle{ecologyNewFR.bst}
