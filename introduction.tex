\chapter*{Introduction générale}         % ne pas numéroter
\label{chap-introduction}       % étiquette pour renvois
\phantomsection\addcontentsline{toc}{chapter}{\nameref{chap-introduction}} % inclure dans TdM

% \usepackage[french]{babel}

\section*{Mise en contexte}
\label{sec:contexte}
\phantomsection\addcontentsline{toc}{section}{\nameref{sec:contexte}}

% 1. **Contexte général : Importance des forêts et de la biodiversité**
%    - Présentation des écosystèmes forestiers comme des réservoirs de biodiversité essentiels à la stabilité environnementale.
%    - Rôle des forêts dans la fourniture de services écosystémiques : séquestration du carbone, régulation du climat, conservation de l'eau, et soutien à la biodiversité.

Les écosystème forestiers jouent un rôle essentiel dans la biosphère à travers leur rôle économique et leur valeur écosystèmiques en fournissant une multitude de produit et services \citep{Balvanera2006Quantifyingevidence}. 
Leur présence permet de réguler les flux de nutriments et d'énergie, notamment à travers la séquestration du carbone, la régulation du climat, la rétention de l'eau et le maintient d'une biodiversité \citep{Balvanera2006Quantifyingevidence,Diaz2006BiodiversityLoss,Canadell2008Managingforests,Pawson2013Plantationforests}. 

% 2. **Pressions exercées par les activités humaines sur les forêts**
%    - Mention des pratiques d’exploitation forestière et leur intensification au fil du temps.
%    - Présentation des types de coupes forestières : coupes à blanc, coupes partielles, sélectives, etc.
%    - Implication croissante des coupes forestières dans la gestion durable des forêts et les objectifs économiques.

Cependant, l'augmentation de la population mondiale et de l'utilisation des terres pour répondre au besoin en produits forestiers et autres services a créer une intensification des pratiques d’exploitation forestière au fil du temps \citep{Foley2005GlobalConsequences}.
Aujourd'hui, le besoin de récolter les ressources naturelles pour répondre au besoin indutriel et la volonté de préserver une certaine qualité environnementale excèdent ce que les terres disponibles peuvent offrir avec les méthodes d'exploitation traditionnelles \citep{Sala2000Globalbiodiversity,Newbold2015Globaleffects}. 

L'impacte de l'exploitation forestière vient modifier le milieu forestier à différents niveaux.

À l'échelle du paysage, l'aménagement des forêts peut entrainer une perte de connectivité entre les habitats, les communautés et les processus écologiques \citep{Lindenmayer2006Generalmanagement}.
Cette perte peut avoir un impacts majeur sur la conservation de la biodiversité en altérant la capacité d'une population à récupérer et à persister après une pertubation \citep{Lamberson1994ReserveDesign}. 
La connectivité favorise le déplacement des individus à travers leur habitat et permet ainsi le brassage génétique au sein d'une population \citep{Saccheri1998Inbreedingextinction}.

À l'échelle du peuplement, l'exploitation forestière peut entrainer une diminution de la complexité structurelle, notamment au niveau de la composition des espèces d'arbres, de la stratification verticale, de la structure d'age, de dynamique de succession et de la fréquence de perturbation \citep{Commarmot2005Structurevirgin}. 
L'hétérogénéité structurelle d'une forêt joue un rôle important pour la biodiversité puisqu'elle fournit une plus grande diversité d'habitat et participe à la connectivité en favoridant la dispersition de certaine espèces. 
Lors d'une perturbation naturelle ou artificiel, un peuplement avec une structure plus hétérogène se régénérera plus rapidement, car il favorise la survie des espèces présentes et facilite le retour des espèces qui ont disparu. 

De plus, le changement structurel d'une forêt modifiera plusieurs conditions environnementales tels la lumière, la température, l'humidité, la litière et les conditions du sol \citep{Sebastia2005Plantdiversity,Michal2014Responsessmall,James2016effectharvest,Zhang2022Intensiveforest}. 
On observe également une réduction de la présence de microhabitats tels que le bois mort, les arbres matures, les cavités et les plaques racinaires \citep{Berg1994ThreatenedPlant,Spies1999Dynamicforest,Bouget2005Shorttermeffect,Christensen2005Deadwood,Brassard2008EffectsForest}, souvent remplacés dans les forêts aménagées par des infrastructures comme les pistes de débardage et les chemins d'exploitation \citep{Hansen1991ConservingBiodiversity,}. 

Parmis les traitements sylvicoles pouvant avoir un effets majeur sur les milieux naturels on retrouve les traitements de coupes forestières.
Cependant, l'objectif économique et l'impacts écologique des coupes forestières varie selon le type traitement, allant d'un niveau de pertubation naturel ou semi-naturelle dans le cas de la gestion forestière extensive à artificiel dans le cas gestion forestière intensive \citep{Ameray2021Forestcarbon}. 

Historiquement, les coupes totales font partie des pratiques sylvicoles les plus courantes dans les forêts tempérés et boréaux \citep{Fedrowitz2014Canretention}. 
Elles font partie d'une gestion forestière intensive largement utilisée pour accroître la productivité et la qualité du bois à court terme, afin de répondre aux besoins croissants de l'industrie et d'augmenter la rentabilité \citep{Irland2011TimberProductivitya}.
Ces coupes implique l'abattage de tous les arbres dans une zone définis.
Elles sont techniquement facile à exécuter, car l'ensemble du couvert supérieur de la parcelle est retiré en une seule récolte, entraînant une déforestation temporaire d'une zone auparavant boisée. 
Les coupes totale utilise une structure équienne avec une seule espèce, ce qui simplifie la structure forestière et réduit la diversité biologique entrainant une homogénéisation des peuplements \citep{Rosenvald2008whatwhen}. 
Cette simplification du milieu perturbe des processus écologiques et évolutifs importants et entraine un déclin de la résilience des forêts \citep{Holling2001UnderstandingComplexity}. 
De plus, la période de rotation est plus courte pour ce type de traitement ce qui amène un fréquence des perturbation plus élevé. 
Le maintient de pratiques de gestion forestière transformant radicalement les structures des écosystèmes par rapport à celles observées naturellement augmente de façon importante les probabilités d'observer un déclin progressif de la biodiversité ainsi que l'extinction locale d'espèces \citep{Hanski2000Extinctiondebt}.  
Cependant, certains auteurs souligne que les coupes totales peuvent être utiliser pour imitation des perturbations naturelles majeures \citep{Greenberg1995comparisonbird}. 

Ces dernières décennies, des pratiques sylvicoles combinant la récolte de bois et la préservation de la biodiversité ont été promues pour atténuer les impacts des coupes totales \citep{Gustafsson2012Retentionforestry}.
Une gestion écosystemique caractérisé par l'émulation des perturbations naturelles a été proposée comme une stratégie prometteuse pour une gestion durable des forêts \citep{Perry1998scientificbasis,Kuuluvainen2002Naturalvariabilitya}. 
Selon cette stratégie, les actions de gestion sont planifiées de manière à émuler les perturbations naturelles et leurs résultats, y compris les structures des peuplements et les successions. 
La gestion écosytémique a pour but de préserver la biodiversité et de maintenir la résilience des peuplements, tout en garantissant la disponibilité d'une grande variété de services écosystémiques \citep{Szaro1998emergenceecosystem,MacDicken2015Globalprogress}.

Les coupes partielles font partie d'une gestion écosystémique visant à préserver la composition et la structure forestières \citep{Bergeron1999Forestmanagementa}.
Elles sont habituellement utilisées dans le cadre d'un plan d'aménagement extensif qui favorise la régénération naturelle et reproduit les perturbations naturelles \citep{Irland2011Timberproductivity}. 
Les coupes partielles consistuent une suppression sélective d'arbres tout en laissant une partie du peuplement intacte \citep{Ameray2021Forestcarbon}. 
La rétention d'arbres dans les coupes partielles offre des structures de succession tardive, favorisant ainsi la préservation d'une plus grande biodiversité \citep{Ameray2021Forestcarbon}.
Elles sont souvent employées pour stimuler la croissance des arbres les plus vigoureux, encourager la diversité des espèces ou préserver une canopée ouverte \citep{Irland2011Timberproductivity}.
Ce type de traitement repose sur une structure multi-âge, avec ou sans mélanges d'espèces, et est caractérisée par des rotations plus longues \citep{Kuuluvainen2009Forestmanagement}. 
La conservation d'arbres en coupe partielle et l'allongement de la période de rotation offrent d'autres avantages sur le plan économique et écosystémique. 
Cela favorise la séquestration du carbone, préserve l'apport de matière organique ainsi que le cycle de nutriment, tout en maintenant un structure hétérogène et différentes niches écologiques pour la faune \citep{Barg1999Influencepartial,Tong2020Forestmanagement,Ameray2021Forestcarbon}.

Comme mentionné précédemment, l'impact de la gestion forestière sur la biodiversité dépend du type de traitement sylvicole utilisé. Cependant, la réaction de la faune à un traitement peut varier selon le type de forêt et les groupes d'espèces concernés \citep{Paillet2010Biodiversitydifferences,Kudrin2023metaanalysiseffects}.


%restant
Les coupe totales et les coupes partielle font partie des coupes les plus courrament utiliser dans l'amenagement forestiers et répondent  à différents besoins en terme de gestion \citep{Man2008Elevenyearresponses,Chaudhary2016Impactforest,MontoroGirona2018ConiferRegeneration,Ameray2021Forestcarbon}.
Dans l'ensemble, il apparaît clairement que pour imiter les perturbations partielles et à petite échelle, une gestion visant à maintenir un couvert forestier continu et une hétérogénéité structurelle sur une grande partie du paysage est nécessaire \citep{Kuuluvainen2009Forestmanagement}.  
Le maintien de l'hétérogénéité des paysages.
Les écosystèmes sont naturellement hétérogènes, et l'hétérogénéité des paysages est une caractéristique des forêts naturelles à travers le monde. 
Les régimes de perturbation peuvent créer une couverture terrestre hétérogène, avec par exemple différentes étapes de succession suivant un incendie de forêt (Whelan et al., 2002). 
En outre, les paysages sont caractérisés par des gradients environnementaux naturels (comme la topographie, le climat, ou le type et la profondeur du sol ; voir Austin et Smith, 1989). 
L'hétérogénéité des paysages correspond à une mosaïque de parcelles représentant différentes compositions et classes d'âge forestières, où des conditions structurelles variées existent (Forman, 1995). 
Différentes espèces habitent dans diverses conditions environnementales des paysages naturels, et la diversité, la taille et la disposition spatiale des parcelles d'habitat sont importantes pour de nombreuses taxons (e.g., Hanski, 1994 ; Saab, 1999 ; Debinski et al., 2001).
L'utilisation des connaissances sur les régimes de perturbation naturelle dans les forêts naturelles pour guider les pratiques de gestion forestière hors réserve. 
Les stratégies de conservation de la biodiversité sont plus susceptibles de réussir lorsque les régimes de perturbation humaine (comme l'exploitation forestière) ont des effets similaires à ceux des perturbations naturelles (Hunter, 1993 ; Korpilahti et Kuuluvainen, 2002), par exemple, les types et nombres de legs biologiques (sensu Franklin et al., 2000) et les schémas spatiaux des conditions environnementales (par exemple, types de parcelles) qui restent après la perturbation (Delong et Kessler, 2000). 
Les organismes sont probablement mieux adaptés aux régimes de perturbation dans lesquels ils ont évolué (Bergeron et al., 1999 ; Hobson et Schieck, 1999), mais peuvent être vulnérables à des formes de perturbation nouvelles (ou des combinaisons de perturbations) qui sont plus ou moins fréquentes et/ou plus ou moins intensives que celles qui se produiraient normalement (Lindenmayer et McCarthy, 2002). 
Par conséquent, les régimes de perturbation naturelle peuvent être des bases de référence appropriées et des plages de variabilité contre lesquelles les régimes de perturbation humaine peuvent être comparés (Hunter, 1993 ; Angelstam et al., 1995).
De même, la création et/ou le maintien de la complexité structurelle des peuplements est essentielle pour la conservation de la biodiversité dans toutes les forêts, y compris celles ayant une longue histoire de gestion (par exemple, Linder et Ostlund, 1998). 
En outre, l'hétérogénéité des paysages est préférable à une gestion forestière intensive qui résulte en une homogénéité des paysages (Lindenmayer et Fischer, en presse).


% 3. **Impacts potentiels des coupes forestières sur la biodiversité**
%    - Dérèglement des structures des communautés animales et végétales après une coupe.
%    - Effets sur la fragmentation des habitats, la modification de la composition et la structure du peuplement forestier.
%    - Rôle clé des coupes dans la réduction de la diversité spécifique et fonctionnelle à différents niveaux trophiques.

Les activité humaine et l'utilisation des terres entraine des changements iomportants au niveau la biodiversité terrestre \citep{Newbold2015Globaleffects}. 
L'un des défis les plus importants en conservation actuellement est de préserver la biodiversité malgré les effets croissants des activités humaines sur les habitats de la faune.
Plusieurs stratégies ont été proposées pour maintenir les populations animales tout en en répondant aux besoins de production humaines anthropiques \citep{Lindenmayer2006Generalmanagement}.

Pour les écosystèmes terrestres, les modifications de l'utilisation des terres, notamment dues aux récoltes forestières, à l'agriculture et à l'urbanisation, sont en grande partie responsables de la perte de biodiversité et de la diminution des écosystèmes forestiers \citep{Sala2000Globalbiodiversity,Naeem2012functionsbiological,Bichet2016Maintaininganimal}.

L'exploitation forestière entraine ainsi un changement dans le milieu et une perte d'habitat pour la faune, limitant l'accès à la nourriture, aux refuges et aux zones de reproduction. 
À l'échelle du paysage, l'addition de plusieurs coupes amène une fragmentation de l'habitat qui complique les déplacements et la reproduction de certaine espèces, réduisant ainsi la diversité génétique des populations. 
Les coupes peuvent aussi perturber les corridors écologiques, limitant la mobilité des animaux, augmentant ainsi le risque d'extinction locale. 

Human exploitation of natural resources can induce animal and plant mortality, force species displacement, increase biotic interactions, modify their life-history traits, change their morphology and physiology, and even their polymorphic forms \citep{Sergio2018Animalresponses}
Moreover, the exploitation of natural habitats by humans results in habitat losses and biotope fragmentation, which constrains the size of animal and plant populations \citep{Coelho2020Effectsanthropogenic}
For instance, by returning stands to an early successional stage, forest harvesting homogenizes forest landscape, which results in an overrepresentation of early-succession forests at the expense of the older states \citep{Cyr2009Forestmanagement,Boucher2017Cumulativepatterns}
Thus, forest harvesting may threaten vertebrate species associated with old-growth forest attributes but also organisms (insects, fungi, non-vascular plants) that have low-dispersal capacity and for show extinction risk increases rapidly when ecological continuity is broken \citep{Norden2001Conceptualproblems,Martin2021indicatorspecies}.
These alterations are predicted to be detrimental to biodiversity especially those associated with older mixedwood and coniferous forests. \citep{Tremblay2018Harvestinginteracts,Cadieux2020Projectedeffects}



La diversité des espèces en forêt est aujourd'hui reconnu comme essentielle au bon fonctionnement des écosystèmes, et les engagements internationaux récents ont souligné l'urgence d'endiguer la perte de biodiversité en plus de favoriser une gestion durable des forêts \citep{Scherer-Lorenzen2005ForestDiversity,Parviainen2007Maintenanceconservation}. 
Cependant préserver cette biodiversité tout en répondant aux exigences économiques constitue un défi majeur pour les pays disposant de ressources forestières. 
En effet, l'exploitation de bois permettant de répondre aux besoins industriels est reconnue pour participer au déclin de nombreuses espèces qui dépendent des habitats forestiers \citep{Bengtsson2000Biodiversitydisturbances}. 
Plusieurs études ont notamment démontrer l'effets des coupes d'arbres sur différent taxon tel que les oiseaux, les chauve-souris, les papillons, les tortues, les petits mammifères et les insectes \citep{Summerville2011Managingforest,Currylow2012ShortTermForest,Kaminski2013EffectsForest,Kellner2013Shorttermresponses,Caldwell2019ComparisonBat}. 

Ces changement dans les attribut forestiers participe une perte de la diversité spécifique et fonctionnelle à différents niveaux trophiques. 
Sur le long terme, cela peut entrainer une diminution de la résilience des forêt à l'échelle locale et engendrer une diminution des services écosystémiques et avoir un impacte négatif pour les sociétés \citep{Hooper2012globalsynthesis,Edwards2014Maintainingecosystem}. 


% réalité des changement climatiques

Le défis de préserver la biodiversité et le écosystèmes forestiers tout en répondant aux exigences économiques est d'autant plus grand dans le contexte acutelle de changements climatique actuelle. 
La hausse de température représentent aujourd'hui un des défis les plus préoccupants pour le maintien des forêts et des écosystèmes tel qu'on les connait \citep{McKenney2009Climatechange,Trumbore2015Foresthealth,Seidl2017Forestdisturbances,Messier2022Warningnatural}.  
Parmi les conséquences attendues des changements climatiques, on prévoit un allongement et une intensification des périodes de sècheresse, une augmentation du nombre de feux de forêt, une altération des régimes de précipitation et une hausse des perturbations biotiques \citep{Parmesan2007Influencesspecies,Joyce2013Climatechange,Gatti2021Amazoniacarbon,Heidari2021Effectsclimate}. 
À cela s'ajoutent des changements dans la phénologie ainsi que dans la distribution des arbres due au manque d'adaptation des végétaux aux nouvelles conditions de leur milieu \citep{Aitken2008Adaptationmigration,Chuine2010Whydoes,Zhu2012Failuremigrate,Gray2013Trackingsuitable}.

Ces changements climatiques demandent une adaptation dans la gestion des forêts afin de préserver la valeur écologique et économique des arbres. 




L'impactes des changement climatiques et de la gestion des forêts 

En plus des impacts des pratiques forestières, les changements climatiques compliquent la gestion des forêts, accentuant parfois l'effet des perturbations causées par l'exploitation sylvicole. 
La gestion des forêt doit ainsi tenir compte de l'effet cumuler de l'amenagement et des changements climatiques qui peuvent se chevauché et avoir un impacte beaucoup plus grand sur la biodiversité.


%    - Adaptation des forêts au changement climatique par des stratégies telles que la migration assistée, un outil utilisé pour aider les espèces forestières à migrer vers des zones climatiquement plus adaptées à leur survie.
%    - Modification de la composition des espèces végétales en réponse au réchauffement climatique et aux projets de migration assistée.
%    - Perturbations des interactions écologiques, y compris celles des espèces locales, avec des effets potentiels sur la structure des communautés, la compétition pour les ressources et l'équilibre des écosystèmes.
%    - Nécessité de pratiques de gestion forestière durables qui intègrent à la fois la migration assistée et la préservation de la biodiversité.



Les changements climatiques représentent aujourd'hui un des défis les plus préoccupants pour le maintien des forêts tel qu'on les connait \citep{McKenney2009Climatechange,Trumbore2015Foresthealth,Seidl2017Forestdisturbances,Messier2022Warningnatural}.  
En raison de ses latitudes nordiques, le Canada est particulièrement vulnérable à l'augmentation des températures et aux perturbations environnementales qui en découlent \citep{Alo2008Potentialfuture,Bush2019Canadachanging}. 

Les forêts de l'Est de l'Amérique du Nord sont et seront donc ainsi particulièrement touchées par cette hausse de température \citep{Park2014Canboreal,Mahony2017closerlook,Sittaro2017Treerange,Messier2022Warningnatural}.


Les changements climatiques se produisent toutefois plus rapidement que la capacité d'adaptation ou de déplacement des arbres \citep{Aitken2008Adaptationmigration,Loarie2009velocityclimate,Vitt2010Assistedmigration,Harrison2020Plantcommunity}, 
menaçant ainsi leur capacité à se maintenir dans des milieux favorables à leur survie et leur croissance \citep{Zhu2012Failuremigrate,Sittaro2017Treerange,Woodall2018Decadalchanges}.
Des changements pourraient ainsi être observés dans la composition des forêts menant à des modifications dans la conduite de l'aménagement forestier et de la conservation \citep{McKenney2009Climatechange,Chmura2011Forestresponses,Lo2011Linkingclimate}.




la réaliter des changement climatique a un impacts encore plus important sur les foret et la biodiversité en s'additionnant èa l'amenagement forestier. \citep{Tremblay2018Harvestinginteracts,Ochs2022Responseterrestrial,Bouderbala2023Longtermeffect}




\section*{Migration assistée}
\label{sec:fam}
\phantomsection\addcontentsline{toc}{section}{\nameref{sec:fam}}

Plusieurs appels à adapter l'aménagement ont été proposés afin de préserver les milieux forestiers et leurs bienfaits \citep{Nagel2017Adaptivesilviculture,Messier2021sakeresilience}.
Différentes mesures d'atténuation ont été proposées pour prévenir la perte des forêts et améliorer la résilience de celle-ci comme l'augmentation de la diversité fonctionnelle et l'amélioration de la connectivité à travers le paysage forestier \citep{Messier2019functionalcomplex}.
Parmi toutes les solutions envisagées, la migration assistée d'arbres est proposée comme une mesure d'atténuation permettant le déplacement d'individus ou de matériel génétique depuis un territoire climatique originel vers une zone de climat futur plus propice à la croissance des arbres \citep{Vitt2010Assistedmigration,Dumroese2015Considerationsrestoring,Park2018Informationunderload,Park2023Provenancetrials}. 
La migration assistée d'arbre permettrait de modifier rapidement la composition des peuplements pour mieux convenir au climat futur de celui-ci \citep{Pedlar2011implementationassisted} 
répondant ainsi aux besoins de conservation, maintenant les services écosystémiques et préservant la valeur économique. \citep{Ste-Marie2011Assistedmigration,Winder2011Ecologicalimplications}.
Un manque de connaissances et un degré d'incertitude subsistent toutefois autour de la migration assistée \citep{Klenk2015assistedmigration,Park2018Informationunderload}. 
La principale préoccupation porte sur le compromis entre la préservation d'une espèce et les risques pour l'écosystème du territoire hôte \citep{Ricciardi2009Assistedcolonization}, 
tels que l'introduction d'espèces invasives ou la perte de diversité génétique pour des adaptations locales \citep{McLachlan2007frameworkdebate,Vitt2010Assistedmigration,Hewitt2011Takingstock,VanDaele2022Genomicanalyses}.

Afin d'améliorer nos connaissances et ainsi réduire l'incertitude, divers scénarios sylvicoles sont à l'heure actuelle étudiés pour limiter les risques associés à la migration assistée \citep{royoDesiredREgenerationAssisted2023}.




% 4. **Focus sur la faune du sol : une composante souvent négligée de la biodiversité**
%    - Présentation des espèces de la faune du sol (invertébrés, micro-organismes, etc.) et de leur rôle fondamental dans le fonctionnement des écosystèmes forestiers : décomposition de la matière organique, recyclage des nutriments, régulation des populations.
%    - Importance de la faune du sol comme bio-indicateurs sensibles aux perturbations environnementales.

Cette déforestation entraine des changements brusque et drastique dans les propriétés de l'habitat forestiers, notamment une ouverture du couvert et une diminution important de volume de débris ligneux, ce qui est une grosse menace pour la biodiversité étant donnée puis que le volume le bois mort habrite une importante proportion d'espèces vivant au niveau des sols frestier (55) Kuuluvainen2009Forestmanagement

L'utilisation de coupes forestières peut toutefois engendrer des modifications dans les conditions environnementales des forêts, 
telles que l'augmentation de l'exposition au rayonnement solaire, la hausse de vitesse des vents et la réception intensifiée de précipitations au sol en réponse à l'élimination de la canopée, 
menant à un accroissement de l'ensoleillement, de la température et de l'humidité du sol \citep{Keenan1993ecologicaleffects,Lindo2003Microbialbiomass,Heithecker2007Edgerelatedgradients}.
De plus, les coupes peuvent affecter la disponibilité en nutriment et augmenter le degré de compaction au niveau des sols \citep{Covington1981Changesforest,Lindo2003Microbialbiomass,Battigelli2004Shorttermimpact,rousseauLongtermEffectsBiomass2018}. 
À terme, les changements environnementaux résultant d'une coupe peuvent affecter la biodiversité. \citep{Paillet2010Biodiversitydifferences,Fedrowitz2014Canretention,Chaudhary2016Impactforest}, 
tout particulièrement sur la faune vivant à la hauteur du sol \citep{Lindo2003Microbialbiomass,Chaudhary2016Impactforest,Kudrin2023metaanalysiseffects}.

peux modifier l'abondance et la diversité specifique
Taxonomic and ecological groups displayed contrasting responses to forest management (paillet)

En conséquence, cela peut entraîner des modifications ou la disparition complète de microhabitats essentiels (comme le bois mort, les cavités ou les arbres matures) qui sont des refuges pour la biodiversité forestière \citep{Paillet2010Biodiversitydifferences}. 
De plus, certains types de gestion forestière peuvent avoir des répercussions plus importantes que d'autres sur les espèces forestières, en raison de différences dans la structure de l'habitat, la continuité ou les conditions microclimatiques après la coupe. 
Certains régimes de gestion peuvent également amplifier des impacts indirects sur la biodiversité, tels que l'augmentation de la chasse ou la fréquence des incendies.

Ainsi, différents groupes taxonomiques peuvent réagir de manière différente aux opérations forestières en fonction de leur taille, mobilité ou régime alimentaire \citep{Barlow2007Quantifyingbiodiversity,Stork2009Vulnerabilityresilience}.


Soil fauna comprises a diverse array of taxa that exhibit significant variations in biology and ecology. As a result, their response to forest harvesting can be highly heterogeneous [13] kudrin

However, existing research on this topic has primarily focused on one or two large taxa and rarely includes multiple groups [14,15] kudrin

Consequently, our comprehension of the differences in the responses of various soil fauna groups to logging and the reasons for these differences is limited. kudrin

Clearcutting, which is historically the most common example of even-aged silviculture practice in temperate and boreal biomes [11], may result in significant changes in environmental conditions. This includes altered light, humidity, wind speed, and other conditions which can constrain forest biota, e.g., [16].

Partial cutting or retention forestry is another practice in which some parts of the trees are left on-site to maintain organic matter inputs and nutrient cycles [17] and provide a refuge for belowground organisms [18]. The response of soil fauna to harvesting may differ depending on the practice used [9–11]. However, we still have a poor understanding of the differences in the response of individual groups of soil fauna to harvesting practices. The forest type is another important factor that can modify the impact of forest harvesting on soil fauna. Coniferous and deciduous forests, for example, are quite different from each other in terms of soil and microclimatic conditions. This is reflected in the dissimilarity of the composition and structure of soil fauna [14,19,20].

In general, forest harvesting significantly decreases the abundance of soil fauna by 17\% in comparison to the control (Figure 2) but does not change its richness. kudrin

du a leur taille la petites faune a une capacité de dispertion plus limité, la méttant plus èa risque face au perturbation du milieu
l'effet cumuler des perturbation du milieu et des changement de condition environnementale peuvent d'autant plus exacerber la biodiversité
 effects of forest harvesting on the occupancy of various groups of soil fauna

La faune constitue une composante essentielle de la biota du sol et joue un rôle clé dans les écosystèmes forestiers en assurant les flux de matière et d'énergie à travers le réseau trophique et en recyclant les nutriments. 
Il est donc impératif de comprendre comment l'exploitation forestière influence la faune du sol.

l'importance de suivre la faune du sol 
    Cette gestion forestière aura un impacte significatif sur la faune du sol.
    La faune du sol regroupe un ensemble d'espèces souvent peu étudier dans les études d'impact mais joue pourtant un rôle primordiale dans les mécanismes des écosystèmes forestiers
    Elle joue un rôle dans mise en circulation des nutriments, dans l'alimentations des espèces plus grandes et représente souvent une importante biomasse.

The meta-regression analysis shows that the types of harvesting and forest can modify the soil fauna responses to forest harvesting. Among the groups included in the analysis, Collembola, Oribatida, Coleoptera, and Araneae changed their response to harvesting (Table 1). The abundance of Collembola and Coleoptera was negatively affected by clear-cutting but not by partial cutting (Figure 4). The richness of these groups positively responded to clear-cutting, but in the case of Collembola, such an effect was not statistically significant. Partial cutting leads to an increase in Collembola richness but not in Coleoptera (Figure 4). Collembola and Oribatida decreased their abundance after harvesting in coniferous forests, while in deciduous and mixed forests, they did not change (Figure 5) kudrin

Forest harvesting is traditionally believed to have significant negative impacts on soil biota, through fragmentation of vegetation, modification of the quality and quantity of litter, alteration of root exudates, leaching of some plant nutrients, and changes in the microclimate and chemical properties of soil [12]. kudrin

Confirming these negative impacts, our meta-analysis showed that, in general, harvesting leads to a reduction in the abundance of soil fauna. kudrin

We did not observe a significant reduction in soil fauna richness. However, the lack of changes in richness does not necessarily mean no changes in taxonomic composition. For example, closed-canopy species of Carabidae beetles may decrease or even disappear from clear-cut plots, while open-area-associated species rapidly colonize them [30,31]

Partial cutting is believed to have a less pronounced negative effect on biota compared to clear-cutting [10,11] due to fewer changes in soil hydrothermal conditions [56] (Londo et al., 1999), the maintenance of microbial biomass [57], and the conservation of refugia that preserve the structure, composition, and functional characteristics of undisturbed forests [58]. However, the modifying effect of harvesting type on soil fauna strongly depends on taxonomic groups. kudrin

Our findings also suggest that partial cutting may have less dramatic effects on soil fauna than clear-cutting, which confirms the potential of this forestry practice to reduce disturbances in forest ecosystems. kudrin

% 5. **Effets spécifiques des coupes forestières sur la faune du sol**
%    - Impacts des modifications microclimatiques (température, humidité, lumière) résultant de la perte de la canopée sur les organismes du sol.
%    - Réduction des ressources comme le bois mort et la litière après des coupes intensives.
%    - Perturbation des chaînes trophiques qui influence la dynamique des populations et des interactions dans les sols forestiers.

L'enlèvement de la canopée, principalement dans les coupe totales, peut augmenter la quantité de rayonnement solaire et de précipitations atteignant la surface du sol, ce qui élève la température et l'humidité du sol.

(deMaynadier and Hunter, 1995). Woodland salamanders (genus Plethodon) are one species group that was historically overlooked when developing forest management guidelines 


(Homyack et al., 2010; Barrett et al., 2014; Hocking and Babbitt, 2014) Forest managers and researchers now have a greater understanding of the ecological roles these salamanders play in forest systems 

Woodland salamanders are numerically dominant in forested systems of the eastern United States (Burton and Likens, 1975a), which allows them to exert a strong influence on the food web in forests (Hairston, 1987).

(Wyman, 1998; Walton, 2013; Hickerson et al., 2017). Salamanders exhibit strong top-down pressures on detritivorous forest floor invertebrates which affects forest litter decomposition rates and carbon dynamics 

Further, woodland salamanders represent a high-quality food source for many predators such as birds, mammals, reptiles, and other amphibians (Burton and Likens, 1975b; Petranka, 1998)

Because of these factors, salamanders represent important linkages in the forest food web dynamic and have been considered important for ecological function and integrity (Welsh Jr. and Droege, 2001; Davic and Welsh, 2004).

Previous research has focused on certain forest characteristics (e.g., overstory cover, coarse woody debris, leaf litter depth) and their influence on salamanders post-harvest in order to guide forest management (Semlitsch, 2002; McKenny et al., 2006).
Even-aged forest management (e.g., clearcutting and shelterwood) is often thought to be the most detrimental to woodland salamanders in the short-term (Duguay and Wood, 2002; Hocking et al., 2013a) 
because these techniques greatly reduce overstory cover which increases the amount of solar radiation reaching the forest floor and subsequently dries out leaf litter and increases soil temperature (Zheng et al., 2000; Brooks and Kyker-Snowman, 2008).
Due to physiological constraints (e.g., thin, vascular skin; Hillman et al. 2009), woodland salamanders are at increased risk of desiccation when exposed to drier environments at the forest floor.
To cope with the environmental stress, individuals will often remain belowground where soils are cooler and moisture is not as limiting. 
However, heavy machinery used during operational silviculture compacts soils creating physical barriers for fossorial salamanders (Steinbrenner, 1995).
Further, the removal of surface refugia such as tree limbs, treetops, and other coarse woody debris (CWD) following harvesting can reduce habitat quality for salamanders and reduce the amount of time they spend at the forest floor surface (Achat et al., 2015; Peele et al., 2017)
Consequently, because woodland salamanders forage and breed at the forest floor surface, harsh surface conditions, soil compaction, and low levels of CWD can affect population dynamics (Peterman and Semlitsch, 2014).
Previous work in hardwood forests of eastern North America and the Appalachian region has shown that increased retention of overstory cover through partial harvesting approaches, such as uneven-aged management (e.g., group selection, single-tree selection), 
can result in higher relative densities of salamanders compared to even-aged treatments that remove greater amounts of overstory (Hocking et al., 2013a; Harper et al., 2015; Mahoney et al., 2016).
Greater retention of the overstory limits the amount of solar radiation reaching the forest floor which retains microclimates generally compatible with the physiology of salamanders (e.g., wetter, cooler; Homyack et al., 2011).
Greater amounts of CWD can also mitigate the negative effects of even-aged forest management by providing surface refugia that maintains suitable environmental conditions underneath logs and rocks and allows individuals to remain at the surface for longer periods of time (Grover, 1998; Moseley et al., 2009; Strojny and Hunter, 2009; Carusco, 2016).
A common outcome of all forms of forest management is the reduction of the overstory layer due to the removal of standing timber, and percent overstory cover and woodland salamander density are often strongly positively correlated (Tilghman et al., 2012).


\section*{Espèces à l'étude}
\label{sec:species}
\phantomsection\addcontentsline{toc}{section}{\nameref{sec:species}}

La faune du sol possède une importance primordiale dans les écosystèmes forestiers, en contribuant entre autres à la circulation de la matière et de l'énergie à travers la chaîne alimentaire, ainsi qu'au recyclage des nutriments \citep{Seibold2021contributioninsects,Kudrin2023metaanalysiseffects}.
Parmi les espèces vivant au sol, les amphibiens et les arthropodes font partie des taxons majoritairement impactés durant des perturbations environnementales, 
surtout lors de traitements sylvicoles \citep{Stuart2004Statustrends,Semlitsch2009Effectstimber,Hartshorn2021reviewforest} et dans le cadre des changements climatiques \citep{Alford1999Globalamphibian,Houlahan2000Quantitativeevidence,Pounds2006Widespreadamphibian,Warren2018projectedeffect}. 
Ces groupes sont souvent utilisés comme indicateurs de l'état des forêts et du degré de perturbation de celles-ci \citep{pongeVerticalDistributionCollembola2000,birdChangesSoilLitter2004,Maleque2009Arthropodsbioindicators}.
Afin de mieux comprendre l'impact des traitements sylvicoles sur la faune du sol, nous avons choisi d'étudier trois groupes d'espèces : la salamandre cendrée de l'Est (\textit{Plethodon cinereus} (Green, 1818)), 
les carabes (Carabidae) et les collemboles (Collembola).

\subsection*{Salamandre cendrée}

La salamandre cendrée constitue une des plus importantes biomasses chez les vertébrés des forêts nord-américaines \citep{Burton1975Salamanderpopulations,Petranka1993Effectstimber,semlitschAbundanceBiomassProduction2014a}.
Parmis les Plethodontidae, cette salamandre exclusivement terrestre respire de façon cutanée puisqu'elle est dépourvue de poumons. 
Elle dépend ainsi de l'humidité présente dans son environnement pour effectuer des échanges gazeux \citep{Heatwole1961Relationsubstrate}. 
Cette salamandre occupe les sols forestiers lorsque la température et le niveau d'humidité sont optimaux pour effectuer sa respiration cutanée. 
En dehors de ces périodes, elle se déplacera verticalement dans le sol pour maintenir des conditions propices à sa survie \citep{Grizzell1949HibernationSite}.  
Le déplacement vertical des salamandres cendrées suit un cycle saisonnier et l’abondance des individus à la surface est la plupart du temps plus élevée au printemps et à l’automne \citep{FraserEmpiricalEvaluation1976,Jaeger1980MicrohabitatsTerrestrial}.
Cette espèce possède un petit domaine vital et adopte d’ordinaire un comportement philopatrique \citep{Yurewicz2004ResourceAvailability}.
Son rôle dans les écosystèmes forestiers est prédominant en tant que prédateur généraliste. 
La salamandre cendrée participe de façon importante à la régulation des invertébrés détritivores \citep{Burton1975Energyflow,Walton2013Topdownregulation,Hickerson2017Easternredbacked}, 
ce qui influe directement sur les processus de décomposition de la matière organique et la mobilisation des nutriments en forêt \citep{Burton1975Energyflow,Wyman1998Experimentalassessment}. 
La salamandre cendrée joue de la même façon le rôle de proie au sein des réseaux trophiques, constituant une source de nourriture très énergétique et riche en protéines pour de nombreuses espèces \citep{Burton1975Energyflow,Pough1987abundancesalamanders}.
La sensibilité de cette salamandre aux perturbations environnementales en raison de sa respiration cutanée en fait un bon indicateur de la qualité des sols forestier \citep{Welsh2001caseusing}.
Elle est ainsi couramment utilisée pour évaluer les effets associées aux traitements sylvicoles, comme l’utilisation de l’habitat \citep{Heatwole1962EnvironmentalFactors,gibbsDistributionWoodlandAmphibians1998,Baecher2018Environmentalgradients,Mossman2019Twosalamander}, 
l'abondance de la salamandre cendrée \citep{Harpole1999Effectsseven,Grialou2000effectsforest,Homyack2009Longtermeffects,Hocking2013Effectsexperimental,Mazerolle2021Woodlandsalamander} 
ou encore à la richesse spécifique des urodèles \citep{Petranka1993Effectstimber,Welsh2001caseusing}.

\subsection*{Carabes}

Les carabes sont des coléoptères particulièrement actifs, vivant au niveau du sol forestier \citep{loveiEcologyBehaviorGround1996,Rochefort2006GroundBeetle}.
Cette famille rassemble la plus forte diversité spécifique parmi les coléoptères avec 40 000 espèces identifiées \citep{Erwin1985taxonpulse} 
et représente une des plus grandes abondances au sein des arthropodes vivant au sol \citep{loveiEcologyBehaviorGround1996,Rochefort2006GroundBeetle}.
La majorité des carabes sont carnivores et polyphages en plus d’être des prédateurs voraces \citep{loveiEcologyBehaviorGround1996}. 
Ils agissent comme des régulateurs des populations d’invertébrés dans la chaîne trophique et consomment surtout des aphides, des collemboles et des escargots \citep{loveiEcologyBehaviorGround1996}. 
Les carabes représentent aussi des proies pour plusieurs espèces d’amphibiens, de reptiles, d’oiseaux et de mammifères \citep{loveiEcologyBehaviorGround1996}. 
Ce groupe est largement répandu dans les écosystèmes terrestres et se retrouve dans une grande variété de milieux naturels, tels que les forêts, les cultures, les marais ou encore les sablières \citep{Larochelle2003naturalhistory}. 
Toutefois, la sélection d'habitats varie selon l'espèce, ce qui entraîne une différence de communauté en fonction du type de milieu.
En milieu forestier, les carabes sont le plus souvent classés selon trois types de communautés : les espèces des milieux forestiers matures et fermés, les espèces des milieux ouverts et les espèces généralistes \citep{Niemela2007effectsforestry}. 
Cette variation dans la sélection d'habitats fait des carabes un des taxons les plus intéressants à étudier lors de perturbations environnementales.
Ce groupe a été utilisé dans des travaux portant sur les traitements sylvicoles, notamment les coupes totales \citep{Niemela1993Effectsclearcut,Heliola2001Distributioncarabid,koivulaBorealCarabidbeetleColeoptera2002a}, 
les coupes partielles et l'élagage \citep{Lemieux2004Groundbeetle,mooreEffectsTwoSilvicultural2004,Peck2004Longertermeffects}.
D'autres études se sont interresées à la fragmentation forestière et à la pollution environnementale pour documenter l'utilisation de l'habitats par les carabes et pour évaluer la valeurs nutritives des sols forestiers propices à ces invertébrés \citep{bouchardBeetleCommunityResponse2016b,Luff1992Classificationprediction,Rainio2003Groundbeetles,Work2008Evaluationcarabid}.

\subsection*{Collemboles}

Les collemboles sont un regroupement polyphylétique d'arthropodes faisant partie de la mésofaune établie dans les sols forestiers.
Ces invertébrés comportent une grande richesse spécifique et sont très abondants \citep{rusekBiodiversityCollembolaTheir1998}, 
malgré leur capacité limitée de dispersion due à leur petite taille, leur absence d’ailes et leur niche écologique \citep{Ojala2001Dispersalmicroarthropods}.
Différentes communautés de collemboles occupent un ensemble de niches écologiques allant de la litière aux différents horizons du sol \citep{pongeVerticalDistributionCollembola2000}.
La répartition verticale de ces communautés dépend essentiellement des conditions abiotiques du sol telles que la luminosité, le taux d’humidité ou encore la porosité.
Les collemboles peuvent ainsi servir à caractériser un substrat en fonction de la communauté qu’on y retrouve \citep{rusekBiodiversityCollembolaTheir1998}.
De plus, les collemboles contribuent à la formation de microstructures dans les sols. 
Ce taxon joue un rôle prédominant dans de nombreux processus écologiques. 
Principalement fongivores et détritivores, ces organismes se nourrissent en grande partie de champignons, de bactéries, d'actinomycètes et d'algues. 
Chaque espèce est cependant spécialisée à un type spécifique de ressource alimentaire \citep{Chen1995Foodpreference,rusekBiodiversityCollembolaTheir1998}.
Ces arthropodes contribuent de façon importante à la décomposition de la matière organique, à la transformation de nutriments et 
au transfert d’énergie dans les écosystèmes terrestres \citep{rusekBiodiversityCollembolaTheir1998,Hattenschwiler2005Biodiversitylitter,Cuchta2019importantrole,Marsden2020Howagroforestry}.
Avec les acariens, les collemboles accélèrent de 10 à 20\% la décomposition de la matière organique \citep{Hattenschwiler2005Biodiversitylitter}.
Ils participent également à la chaîne trophique en tant que proie pour plusieurs espèces d’arachnides, de coléoptères, d’amphibiens, de reptiles et d’oiseaux. 
D’autre part, ils servent d’hôtes à des parasites incluant les nématodes, les trématodes, les protozoaires, les bactéries et les champignons \citep{rusekBiodiversityCollembolaTheir1998}.
L'étude des collemboles est pertinente pour ce projet en raison de la relation proie-prédateur existante entre ce taxon, les salamandres et les carabes. 
De plus, le groupe des collemboles est couramment utilisé dans le cadre de recherches portant sur les répercussions des traitements sylvicoles sur la mésofaune \citep{Salmon2008Relationshipssoil,farskaManagementIntensityAffects2014,rousseauWoodyBiomassRemoval2019}.

En résumé, le choix de ces trois groupes d'espèces est pertinent pour étudier l'impact des traitements sylvicoles. 
Leur sensibilité aux perturbations environnementales et leurs relations trophiques en font des candidats idéaux pour analyser l'effet des coupes forestières sur la dynamique de la faune du sol.


% 6. **Justification de l’étude**
%    - Nécessité d’études plus approfondies sur l’impact des coupes forestières à différentes échelles (spatiales et temporelles) sur la faune du sol.
%    - Lacunes dans la littérature scientifique concernant les réponses des communautés de la faune du sol face aux pratiques sylvicoles.
%    - Importance de mieux comprendre ces interactions pour proposer des recommandations en matière de gestion forestière durable.


Malgré l'importance de la faune du sol et la multitude d'études ayant évalué la réponse de différents groupes d'invertébrés du sol à l'exploitation forestière, il existe encore peu de revues sur cette problématique. 
Marshall a publié une revue traditionnelle sur l'influence de l'exploitation forestière sur les processus biologiques il y a plus de 20 ans, et depuis, de nombreuses nouvelles recherches ont été menées. 
Bien que les méta-analyses évaluant les changements de biodiversité liés à la gestion forestière incluent les invertébrés du sol, elles ne prennent pas en compte les données sur l'abondance et la réaction des différents groupes de faune du sol.


There are urgent needs to identify the most appropriate forest harvesting scenarios to achieve management objectives, such as species recovery [25] or habitat restoration [26].

Ce contexte souligne l'importance de mieux comprendre et minimiser les impacts des pratiques forestières sur la biodiversité, notamment en ce qui concerne la faune du sol, qui joue un rôle essentiel dans le maintien des écosystèmes forestiers.


There is a need to better understand the underlying mechanisms driving responses of soil fauna cooccurence to forest management in order to provide managers with harvesting strategies that are compatible with soil fauna occupancy.
A greater understanding of the mechanistic drivers from forestry prescriptions will allow managers to directly incorporate these measures into their forest planning to help mitigate negative effects from forest management.

Such single-species approaches, however, have often been criticized for their poor efficiency in maintaining biodiversity in managed landscapes (Roberge and Angelstam 2004, Branton and Richardson 2011)

peu d'étude se sont interréssé au interaction entre plusieurs groupe tel que les amphibien et les arthropodes, il sont souvent étudié séparément.

La plupart des travaux qui se sont intéressés aux impacts des traitements sylvicoles sur la faune discutent généralement des effets directs des perturbations sur un ou plusieurs groupes d'espèces, 
sans tenir compte des relations existantes entre les variables environnementales et les différents groupes d'espèces, 
négligeant ainsi les effets indirects des coupes sur la faune du sol \citep{josephIntegratingOccupancyModels2016,Pollierer2021Diversityfunctional,Kudrin2023metaanalysiseffects}. 
Mon projet comblait justement cette lacune en essayant de comprendre comment les effets des traitements sylvicoles se propagent à l’intérieur du réseau écologique forestier et influence la dynamique de la faune du sol.  
Ultimement, ce gain de connaissances fournira des outils précieux pour faciliter la gestion durable des forêts.

% 7. **Objectifs de l’étude**
%    - Identifier et quantifier les effets directs et indirects des différents types de coupes forestières sur les communautés de la faune du sol.
%    - Explorer les changements dans l’utilisation de l’habitat et la structure des populations d’espèces clés du sol après différentes intensités de coupes.
%    - Proposer des pistes pour intégrer ces connaissances dans les pratiques de gestion forestière, tout en maintenant les services écosystémiques essentiels.


\section*{Objectifs et hypothèses}
\label{sec:objectifs}
\phantomsection\addcontentsline{toc}{section}{\nameref{sec:objectifs}}

Le but de mon étude était de comprendre comment les traitements sylvicoles, effectués dans un contexte de migration assistée, 
affecte la dynamique des écosystèmes du sol forestier. Les objectifs qui s’y rattachaient étaient :

\begin{enumerate}
    \item De quantifier l'effet des traitements de coupes forestières sur les variables environnementales qui influencent l'utilisation de l'habitat par la faune du sol.
    \item De mesurer l'impact des coupes forestières sur l'utilisation de l'habitat par la faune du sol.
\end{enumerate}

L'hypothèse 1.1 liée à notre deuxième objectif soutenait que les variables environnementales favorables à l'utilisation de l'habitat par les espèces fluctuent 
en fonction de l'intensité des coupes forestières. Ainsi, les traitements de coupes forestières constituent une variable englobant 
les changements de conditions environnementales.

L'hypothèse 2.1 attachée à notre premier objectif stipulait que les traitements de coupe forestière entraînent une modification de l'utilisation de l'habitat 
par la faune du sol et se propage à travers le réseau trophique. Spécifiquement, les coupes affectent l'utilisation de l'habitat par les salamandres et 
les grands carabes (compétiteurs de salamandres), ce qui modifie ensuite la sélection d'habitat des petits carabes (proies de salamandres), puis enfin des collemboles (proies de grands carabes et salamandres).



\cleardoublepage

\bibliography{References.bib}
\bibliographystyle{ecologyNewFR.bst}
