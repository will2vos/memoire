\chapter*{Introduction générale}         % ne pas numéroter
\label{chap-introduction}       % étiquette pour renvois
\phantomsection\addcontentsline{toc}{chapter}{\nameref{chap-introduction}} % inclure dans TdM

% \usepackage[french]{babel}

\section*{Mise en contexte}
\label{sec:contexte}
\phantomsection\addcontentsline{toc}{section}{\nameref{sec:contexte}}

Les écosystème forestiers jouent un rôle essentiel dans la biosphère à travers leur rôle économique et leur valeur écosystèmiques en fournissant une multitude de produit et services \citep{Balvanera2006Quantifyingevidence}. 
Leur présence permet de réguler les flux de nutriments et d'énergie, notamment à travers la séquestration du carbone, la régulation du climat, la rétention de l'eau et la conservation de la biodiversité \citep{Balvanera2006Quantifyingevidence,Diaz2006BiodiversityLoss,Canadell2008Managingforests,Pawson2013Plantationforests}. 
Cependant la croissance démographique mondiale, l'augmentation des besoins en produits forestiers et autres services ont conduit à une intensification des pratiques d'exploitation forestière au cours des dernières décennies \citep{Foley2005GlobalConsequences}. 
Ces changements d'affectations des terres, comme la récoltes de bois, l'agriculture et l'urbanisation, entrainent ainsi la perte d'écosystèmes forestiers et de biodiversité \citep{Bengtsson2000Biodiversitydisturbances,Sala2000Globalbiodiversity,Naeem2012functionsbiological,Bichet2016Maintaininganimal}. 

Cette biodiversité est reconnue comme essentielle pour le bon fonctionnement des écosystèmes terrestres et les engagements internationaux ont souligné l'urgence de freiner cette perte tout en promouvant une gestion durable des forêts \citep{Scherer-Lorenzen2005ForestDiversity,Parviainen2007Maintenanceconservation,Newbold2015Globaleffects}. 
De nombreuses recherches ont mis en évidence l'impact de la gestion forestière sur divers groupes fauniques, tels que les oiseaux, chauves-souris, papillons, tortues, petits mammifères et insectes \citep{Summerville2011Managingforest, Currylow2012ShortTermForest, Kaminski2013EffectsForest, Kellner2013Shorttermresponses, Caldwell2019ComparisonBat}. 
Les pratiques sylvicoles peuvent provoquer la mortalité des animaux, perturber ou restreindre leurs déplacements, intensifier les interactions biotiques, modifier leurs cycles de vie, altérer leur morphologie et leur physiologie, voire affecter leurs formes polymorphiques \citep{Sergio2018Animalresponses}. 
De plus, l'exploitation forestière entraîne des pertes d'habitat et la fragmentation des milieux naturels, réduisant ainsi l'accès à la nourriture, aux refuges et aux sites de reproduction, tout en diminuant la taille et la diversité génétique des populations \citep{Bouderbala2023Longtermeffect}.
Les coupes forestières peuvent également réduire la connectivité entre les habitats, les communautés et les processus écologiques ce qui peut avoir des conséquences grave pour la conservation de la faune \citep{Lindenmayer2006Generalmanagement}. 
Une diminution de la connectivité peut compromettre la capacité des populations à se rétablir et à se maintenir après une perturbation \citep{Lamberson1994ReserveDesign}. 
Cette diminution restreint aussi le déplacement des individus dans leur habitat et le long des corridors écologiques, réduisant ainsi le flux génétique entre populations et augmentant le risque d'extinction locale \citep{Saccheri1998Inbreedingextinction}.

En ramenant les peuplements forestiers à des stades précoces de succession, l'exploitation forestière tend à homogénéiser le paysage, favorisant ainsi une surreprésentation des forêts en début de succession 
au détriment des stades plus avancés \citep{Cyr2009Forestmanagement,Boucher2017Cumulativepatterns}. 
Cette pratique réduit la complexité structurelle des forêts, affectant des aspects tels que la composition des espèces d'arbres, la stratification verticale, la structure d'âges, 
la dynamique de succession et la fréquence des perturbations \citep{Commarmot2005Structurevirgin}. 
Ces modifications sont particulièrement préoccupantes pour la biodiversité, notamment pour les espèces associées aux forêts mixtes et conifériennes matures \citep{Tremblay2018Harvestinginteracts,Cadieux2020Projectedeffects}.
L'hétérogénéité structurelle des forêts joue un rôle significatif dans la conservation de la biodiversité, car elle offre une variété d'habitats et renforce la connectivité en favorisant la dispersion de certaines espèces. 
Par conséquent, l'exploitation forestière peut représenter une menace et un risque d'extinction pour les espèces dépendant des caractéristiques des forêts anciennes et pour celles ayant une faible capacité de dispersion \citep{Norden2001Conceptualproblems,Martin2021indicatorspecies}. 
De plus, un peuplement avec une structure plus hétérogène est plus résilient lors d'une perturbation, car il facilite la régénération des espèces arbres présentes et le retour de celles qui ont disparu \citep{Kuuluvainen2009Forestmanagement}. 

De manière générale, les modifications des attributs forestiers contribuent à une perte de la diversité spécifique et fonctionnelle. 
À long terme, cela peut réduire la résilience des forêts à l'échelle locale et entraîner une diminution des services écosystémiques \citep{Hooper2012globalsynthesis,Edwards2014Maintainingecosystem}. 

\section*{Traitements sylvicoles}
\label{sec:sylvicole}
\phantomsection\addcontentsline{toc}{section}{\nameref{sec:sylvicole}}

Parmi les interventions sylvicoles ayant un impact significatif sur les milieux naturels, les coupes forestières occupent une place importante. 
Toutefois, le degré de perturbation causé par ces coupes varie en fonction du type de traitement utilisé, allant de perturbations proches du naturel ou semi-naturel dans le cadre de la gestion forestière extensive, à des perturbations plus artificielles dans le cas de la gestion forestière intensive \citep{Ameray2021Forestcarbon}. 
Le choix des traitements appliqués dépendra principalement des objectifs économiques et écologiques fixés par le plan d'aménagement.

Historiquement, les coupes totales sont parmi les pratiques sylvicoles les plus répandues dans les forêts tempérées et boréales \citep{Fedrowitz2014Canretention,Chaudhary2016Impactforest}. 
Elles font partie d'une gestion forestière intensive largement utilisée pour accroître la productivité et la qualité du bois à court terme, afin de répondre aux besoins croissants de l'industrie et d'augmenter la rentabilité \citep{Irland2011TimberProductivitya}.
Ce type de coupe implique l'abattage de l'ensemble des arbres dans une zone définis.
Sur le plan technique, elles sont faciles à réaliser, car la totalité du couvert forestier est retiré en une seule récolte, entraînant une déforestation temporaire de la parcelle.
Les coupes totale utilise généralement une structure équienne avec une seule espèce, ce qui simplifie la structure forestière et réduit la diversité biologique entrainant une homogénéisation drastique des peuplements \citep{Rosenvald2008whatwhen}. 
Cette simplification perturbe des processus écologiques et évolutifs essentiels et entraine une diminution de la résilience des forêts \citep{Holling2001UnderstandingComplexity}. 
De plus, les rotations forestières sont plus courtes avec ce type de gestion, ce qui augmente la fréquence des perturbations. 
L’utilisation de pratiques modifiant radicalement la structure des écosystèmes par rapport aux conditions naturelles accroît le risque de déclin de la biodiversité et d’extinctions locales \citep{Hanski2000Extinctiondebt}.  
Cependant, certains auteurs estiment que les coupes totales peuvent imiter des perturbations de grande ampleur, telles que des incendies ou des tempêtes \citep{Greenberg1995comparisonbird}. 

Au cours des dernières décennies, des pratiques sylvicoles intégrant à la fois la récolte de bois et la préservation de la biodiversité ont été encouragées pour réduire les effets négatifs des coupes totales \citep{Gustafsson2012Retentionforestry}.
La gestion écosystémique, qui s'inspire des perturbations naturelles, a été proposée comme une stratégie prometteuse pour une gestion forestière durable \citep{Perry1998scientificbasis,Kuuluvainen2002Naturalvariabilitya}. 
Cette approche vise à émuler les perturbations naturelles ainsi que les structures des peuplements et les dynamiques de succession qui en résultent.  
L'objectif est de préserver la biodiversité et de maintenir la résilience des écosystèmes forestiers tout en assurant la disponibilité d'une diversité de services écosystémiques \citep{Szaro1998emergenceecosystem,MacDicken2015Globalprogress}.

Les coupes partielles s'inscrivent dans une gestion écosystémique visant à maintenir la composition et la structure des forêts \citep{Bergeron1999Forestmanagementa}.
Elles sont généralement intégrées dans un plan de gestion extensive qui privilégie la régénération et imite les perturbations naturelles \citep{Irland2011Timberproductivity}. 
Ces coupes impliquent une suppression sélective d'arbres, tout en conservant une partie du peuplement \citep{Ameray2021Forestcarbon}. 
La rétention d'arbres dans ce type de coupes favorise des structures de succession tardive, ce qui contribue à la préservation d'une biodiversité plus riche \citep{Ameray2021Forestcarbon}.
Elles sont souvent employées pour stimuler la croissance des arbres les plus vigoureux, encourager la diversité des espèces ou maintenir une canopée ouverte \citep{Irland2011Timberproductivity}.
Ces traitements reposent sur une structure multi-âge, pouvant inclure des mélanges d'espèces, et se caractérisent par des rotations plus longues \citep{Kuuluvainen2009Forestmanagement}. 
La conservation d'arbres lors de coupes partielles, ainsi que l'allongement des périodes de rotation, présentent également des avantages économiques et écologiques \citep{Ameray2021Forestcarbon}. 
Cela favorise la séquestration du carbone, maintient l'apport de matière organique et régule le cycle des nutriments, tout en préservant une structure hétérogène et diverses niches écologiques pour la faune \citep{Dahlgren1994effectswholetree,Barg1999Influencepartial,Tong2020Forestmanagement,Ameray2021Forestcarbon}. 

En plus du défi de préserver la biodiversité et les écosystèmes forestiers tout en répondant aux exigences économiques, les changements climatiques ajoutent une complexité accrue à la gestion des peuplements.
La hausse globale de température constitue une menace additionnelle pour la pérennité de la faune et de la flore en modifiant de manière significative les conditions environnementales \citep{McKenney2009Climatechange,Trumbore2015Foresthealth,Seidl2017Forestdisturbances,Messier2022Warningnatural}. 
Parmi ces modifications, on anticipe un allongement et une intensification des périodes de sécheresse, une hausse des feux de forêt, une altération des régimes de précipitations, et une augmentation des perturbations biotiques \citep{Parmesan2007Influencesspecies,Joyce2013Climatechange,Gatti2021Amazoniacarbon,Heidari2021Effectsclimate}. 

En parallèle, des changements dans la phénologie et la distribution des arbres pourraient survenir en raison de leur faible capacité d'adaptation aux nouvelles conditions climatiques \citep{Aitken2008Adaptationmigration,Chuine2010Whydoes,Zhu2012Failuremigrate,Gray2013Trackingsuitable}.
De plus, les stress climatiques agissent souvent de manière additive ou synergique avec les activités forestières, ce qui amplifie leur impact sur la biodiversité et les écosystèmes \citep{Brook2008Synergiesextinction,Tremblay2018Harvestinginteracts,Ochs2022Responseterrestrial,Bouderbala2023Longtermeffect}. 
Par conséquent, la composition des forêts pourrait être altérée, nécessitant des ajustements dans les pratiques de gestion forestière et les stratégies de conservation \citep{McKenney2009Climatechange,Chmura2011Forestresponses,Lo2011Linkingclimate}.
Des mesures d'atténuation telles que l'augmentation de la diversité fonctionnelle et l'amélioration de la connectivité au sein du paysage forestier ont été proposées pour renforcer la résilience des forêts face aux changements climatiques \citep{Messier2019functionalcomplex}.
Parmi ces mesures, la migration assistée des arbres, qui consiste à déplacer des individus ou du matériel génétique d'une région climatique originelle vers une zone mieux adaptée aux conditions futures, 
est suggérée comme une solution pour maintenir les services écosystémiques et la valeur économique des forêts \citep{Vitt2010Assistedmigration,Pedlar2011implementationassisted,Ste-Marie2011Assistedmigration,Winder2011Ecologicalimplications}. 
Cette approche permettrait de modifier rapidement la composition des peuplements pour les adapter aux climats futurs, répondant ainsi aux besoins de conservation \citep{Dumroese2015Considerationsrestoring,Park2018Informationunderload,Park2023Provenancetrials}. 
Cependant un manque de connaissances et un degré d'incertitude subsistent toutefois autour des nouvelles mesures d'adaptation \citep{Klenk2015assistedmigration,Park2018Informationunderload}. 
Il est donc essentiel de mieux comprendre l'impact des traitements sylvicoles sur la biodiversité, dans un contexte où la gestion forestière doit s'adapter aux changements climatiques et à la perte de biodiversité, tout en satisfaisant les exigences économiques. 


\section*{Faune du sol}
\label{sec:soilfauna}
\phantomsection\addcontentsline{toc}{section}{\nameref{sec:soilfauna}}

Au sein de la biodiversité forestière, la faune du sol joue un rôle essentiel dans les écosystèmes en contribuant à la circulation de la matière et de l'énergie à travers la chaîne alimentaire, ainsi qu'au recyclage des nutriments \citep{Seibold2021contributioninsects,Kudrin2023metaanalysiseffects}.
Cependant, ce groupe figure parmi les espèces les plus affecté par les pertubations environnementales au sein de la biodiveristé forestières \citep{Marshall2000Impactsforest,Coyle2017Soilfauna}. 
En raison de leur petite taille, les organismes du sol disposent d'une capacité de dispersion limitée, ce qui les rend particulièrement vulnérables aux altérations de leur habitat \citep{Kudrin2023metaanalysiseffects}.

Les pratiques sylvicoles, telles que les coupes forestières, provoquent des modifications soudaines et importantes des caractéristiques de l'habitat forestier. 
Ces changements peuvent affecter négativement la biodiversité, notamment la faune vivant à hauteur du sol \citep{Lindo2003Microbialbiomass,Paillet2010Biodiversitydifferences,Fedrowitz2014Canretention,Chaudhary2016Impactforest}. 
Le retrait de la canopée expose davantage la surface du sol au rayonnement solaire, entraînant une hausse des températures et des modifications de l'humidité du sol \citep{Lindo2003Microbialbiomass,Brook2008Synergiesextinction,Zhang2022Intensiveforest}. 
Ces effets sont amplifiés par une augmentation de la vitesse du vent et une intensification des précipitations atteignant le sol \citep{Keenan1993ecologicaleffects,Heithecker2007Edgerelatedgradients}.

La structure du sol subit également des altérations, notamment une compaction accrue causée par le passage des machines forestières, ce qui affecte la porosité du sol et, par conséquent, la faune qui y réside \citep{Battigelli2004Shorttermimpact,Mazerolle2021Woodlandsalamander}. 
De plus, les opérations forestières perturbent la disponibilité des nutriments en modifiant la qualité et la quantité de la litière, en altérant les sécrétions racinaires, en favorisant le lessivage et en modifiant les propriétés chimiques du sol \citep{Covington1981Changesforest,Marshall2000Impactsforest,Lindo2003Microbialbiomass,Battigelli2004Shorttermimpact}. 
Ces bouleversements des conditions microclimatiques et physico-chimiques du sol ont des répercussions directes sur la biodiversité, 
notamment sur les organismes dépendant de microhabitats tels que le bois mort, les cavités des arbres matures ou les plaques racinaires \citep{Spies1999Dynamicforest,Christensen2005Deadwood,Brassard2008EffectsForest}. 
Ces éléments structurels, souvent réduits ou éliminés par les opérations de coupe, sont généralement remplacés par des pistes de débardage et des chemins d'exploitation \citep{Hansen1991ConservingBiodiversity}.

Les microhabitats jouent pourtant un rôle essentiel pour une grande partie de la faune du sol, notamment les arthropodes, les amphibiens et les microorganismes \citep{Paillet2010Biodiversitydifferences,Fedrowitz2014Canretention,Chaudhary2016Impactforest}. 
Leur disparition augmente la vulnérabilité de nombreuses espèces, tout en créant des conditions environnementales défavorables à leur survie. 
Par conséquent, les espèces forestières qui dépendent de la fraîcheur et de l'humidité caractéristiques des sols non perturbés peuvent voir leurs populations décliner, voire disparaître localement \citep{Kudrin2023metaanalysiseffects}. 
À l'inverse, les espèces adaptées aux milieux plus ouverts et secs peuvent coloniser les parcelles récemment coupées, entraînant une modification de la composition spécifique des communautés fauniques.

Plusieurs études se sont intéressés aux changements des caractéristiques forestières comme le débris ligneux grossiers, la profondeur de la litière, l'ouverture de la canopée, 
ainsi que sur leur impact sur la faune du sol après les opérations de récolte, dans le but d'orienter la gestion forestière \citep{Semlitsch2002CriticalElements,McKenny2006Effectsstructural}. 
Cependant, la faune du sol englobe une grande diversité de taxons, présentant des différences biologiques et écologiques significatives \citep{Kudrin2023metaanalysiseffects}. 
En conséquence, leur réponse à l'exploitation forestière varie selon le type de traitement appliqué et le groupe d'espèces étudié \citep{Malmstrom2009Dynamicssoil,Paillet2010Biodiversitydifferences}.

Bien qu'ils soient communs dans les forêts nord-américaines, les amphibiens et les arthropodes ont souvent été omis des stratégies de gestion forestière \citep{deMaynadier1995relationshipforest}. 
Cependant, des études ont mis en évidence leur rôle écologique crucial, tant dans le fonctionnement des réseaux trophiques que dans la dynamique du carbone dans les sols forestiers. 
De surcroît, les amphibiens et les arthropodes ont connu un déclin significatif au cours des dernières décennies, principalement en raison des pratiques d'aménagement forestier et des changements climatiques \citep{Houlahan2000Quantitativeevidence,Stuart2004Statustrends,Warren2018projectedeffect,Wagner2021Insectdecline}. 
Leur sensibilité aux changements environnementaux et leur capacité de dispersion limitée en font des modèles d'étude pertinents pour mieux appréhender les effets des pratiques forestières 
sur la biodiversité et l'intégrité écologique des forêts \citep{pongeVerticalDistributionCollembola2000,Ojala2001Dispersalmicroarthropods,birdChangesSoilLitter2004,Maleque2009Arthropodsbioindicators}. 
Parmi les groupes d'espèces fréquemment étudiés chez ces taxons, on retrouve la salamandre cendrée de l'Est (\textit{Plethodon cinereus} (Green, 1818)), les carabes (Carabidae) et les collemboles (Collembola).

\subsection*{Salamandre cendrée}

La salamandre cendrée, un membre des Plethodontidae, représente l'une des biomasses les plus importantes chez les vertébrés des forêts nord-américaines \citep{Burton1975Salamanderpopulations,Petranka1993Effectstimber,semlitschAbundanceBiomassProduction2014a}. 
En tant qu'espèce exclusivement terrestre et dépourvue de poumons, elle dépend de la respiration cutanée et donc des conditions d'humidité pour assurer ses échanges gazeux \citep{Heatwole1961Relationsubstrate}. 
Ce mode de respiration la contraint à occuper des microhabitats spécifiques, en surface lorsque la température et l'humidité sont favorables, ou en profondeur dans le sol durant des périodes moins propices \citep{Grizzell1949HibernationSite,FraserEmpiricalEvaluation1976,Jaeger1980MicrohabitatsTerrestrial}. 
Le retrait des refuges à la surface, tels que les débris ligneux, après la récolte peut réduire la qualité de l'habitat pour les salamandres et diminuer le temps qu'elles passent à la surface du sol forestier \citep{Achat2015Quantifyingconsequences,Peele2017EffectsWoody}. 
Étant donné que les salamandres cendrées se nourrissent et se reproduisent à la surface du sol forestier, des conditions de surface défavorables, un compactage des sols et de faibles niveaux de CWD peuvent affecter la dynamique des populations \citep{Peterman2014Spatialvariation}. 
Prédateur généraliste, la salamandre cendrée contribue de manière significative à la régulation des invertébrés détritivores, influençant directement les processus de décomposition, la circulation des nutriments dans les sols et la dynamique du carbone \citep{Burton1975Energyflow,Wyman1998Experimentalassessment,Walton2013Topdownregulation,Hickerson2017Easternredbacked}. 
Par ailleurs, elle constitue une proie de haute valeur nutritive pour divers prédateurs tels que les oiseaux, mammifères et reptiles, renforçant ainsi son rôle dans les dynamiques trophiques forestières \citep{Burton1975Energyflow,Pough1987abundancesalamanders}. 
Compte tenu de sa sensibilité aux perturbations environnementales, notamment en raison de sa respiration cutanée, la salamandre cendrée est souvent utilisée comme indicateur de la qualité des sols forestiers \citep{Welsh2001caseusing}. 
Son abondance et sa présence sont ainsi des paramètres couramment évalués pour mesurer l'impact des pratiques sylvicoles sur la biodiversité \citep{Harpole1999Effectsseven,Grialou2000effectsforest,Homyack2009Longtermeffects,Hocking2013Effectsexperimental,Mazerolle2021Woodlandsalamander}. 

\subsection*{Carabes}

Nous utilisons les carabes comme deuxième organismes modèles en raison de leur documentation détaillé sur les plans taxonomique et écologique \citep{loveiEcologyBehaviorGround1996}. 
Ces insectes ont une courte durée de vie, occupent une position élevée dans la chaîne alimentaire et réagissent rapidement et de manière complexe aux modifications de leur environnement \citep{loveiEcologyBehaviorGround1996}.
Les carabes jouent un rôle écologique dans la régulation des populations d'invertébrés, se nourrissant principalement d'aphides, de collemboles et d'escargots, tout en étant à leur tour des proies pour divers amphibiens, reptiles, oiseaux et mammifères \citep{loveiEcologyBehaviorGround1996}. 
Ils sont particulièrement sensibles à la complexité structurelle des peuplements forestiers, et ce, à différentes échelles temporelles et spatiales \citep{Butterfield1995Carabidbeetle,loveiEcologyBehaviorGround1996,Niemela2007effectsforestry}.
Avec environ 40 000 espèces répertoriées, les carabes constituent l'une des familles les plus diversifiées parmi les coléoptères et représentent l'une des plus fortes abundances d'arthropodes du sol \citep{Erwin1985taxonpulse,loveiEcologyBehaviorGround1996,Rochefort2006GroundBeetle}. 
Ils sont largement distribués et présents en nombre significatif dans presque tous les écosystèmes terrestres, bien que les espèces varient dans leur sélection d'habitats \citep{loveiEcologyBehaviorGround1996,kotzeFortyYearsCarabid2011a,Larochelle2003naturalhistory}. 
Les changements environnementaux et les modifications d'habitat favorisent certaines espèces au détriment d'autres, ce qui rend leur diversité et leur sensibilité particulièrement intéressantes pour étudier l'impact des perturbations environnementales \citep{Rainio2003Groundbeetles}.

\subsection*{Collemboles}

Parmi les taxons de la mésofaune du sol, les collemboles représentent un des groupes les plus abondants et les plus diversifiés de petits arthropodes \citep{rusekBiodiversityCollembolaTheir1998}. 
Différentes communautés de collemboles occupent un ensemble de niches écologiques allant de la litière aux différents horizons du sol \citep{pongeVerticalDistributionCollembola2000}.
La répartition verticale de ces communautés dépend essentiellement des conditions abiotiques du sol telles que la luminosité, le taux d’humidité ou encore la porosité.
Ces invertébrés exercent une influence significative sur l'écologie forestière à travers leur influence sur les micro-organismes, sur la décomposition et le cycle des nutriments.
Étant principalement fongivore et détritivore, les différentes communautés de collemboles influencent les taux de décomposition du bois, affectent la structure physique et les taux de minéralisation de la litière, 
influencent l'absorption des nutriments et régulent la communauté microbienne, en plus de participer à la formation de microstructure du sol \citep{Petersen1982comparativeanalysis,Neher2012Linkinginvertebrate,Maass2015Functionalrole,Potapov2016Connectingtaxonomy}. 
Les collemboles participent également à la chaine trophique en constituant une ressource alimentaire abondantes pour plusieurs organismes comme les amphibiens, les coléoptères, les arachnides, les oiseaux et les reptiles.

Grâce aux relations trophiques qui les lient, leur sensibilité aux variations des conditions environnementales, et leur dépendance aux caractéristiques forestières comme les débris ligneux et la litière, 
ces trois groupes d'espèces constituent des modèles pertinents pour étudier l'impact des pratiques forestières sur la faune du sol. 

La plupart des travaux qui se sont intéressés aux impacts des traitements sylvicoles sur la faune discutent généralement des effets directs des perturbations sur un ou plusieurs groupes d'espèces, 
sans tenir compte des relations existantes entre les variables environnementales et les différents groupes d'espèces, 
négligeant ainsi les effets indirects des coupes sur la faune du sol \citep{josephIntegratingOccupancyModels2016,Pollierer2021Diversityfunctional,Kudrin2023metaanalysiseffects}. 
Mon projet comblait justement cette lacune en essayant de comprendre comment les effets des traitements sylvicoles se propagent à l’intérieur du réseau écologique forestier et influence la dynamique de la faune du sol.  
Ultimement, ce gain de connaissances fournira des outils précieux pour faciliter la gestion durable des forêts.


\section*{Objectifs et hypothèses}
\label{sec:objectifs}
\phantomsection\addcontentsline{toc}{section}{\nameref{sec:objectifs}}

Le but de mon étude était de comprendre comment les traitements sylvicoles, effectués dans un contexte de migration assistée, 
affecte la dynamique des écosystèmes du sol forestier. Les objectifs qui s’y rattachaient étaient :

\begin{enumerate}
    \item De quantifier l'effet des traitements de coupes forestières sur les variables environnementales qui influencent l'utilisation de l'habitat par la faune du sol.
    \item De mesurer l'impact des coupes forestières sur l'utilisation de l'habitat par la faune du sol.
\end{enumerate}

L'hypothèse 1.1 liée à notre deuxième objectif soutenait que les variables environnementales favorables à l'utilisation de l'habitat par les espèces fluctuent 
en fonction de l'intensité des coupes forestières. Ainsi, les traitements de coupes forestières constituent une variable englobant 
les changements de conditions environnementales.

L'hypothèse 2.1 attachée à notre premier objectif stipulait que les traitements de coupe forestière entraînent une modification de l'utilisation de l'habitat 
par la faune du sol et se propage à travers le réseau trophique. Spécifiquement, les coupes affectent l'utilisation de l'habitat par les salamandres et 
les grands carabes (compétiteurs de salamandres), ce qui modifie ensuite la sélection d'habitat des petits carabes (proies de salamandres), puis enfin des collemboles (proies de grands carabes et salamandres).



\cleardoublepage

\bibliography{References.bib}
\bibliographystyle{ecologyNewFR.bst}
