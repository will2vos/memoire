\chapter*{Introduction générale}         % ne pas numéroter
\label{chap-introduction}       % étiquette pour renvois
\phantomsection\addcontentsline{toc}{chapter}{\nameref{chap-introduction}} % inclure dans TdM

% \usepackage[french]{babel}

\section*{Mise en contexte}
\label{sec:contexte}
\phantomsection\addcontentsline{toc}{section}{\nameref{sec:contexte}}

Les écosystème forestiers jouent un rôle essentiel dans la biosphère à travers leur rôle économique et leur valeur écosystèmiques en fournissant une multitude de produit et services \citep{Balvanera2006Quantifyingevidence}. 
Leur présence permet de réguler les flux de nutriments et d'énergie, notamment à travers la séquestration du carbone, la régulation du climat, la rétention de l'eau et la conservation de la biodiversité \citep{Balvanera2006Quantifyingevidence,Diaz2006BiodiversityLoss,Canadell2008Managingforests,Pawson2013Plantationforests}. 

La diversité des espèces forestières est désormais reconnue comme essentielle pour le bon fonctionnement des écosystèmes terrestres, cependant les activités humaines et l'utilisation des terres entraînent des changements majeurs qui affectent cette biodiversité \citep{Newbold2015Globaleffects}.
Par ailleurs, les engagements internationaux ont souligné l'urgence de freiner la perte de biodiversité tout en promouvant une gestion durable des forêts \citep{Scherer-Lorenzen2005ForestDiversity,Parviainen2007Maintenanceconservation}. 
En effet, la croissance démographique mondiale et l'accroissement des besoins en produits forestiers et autres services ont conduit à une intensification des pratiques d'exploitation forestière au cours des dernières décennies \citep{Foley2005GlobalConsequences}. 
Plus précisement, les modifications de l'utilisation des terres, telles que les récoltes forestières, l'agriculture et l'urbanisation, sont des facteurs associées à la perte de biodiversité et à la réduction des écosystèmes forestiers \citep{Sala2000Globalbiodiversity,Naeem2012functionsbiological,Bichet2016Maintaininganimal}.
Un défi en biologie de la conservation est aujourd'hui de préserver la biodiversité face aux impacts croissants des activités humaines sur les habitats fauniques. 
L'exploitation de bois permettant de répondre aux besoins industriels est ainsi reconnue pour participer au déclin de nombreuses espèces qui dépendent des habitats forestiers \citep{Bengtsson2000Biodiversitydisturbances}. 
Des recherches ont notamment démontré les effets des coupes forestières sur divers groupes taxonomiques, tels que les oiseaux, chauves-souris, papillons, tortues, petits mammifères et insectes \citep{Summerville2011Managingforest,Currylow2012ShortTermForest,Kaminski2013EffectsForest,Kellner2013Shorttermresponses,Caldwell2019ComparisonBat}.

L'exploitation des ressources naturelles par les humains peut entraîner la mortalité des animaux et des plantes, forcer ou limiter le déplacement des espèces, intensifier les interactions biotiques, modifier leurs traits de vie, altérer leur morphologie et leur physiologie, et même affecter leurs formes polymorphiques \citep{Sergio2018Animalresponses}. 
De plus, l'exploitation forestière entraîne des pertes d'habitats et une fragmentation des milieux naturels, ce qui limite l'accès à la nourriture, aux refuges et aux zones de reproduction ainsi que la taille et la diversité génétique des populations \citep{Coelho2020Effectsanthropogenic}.

La gestion forestière peut entraîner une réduction de la connectivité entre les habitats, les communautés et les processus écologiques \citep{Lindenmayer2006Generalmanagement}.
Cette diminution peut avoir un impacts majeur sur la conservation de la biodiversité en compromettant la capacité des populations à se rétablir et à se maintenir après une perturbation \citep{Lamberson1994ReserveDesign}. 
Le manque de connectivité limite les déplacements des individus au sein de leur habitat et des corridors écologique, réduisant ainsi le flux génétique entre populations, ce qui accroît le risque d'extinction locale \citep{Saccheri1998Inbreedingextinction}.

En ramenant les peuplements à un stade de succession précoce, l'exploitation forestière homogénéise le paysage forestier, ce qui conduit à une surreprésentation des forêts en début de succession au détriment des stades plus avancés \citep{Cyr2009Forestmanagement,Boucher2017Cumulativepatterns}. 
Plus précisément, l'exploitation forestière diminue la complexité structurelle, notamment au niveau de la composition des espèces d'arbres, de la stratification verticale, de la structure d'age, de dynamique de succession et de la fréquence de perturbation \citep{Commarmot2005Structurevirgin}. 
L'hétérogénéité structurelle d'une forêt joue un rôle important pour la biodiversité puisqu'elle fournit une plus grande variété d'habitat et participe à la connectivité en favoridant la dispersition de certaine espèces. 
Lors d'une perturbation naturelle ou artificiel, un peuplement avec une structure plus hétérogène se régénérera plus rapidement, car il favorise la survie des espèces présentes et facilite le retour des espèces qui ont disparu. 
Ainsi, la récolte forestière peut menacer les espèces associées aux attributs des vieilles forêts, ainsi que les organismes ayant une faible capacité de dispersion, pour lesquels le risque d'extinction augmente rapidement lorsque la continuité écologique est rompue \citep{Norden2001Conceptualproblems,Martin2021indicatorspecies}.  
Ces altérations sont prévues pour être particulièrement néfastes pour la biodiversité, notamment pour les espèces associées aux forêts mixtes et conifériennes plus anciennes \citep{Tremblay2018Harvestinginteracts,Cadieux2020Projectedeffects}.

Ces changement dans les attribut forestiers participe une perte de la diversité spécifique et fonctionnelle à différents niveaux trophiques. 
Sur le long terme, cela peut entrainer une diminution de la résilience des forêt à l'échelle locale et engendrer une diminution des services écosystémiques et avoir un impacte négatif pour les sociétés \citep{Hooper2012globalsynthesis,Edwards2014Maintainingecosystem}. 

Parmis les traitements sylvicoles pouvant avoir un effets significatif sur les milieux naturels on retrouve les traitements de coupes forestières.
Cependant, l'objectif économique et l'impacts écologique des coupes forestières varie selon le type traitement, allant d'un niveau de pertubation naturel ou semi-naturelle dans le cas de la gestion forestière extensive à artificiel dans le cas gestion forestière intensive \citep{Ameray2021Forestcarbon}. 

Historiquement, les coupes totales font partie des pratiques sylvicoles les plus courantes dans les forêts tempérés et boréaux \citep{Fedrowitz2014Canretention,Chaudhary2016Impactforest}. 
Elles font partie d'une gestion forestière intensive largement utilisée pour accroître la productivité et la qualité du bois à court terme, afin de répondre aux besoins croissants de l'industrie et d'augmenter la rentabilité \citep{Irland2011TimberProductivitya}.
Ces coupes implique l'abattage de tous les arbres dans une zone définis.
Elles sont techniquement facile à exécuter, car l'ensemble du couvert supérieur de la parcelle est retiré en une seule récolte, entraînant une déforestation temporaire d'une zone auparavant boisée. 
Les coupes totale utilise une structure équienne avec une seule espèce, ce qui simplifie la structure forestière et réduit la diversité biologique entrainant une homogénéisation des peuplements \citep{Rosenvald2008whatwhen}. 
Cette simplification du milieu perturbe des processus écologiques et évolutifs importants et entraine un déclin de la résilience des forêts \citep{Holling2001UnderstandingComplexity}. 
De plus, la période de rotation est plus courte pour ce type de traitement ce qui amène un fréquence des perturbation plus élevé. 
Le maintient de pratiques de gestion forestière transformant radicalement les structures des écosystèmes par rapport à celles observées naturellement augmente de façon importante les probabilités d'observer un déclin progressif de la biodiversité ainsi que l'extinction locale d'espèces \citep{Hanski2000Extinctiondebt}.  
Cependant, certains auteurs souligne que les coupes totales peuvent être utiliser pour imitation des perturbations naturelles de grande ampleur \citep{Greenberg1995comparisonbird}. 

Ces dernières décennies, des pratiques sylvicoles combinant la récolte de bois et la préservation de la biodiversité ont été promues pour atténuer les impacts des coupes totales \citep{Gustafsson2012Retentionforestry}.
Une gestion écosystemique caractérisé par l'émulation des perturbations naturelles a été proposée comme une stratégie prometteuse pour une gestion durable des forêts \citep{Perry1998scientificbasis,Kuuluvainen2002Naturalvariabilitya}. 
Selon cette stratégie, les actions de gestion sont planifiées de manière à émuler les perturbations naturelles et leurs résultats, y compris les structures des peuplements et les successions. 
La gestion écosytémique a pour but de préserver la biodiversité et de maintenir la résilience des peuplements, tout en garantissant la disponibilité d'une grande variété de services écosystémiques \citep{Szaro1998emergenceecosystem,MacDicken2015Globalprogress}.

Les coupes partielles font partie d'une gestion écosystémique visant à préserver la composition et la structure forestières \citep{Bergeron1999Forestmanagementa}.
Elles sont habituellement utilisées dans le cadre d'un plan d'aménagement extensif qui favorise la régénération naturelle et reproduit les perturbations naturelles \citep{Irland2011Timberproductivity}. 
Les coupes partielles consistuent une suppression sélective d'arbres tout en laissant une partie du peuplement intacte \citep{Ameray2021Forestcarbon}. 
La rétention d'arbres dans les coupes partielles offre des structures de succession tardive, favorisant ainsi la préservation d'une plus grande biodiversité \citep{Ameray2021Forestcarbon}.
Elles sont souvent employées pour stimuler la croissance des arbres les plus vigoureux, encourager la diversité des espèces ou préserver une canopée ouverte \citep{Irland2011Timberproductivity}.
Ce type de traitement repose sur une structure multi-âge, avec ou sans mélanges d'espèces, et est caractérisée par des rotations plus longues \citep{Kuuluvainen2009Forestmanagement}. 
La conservation d'arbres en coupe partielle et l'allongement de la période de rotation offrent d'autres avantages sur le plan économique et écosystémique \citep{Ameray2021Forestcarbon}. 
Cela favorise la séquestration du carbone, préserve l'apport de matière organique ainsi que le cycle de nutriment, tout en maintenant un structure hétérogène et différentes niches écologiques pour la faune \citep{Dahlgren1994effectswholetree,Barg1999Influencepartial,Tong2020Forestmanagement,Ameray2021Forestcarbon}.

Comme mentionné précédemment, l'impact de la gestion forestière sur la biodiversité dépend du type de traitement sylvicole utilisé. Cependant, la réaction de la faune à un traitement peut varier selon le type de forêt et les groupes d'espèces concernés \citep{Paillet2010Biodiversitydifferences,Kudrin2023metaanalysiseffects}.

Le défis de préserver la biodiversité et les écosystèmes forestiers tout en répondant aux exigences économiques est d'autant plus grand dans le contexte actuelle des changements climatiques. 
La hausse globale de température représente une menace de plus pour la pérennité de la faune et de la flore en modifiant de façon importantes les conditions environnementales \citep{McKenney2009Climatechange,Trumbore2015Foresthealth,Seidl2017Forestdisturbances,Messier2022Warningnatural}. 
Parmi ces modifications, on prévoit un allongement et une intensification des périodes de sècheresse, une augmentation du nombre de feux de forêt, une altération des régimes de précipitation et une hausse des perturbations biotiques \citep{Parmesan2007Influencesspecies,Joyce2013Climatechange,Gatti2021Amazoniacarbon,Heidari2021Effectsclimate}. 
À cela s'ajoutent des changements dans la phénologie ainsi que dans la distribution des arbres due au manque d'adaptation des végétaux aux nouvelles conditions de leur milieu \citep{Aitken2008Adaptationmigration,Chuine2010Whydoes,Zhu2012Failuremigrate,Gray2013Trackingsuitable}.
De plus, les stresseurs climatiques agissent de façon additive ou synergique avec les activités forestières et l'interactions entre ces facteurs exerce un impact beaucoup plus grand sur la biodiversité et l'environnement \citep{Brook2008Synergiesextinctiona,Tremblay2018Harvestinginteracts,Ochs2022Responseterrestrial,Bouderbala2023Longtermeffect}. 
La composition des forêts pourrait ainsi être altérée, entraînant des ajustements dans les pratiques d'aménagement forestier et les stratégies de conservation \citep{McKenney2009Climatechange,Chmura2011Forestresponses,Lo2011Linkingclimate}.
Différentes mesures d'atténuation ont été proposées pour prévenir la perte des forêts et améliorer la résilience de celle-ci comme l'augmentation de la diversité fonctionnelle et l'amélioration de la connectivité à travers le paysage forestier \citep{Messier2019functionalcomplex}.
D'autres solutions tel que la migration assistée d'arbres ont été proposée comme une mesure d'atténuation permettant le déplacement d'individus ou de matériel génétique depuis un territoire climatique originel vers une zone de climat futur plus propice à la croissance des arbres \citep{Vitt2010Assistedmigration,Dumroese2015Considerationsrestoring,Park2018Informationunderload,Park2023Provenancetrials}. 
La migration assistée d'arbre permettrait de modifier rapidement la composition des peuplements pour mieux convenir au climat futur de celui-ci répondant ainsi aux besoins de conservation, maintenant les services écosystémiques et préservant la valeur économique \citep{Pedlar2011implementationassisted,Ste-Marie2011Assistedmigration,Winder2011Ecologicalimplications}.
Cependant un manque de connaissances et un degré d'incertitude subsistent toutefois autour des nouvelles mesures d'adaptation \citep{Klenk2015assistedmigration,Park2018Informationunderload}. 
Il est donc essentiel de comprendre l'impact des traitements sylvicoles sur la biodiversité, dans un contexte où l'aménagement forestier doit adapter sa pratique aux changements climatiques et aux déclin de la biodiversité, tout en répondant aux besoin économiques.


\section*{Espèces à l'étude}
\label{sec:species}
\phantomsection\addcontentsline{toc}{section}{\nameref{sec:species}}

Au sein de la biodiversité forestières, la faune du sol possède une importance primordiale dans les écosystèmes forestiers, en contribuant entre autres à la circulation de la matière et de l'énergie à travers la chaîne alimentaire, ainsi qu'au recyclage des nutriments \citep{Seibold2021contributioninsects,Kudrin2023metaanalysiseffects}.
Cependant cette communauté fait partie des groupes d'espèces les plus affecté par les pertubations environnementales au sein de la biodiveristé forestières \citep{Marshall2000Impactsforest,Coyle2017Soilfauna}. 
En raison de leur petite taille, la faune du sol possède une capacité de dispersion plus restreinte, ce qui les rend plus vulnérables face aux perturbations de leur environnement \citep{Kudrin2023metaanalysiseffects}.

Les pratiques sylvicoles comme les coupes forestières peuvent induisent des modifications brusque et drastique dans les propriétés de l'habitat forestiers. 
Ces altérations peuvent affecter négativement la biodiversité, particulièrement la faune vivant à la hauteur du sol \citep{Lindo2003Microbialbiomass,Paillet2010Biodiversitydifferences,Fedrowitz2014Canretention,Chaudhary2016Impactforest}. 
Le retrait de la canopée augmente l'exposition de la surface du sol au rayonnement solaire, ce qui entraîne une élévation de la température et une modification de l'humidité du sol \citep{Lindo2003Microbialbiomass,Brook2008Synergiesextinction,Zhang2022Intensiveforest}. 
Ces changements sont exacerbés par une augmentation de la vitesse du vent et une intensification des précipitations atteignant le sol \citep{Keenan1993ecologicaleffects,Heithecker2007Edgerelatedgradients}. 

La structure du sol subit également des altérations, notamment une compaction accrue due au passage des machines forestières, ce qui peut affecter la porosité du sol et, par conséquent, la faune qui y vit \citep{Battigelli2004Shorttermimpact,Mazerolle2021Woodlandsalamander}. 
De plus, les opérations forestières perturbent la disponibilité des nutriments en modifiant la qualité et la quantité de litière, en altérant les sécrétions racinaires, en provoquant un lessivage, et en changeant les propriétés chimiques du sol \citep{Covington1981Changesforest,Marshall2000Impactsforest,Lindo2003Microbialbiomass,Battigelli2004Shorttermimpact}. 
Ce bouleversement des conditions microclimatiques et physico-chimiques du sol peut entraîner des répercussions directes sur la biodiversité, 
en particulier sur les organismes qui dépendent des microhabitats comme le bois mort, les cavités dans les arbres matures ou les plaques racinaires \citep{Berg1994ThreatenedPlant,Spies1999Dynamicforest,Bouget2005Shorttermeffect,Christensen2005Deadwood,Brassard2008EffectsForest}.
Ces éléments structurels, souvent éliminés ou réduits lors des opérations de coupe sont généralement remplacés par des pistes de débardage et les chemins d'exploitation \citep{Hansen1991ConservingBiodiversity}. 

Les microhabitats jouent pourtant un rôle crucial pour une grande partie de la faune du sol, comme les arthropodes, les amphibiens et les microorganismes \citep{Paillet2010Biodiversitydifferences,Fedrowitz2014Canretention,Chaudhary2016Impactforest}. 
La disparition de ces refuges accroît la vulnérabilité de nombreuses espèces, tout en favorisant l’émergence de conditions environnementales défavorables à leur survie. 
Ainsi, les espèces forestières qui dépendent des conditions de fraîcheur et d’humidité typiques des sols forestiers non perturbés peuvent voir leur population diminuer, voire disparaître localement \citep{Kudrin2023metaanalysiseffects}. 
En revanche, des espèces associées aux zones ouvertes et plus sèches peuvent coloniser les parcelles récemment coupées, modifiant ainsi la composition spécifique des communautés fauniques.

Plusieurs recherches se sont intéressés aux changements des caractéristiques forestières comme le débris ligneux grossiers, la profondeur de la litière, l'ouverture de la canopée 
et leur influence sur la faune du sol après la récolte forestière afin de guider la gestion forestière \citep{Semlitsch2002CriticalElements,McKenny2006Effectsstructural}. 

La faune du sol regroupe cependant une grande diversité de taxons, caractérisés par des différences significatives sur le plan biologique et écologique \citep{Kudrin2023metaanalysiseffects}. 
Par conséquent, leur réponse à l'exploitation forestière varier selon le type de traitement appliqué et le groupe d'espèce étudié \citep{Malmstrom2009Dynamicssoil,Paillet2010Biodiversitydifferences}.

Malgré leur abondance dans les écosystèmes forestiers de l'est de l'Amérique du Nord, les amphibiens et les arthropodes ont longtemps été négligés dans les stratégies de gestion forestière \citep{deMaynadier1995relationshipforest}. 
Or, des recherches ont souligné leur importance écologique, tant pour le fonctionnement des réseaux trophiques que pour la dynamique du carbone au sein des sols forestiers. 
Leur sensibilité aux changements environnementaux et leurs capacité de dispersion restrainte en font des modèles d’étude pertinent pour mieux comprendre les effets des pratiques forestières sur la biodiversité et l'intégrité écologique des forêts \citep{pongeVerticalDistributionCollembola2000,Ojala2001Dispersalmicroarthropods,birdChangesSoilLitter2004,Maleque2009Arthropodsbioindicators}.
De plus, les amphibiens et les arthropodes ont subi un déclin majeur au cours des dernières décennies, principalement dû aux pratiques d'aménagement forestier et aux changements climatiques \citep{Houlahan2000Quantitativeevidence,Stuart2004Statustrends,Warren2018projectedeffect,Wagner2021Insectdecline}. 

Parmi les groupes d'espèces fréquemment étudiés chez ces taxons, on retrouve la salamandre cendrée de l'Est (\textit{Plethodon cinereus} (Green, 1818)), les carabes (Carabidae) et les collemboles (Collembola).

\subsection*{Salamandre cendrée}

La salamandre cendrée, un membre des Plethodontidae, représente l'une des biomasses les plus importantes chez les vertébrés des forêts nord-américaines \citep{Burton1975Salamanderpopulations,Petranka1993Effectstimber,semlitschAbundanceBiomassProduction2014a}. 
En tant qu'espèce exclusivement terrestre et dépourvue de poumons, elle dépend de la respiration cutanée et donc des conditions d'humidité pour assurer ses échanges gazeux \citep{Heatwole1961Relationsubstrate}. 
Ce mode de respiration la contraint à occuper des microhabitats spécifiques, en surface lorsque la température et l'humidité sont favorables, ou en profondeur dans le sol durant des périodes moins propices \citep{Grizzell1949HibernationSite,FraserEmpiricalEvaluation1976,Jaeger1980MicrohabitatsTerrestrial}. 
Le retrait des refuges à la surface, tels que les débris ligneux, après la récolte peut réduire la qualité de l'habitat pour les salamandres et diminuer le temps qu'elles passent à la surface du sol forestier \citep{Achat2015Quantifyingconsequences,Peele2017EffectsWoody}. 
Étant donné que les salamandres cendrées se nourrissent et se reproduisent à la surface du sol forestier, des conditions de surface défavorables, un compactage des sols et de faibles niveaux de CWD peuvent affecter la dynamique des populations \citep{Peterman2014Spatialvariation}. 
Prédateur généraliste, la salamandre cendrée contribue de manière significative à la régulation des invertébrés détritivores, influençant directement les processus de décomposition, la circulation des nutriments dans les sols et la dynamique du carbone \citep{Burton1975Energyflow,Wyman1998Experimentalassessment,Walton2013Topdownregulation,Hickerson2017Easternredbacked}. 
Par ailleurs, elle constitue une proie de haute valeur nutritive pour divers prédateurs tels que les oiseaux, mammifères et reptiles, renforçant ainsi son rôle dans les dynamiques trophiques forestières \citep{Burton1975Energyflow,Pough1987abundancesalamanders,Petranka1998SalamandersUnited}. 
Compte tenu de sa sensibilité aux perturbations environnementales, notamment en raison de sa respiration cutanée, la salamandre cendrée est souvent utilisée comme indicateur de la qualité des sols forestiers \citep{Welsh2001caseusing}. 
Son abondance et sa présence sont ainsi des paramètres couramment évalués pour mesurer l'impact des pratiques sylvicoles sur la biodiversité \citep{Harpole1999Effectsseven,Grialou2000effectsforest,Homyack2009Longtermeffects,Hocking2013Effectsexperimental,Mazerolle2021Woodlandsalamander}. 

\subsection*{Carabes}

Nous utilisons les carabes comme deuxième organismes modèles en raison de leur documentation détaillé sur les plans taxonomique et écologique \citep{loveiEcologyBehaviorGround1996}. 
Ces insectes ont une courte durée de vie, occupent une position élevée dans la chaîne alimentaire et réagissent rapidement et de manière complexe aux modifications de leur environnement \citep{loveiEcologyBehaviorGround1996}.
Les carabes jouent un rôle écologique dans la régulation des populations d'invertébrés, se nourrissant principalement d'aphides, de collemboles et d'escargots, tout en étant à leur tour des proies pour divers amphibiens, reptiles, oiseaux et mammifères \citep{loveiEcologyBehaviorGround1996}. 
Ils sont particulièrement sensibles à la complexité structurelle des peuplements forestiers, et ce, à différentes échelles temporelles et spatiales \citep{Butterfield1995Carabidbeetle,loveiEcologyBehaviorGround1996,Niemela2007effectsforestry}.
Avec environ 40 000 espèces répertoriées, les carabes constituent l'une des familles les plus diversifiées parmi les coléoptères et représentent l'une des plus fortes abundances d'arthropodes du sol \citep{Erwin1985taxonpulse,loveiEcologyBehaviorGround1996,Rochefort2006GroundBeetle}. 
Ils sont largement distribués et présents en nombre significatif dans presque tous les écosystèmes terrestres, bien que les espèces varient dans leur sélection d'habitats \citep{loveiEcologyBehaviorGround1996,kotzeFortyYearsCarabid2011a,Larochelle2003naturalhistory}. 
Les changements environnementaux et les modifications d'habitat favorisent certaines espèces au détriment d'autres, ce qui rend leur diversité et leur sensibilité particulièrement intéressantes pour étudier l'impact des perturbations environnementales \citep{Rainio2003Groundbeetles}.

\subsection*{Collemboles}

Parmi les taxons de la mésofaune du sol, les collemboles représentent un des groupes les plus abondants et les plus diversifiés de petits arthropodes \citep{rusekBiodiversityCollembolaTheir1998}. 
Différentes communautés de collemboles occupent un ensemble de niches écologiques allant de la litière aux différents horizons du sol \citep{pongeVerticalDistributionCollembola2000}.
La répartition verticale de ces communautés dépend essentiellement des conditions abiotiques du sol telles que la luminosité, le taux d’humidité ou encore la porosité.
Ces invertébrés exercent une influence significative sur l'écologie forestière à travers leur influence sur les micro-organismes, sur la décomposition et le cycle des nutriments.
Étant principalement fongivore et détritivore, les différentes communautés de collemboles influencent les taux de décomposition du bois, affectent la structure physique et les taux de minéralisation de la litière, 
influencent l'absorption des nutriments et régulent la communauté microbienne, en plus de participer à la formation de microstructure du sol \citep{Petersen1982ComparativeAnalysisa,Neher2012Linkinginvertebrate,Maass2015Functionalrole,Potapov2016Connectingtaxonomy}. 
Les collemboles participent également à la chaine trophique en constituant une ressource alimentaire abondantes pour plusieurs organismes comme les amphibiens, les coléoptères, les arachnides, les oiseaux et les reptiles.

Grâce aux relations trophiques qui les lient, leur sensibilité aux variations des conditions environnementales, et leur dépendance aux caractéristiques forestières comme les débris ligneux et la litière, 
ces trois groupes d'espèces constituent des modèles pertinents pour étudier l'impact des pratiques forestières sur la faune du sol. 

La plupart des travaux qui se sont intéressés aux impacts des traitements sylvicoles sur la faune discutent généralement des effets directs des perturbations sur un ou plusieurs groupes d'espèces, 
sans tenir compte des relations existantes entre les variables environnementales et les différents groupes d'espèces, 
négligeant ainsi les effets indirects des coupes sur la faune du sol \citep{josephIntegratingOccupancyModels2016,Pollierer2021Diversityfunctional,Kudrin2023metaanalysiseffects}. 
Mon projet comblait justement cette lacune en essayant de comprendre comment les effets des traitements sylvicoles se propagent à l’intérieur du réseau écologique forestier et influence la dynamique de la faune du sol.  
Ultimement, ce gain de connaissances fournira des outils précieux pour faciliter la gestion durable des forêts.


\section*{Objectifs et hypothèses}
\label{sec:objectifs}
\phantomsection\addcontentsline{toc}{section}{\nameref{sec:objectifs}}

Le but de mon étude était de comprendre comment les traitements sylvicoles, effectués dans un contexte de migration assistée, 
affecte la dynamique des écosystèmes du sol forestier. Les objectifs qui s’y rattachaient étaient :

\begin{enumerate}
    \item De quantifier l'effet des traitements de coupes forestières sur les variables environnementales qui influencent l'utilisation de l'habitat par la faune du sol.
    \item De mesurer l'impact des coupes forestières sur l'utilisation de l'habitat par la faune du sol.
\end{enumerate}

L'hypothèse 1.1 liée à notre deuxième objectif soutenait que les variables environnementales favorables à l'utilisation de l'habitat par les espèces fluctuent 
en fonction de l'intensité des coupes forestières. Ainsi, les traitements de coupes forestières constituent une variable englobant 
les changements de conditions environnementales.

L'hypothèse 2.1 attachée à notre premier objectif stipulait que les traitements de coupe forestière entraînent une modification de l'utilisation de l'habitat 
par la faune du sol et se propage à travers le réseau trophique. Spécifiquement, les coupes affectent l'utilisation de l'habitat par les salamandres et 
les grands carabes (compétiteurs de salamandres), ce qui modifie ensuite la sélection d'habitat des petits carabes (proies de salamandres), puis enfin des collemboles (proies de grands carabes et salamandres).



\cleardoublepage

\bibliography{References.bib}
\bibliographystyle{ecologyNewFR.bst}
