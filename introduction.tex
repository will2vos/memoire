\chapter*{Introduction}         % ne pas numéroter
\label{chap-introduction}       % étiquette pour renvois
\phantomsection\addcontentsline{toc}{chapter}{\nameref{chap-introduction}} % inclure dans TdM

% \usepackage[french]{babel}

\section*{Mise en contexte}
\label{sec:contexte}
\phantomsection\addcontentsline{toc}{section}{\nameref{sec:contexte}}

Les environnements forestiers jouent un rôle majeur à l’échelle planétaire à travers leurs fonctions écosystémiques et l'apport \hl{économique qu’ils représentent} en fournissant de nombreux biens et services \citep{Balvanera2006Quantifyingevidence}. 
Ces milieux permettent de réguler les flux de nutriments et d’énergie en participant notamment à la séquestration du carbone, à la stabilisation du climat, à la rétention de l’eau ainsi qu’à la conservation de la biodiversité \citep{Balvanera2006Quantifyingevidence,Diaz2006BiodiversityLoss,Canadell2008Managingforests,Pawson2013Plantationforests}. 
Toutefois, la croissance démographique et l’augmentation des besoins en produits ligneux amènent une hausse de l’exploitation forestière depuis maintenant plusieurs décennies \citep{Foley2005GlobalConsequences}. 
Par conséquent, le changement d’affectations des sols, causé par la récolte de bois, de l’agriculture et de l’urbanisation, occasionne une diminution globale du territoire forestier, entraînant une perte de biodiversité \citep{Bengtsson2000Biodiversitydisturbances,Sala2000Globalbiodiversity,Naeem2012functionsbiological,Bichet2016Maintaininganimal}. 
La biodiversité est cependant essentielle pour le bon fonctionnement des écosystèmes terrestres, et plusieurs engagements internationaux ont souligné l'urgence de freiner ce déclin et encouragent une gestion durable des forêts \citep{Newbold2015Globaleffects}. 

De nombreuses recherches ont démontré l'impact de la gestion forestière sur différents groupes fauniques, tels que les oiseaux, les mammifères, les reptiles, les amphibiens et les insectes \citep{Drapeau2000LandscapeScaleDisturbances,Suffice2015Shorttermeffects,bouchardBeetleCommunityResponse2016b,Hughes2019Impactnatural,Mazerolle2021Woodlandsalamander}. 
Les pratiques sylvicoles peuvent provoquer une hausse de mortalité chez les espèces, perturber ou restreindre leurs déplacements, intensifier les interactions biotiques, modifier leurs cycles de vie et altérer leur morphologie ainsi que leur physiologie \citep{Sergio2018Animalresponses}. 
De plus, l'exploitation des peuplements entraîne des pertes d'habitat et la fragmentation des milieux naturels, diminuant ainsi l'accès à la nourriture, aux refuges et aux sites de reproduction, tout en réduisant la taille et la diversité génétique des populations \citep{Bouderbala2023Longtermeffect}.
Ces pratiques causent également des problèmes de connectivité entre les habitats, les groupes d'espèces et les processus écologiques \citep{Lindenmayer2006Generalmanagement}. 
Une baisse de la connectivité peut compromettre la capacité des populations à se rétablir et à se maintenir après une perturbation, ce qui affecte négativement la conservation de la faune \citep{Lamberson1994ReserveDesign}. 
Cela restreint de la même façon le déplacement des individus au sein de leur habitat et le long des corridors écologiques, limitant ainsi le flux génétique entre populations et augmentant le risque d'extinction locale \citep{Saccheri1998Inbreedingextinction}. 

De plus, à l’échelle du paysage, l'exploitation forestière peut amener une homogénéisation du paysage en ramenant les peuplements à des stades précoces de succession, ce qui favorise une sous-représentation des forêts des stades de succession plus avancés \citep{Cyr2009Forestmanagement,Boucher2017Cumulativepatterns}. 
Cette pratique peut réduire la complexité structurelle des forêts, affectant parfois des aspects tels que la composition des espèces d'arbres, la stratification verticale, la structure d'âges, la dynamique de succession et la fréquence des perturbations \citep{Bergeron2000Speciesstand,Commarmot2005Structurevirgin,Varga2005Treesizediversity}. 
Ces modifications sont particulièrement préoccupantes pour la biodiversité, notamment pour les espèces qui vivent en forêts mixtes et conifériennes matures \citep{Tremblay2018Harvestinginteracts,Cadieux2020Projectedeffects}.
Le maintien de l'hétérogénéité structurelle joue un rôle important, car cela amène une variété d'habitats et permet de renforcer la connectivité en améliorant la dispersion de certaines espèces. 
C’est pourquoi l’aménagement forestier peut représenter un risque particulièrement important pour les espèces dépendantes des attributs de forêts complexes ainsi que pour celles ayant une faible capacité de dispersion  \citep{Norden2001Conceptualproblems,Martin2021indicatorspecies}. 
De plus, un peuplement avec une structure plus hétérogène est plus résilient lors d'une perturbation, car il facilite la régénération des espèces d'arbres présentes et le retour de celles qui ont disparu \citep{Kuuluvainen2009Forestmanagement}. 

Les modifications des caractéristiques forestières peuvent participer à la perte d'une partie de la diversité spécifique et fonctionnelle associée au milieu forestier. 
Certaines pratiques de gestion peuvent ainsi réduire la résilience des forêts à l'échelle locale et entraîner une diminution des services écosystémiques \citep{Hooper2012globalsynthesis,Edwards2014Maintainingecosystem}. 


\section*{Traitements sylvicoles}
\label{sec:sylvicole}
\phantomsection\addcontentsline{toc}{section}{\nameref{sec:sylvicole}}

Au cours des dernières décennies, les pratiques sylvicoles qui favorisent la sauvegarde de la biodiversité ont été encouragées afin de réduire les effets néfastes de la gestion intensive \citep{Gustafsson2012Retentionforestry}. 
\hl{L'aménagement écosystémique} a ainsi été suggérée comme une stratégie pour une gestion forestière durable \citep{Perry1998scientificbasis,Kuuluvainen2002Naturalvariabilitya}. 
Cette approche a pour but d'imiter les perturbations naturelles, les structures des peuplements et les dynamiques de succession qui en découlent. 
Elle s'inspire des processus naturels de la forêt, comme les ouvertures créées par les chablis, la mortalité partielle ou les perturbations à plus grande échelle comme les feux de forêt ou les épisodes cycliques de ravageurs \citep{Bergeron1999Forestmanagementa,Bergeron2007Usingknowledge}.
Elle permet également de préserver la biodiversité et d'entretenir la résilience des écosystèmes tout en assurant une variété de services écosystémiques \citep{Szaro1998emergenceecosystem,MacDicken2015Globalprogress}. 
\hl{Plus précisément, l'aménagement écosystémique utilise les traitements sylvicoles à l'échelle des peuplements, comme la coupe totale et la coupe partielle, pour reproduire les impacts écologiques associés aux perturbations naturelles. }
Toutefois, le niveau de perturbation varie selon le type de traitement appliqué, allant de perturbations plus légères à des perturbations plus intensives dans le milieu \citep{Thiffault2006Harvestingintensity}. 

Historiquement, les coupes totales sont parmi les pratiques sylvicoles les plus répandues dans les forêts tempérées et boréales \citep{Fedrowitz2014Canretention,Chaudhary2016Impactforest}. 
Elles font partie d’une gestion forestière permettant d’accroître la rentabilité des opérations forestières à relativement court terme, afin de répondre aux besoins des industries de transformation du bois \citep{Irland2011TimberProductivitya}. 
Sur le plan technique, elles sont relativement simples à réaliser, car l'intégralité du couvert forestier est retirée en une seule récolte, occasionnant une déforestation temporaire de la parcelle. 
Les coupes totales entraînent généralement une homogénéisation de la structure des peuplements \citep{Brashears2004AssessmentCanopy,Martin2020Forestmanagement}. 
\hl{Cette pratique peut ainsi être intégrée dans une approche de gestion écosystémique visant à imiter des perturbations naturelles de grande ampleur, telles que les feux de forêt, et à favoriser le maintien des stades précoces de la succession forestière} \citep{Bergeron1999Forestmanagementa}. 
Toutefois la simplification engendré par ce type de traitement peu parfois bouleverser des processus écologiques et évolutifs en amenant un changement drastique au niveau de la diversité biologique \citep{Holling2001UnderstandingComplexity}. 
Ainsi, l’utilisation de méthodes qui modifient radicalement la structure des milieux par rapport à leur état d’origine peut accroître le risque de déclin de certaines communautés biologiques \citep{Hanski2000Extinctiondebt}. 

Les coupes partielles sont un autre exemple d'\hl{aménagement écosystémique} visant à maintenir la composition, la structure et les fonctions de l'écosystème dans les limites de leur variabilité naturelle \citep{Bergeron1999Forestmanagementa,Raymond2009irregularshelterwood}.
Ce type de coupe implique une récolte sélective d'arbres, souvent orientée par des critères écologiques ou économiques spécifiques. 
Cela permet de conserver une couverture forestière partielle qui protège les semis et favorise une régénération naturelle \citep{Raymond2009irregularshelterwood}.
Les coupes partielles favorisent une structure forestière irrégulière en maintenant plusieurs classes d'âge ou de tailles d'arbres. 
Cette hétérogénéité peut inclure une diversité verticale et horizontale à l'intérieur des peuplements forestiers \citep{Raymond2009irregularshelterwood}.
\hl{Ainsi, les coupes partielles peuvent être appliquées afin de maintenir les stades de succession naturelle à l'intérieur des peuplements} \citep{Bergeron1999Forestmanagementa}. 
Contrairement aux coupes totales, les coupes partielles ont une période prolongée de régénération, 
permettant une gestion à long terme et l'intégration d'objectifs de conservation \citep{Raymond2009irregularshelterwood}. 
Les coupes partielles \hl{peuvent permettre} une meilleure conservation des espèces associées au milieu forestier en maintenant des attributs importants comme les débris ligneux, 
plusieurs strates de canopée, différentes tailles d'arbres et un microclimat favorable, 
ce qui est bénéfique pour les espèces sensibles aux variations climatiques ou à la perte d'habitat \citep{Hansen1991Conservingbiodiversity}. 


\section*{Faune du sol}
\label{sec:soilfauna}
\phantomsection\addcontentsline{toc}{section}{\nameref{sec:soilfauna}}

Au sein de la biodiversité forestière, la faune du sol joue un rôle écologique prédominant en contribuant à la circulation de la matière et de l'énergie à travers la chaine alimentaire et en participant au recyclage des nutriments \citep{Seibold2021contributioninsects,Kudrin2023metaanalysiseffects}. 
Toutefois, ces organismes ont souvent une capacité de dispersion limitée en raison de leur petite taille, ce qui les rend vulnérables face aux pertes locales d’habitats \citep{Kudrin2023metaanalysiseffects}. 
Cette communauté est ainsi identifiée comme l'une des plus touchées par les perturbations environnementales \citep{Marshall2000Impactsforest,Coyle2017Soilfauna}. 

Les coupes forestières amènent généralement des changements brusques et importants dans les habitats de la faune du sol \citep{Lindo2003Microbialbiomass,Paillet2010Biodiversitydifferences,Fedrowitz2014Canretention,Chaudhary2016Impactforest}. 
Par exemple, le retrait de la canopée augmente l'exposition de la surface du sol au rayonnement solaire, entraînant une hausse des températures ainsi qu'une modification de l'humidité du sol \citep{Lindo2003Microbialbiomass,Brook2008Synergiesextinction,Zhang2022Intensiveforest}. 
Ces effets sont amplifiés par une augmentation de la vitesse du vent et une intensification des précipitations atteignant le sol \citep{Keenan1993ecologicaleffects,Heithecker2007Edgerelatedgradients}. 
Les modifications de la structure du sol, telles qu'une compaction accrue due au passage des machines forestières, réduisent également la porosité du sol et affectent négativement la faune qui y vit \citep{Battigelli2004Shorttermimpact}. 
De plus, les opérations forestières perturbent la disponibilité des nutriments en modifiant la qualité et la quantité de la litière, en altérant les sécrétions racinaires, en favorisant le lessivage et en modifiant les propriétés chimiques du sol \citep{Covington1981Changesforest,Marshall2000Impactsforest,Lindo2003Microbialbiomass,Battigelli2004Shorttermimpact}. 
Ces bouleversements des conditions microclimatiques et physico-chimiques ont des répercussions directes sur la biodiversité, 
notamment sur les organismes dépendant de microhabitats tels que le bois mort, les cavités des arbres matures ou les plaques racinaires \citep{Spies1999Dynamicforest,Christensen2005Deadwood,Brassard2008EffectsForest}. 
Ces éléments structurels, souvent réduits ou éliminés par les opérations de coupe, sont généralement remplacés par des pistes de débardage et des chemins d'exploitation \citep{Hansen1991Conservingbiodiversity}. 

La présence de microhabitats joue pourtant un rôle essentiel pour une grande partie de la faune du sol, notamment les arthropodes, les amphibiens et les microorganismes \citep{Paillet2010Biodiversitydifferences,Fedrowitz2014Canretention,Chaudhary2016Impactforest}. 
La disparition de ces structures augmente la vulnérabilité de nombreuses espèces, tout en créant des conditions environnementales défavorables à leur survie. 
Par conséquent, les espèces forestières qui dépendent de la fraîcheur et de l'humidité caractéristiques des sols non perturbés peuvent voir leurs populations décliner, voire disparaître localement \citep{Kudrin2023metaanalysiseffects}. 
À l'inverse, les espèces adaptées aux milieux plus ouverts et secs peuvent coloniser les parcelles récemment coupées, entraînant ainsi une modification de la composition spécifique. 

Plusieurs études se sont intéressées aux changements des caractéristiques forestières comme le débris ligneux grossier, la profondeur de la litière, l'ouverture de la canopée, 
ainsi que sur leur impact sur la faune du sol après les opérations de récolte, dans le but d'orienter la gestion forestière \citep{Semlitsch2002CriticalElements,McKenny2006Effectsstructural}. 
Cependant, la faune du sol englobe une grande diversité de taxons, présentant des différences biologiques et écologiques significatives \citep{Kudrin2023metaanalysiseffects}. 
En conséquence, leur réponse à l'exploitation forestière varie selon le type de traitement appliqué et le groupe d'espèces étudiées \citep{Malmstrom2009Dynamicssoil,Paillet2010Biodiversitydifferences}. 

Les amphibiens et les arthropodes jouent un rôle important dans l'écologie forestière, principalement au niveau de la chaine trophique et dans la circulation du carbone au niveau des sols \citep{Burton1975Energyflow,loveiEcologyBehaviorGround1996,Handa2014Consequencesbiodiversity}. 
Malgré le fait qu'ils soient abondants dans les forêts nord-américaines, ces groupes ne sont pas souvent pris en compte dans les plans de gestion forestière \citep{deMaynadier1995relationshipforest}. 
Pourtant, les amphibiens et les arthropodes ont connu un fort déclin au cours des dernières décennies, principalement en raison des pratiques d'aménagement forestier \citep{Houlahan2000Quantitativeevidence,Stuart2004Statustrends,Wagner2021Insectdecline}. 
Leur sensibilité aux changements environnementaux et leur capacité de dispersion limitée en font des modèles d'étude pertinents pour mieux comprendre les effets des pratiques forestières 
sur la biodiversité et l'intégrité écologique des forêts \citep{pongeVerticalDistributionCollembola2000,Ojala2001Dispersalmicroarthropods,birdChangesSoilLitter2004,Maleque2009Arthropodsbioindicators}. 
Parmi les groupes d'espèces fréquemment étudiés chez ces taxons, on retrouve la salamandre cendrée de l'Est (\textit{Plethodon cinereus} (Green, 1818)), les carabes (Carabidae) et les collemboles (Collembola). 

\subsection*{Salamandre cendrée}

La salamandre cendrée, de la famille des Plethodontidae, représente l’une des biomasses les plus importantes chez les vertébrés des forêts de l’est de l’Amérique du Nord \citep{Burton1975Salamanderpopulations,Petranka1993Effectstimber,semlitschAbundanceBiomassProduction2014a}. 
En tant qu'espèce exclusivement terrestre et dépourvue de poumons, elle dépend de la respiration cutanée et donc des conditions d'humidité pour assurer ses échanges gazeux \citep{Heatwole1961Relationsubstrate}. 
Ce mode de respiration la contraint à occuper des microhabitats spécifiques, en surface lorsque la température et l'humidité sont favorables, ou en profondeur dans le sol durant des périodes moins propices \citep{Grizzell1949HibernationSite,FraserEmpiricalEvaluation1976,Jaeger1980MicrohabitatsTerrestrial}. 
Le retrait des refuges à la surface, tels que les débris ligneux, après la récolte peut réduire la qualité de l'habitat pour les salamandres et diminuer le temps qu'elles passent à la surface du sol forestier \citep{Achat2015Quantifyingconsequences,Peele2017Effectswoody}. 
Étant donné que les salamandres cendrées se nourrissent et se reproduisent à la surface du sol forestier, des conditions de surface défavorables, un compactage des sols et de faibles niveaux de débris ligneux peuvent affecter la dynamique des populations \citep{Peterman2014Spatialvariation}. 
Prédateur généraliste, la salamandre cendrée contribue de manière significative à la régulation des invertébrés détritivores, influençant directement les processus de décomposition, la circulation des nutriments dans les sols et la dynamique du carbone \citep{Burton1975Energyflow,Wyman1998Experimentalassessment,Walton2013Topdownregulation,Hickerson2017Easternredbacked}. 
Par ailleurs, elle constitue une proie de haute valeur nutritive pour divers prédateurs tels que les oiseaux, les mammifères et les reptiles, renforçant ainsi son rôle dans les dynamiques trophiques forestières \citep{Burton1975Energyflow,Pough1987abundancesalamanders}. 
Compte tenu de sa sensibilité aux perturbations environnementales, notamment en raison de sa respiration cutanée, la salamandre cendrée est souvent utilisée comme indicateur de la qualité des sols forestiers \citep{Welsh2001caseusing,Fisher-Reid2024Easternredbacked}. 
Son abondance et sa présence sont des paramètres couramment évalués pour mesurer l'impact des pratiques sylvicoles sur la biodiversité \citep{Harpole1999Effectsseven,Grialou2000effectsforest,Homyack2009Longtermeffects,Hocking2013Effectsexperimental,Mazerolle2021Woodlandsalamander}. 

\subsection*{Carabes}

\hl{Nous utilisons les carabes comme deuxième organisme modèle en raison de la documentation abondante de leur taxonomie et leur écologie} \citep{loveiEcologyBehaviorGround1996}. 
Ces insectes ont une courte durée de vie, occupent une position élevée dans la chaine alimentaire et réagissent rapidement et de manière complexe aux modifications de leur environnement \citep{loveiEcologyBehaviorGround1996}.
Les carabes jouent un rôle écologique dans la régulation des populations d'invertébrés, se nourrissant principalement d'aphides, de collemboles et de gastéropodes, tout en étant à leur tour des proies pour divers amphibiens, reptiles, oiseaux et mammifères \citep{loveiEcologyBehaviorGround1996}. 
Les carabes sont particulièrement sensibles à la complexité structurelle des peuplements forestiers, et ce, à différentes échelles temporelles et spatiales \citep{Butterfield1995Carabidbeetle,loveiEcologyBehaviorGround1996,Niemela2007effectsforestry}.
Avec environ 40 000 espèces répertoriées, les carabes constituent l'une des familles les plus diversifiées parmi les coléoptères et représentent l'une des plus fortes abondances d'arthropodes du sol \citep{Erwin1985taxonpulse,loveiEcologyBehaviorGround1996,Rochefort2006GroundBeetle}. 
Ils sont largement distribués et abondants dans presque tous les écosystèmes terrestres, bien que les espèces varient dans leur sélection d'habitats \citep{loveiEcologyBehaviorGround1996,kotzeFortyYearsCarabid2011a,Larochelle2003naturalhistory}. 
Les changements environnementaux et les modifications d'habitat favorisent certaines espèces au détriment d'autres, ce qui rend leur diversité et leur sensibilité particulièrement intéressantes pour étudier l'impact des perturbations environnementales \citep{Rainio2003Groundbeetles}.

\subsection*{Collemboles}

Parmi les taxons de la mésofaune du sol, les collemboles représentent un des groupes les plus abondants et les plus diversifiés de petits arthropodes \citep{rusekBiodiversityCollembolaTheir1998}. 
Différentes communautés de collemboles occupent un ensemble de niches écologiques allant de la litière aux différents horizons du sol \citep{pongeVerticalDistributionCollembola2000}.
La répartition verticale de ces communautés dépend essentiellement des conditions abiotiques du sol telles que la luminosité, le taux d’humidité ou encore la porosité.
Ces invertébrés exercent une influence sur l'écologie forestière à travers leur influence sur les micro-organismes, sur la décomposition et le cycle des nutriments.
Étant principalement fongivores et détritivores, les différentes communautés de collemboles influencent les taux de décomposition du bois, affectent la structure physique et les taux de minéralisation de la litière, 
influencent l'absorption des nutriments et régulent la communauté microbienne, en plus de participer à la formation de microstructure du sol \citep{Petersen1982comparativeanalysis,Neher2012Linkinginvertebrate,Maass2015Functionalrole,Potapov2016Connectingtaxonomy}. 
Les collemboles participent également à la chaine trophique en constituant une ressource alimentaire abondante pour plusieurs organismes comme les amphibiens, les coléoptères, les arachnides, les oiseaux et les reptiles \citep{Burton1975Energyflow,Bauer1982Predationcarabid,rusekBiodiversityCollembolaTheir1998}.

\section*{Objectifs et hypothèses}
\label{sec:objectifs}
\phantomsection\addcontentsline{toc}{section}{\nameref{sec:objectifs}}

Grâce à leurs relations trophiques, leur sensibilité aux variations des conditions environnementales et leur dépendance aux caractéristiques forestières, ces trois groupes d'espèces constituent des modèles pertinents pour étudier l'impact des pratiques forestières sur la faune du sol. 
La plupart des travaux qui se sont intéressés aux impacts des traitements sylvicoles sur la faune discutent généralement des effets directs des perturbations sur un ou plusieurs groupes d'espèces, 
sans tenir compte des relations existantes entre les variables environnementales et les différents groupes d'espèces, 
négligeant ainsi les effets indirects des coupes sur la faune du sol \citep{josephIntegratingOccupancyModels2016,Pollierer2021Diversityfunctional,Kudrin2023metaanalysiseffects}. 
\hl{Mon projet a pour but de combler cette lacune en essayant de comprendre comment les effets à court terme de différents types de récolte forestière se propagent à l’intérieur du réseau écologique forestier et influencent la dynamique de la faune du sol. }
Ultimement, ces nouvelles connaissances permettront de mieux informer la gestion durable des forêts.
Les objectifs liés à cette étude sont :

\begin{enumerate}
    \item De quantifier l'effet des traitements de coupes forestières sur les variables environnementales qui influencent l'utilisation de l'habitat par la faune du sol. 
    \item De mesurer l'impact des coupes forestières sur l'utilisation de l'habitat par la faune du sol.
\end{enumerate}

L’hypothèse 1.1 attachée à notre premier objectif soutient que les variables environnementales (volume de débris ligneux, ouverture de la canopée, profondeur de litière) favorables à l’utilisation de l’habitat par les espèces varient en fonction de l’intensité des coupes forestières. 
En ce sens, l’intensité du traitement de coupe est utilisée comme indicateur global représentant les changements dans les conditions environnementales.

L'hypothèse 2.1 liée à notre deuxième objectif stipule que les traitements de coupe forestière entraînent une modification de l'utilisation de l'habitat 
par la faune du sol et que ces effets se propagent à travers le réseau trophique, allant des prédateurs vers les proies. 
Spécifiquement, les coupes affectent l'utilisation de l'habitat par les salamandres et les grands carabes (compétiteurs de salamandres), 
ce qui modifie ensuite la sélection d'habitats des petits carabes (proies de salamandres), puis enfin des collemboles (proies de grands carabes et salamandres). 
Cette hypothèse repose sur le fait que la salamandre cendrée, en tant que prédateur sensible aux perturbations, est reconnue pour son rôle important dans la régulation des communautés d'invertébrés du sol forestier \citep{Wyman1998Experimentalassessment,MichaelWalton2005Salamandersforestfloor,Walton2006Salamandersforestfloor,Walton2013Topdownregulation,Hickerson2017Easternredbacked}. 
Parallèlement, les carabes ont une relation complexe avec les salamandres, se positionnant à la fois comme compétiteurs et proies de celles-ci \citep{Jaeger1980MicrohabitatsTerrestrial,loveiEcologyBehaviorGround1996,Gall2003BehavioralInteractions}. 


\cleardoublepage

\bibliography{References.bib}
\bibliographystyle{ecologyNewFR.bst}
