\chapter*{Introduction générale}         % ne pas numéroter
\label{chap-introduction}       % étiquette pour renvois
\phantomsection\addcontentsline{toc}{chapter}{\nameref{chap-introduction}} % inclure dans TdM

\section*{Mise en contexte}
\label{sec:contexte}
\phantomsection\addcontentsline{toc}{section}{\nameref{sec:contexte}}

\section*{Objectifs et hypothèses}
\label{sec:objectifs}
\phantomsection\addcontentsline{toc}{section}{\nameref{sec:objectifs}}

Notre étude vise principalement à comprendre de façon intégrative l'impacte de la préparation de sites en forêt mixte sur la dynamique des écosystème du sol forestier, dans un contexte de migration assistée d'arbres.
Notre objectif est de dévelloper un modèle d'équations structurelles intégrant des modèles d'occupations pour quantifier les impactes directes et indirectes des traitements de coupes forestières (coupe totale, coupe partielle) sur : 

\begin{enumerate}
    \item Les attributs forestiers propices à la sélection d'habitats par la faune du sol tel que le volume de débris ligneux, la profondeur de litière ainsi que l'ouverture de la canopée. 
    \item Le relations de cooccurences entre la salamandres cendrée de l'Est (\textit{Plethodon cinereus}), les carabes (Carabidae) et les collemboles (Collembola).
\end{enumerate}

L'hypothèse 1.1 associé à notre objectif soutient que les traitements de coupes forestières influence directement les attributs environnementaux des forêts en 
entrainant une réduction de la profondeur de litière ainsi qu'une baisse le volume de débris ligneux, suite à diminution de recrutement de matériel. 
À l'inverse, une coupe forestière plus intensive entrainera une hausse de l'ouverture de la canopée.

L'hypothèse 1.2 liée à notre objectif atteste que les traitements de coupes forestières entraine par un effet cascade un changement dans l'utilisation de l'habitat par la faune du sol. 
En premier lieu, l'influences des coupes forestières modifient la sélection d'habitat du haut de la chaîne trophique (salamandres, carabes de grande taille)
puis entraînent un effet en cascade qui influence ensuite les niveaux trophiques inférieurs tels que les petits carabes, 
et enfin, les collemboles.

\cleardoublepage

\bibliography{references.bib}
\bibliographystyle{ecologyNewFR.bst}
