\chapter*{Introduction générale}         % ne pas numéroter
\label{chap-introduction}       % étiquette pour renvois
\phantomsection\addcontentsline{toc}{chapter}{\nameref{chap-introduction}} % inclure dans TdM

\section*{Mise en contexte}
\label{sec:contexte}
\phantomsection\addcontentsline{toc}{section}{\nameref{sec:contexte}}


\section*{Migration assitée et préparation de sites}
\label{sec:fam}
\phantomsection\addcontentsline{toc}{section}{\nameref{sec:fam}}


\section*{Espèces à l'étude}
\label{sec:species}
\phantomsection\addcontentsline{toc}{section}{\nameref{sec:species}}


\section*{Objectifs et hypothèses}
\label{sec:objectifs}
\phantomsection\addcontentsline{toc}{section}{\nameref{sec:objectifs}}



Notre étude vise à comprendre de façon intégrative l'impact de la préparation de sites, en vue d'une migration assistée d'arbres en forêt mixte, sur la dynamique des écosystèmes du sol forestier.
L'objectif est de quantifier les impacts des traitements de coupes forestières sur la faune du sol et son habitat. 
Le moyen d'atteindre cet objectif est de construire un modèle un modèle d'équations structurelles pour mesurer les effets directs et indirects des traitements de coupes coupe totale et de coupe partielle sur : 

\begin{enumerate}
    \item Le relations de cooccurences entre la salamandres cendrée de l'Est (\textit{Plethodon cinereus}), les carabes (Carabidae) et les collemboles (Collembola).
    \item Les attributs forestiers propices à la sélection d'habitats par la faune du sol tel que le volume de débris ligneux, la profondeur de litière ainsi que l'ouverture de la canopée. 
\end{enumerate}

L'hypothèse 1.1 liée à notre objectif atteste que les traitements de coupes forestières entraînent un changement dans l'utilisation de l'habitat par la faune du sol en se propageant via le réseau trophique. 
Les coupes affectent en premier lieu l'utilisation d'habitat des salamandres et des carabes de grandes tailles et modifie par la suite la sélection d'habitat de carabes de petite taille puis enfin des collemboles. 

L'hypothèse 1.2 associé à notre objectif soutient que les traitements de coupes forestières influencent directement les attributs environnementaux des forêts en 
réduisant la profondeur de litière ainsi qu'en diminuant le volume de débris ligneux, suite à une baisse de recrutement de feuilles et de débris ligneux au sol. 
À l'inverse, nous pensons qu'une coupe forestière plus intensive amène une hausse dans l'ouverture de la canopée.


\cleardoublepage

\bibliography{References.bib}
\bibliographystyle{ecologyNewFR.bst}
