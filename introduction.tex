\chapter*{Introduction générale}         % ne pas numéroter
\label{chap-introduction}       % étiquette pour renvois
\phantomsection\addcontentsline{toc}{chapter}{\nameref{chap-introduction}} % inclure dans TdM

% \usepackage[french]{babel}

\section*{Mise en contexte}
\label{sec:contexte}
\phantomsection\addcontentsline{toc}{section}{\nameref{sec:contexte}}

%1 importance des forêt

Les écosystème forestiers jouent un rôle essentiel dans la biosphère à travers leur rôle économique et leur valeur écosystèmiques en fournissant une multitude de de produit et services \citep{Balvanera2006Quantifyingevidence}. 
Leur présence permet de réguler les flux de nutriments et d'énergie, notamment à travers la séquestration du carbone, la régulation du climat et la rétention de l'eau \citep{Balvanera2006Quantifyingevidence,Canadell2008Managingforests,Pawson2013Plantationforests}. 
De plus, les forêts agissent comme d'important reservoir de biodiversité essentiels à la stabilité environnementale \citep{Diaz2006BiodiversityLoss}. 

La diversité des espèces est désormais largement considérée comme essentielle au bon fonctionnement des écosystèmes, et les engagements internationaux récents ont mis en avant l'urgence d'arrêter la perte de biodiversité et d'encourager une gestion durable des forêts \citep{Scherer-Lorenzen2005ForestDiversity,Parviainen2007Maintenanceconservation}. 
Cependant préserver la biodiversité forestière tout en répondant aux exigences économiques constitue un défi majeur pour les pays disposant de ressources forestières.
Cependant, la gestion forestière axée sur la production de bois pour répondre aux besoins industriels participe au déclin nombreuses espèces qui dépendent des habitats forestiers naturels \citep{Bengtsson2000Biodiversitydisturbances}. 
Les forêts naturelles possèdent des caractéristiques distinctives, telles qu'une grande quantité de bois mort et d'arbres en décomposition, des arbres vieux et de grande taille, ainsi que des fosses et monticules autour des racines \citep{Spies1999Dynamicforest}. 
En revanche, les forêts exploitées sont souvent marqués par des perturbations fréquentes avec une faible variabilité dans la taille de celles-ci. 
Ce type de forêts se charactérise aussi par une plus grande homogénéité dans la composition d'espèces d'arbres, la stratification verticale, la structure d'age, dynamique de succession ainsi qu'un manque de phase sénescentes \citep{Commarmot2005Structurevirgin}.


Ces altérations influencent les conditions environnementales locales, notamment la température, l'humidité, la lumière et les caractéristiques du sol et de la litière. 
Il a été démontré que les coupes forestières influencent les propriétés du sol en réduisant la teneur en carbone et en augmentant de manière significative la concentration de NO3-N, le taux de nitrification, le pH ainsi que les flux de N2O \citep{Jerabkova2011metaanalysiseffects,Michal2014Responsessmall,James2016effectharvest,Zhang2022Intensiveforest}.

En conséquence, cela peut entraîner des modifications ou la disparition complète de microhabitats essentiels (comme le bois mort, les cavités ou les arbres matures) qui sont des refuges pour la biodiversité forestière \citep{Paillet2010Biodiversitydifferences}. 
Dans une même région, différents groupes taxonomiques peuvent réagir de manière différente aux opérations forestières en fonction de leur taille, mobilité ou régime alimentaire \citep{Barlow2007Quantifyingbiodiversity,Stork2009Vulnerabilityresilience}.

De plus, certains types de gestion forestière peuvent avoir des répercussions plus importantes que d'autres sur les espèces forestières, en raison de différences dans la structure de l'habitat, la continuité ou les conditions microclimatiques après la coupe. 
Certains régimes de gestion peuvent également amplifier des impacts indirects sur la biodiversité, tels que l'augmentation de la chasse ou la fréquence des incendies.


    % exemple d'effet des coupes   

Ces changement dans les attribut forestiers participe à perte la biodiversité suite à une réduction de la diversité spécifique et fonctionnelle à différents niveaux trophiques.
Sur le long terme, cela peut entrainer une diminution de la résilience des forêt à l'échelle locale et engendrer une diminution des services écosystémiques et avoir un impacte négatif pour les sociétés \citep{Hooper2012globalsynthesis,Edwards2014Maintainingecosystem}.

    % effet des changement climatiques

Depuis plusieurs décennies, la gestion écosystémique intégrée aux plans d’aménagement forestier cherche à atténuer les impacts des perturbations liées à l’exploitation forestière sur les écosystèmes.
Cette gestion privilégie le maintien des attributs environnementaux des forêts, de leur biodiversité, ainsi que des services écologiques associés, tout en répondant aux exigences de productivité des industries forestières \citep{Perry1998scientificbasis,Szaro1998emergenceecosystem,Kuuluvainen2009Forestmanagement,MacDicken2015Globalprogress}.


L

peux modifier l'abondance et la diversité specifique

Species richness was higher in unmanaged than in managed forests, as indicated by the negative grand mean calculated for the entire data set : paillet
Taxonomic and ecological groups displayed contrasting responses to forest management (Table 2). Globally, the absolute values of effect size fell between 0.4 and 0.7 (except for carabids), which corresponds to a medium intensity effect (Cohen 1969)
All the other groups exhibited higher species richness in the unmanaged forests, but the results were only significant for fungi, lichens, carabids, and saproxylic beetles. :paillet

Several mechanisms may explain the effect of management on forest biodiversity: changes in tree age structure, vertical stratification, and composition of tree species, which affect light, temperature, moisture, litter, and topsoil conditions (Sebastia et al. 2005; Standovar et al. 2006); presence of microhabitats (e.g., dead wood, veteran trees, cavities, root plates) specific to unmanaged (Berg et al. 1994; Bouget 2005a; Christensen et al. 2005; Gibb et al. 2005) or managed forests (e.g., skid trails and haul roads) (Hansen et al. 1991; Gosselin 2004); and forest cover continuity and features resulting from extensive management in the past (Hjalten et al. 2007). The pattern of response may therefore depend on which of the above mechanisms, or which combinations of them, have the strongest effects on different taxonomic or functional groups. paillet

%3 Effet des coupes

La réponse de la biodiversité aux coupes forestières varie en fonction des milieux, des traitements appliqués et des groupes d'espèces étudiés \citep{Kudrin2023metaanalysiseffects}.
Certaines espèces réagissent positivement aux coupes, mais la plupart du temps, ces interventions entraînent un déclin de la faune. 
Les études montrent souvent une diminution de la richesse spécifique après des coupes à blanc, ou une stabilité dans le cas des coupes partielles.

Les coupes forestières provoquent la perte d'habitat, perturbant l'accès des espèces à la nourriture, aux refuges et aux zones de reproduction, et altérant les chaînes alimentaires. 
Elles entraînent également une fragmentation des habitats, compliquant les déplacements et la reproduction, et réduisant la diversité génétique.

En perturbant les corridors écologiques, les coupes limitent la mobilité des animaux, augmentant ainsi le risque d'extinction locale. 
De plus, l'abattage modifie le microclimat forestier, rendant les conditions moins favorables pour certaines espèces. 
Les animaux peuvent également subir des impacts directs, comme la destruction de leurs habitats par les machines d'abattage, affectant particulièrement les espèces peu mobiles.

Enfin, en plus des impacts des pratiques forestières, les changements climatiques compliquent la gestion des forêts, accentuant parfois les perturbations causées par l'exploitation sylvicole. 
L'effet des coupes sur la faune du sol dépend des groupes taxonomiques, des types de traitement et du type de forêt.

Chaudhary : Second, within the same region, different taxonomic groups may respond in different ways to forestry operations due to variation in, for example, body size, mobility, and diet 12,13.
At an extreme end of the management intensity range is clear-cutting, which results in temporary deforestation of a previously forested area
Clear-cutting is historically the most common example of even-aged  silviculture practice in temperate and boreal biomes15. It is technically easy to execute, as the entire stand overstorey is removed in one harvest. 
Clear-cutting has been criticized for simplifying forest structure and reducing  biological diversity, leading to homogeneous forests16 (but see Greenberg et al. for exceptions, where clear-cutting  is found to mimic high intensity natural disturbance regimes)17
In recent decades, silvicultural practices that combine timber harvesting and biodiversity preservation have been promoted to mitigate the impacts of clear-cuts19 
In recent decades, silvicultural practices that combine timber harvesting and biodiversity preservation have been promoted to mitigate the impacts of clear-cuts19. 
This has led to other variations of even-aged silviculture, in which individuals (dispersed retention) or groups of trees (aggregated retention) are left on-site to maintain structural diversity (such as patch-cut or green tree retention systems), supply seeds for the next crop (seed tree retention) or to protect the regenerating understorey (shelterwood  system)18,19. 
Ideally, this category will be subdivided in the future according to the amount and configuration of trees retained, to assess the impact of individual retention practices.
coupe total réduit la richesse specifique pour l'ensensemble des espèces, mais a un effet opitif dsur le mammifaire.
coupe partielle n'ont pas eu d'iumpacte sur la richesse specifique pour l'ensemble des espèce ( oiseau, ), effet positif pour les mammifaire.
pour les arthropode la coupe partiellea un impâctes négatif sur la biodiversité, mais la coupe total n'a pas d'effet significatif sur la richesse specifique
coupe total a un effet negatif sur les amphibian
All types of management change some properties of the forest, such as tree age structure, microclimate, or soil conditions
selection and retention systems in temperate and boreal regions, reduced impact selective logging (RIL) followed by conventional selective logging in tropical regions, clear-cutting in temperate regions, and timber plantations
Management regimes that mimic small-scale disturbance such as selection harvest, retention, and RIL, result in increased environmental heterogeneity compared to clear cutting, timber plantations or high intensity conventional logging18,39. Selection and retention systems appear to be similarly good at maintaining high species richness in managed forests.
The retention trees, areas, and unharvested portions of selections systems provide late successional structures that are more likely to maintain high diversity.

%4 défis des changement climatique des et des solutions comme la migration assistée

Margenau 2023 : Forest management in the twenty-first century seeks to balance multiple ecological goals, including promoting biodiversity and maintaining ecosystem function postharvest.

Cependant, le défi consiste à gérer la capacité régénératrice de la forêt de manière à produire des bénéfices maintenant sans compromettre les avantages et les choix futurs.

Les coupes forestières ne sont pas la seul raison pour expliquer le déclin de la bidiversité dans les milieux forestio
Le defis des changements climatiques a travers tout ça et l'adaptation par la migration assistée

    la réalité des changements climatiques vient expliquer une raison du déclin de la biodiversité et s'ajoute au défis de repenser les pratiques forestières pour préserver les écostème tout en f
    De plus, l'equation pour adapter les pratiques forestières afin de préservé les écosystème doit tenir compte de la réalité des changement climatique qui vient influence non seulement la survie et la distributions des espèces mais oblige aussi 
    En plus de devoir répondre aux besoins de préservéer les écosystèmes les nouvelles pratique forestières doivent tenir compte de la réalité des changements climatiques.
    La hausse de température menace la pérénité des forêts et la survie de certains peuplement incapable de se déplacer suffisemment vites
    les pratique forestière on besoinde répondre a se défis pour préservé la valeurs écologique et économique des arbres.
    Dans ca gestion des foret, elle doit tenir compte de l'effet cumuler de l'amenagement et des changement climatique qui peuvent se chevauché et avoir un impacte beaucoup plus grand sur la biodiversité.

    Ajouté à cela, les milieu forestiers et la biodiversité font actuellement face à la réaliter des changement climatique qui 

Cette problématique est d'autant plus importante si l'on tient compte des changements climatique. 
En effet, la hausse de température associé aux changements climatiques vient modifier les milieux naturels en ....

% 3. **Impact accrue par les changements climatiques et de la migration assistée sur les écosystèmes forestiers**
%    - Impact du changement climatique sur les écosystèmes forestiers : altération des régimes de précipitation, augmentation des températures, stress hydrique accru.
%    - Adaptation des forêts au changement climatique par des stratégies telles que la migration assistée, un outil utilisé pour aider les espèces forestières à migrer vers des zones climatiquement plus adaptées à leur survie.
%    - Modification de la composition des espèces végétales en réponse au réchauffement climatique et aux projets de migration assistée.
%    - Perturbations des interactions écologiques, y compris celles des espèces locales, avec des effets potentiels sur la structure des communautés, la compétition pour les ressources et l'équilibre des écosystèmes.
%    - Nécessité de pratiques de gestion forestière durables qui intègrent à la fois la migration assistée et la préservation de la biodiversité.

Aujourd'hui le pratique forestière le praitque forestière tiennent de plus en plus compte des besoin de minimiser l'impacte des coupes sur la biodiversité et les écosystèmes.
Se besoin est d'autant plus nécesaaire dans le contexte climatique actuelle ou la hausse de température globale vient ajouter un stress sur les ecosystèeme terrestre en menacant la pérénité des forets et de la biodiversité

POurtant la foresterie fait aujourd'hui face a de nombreux défis tel que les changements climatique et doit repenser sa pratiques
compound effect entre changement climatique et foresterie peux exacerber encore plus la biodiversité

Cependant, les changements climatiques représentent aujourd'hui un des défis les plus préoccupants pour le maintien des forêts tel qu'on les connait \citep{McKenney2009Climatechange,Trumbore2015Foresthealth,Seidl2017Forestdisturbances,Messier2022Warningnatural}.  
Malgré les engagements internationaux pris au courant des dernières décennies pour diminuer la hausse de température lié aux activités anthropiques, 
les projections climatiques actuelles nous indiquent une hausse future de température globale dépassant 1.5\up{o}C par rapport à l'air préindustriel \citep{Matthews2022Currentglobal}.
En raison de ses latitudes nordiques, le Canada est particulièrement vulnérable à l'augmentation des températures et aux perturbations environnementales qui en découlent \citep{Alo2008Potentialfuture,Bush2019Canadachanging}. 

Les forêts de l'Est de l'Amérique du Nord sont et seront donc ainsi particulièrement touchées par cette hausse de température \citep{Park2014Canboreal,Mahony2017closerlook,Sittaro2017Treerange,Messier2022Warningnatural}.
Parmi les conséquences attendues, on prévoit un allongement et une intensification des périodes de sècheresse, une augmentation du nombre de feux de forêt et une hausse des perturbations biotiques \citep{Parmesan2007Influencesspecies,Joyce2013Climatechange,Gatti2021Amazoniacarbon,Heidari2021Effectsclimate}. 
À cela s'ajoutent des changements dans la phénologie \citep{Chuine2010Whydoes} ainsi que dans la distribution des arbres \citep{Zhu2012Failuremigrate,Gray2013Trackingsuitable} due au manque d'adaptation des végétaux aux nouvelles conditions de leur milieu \citep{Aitken2008Adaptationmigration}.
Les changements climatiques se produisent toutefois plus rapidement que la capacité d'adaptation ou de déplacement des arbres \citep{Aitken2008Adaptationmigration,Loarie2009velocityclimate,Vitt2010Assistedmigration,Harrison2020Plantcommunity}, 
menaçant ainsi leur capacité à se maintenir dans des milieux favorables à leur survie et leur croissance \citep{Zhu2012Failuremigrate,Sittaro2017Treerange,Woodall2018Decadalchanges}.
Des changements pourraient ainsi être observés dans la composition des forêts menant à des modifications dans la conduite de l'aménagement forestier et de la conservation \citep{McKenney2009Climatechange,Chmura2011Forestresponses,Lo2011Linkingclimate}.


\section*{Migration assistée}
\label{sec:fam}
\phantomsection\addcontentsline{toc}{section}{\nameref{sec:fam}}

Plusieurs appels à adapter l'aménagement ont été proposés afin de préserver les milieux forestiers et leurs bienfaits \citep{Nagel2017Adaptivesilviculture,Messier2021sakeresilience}.
Différentes mesures d'atténuation ont été proposées pour prévenir la perte des forêts et améliorer la résilience de celle-ci comme l'augmentation de la diversité fonctionnelle et l'amélioration de la connectivité à travers le paysage forestier \citep{Messier2019functionalcomplex}.
Parmi toutes les solutions envisagées, la migration assistée d'arbres est proposée comme une mesure d'atténuation permettant le déplacement d'individus ou de matériel génétique depuis un territoire climatique originel vers une zone de climat futur plus propice à la croissance des arbres \citep{Vitt2010Assistedmigration,Dumroese2015Considerationsrestoring,Park2018Informationunderload,Park2023Provenancetrials}. 
La migration assistée d'arbre permettrait de modifier rapidement la composition des peuplements pour mieux convenir au climat futur de celui-ci \citep{Pedlar2011implementationassisted} 
répondant ainsi aux besoins de conservation, maintenant les services écosystémiques et préservant la valeur économique. \citep{Ste-Marie2011Assistedmigration,Winder2011Ecologicalimplications}.
Un manque de connaissances et un degré d'incertitude subsistent toutefois autour de la migration assistée \citep{Klenk2015assistedmigration,Park2018Informationunderload}. 
La principale préoccupation porte sur le compromis entre la préservation d'une espèce et les risques pour l'écosystème du territoire hôte \citep{Ricciardi2009Assistedcolonization}, 
tels que l'introduction d'espèces invasives ou la perte de diversité génétique pour des adaptations locales \citep{McLachlan2007frameworkdebate,Vitt2010Assistedmigration,Hewitt2011Takingstock,VanDaele2022Genomicanalyses}.

Afin d'améliorer nos connaissances et ainsi réduire l'incertitude, divers scénarios sylvicoles sont à l'heure actuelle étudiés pour limiter les risques associés à la migration assistée \citep{royoDesiredREgenerationAssisted2023}.
Les traitements sylvicoles sont en général utilisés pour influencer la croissance, la santé et la composition des peuplements forestiers.
Parmi ces traitements, les coupes totales et les coupes partielles font partie des opérations couramment utilisées en foresterie \citep{Man2008Elevenyearresponses,Chaudhary2016Impactforest,MontoroGirona2018ConiferRegeneration,Ameray2021Forestcarbon}. 
La coupe totale implique l'abattage de tous les arbres dans une zone déterminée.
Elles sont le plus souvent employées dans le cadre d'un plan d'aménagement intensif visant à accroître la productivité et la qualité du bois 
sur un court intervalle de temps afin de répondre aux besoins de l'industrie et d'optimiser les bénéfices \citep{Ameray2021Forestcarbon}.
Pour leur part, les coupes partielles consistuent une suppression sélective d'arbres tout en laissant une partie du peuplement intacte.
Ce type de coupe est habituellement utilisé dans le cadre d'un plan d'aménagement extensif qui favorise la régénération naturelle et imite les perturbations naturelles.
Les coupes partielles sont souvent utilisées pour favoriser la croissance des arbres les plus vigoureux, pour encourager la diversité spécifique ou pour maintenir une canopée ouverte \citep{Irland2011Timberproductivity,Ameray2021Forestcarbon}.
La conservation d'arbres en coupe partielle et l'allongement de la période de rotation offrent d'autres avantages sur le plan économique et écosystémique. 
Cela favorise la séquestration du carbone, préserve l'apport de matière organique ainsi que le cycle de nutriment, 
tout en maintenant différentes niches écologiques pour la faune \citep{Barg1999Influencepartial,Tong2020Forestmanagement,Ameray2021Forestcarbon}.

L'utilisation de coupes forestières peut toutefois engendrer des modifications dans les conditions environnementales des forêts, 
telles que l'augmentation de l'exposition au rayonnement solaire, la hausse de vitesse des vents et la réception intensifiée de précipitations au sol en réponse à l'élimination de la canopée, 
menant à un accroissement de l'ensoleillement, de la température et de l'humidité du sol \citep{Keenan1993ecologicaleffects,Lindo2003Microbialbiomass,Heithecker2007Edgerelatedgradients}.
De plus, les coupes peuvent affecter la disponibilité en nutriment et augmenter le degré de compaction au niveau des sols \citep{Covington1981Changesforest,Lindo2003Microbialbiomass,Battigelli2004Shorttermimpact,rousseauLongtermEffectsBiomass2018}. 
À terme, les changements environnementaux résultant d'une coupe peuvent affecter la biodiversité. \citep{Paillet2010Biodiversitydifferences,Fedrowitz2014Canretention,Chaudhary2016Impactforest}, 
tout particulièrement sur la faune vivant à la hauteur du sol \citep{Lindo2003Microbialbiomass,Chaudhary2016Impactforest,Kudrin2023metaanalysiseffects}.


%5 Focus sur la faune du sol : une composante négligée de la biodiversité

% - Présentation de la faune du sol (invertébrés, micro-organismes) et de leur rôle essentiel dans le fonctionnement des écosystèmes forestiers : décomposition, transformation des nutriments, maintien de la fertilité des sols.
% - Importance accrue de ces organismes face aux perturbations dues aux coupes forestières, au changement climatique et à la migration assistée, notamment à cause des variations des conditions microclimatiques et des ressources disponibles.

Soil fauna comprises a diverse array of taxa that exhibit significant variations in biology and ecology. As a result, their response to forest harvesting can be highly heterogeneous [13] kudrin

However, existing research on this topic has primarily focused on one or two large taxa and rarely includes multiple groups [14,15] kudrin

Consequently, our comprehension of the differences in the responses of various soil fauna groups to logging and the reasons for these differences is limited. kudrin

Clearcutting, which is historically the most common example of even-aged silviculture practice in temperate and boreal biomes [11], may result in significant changes in environmental conditions. This includes altered light, humidity, wind speed, and other conditions which can constrain forest biota, e.g., [16].

Partial cutting or retention forestry is another practice in which some parts of the trees are left on-site to maintain organic matter inputs and nutrient cycles [17] and provide a refuge for belowground organisms [18]. The response of soil fauna to harvesting may differ depending on the practice used [9–11]. However, we still have a poor understanding of the differences in the response of individual groups of soil fauna to harvesting practices. The forest type is another important factor that can modify the impact of forest harvesting on soil fauna. Coniferous and deciduous forests, for example, are quite different from each other in terms of soil and microclimatic conditions. This is reflected in the dissimilarity of the composition and structure of soil fauna [14,19,20].

In general, forest harvesting significantly decreases the abundance of soil fauna by 17\% in comparison to the control (Figure 2) but does not change its richness. kudrin

du a leur taille la petites faune a une capacité de dispertion plus limité, la méttant plus èa risque face au perturbation du milieu
l'effet cumuler des perturbation du milieu et des changement de condition environnementale peuvent d'autant plus exacerber la biodiversité
 effects of forest harvesting on the occupancy of various groups of soil fauna

La faune constitue une composante essentielle de la biota du sol et joue un rôle clé dans les écosystèmes forestiers en assurant les flux de matière et d'énergie à travers le réseau trophique et en recyclant les nutriments. 
Il est donc impératif de comprendre comment l'exploitation forestière influence la faune du sol.

l'importance de suivre la faune du sol 
    Cette gestion forestière aura un impacte significatif sur la faune du sol.
    La faune du sol regroupe un ensemble d'espèces souvent peu étudier dans les études d'impact mais joue pourtant un rôle primordiale dans les mécanismes des écosystèmes forestiers
    Elle joue un rôle dans mise en circulation des nutriments, dans l'alimentations des espèces plus grandes et représente souvent une importante biomasse.

The meta-regression analysis shows that the types of harvesting and forest can modify the soil fauna responses to forest harvesting. Among the groups included in the analysis, Collembola, Oribatida, Coleoptera, and Araneae changed their response to harvesting (Table 1). The abundance of Collembola and Coleoptera was negatively affected by clear-cutting but not by partial cutting (Figure 4). The richness of these groups positively responded to clear-cutting, but in the case of Collembola, such an effect was not statistically significant. Partial cutting leads to an increase in Collembola richness but not in Coleoptera (Figure 4). Collembola and Oribatida decreased their abundance after harvesting in coniferous forests, while in deciduous and mixed forests, they did not change (Figure 5) kudrin

Forest harvesting is traditionally believed to have significant negative impacts on soil biota, through fragmentation of vegetation, modification of the quality and quantity of litter, alteration of root exudates, leaching of some plant nutrients, and changes in the microclimate and chemical properties of soil [12]. kudrin

Confirming these negative impacts, our meta-analysis showed that, in general, harvesting leads to a reduction in the abundance of soil fauna. kudrin

We did not observe a significant reduction in soil fauna richness. However, the lack of changes in richness does not necessarily mean no changes in taxonomic composition. For example, closed-canopy species of Carabidae beetles may decrease or even disappear from clear-cut plots, while open-area-associated species rapidly colonize them [30,31]

Partial cutting is believed to have a less pronounced negative effect on biota compared to clear-cutting [10,11] due to fewer changes in soil hydrothermal conditions [56] (Londo et al., 1999), the maintenance of microbial biomass [57], and the conservation of refugia that preserve the structure, composition, and functional characteristics of undisturbed forests [58]. However, the modifying effect of harvesting type on soil fauna strongly depends on taxonomic groups. kudrin

Our findings also suggest that partial cutting may have less dramatic effects on soil fauna than clear-cutting, which confirms the potential of this forestry practice to reduce disturbances in forest ecosystems. kudrin
% 6. **Effets spécifiques des coupes forestières sur la faune du sol dans un contexte de changement climatique**
%    - Changements microclimatiques (température, humidité, exposition à la lumière) résultant des coupes, exacerbés par le réchauffement climatique, affectant les organismes du sol.
%    - Réduction des ressources comme le bois mort et la litière après des coupes intensives, privant les espèces du sol de leurs habitats.
%    - Perturbation des réseaux trophiques souterrains et des interactions entre espèces face à ces perturbations combinées.

(deMaynadier and Hunter, 1995). Woodland salamanders (genus Plethodon) are one species group that was historically overlooked when developing forest management guidelines 


(Homyack et al., 2010; Barrett et al., 2014; Hocking and Babbitt, 2014) Forest managers and researchers now have a greater understanding of the ecological roles these salamanders play in forest systems 

Woodland salamanders are numerically dominant in forested systems of the eastern United States (Burton and Likens, 1975a), which allows them to exert a strong influence on the food web in forests (Hairston, 1987).

(Wyman, 1998; Walton, 2013; Hickerson et al., 2017). Salamanders exhibit strong top-down pressures on detritivorous forest floor invertebrates which affects forest litter decomposition rates and carbon dynamics 

Further, woodland salamanders represent a high-quality food source for many predators such as birds, mammals, reptiles, and other amphibians (Burton and Likens, 1975b; Petranka, 1998)

Because of these factors, salamanders represent important linkages in the forest food web dynamic and have been considered important for ecological function and integrity (Welsh Jr. and Droege, 2001; Davic and Welsh, 2004).

Previous research has focused on certain forest characteristics (e.g., overstory cover, coarse woody debris, leaf litter depth) and their influence on salamanders post-harvest in order to guide forest management (Semlitsch, 2002; McKenny et al., 2006).
Even-aged forest management (e.g., clearcutting and shelterwood) is often thought to be the most detrimental to woodland salamanders in the short-term (Duguay and Wood, 2002; Hocking et al., 2013a) 
because these techniques greatly reduce overstory cover which increases the amount of solar radiation reaching the forest floor and subsequently dries out leaf litter and increases soil temperature (Zheng et al., 2000; Brooks and Kyker-Snowman, 2008).
Due to physiological constraints (e.g., thin, vascular skin; Hillman et al. 2009), woodland salamanders are at increased risk of desiccation when exposed to drier environments at the forest floor.
To cope with the environmental stress, individuals will often remain belowground where soils are cooler and moisture is not as limiting. 
However, heavy machinery used during operational silviculture compacts soils creating physical barriers for fossorial salamanders (Steinbrenner, 1995).
Further, the removal of surface refugia such as tree limbs, treetops, and other coarse woody debris (CWD) following harvesting can reduce habitat quality for salamanders and reduce the amount of time they spend at the forest floor surface (Achat et al., 2015; Peele et al., 2017)
Consequently, because woodland salamanders forage and breed at the forest floor surface, harsh surface conditions, soil compaction, and low levels of CWD can affect population dynamics (Peterman and Semlitsch, 2014).
Previous work in hardwood forests of eastern North America and the Appalachian region has shown that increased retention of overstory cover through partial harvesting approaches, such as uneven-aged management (e.g., group selection, single-tree selection), 
can result in higher relative densities of salamanders compared to even-aged treatments that remove greater amounts of overstory (Hocking et al., 2013a; Harper et al., 2015; Mahoney et al., 2016).
Greater retention of the overstory limits the amount of solar radiation reaching the forest floor which retains microclimates generally compatible with the physiology of salamanders (e.g., wetter, cooler; Homyack et al., 2011).
Greater amounts of CWD can also mitigate the negative effects of even-aged forest management by providing surface refugia that maintains suitable environmental conditions underneath logs and rocks and allows individuals to remain at the surface for longer periods of time (Grover, 1998; Moseley et al., 2009; Strojny and Hunter, 2009; Carusco, 2016).
A common outcome of all forms of forest management is the reduction of the overstory layer due to the removal of standing timber, and percent overstory cover and woodland salamander density are often strongly positively correlated (Tilghman et al., 2012).


\section*{Espèces à l'étude}
\label{sec:species}
\phantomsection\addcontentsline{toc}{section}{\nameref{sec:species}}

La faune du sol possède une importance primordiale dans les écosystèmes forestiers, en contribuant entre autres à la circulation de la matière et de l'énergie à travers la chaîne alimentaire, ainsi qu'au recyclage des nutriments \citep{Seibold2021contributioninsects,Kudrin2023metaanalysiseffects}.
Parmi les espèces vivant au sol, les amphibiens et les arthropodes font partie des taxons majoritairement impactés durant des perturbations environnementales, 
surtout lors de traitements sylvicoles \citep{Stuart2004Statustrends,Semlitsch2009Effectstimber,Hartshorn2021reviewforest} et dans le cadre des changements climatiques \citep{Alford1999Globalamphibian,Houlahan2000Quantitativeevidence,Pounds2006Widespreadamphibian,Warren2018projectedeffect}. 
Ces groupes sont souvent utilisés comme indicateurs de l'état des forêts et du degré de perturbation de celles-ci \citep{pongeVerticalDistributionCollembola2000,birdChangesSoilLitter2004,Maleque2009Arthropodsbioindicators}.
Afin de mieux comprendre l'impact des traitements sylvicoles sur la faune du sol, nous avons choisi d'étudier trois groupes d'espèces : la salamandre cendrée de l'Est (\textit{Plethodon cinereus} (Green, 1818)), 
les carabes (Carabidae) et les collemboles (Collembola).

\subsection*{Salamandre cendrée}

La salamandre cendrée constitue une des plus importantes biomasses chez les vertébrés des forêts nord-américaines \citep{Burton1975Salamanderpopulations,Petranka1993Effectstimber,semlitschAbundanceBiomassProduction2014a}.
Parmis les Plethodontidae, cette salamandre exclusivement terrestre respire de façon cutanée puisqu'elle est dépourvue de poumons. 
Elle dépend ainsi de l'humidité présente dans son environnement pour effectuer des échanges gazeux \citep{Heatwole1961Relationsubstrate}. 
Cette salamandre occupe les sols forestiers lorsque la température et le niveau d'humidité sont optimaux pour effectuer sa respiration cutanée. 
En dehors de ces périodes, elle se déplacera verticalement dans le sol pour maintenir des conditions propices à sa survie \citep{Grizzell1949HibernationSite}.  
Le déplacement vertical des salamandres cendrées suit un cycle saisonnier et l’abondance des individus à la surface est la plupart du temps plus élevée au printemps et à l’automne \citep{FraserEmpiricalEvaluation1976,Jaeger1980MicrohabitatsTerrestrial}.
Cette espèce possède un petit domaine vital et adopte d’ordinaire un comportement philopatrique \citep{Yurewicz2004ResourceAvailability}.
Son rôle dans les écosystèmes forestiers est prédominant en tant que prédateur généraliste. 
La salamandre cendrée participe de façon importante à la régulation des invertébrés détritivores \citep{Burton1975Energyflow,Walton2013Topdownregulation,Hickerson2017Easternredbacked}, 
ce qui influe directement sur les processus de décomposition de la matière organique et la mobilisation des nutriments en forêt \citep{Burton1975Energyflow,Wyman1998Experimentalassessment}. 
La salamandre cendrée joue de la même façon le rôle de proie au sein des réseaux trophiques, constituant une source de nourriture très énergétique et riche en protéines pour de nombreuses espèces \citep{Burton1975Energyflow,Pough1987abundancesalamanders}.
La sensibilité de cette salamandre aux perturbations environnementales en raison de sa respiration cutanée en fait un bon indicateur de la qualité des sols forestier \citep{Welsh2001caseusing}.
Elle est ainsi couramment utilisée pour évaluer les effets associées aux traitements sylvicoles, comme l’utilisation de l’habitat \citep{Heatwole1962EnvironmentalFactors,gibbsDistributionWoodlandAmphibians1998,Baecher2018Environmentalgradients,Mossman2019Twosalamander}, 
l'abondance de la salamandre cendrée \citep{Harpole1999Effectsseven,Grialou2000effectsforest,Homyack2009Longtermeffects,Hocking2013Effectsexperimental,Mazerolle2021Woodlandsalamander} 
ou encore à la richesse spécifique des urodèles \citep{Petranka1993Effectstimber,Welsh2001caseusing}.

\subsection*{Carabes}

Les carabes sont des coléoptères particulièrement actifs, vivant au niveau du sol forestier \citep{loveiEcologyBehaviorGround1996,Rochefort2006GroundBeetle}.
Cette famille rassemble la plus forte diversité spécifique parmi les coléoptères avec 40 000 espèces identifiées \citep{Erwin1985taxonpulse} 
et représente une des plus grandes abondances au sein des arthropodes vivant au sol \citep{loveiEcologyBehaviorGround1996,Rochefort2006GroundBeetle}.
La majorité des carabes sont carnivores et polyphages en plus d’être des prédateurs voraces \citep{loveiEcologyBehaviorGround1996}. 
Ils agissent comme des régulateurs des populations d’invertébrés dans la chaîne trophique et consomment surtout des aphides, des collemboles et des escargots \citep{loveiEcologyBehaviorGround1996}. 
Les carabes représentent aussi des proies pour plusieurs espèces d’amphibiens, de reptiles, d’oiseaux et de mammifères \citep{loveiEcologyBehaviorGround1996}. 
Ce groupe est largement répandu dans les écosystèmes terrestres et se retrouve dans une grande variété de milieux naturels, tels que les forêts, les cultures, les marais ou encore les sablières \citep{Larochelle2003naturalhistory}. 
Toutefois, la sélection d'habitats varie selon l'espèce, ce qui entraîne une différence de communauté en fonction du type de milieu.
En milieu forestier, les carabes sont le plus souvent classés selon trois types de communautés : les espèces des milieux forestiers matures et fermés, les espèces des milieux ouverts et les espèces généralistes \citep{Niemela2007effectsforestry}. 
Cette variation dans la sélection d'habitats fait des carabes un des taxons les plus intéressants à étudier lors de perturbations environnementales.
Ce groupe a été utilisé dans des travaux portant sur les traitements sylvicoles, notamment les coupes totales \citep{Niemela1993Effectsclearcut,Heliola2001Distributioncarabid,koivulaBorealCarabidbeetleColeoptera2002a}, 
les coupes partielles et l'élagage \citep{Lemieux2004Groundbeetle,mooreEffectsTwoSilvicultural2004,Peck2004Longertermeffects}.
D'autres études se sont interresées à la fragmentation forestière et à la pollution environnementale pour documenter l'utilisation de l'habitats par les carabes et pour évaluer la valeurs nutritives des sols forestiers propices à ces invertébrés \citep{bouchardBeetleCommunityResponse2016b,Luff1992Classificationprediction,Rainio2003Groundbeetles,Work2008Evaluationcarabid}.

\subsection*{Collemboles}

Les collemboles sont un regroupement polyphylétique d'arthropodes faisant partie de la mésofaune établie dans les sols forestiers.
Ces invertébrés comportent une grande richesse spécifique et sont très abondants \citep{rusekBiodiversityCollembolaTheir1998}, 
malgré leur capacité limitée de dispersion due à leur petite taille, leur absence d’ailes et leur niche écologique \citep{Ojala2001Dispersalmicroarthropods}.
Différentes communautés de collemboles occupent un ensemble de niches écologiques allant de la litière aux différents horizons du sol \citep{pongeVerticalDistributionCollembola2000}.
La répartition verticale de ces communautés dépend essentiellement des conditions abiotiques du sol telles que la luminosité, le taux d’humidité ou encore la porosité.
Les collemboles peuvent ainsi servir à caractériser un substrat en fonction de la communauté qu’on y retrouve \citep{rusekBiodiversityCollembolaTheir1998}.
De plus, les collemboles contribuent à la formation de microstructures dans les sols. 
Ce taxon joue un rôle prédominant dans de nombreux processus écologiques. 
Principalement fongivores et détritivores, ces organismes se nourrissent en grande partie de champignons, de bactéries, d'actinomycètes et d'algues. 
Chaque espèce est cependant spécialisée à un type spécifique de ressource alimentaire \citep{Chen1995Foodpreference,rusekBiodiversityCollembolaTheir1998}.
Ces arthropodes contribuent de façon importante à la décomposition de la matière organique, à la transformation de nutriments et 
au transfert d’énergie dans les écosystèmes terrestres \citep{rusekBiodiversityCollembolaTheir1998,Hattenschwiler2005Biodiversitylitter,Cuchta2019importantrole,Marsden2020Howagroforestry}.
Avec les acariens, les collemboles accélèrent de 10 à 20\% la décomposition de la matière organique \citep{Hattenschwiler2005Biodiversitylitter}.
Ils participent également à la chaîne trophique en tant que proie pour plusieurs espèces d’arachnides, de coléoptères, d’amphibiens, de reptiles et d’oiseaux. 
D’autre part, ils servent d’hôtes à des parasites incluant les nématodes, les trématodes, les protozoaires, les bactéries et les champignons \citep{rusekBiodiversityCollembolaTheir1998}.
L'étude des collemboles est pertinente pour ce projet en raison de la relation proie-prédateur existante entre ce taxon, les salamandres et les carabes. 
De plus, le groupe des collemboles est couramment utilisé dans le cadre de recherches portant sur les répercussions des traitements sylvicoles sur la mésofaune \citep{Salmon2008Relationshipssoil,farskaManagementIntensityAffects2014,rousseauWoodyBiomassRemoval2019}.

En résumé, le choix de ces trois groupes d'espèces est pertinent pour étudier l'impact des traitements sylvicoles. 
Leur sensibilité aux perturbations environnementales et leurs relations trophiques en font des candidats idéaux pour analyser l'effet des coupes forestières sur la dynamique de la faune du sol.




%6 importance de l'étude
Malgré l'importance de la faune du sol et la multitude d'études ayant évalué la réponse de différents groupes d'invertébrés du sol à l'exploitation forestière, il existe encore peu de revues sur cette problématique. 
Marshall a publié une revue traditionnelle sur l'influence de l'exploitation forestière sur les processus biologiques il y a plus de 20 ans, et depuis, de nombreuses nouvelles recherches ont été menées. 
Bien que les méta-analyses évaluant les changements de biodiversité liés à la gestion forestière incluent les invertébrés du sol, elles ne prennent pas en compte les données sur l'abondance et la réaction des différents groupes de faune du sol.




Ce contexte souligne l'importance de mieux comprendre et minimiser les impacts des pratiques forestières sur la biodiversité, notamment en ce qui concerne la faune du sol, qui joue un rôle essentiel dans le maintien des écosystèmes forestiers.


% 7. **Justification de l’étude dans un contexte de changement climatique et de migration assistée**
%    - Besoin d’études pour comprendre comment les coupes forestières, combinées aux effets du changement climatique, influencent la faune du sol et la biodiversité forestière.
%    - Lacunes dans la littérature concernant la réponse des communautés de la faune du sol à ces pratiques dans un contexte de gestion forestière orientée vers la migration assistée.
%    - Importance de ces connaissances pour proposer des stratégies de gestion forestière qui minimisent les risques pour les écosystèmes et maximisent leur résilience face au changement climatique.

There is a need to better understand the underlying mechanisms driving responses of soil fauna cooccurence to forest management in order to provide managers with harvesting strategies that are compatible with soil fauna occupancy.
A greater understanding of the mechanistic drivers from forestry prescriptions will allow managers to directly incorporate these measures into their forest planning to help mitigate negative effects from forest management.

peu d'étude se sont interréssé au interaction entre plusieurs groupe tel que les amphibien et les arthropodes, il sont souvent étudié séparément.

La plupart des travaux qui se sont intéressés aux impacts des traitements sylvicoles sur la faune discutent généralement des effets directs des perturbations sur un ou plusieurs groupes d'espèces, 
sans tenir compte des relations existantes entre les variables environnementales et les différents groupes d'espèces, 
négligeant ainsi les effets indirects des coupes sur la faune du sol \citep{josephIntegratingOccupancyModels2016,Pollierer2021Diversityfunctional,Kudrin2023metaanalysiseffects}. 
Mon projet comblait justement cette lacune en essayant de comprendre comment les effets des traitements sylvicoles se propagent à l’intérieur du réseau écologique forestier et influence la dynamique de la faune du sol.  
Ultimement, ce gain de connaissances fournira des outils précieux pour faciliter la gestion durable des forêts.


% 8. **Objectifs de l’étude**
%    - Quantifier les effets directs et indirects des coupes forestières et des changements environnementaux (liés au climat) sur la faune du sol.
%    - Explorer comment la migration assistée et les coupes forestières influencent la structure et la dynamique des populations d’espèces du sol, en tenant compte de leur rôle crucial dans l’écosystème.
%    - Proposer des recommandations pour intégrer ces résultats dans les pratiques de gestion durable des forêts, notamment en tenant compte des effets du changement climatique et de la nécessité de maintenir les services écosystémiques essentiels.


\section*{Objectifs et hypothèses}
\label{sec:objectifs}
\phantomsection\addcontentsline{toc}{section}{\nameref{sec:objectifs}}

Le but de mon étude était de comprendre comment les traitements sylvicoles, effectués dans un contexte de migration assistée, 
affecte la dynamique des écosystèmes du sol forestier. Les objectifs qui s’y rattachaient étaient :

\begin{enumerate}
    \item De quantifier l'effet des traitements de coupes forestières sur les variables environnementales qui influencent l'utilisation de l'habitat par la faune du sol.
    \item De mesurer l'impact des coupes forestières sur l'utilisation de l'habitat par la faune du sol.
\end{enumerate}

L'hypothèse 1.1 liée à notre deuxième objectif soutenait que les variables environnementales favorables à l'utilisation de l'habitat par les espèces fluctuent 
en fonction de l'intensité des coupes forestières. Ainsi, les traitements de coupes forestières constituent une variable englobant 
les changements de conditions environnementales.

L'hypothèse 2.1 attachée à notre premier objectif stipulait que les traitements de coupe forestière entraînent une modification de l'utilisation de l'habitat 
par la faune du sol et se propage à travers le réseau trophique. Spécifiquement, les coupes affectent l'utilisation de l'habitat par les salamandres et 
les grands carabes (compétiteurs de salamandres), ce qui modifie ensuite la sélection d'habitat des petits carabes (proies de salamandres), puis enfin des collemboles (proies de grands carabes et salamandres).



\cleardoublepage

\bibliography{References.bib}
\bibliographystyle{ecologyNewFR.bst}
