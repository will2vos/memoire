\chapter*{Introduction générale}         % ne pas numéroter
\label{chap-introduction}       % étiquette pour renvois
\phantomsection\addcontentsline{toc}{chapter}{\nameref{chap-introduction}} % inclure dans TdM

% \usepackage[french]{babel}

\section*{Mise en contexte}
\label{sec:contexte}
\phantomsection\addcontentsline{toc}{section}{\nameref{sec:contexte}}
À travers les services écosystémiques qu'elles fournissent et leurs valeurs économiques, les forêts jouent un rôle prédominant à l'échelle planétaire \citep{Balvanera2006Quantifyingevidence}.
L'environnement qu'elles apportent dans les écosystèmes terrestres permet le maintien d’une biodiversité importante ainsi qu'une régulation de facteurs biogéochimiques \citep{Pawson2013Plantationforests}. 
Cependant, les changements climatiques représentent aujourd'hui un des défis les plus préoccupants pour le maintien des forêts tel qu'on les connait \citep{McKenney2009Climatechange,Messier2022Warningnatural,Seidl2017Forestdisturbances,Trumbore2015Foresthealth}.  
Malgré les engagements internationaux pris au courant des dernières décennies pour diminuer la hausse de température lié aux activités anthropiques, 
les projections climatiques actuelles nous indiquent une hausse future de température globale dépassant 1.5\up{o}C par rapport à l'air préindustriel \citep{Matthews2022Currentglobal}.
En raison de ses latitudes nordiques, le Canada est particulièrement vulnérable à l'augmentation des températures et aux perturbations environnementales qui en découlent \citep{Alo2008Potentialfuture,Bush2019Canadachanging}. 

Les forêts de l'Est de l'Amérique du Nord sont et seront donc ainsi particulièrement touchées par cette hausse de température \citep{Park2014Canboreal,Mahony2017closerlook,Messier2022Warningnatural,Sittaro2017Treerange}.
Parmi les conséquences attendues, on prévoit un allongement et une intensification des périodes de sècheresse, une augmentation du nombre de feux de forêt et une hausse des perturbations biotiques \citep{Gatti2021Amazoniacarbon,Heidari2021Effectsclimate,Joyce2013Climatechange,Parmesan2007Influencesspecies}. 
À cela s'ajoutent des changements dans la phénologie \citep{Chuine2010Whydoes} ainsi que dans la distribution des arbres \citep{Gray2013Trackingsuitable,Zhu2012Failuremigrate} due au manque d'adaptation des végétaux aux nouvelles conditions de leur milieu \citep{Aitken2008Adaptationmigration}.
Les changements climatiques se produisent toutefois plus rapidement que la capacité d'adaptation ou de déplacement des arbres \citep{Aitken2008Adaptationmigration,Harrison2020Plantcommunity,Loarie2009velocityclimate,Messier2022Warningnatural,Williams2013Preparingclimate,Vitt2010Assistedmigration}, 
menaçant ainsi leur capacité à se maintenir dans des milieux favorables à leur survie et leur croissance \citep{Sittaro2017Treerange,Woodall2018Decadalchanges,Zhu2012Failuremigrate}.
Des changements pourraient ainsi être observés dans la composition des forêts menant à des modifications dans la conduite de l'aménagement forestier et de la conservation \citep{Chmura2011Forestresponses,Lo2011Linkingclimate,McKenney2009Climatechange}.


\section*{Migration assistée}
\label{sec:fam}
\phantomsection\addcontentsline{toc}{section}{\nameref{sec:fam}}

Plusieurs appels à adapter l'aménagement ont été proposés afin de préserver les milieux forestiers et leurs bienfaits \citep{Messier2021sakeresilience,Nagel2017Adaptivesilviculture}.
Différentes mesures d'atténuation ont été proposées pour prévenir la perte des forêts et améliorer la résilience de celle-ci comme l'augmentation de la diversité fonctionnelle et l'amélioration de la connectivité à travers le paysage forestier \citep{Messier2019functionalcomplex}.
Parmi toutes les solutions envisagées, la migration assistée d'arbres est proposée comme une mesure d'atténuation permettant le déplacement d'individus ou de matériel génétique depuis un territoire climatique originel vers une zone de climat futur plus propice à la croissance des arbres \citep{Dumroese2015Considerationsrestoring,Palik2022Operationalizingforestassisted,Park2023Provenancetrials,Park2018Informationunderload,Pedlar2011implementationassisted,Vitt2010Assistedmigration,Williams2013Preparingclimate}. 
La migration assistée d'arbre permettrait de modifier rapidement la composition des peuplements pour mieux convenir au climat futur de celui-ci \citep{Pedlar2011implementationassisted} 
répondant ainsi aux besoins de conservation, maintenant les services écosystémiques et préservant la valeur économique. \citep{Ste-Marie2011Assistedmigration,Winder2011Ecologicalimplications}.

Un manque de connaissances et un degré d'incertitude subsistent toutefois autour de la migration assistée \citep{Park2018Informationunderload,Klenk2015assistedmigration}, 
en particulier en ce qui concerne le dilemme entre les avantages de la préservation d'une espèce et les risques pour les espèces du territoire d'accueil \citep{Hewitt2011Takingstock,McLachlan2007frameworkdebate,Vitt2010Assistedmigration}.
Afin d'améliorer nos connaissances et ainsi réduire l'incertitude, divers scénarios sylvicoles sont à l'heure actuelle étudiés pour limiter les risques associés à la migration assistée \citep{royoDesiredREgenerationAssisted2023}.
Les traitements sylvicoles sont en général utilisés pour influencer la croissance, la santé et la composition des peuplements forestiers.
Parmi ces traitements, les coupes totales et les coupes partielles font partie des opérations couramment utilisées en foresterie \citep{Ameray2021Forestcarbon,Chaudhary2016Impactforest,Man2008Elevenyearresponses,MontoroGirona2018ConiferRegeneration,PamerleauCouture2015Effectthree}. 
La coupe totale implique l'abattage de tous les arbres dans une zone déterminée.
Elles sont le plus souvent employées dans le cadre d'un plan d'aménagement intensif visant à accroître la productivité et la qualité du bois 
sur un court intervalle de temps dans afin de répondre aux besoins de l'industrie et d'optimiser les bénéfices \citep{Ameray2021Forestcarbon}.
Pour leur part, les coupes partielles consistuent une suppression sélective d'arbres tout en laissant une partie du peuplement intacte.
Ce type de coupe est habituellement utilisé dans le cadre d'un plan d'aménagement extensif qui favorise la régénération naturelle et imite les perturbations naturelles.
Les coupes partielles sont souvent utilisées pour favoriser la croissance des arbres les plus vigoureux, pour encourager la diversité spécifique ou pour maintenir une canopée ouverte \citep{Ameray2021Forestcarbon,Irland2011Timberproductivity}.
La conservation d'arbres en coupe partielle et l'allongement de la période de rotation offrent d'autres avantages sur le plan économique et écosystémique. 
Cela favorise la séquestration du carbone, préserve l'apport de matière organique ainsi que le cycle de nutriment, 
tout en maintenant différentes niches écologiques pour la faune \citep{Ameray2021Forestcarbon,Barg1999Influencepartial,Tong2020Forestmanagement}.

L'utilisation de coupes forestières peut toutefois engendrer des modifications dans les conditions environnementales des forêts, 
telles que l'augmentation de l'exposition au rayonnement solaire, la hausse de vitesse des vents et la réception intensifiée de précipitations au sol en réponse à l'élimination de la canopée, 
menant à un accroissement de l'ensoleillement, de la température et de l'humidité du sol \citep{Keenan1993ecologicaleffects,Lindo2003Microbialbiomass,Heithecker2007Edgerelatedgradients}.
De plus, les coupes peuvent affecter la disponibilité en nutriment et augmenter le degré de compaction au niveau des sols \citep{Battigelli2004Shorttermimpact,Covington1981Changesforest,Lindo2003Microbialbiomass,rousseauLongtermEffectsBiomass2018}. 
À terme, les changements environnementaux résultant d'une coupe peuvent affecter la biodiversité. \citep{Chaudhary2016Impactforest,Fedrowitz2014Canretention,Paillet2010Biodiversitydifferences}, 
tout particulièrement sur la faune vivant à la hauteur du sol \citep{Chaudhary2016Impactforest,Lindo2003Microbialbiomass,Kudrin2023metaanalysiseffects}.

\section*{Espèces à l'étude}
\label{sec:species}
\phantomsection\addcontentsline{toc}{section}{\nameref{sec:species}}

La faune du sol possède une importance primordiale dans les écosystèmes forestiers, en contribuant entre autres à la circulation de la matière et de l'énergie à travers la chaîne alimentaire, ainsi qu'au recyclage des nutriments \citep{Kudrin2023metaanalysiseffects,Seibold2021contributioninsects}.
Parmi les espèces vivant au sol, les amphibiens et les arthropodes font partie des taxons majoritairement impactés durant des perturbations environnementales, 
surtout lors de traitements sylvicoles \citep{Hartshorn2021reviewforest,Semlitsch2009Effectstimber,Stuart2004Statustrends} et dans le cadre des changements climatiques \citep{Alford1999Globalamphibian,Houlahan2000Quantitativeevidence,Milanovich2010Projectedloss,Parmesan2006EcologicalEvolutionary,Pounds2006Widespreadamphibian,Warren2018projectedeffect}. 
Ces groupes sont ainsi souvent utilisés comme indicateurs de l'état des forêts et du degré de perturbation de celles-ci \citep{birdChangesSoilLitter2004,Maleque2009Arthropodsbioindicators,pongeVerticalDistributionCollembola2000}.
Afin de mieux comprendre l'impact des traitements sylvicoles sur la faune du sol, nous avons choisi d'étudier trois groupes d'espèces : la salamandre cendrée de l'Est (\textit{Plethodon cinereus} (Green, 1818)), 
les carabes (Carabidae) et les collemboles (Collembola).
% impacte sur le débris ligneux et la profondeur de litières
% parler de la dynamique des espèces ?

\subsection*{Salamandre cendrée}

La salamandre cendrée constitue une des plus importantes biomasses chez les vertébrés des forêts nord-américaines \citep{Burton1975Salamanderpopulations,Petranka1993Effectstimber,semlitschAbundanceBiomassProduction2014a}.
Comme le reste des Plethodontidae, cette salamandre est exclusivement terrestre. Sa respiration se fait de façon cutanée puisqu'elle est dépourvue de poumons. 
Elle dépend ainsi de l'humidité présente dans son environnement pour effectuer des échanges gazeux \citep{Heatwole1961Relationsubstrate}. 
Cette salamandre occupe les sols forestiers lorsque la température et le niveau d'humidité sont optimaux pour effectuer sa respiration cutanée. 
En dehors de ces périodes, elle se déplacera verticalement dans le sol pour maintenir des conditions propices à sa survie \citep{Grizzell1949HibernationSite}.  
Le déplacement vertical des salamandres cendrées suit un cycle saisonnier et l'abondance des individus à la surface est la plupart du temps plus grande durant le printemps et l'automne \citep{FraserEmpiricalEvaluation1976,Jaeger1980MicrohabitatsTerrestrial}.
Cette espèce possède un domaine vital de faible envergure et adopte d'ordinaire un comportement philopatrique \citep{Yurewicz2004ResourceAvailability}.
Son rôle dans les écosystèmes forestiers est prédominant en tant que prédateur généraliste. 
Elle participe de façon importante à la régulation des invertébrés détritivores \citep{Burton1975Energyflow,Hickerson2017Easternredbacked,Walton2013Topdownregulation}, 
ce qui influe directement sur les processus de décomposition des matières organiques et la mobilisation des nutriments en forêt \citep{Burton1975Energyflow,Wyman1998Experimentalassessment}. 
La salamandre cendrée joue de la même façon le rôle de proie au sein des réseaux trophiques, constituant une source de nourriture très énergétique et riche en protéines pour de nombreuses espèces \citep{Burton1975Energyflow,Pough1987abundancesalamanders}.
La sensibilité de cette salamandre aux perturbations environnementales en raison de sa respiration cutanée en fait un bon indicateur de la qualité environnementale des sols \citep{Welsh2001caseusing}.
Elle est de cette façon couramment employée en tant que bio-indicateur pour évaluer les perturbations environnementales associées aux traitements sylvicoles.
Plusieurs études se sont ainsi intéressées à l'utilisation de l'habitat \citep{Baecher2018Environmentalgradients,gibbsDistributionWoodlandAmphibians1998,Heatwole1962EnvironmentalFactors,Mossman2019Twosalamander}, 
à l'abondance de la salamandre cendrée \citep{Harpole1999Effectsseven,Hocking2013Effectsexperimental,Homyack2009Longtermeffects,Grialou2000effectsforest,Mazerolle2021Woodlandsalamander} 
ou encore à la richesse spécifique des urodèles \citep{Petranka1993Effectstimber,Welsh2001caseusing}.

\subsection*{Carabe}

Les carabes sont des coléoptères particulièrement actifs, vivant au niveau du sol forestier \citep{loveiEcologyBehaviorGround1996,Rochefort2006GroundBeetle}.
Cette famille rassemble la plus forte diversité spécifique parmi les coléoptères avec 40 000 espèces identifiées \citep{Erwin1985taxonpulse} 
et représente une des plus grande abondances au sein des arthropodes vivant au sol \citep{loveiEcologyBehaviorGround1996,Rochefort2006GroundBeetle}.
La majorité des carabes sont carnivores et polyphages en plus d’être des prédateurs voraces \citep{loveiEcologyBehaviorGround1996}. 
Ils agissent ainsi comme des régulateurs des populations d’invertébrés dans la chaîne trophique et consomment surtout des aphides, des collemboles et des escargots \citep{loveiEcologyBehaviorGround1996}. 
Les carabes représentent aussi des proies pour plusieurs espèces d’amphibiens, de reptiles, d’oiseaux et de mammifères \citep{loveiEcologyBehaviorGround1996}. 
Ce groupe est largement répandu dans les écosystèmes terrestres et se retrouve dans une grande variété de milieux naturels \citep{Larochelle2003naturalhistory}. 
Toutefois, la sélection d'habitats varie selon l'espèce, ce qui entraîne une différence de communauté en fonction du type de territoire.
Les carabes sont le plus souvent classés selon trois types de communautés : les espèces des milieux forestiers matures et fermés, les espèces des milieux ouverts et les espèces généralistes \citep{Niemela2007effectsforestry}. 
Cette variation dans la sélection d'habitats fait des carabes un des taxons les plus intéressants à étudier lors de perturbations environnementales.
Ce groupe a été utilisé dans des travaux portant sur les traitements sylvicoles, notamment les coupes totales \citep{Heliola2001Distributioncarabid,koivulaBorealCarabidbeetleColeoptera2002a,Niemela1993Effectsclearcut}, 
les coupes partielles et l'élagage \citep{Lemieux2004Groundbeetle,Peck2004Longertermeffects,mooreEffectsTwoSilvicultural2004}, 
ainsi que dans la préparation de sites \citep{Duchesne*1999EffectsClearCutting}.
D'autres études se sont aussi intéressées à la fragmentation, à la pollution environnementale, à la classification d'habitats et à la catégorisation des valeurs nutritives des sols forestiers \citep{bouchardBeetleCommunityResponse2016b,Halme1993Carabidbeetles,Luff1992Classificationprediction,Niemela2001Carabidbeetles,Rainio2003Groundbeetles,Work2008Evaluationcarabid}.

\subsection*{Collembole}

Les collemboles sont un regroupement polyphylétique d'arthropodes faisant partie de la mésofaune établie dans les sols forestiers.
Ces invertébrés comportent une grande richesse spécifique et représentent une abondance importante \citep{rusekBiodiversityCollembolaTheir1998}, 
malgré leur capacité limitée de dispersion due à leur petite taille, leur absence d’ailes et leur niche écologique \citep{Ojala2001Dispersalmicroarthropods}.
Différentes communautés de collemboles occupent un ensemble de niches écologiques allant de la litière aux différents horizons du sol \citep{pongeVerticalDistributionCollembola2000}.
La répartition verticale de ces communautés dépend essentiellement des conditions abiotiques telles que la luminosité, le taux d’humidité ou encore la porosité.
Les collemboles peuvent ainsi servir à caractériser un substrat en fonction de la communauté qu’on y retrouve \citep{rusekBiodiversityCollembolaTheir1998}.
Le mode de vie édaphique des collemboles participe de plus à la formation de microstructures dans les sols.
Ce taxon joue un rôle prédominant dans de nombreux processus écologiques. 
Principalement fongivores et détritivores, ces organismes se nourrissent en grande partie de champignons, de bactéries, d'actinomycètes et d'algues. 
Chaque espèce est cependant spécialisée dans un type spécifique de ressource alimentaire \citep{Chen1995Foodpreference,rusekBiodiversityCollembolaTheir1998}.
Ces arthropodes contribuent ainsi de façon importante à la décomposition de la matière organique, à la transformation de nutriments et 
au transfert d’énergie dans les écosystèmes terrestres \citep{Cuchta2019importantrole,Hattenschwiler2005Biodiversitylitter,Marsden2020Howagroforestry,Petersen2000Collembolapopulations,rusekBiodiversityCollembolaTheir1998,Wolters1991SoilInvertebrates}.
Avec les acariens, les collemboles accélèrent de 10 à 20 \% la décomposition de la matière organique. \citep{Hattenschwiler2005Biodiversitylitter}.
Ils participent également à la chaîne trophique en tant que proie pour plusieurs espèces d’arachnides, de coléoptères, d’amphibiens, 
de reptiles et d’oiseaux. 
D'autre part, ils servent d’hôtes parasitaires à certains nématodes, trématodes, protozoaires, bactéries et champignons \citep{rusekBiodiversityCollembolaTheir1998}.
L'étude des collemboles est pertinente pour ce projet en raison de la relation proie-prédateur existante entre ce taxon, les salamandres et les carabes. 
De plus, ce groupe est couramment utilisé dans le cadre de recherches portant sur les répercussions que peuvent avoir les traitements sylvicoles sur la mésofaune \citep{farskaManagementIntensityAffects2014,rousseauWoodyBiomassRemoval2019,Salmon2008Relationshipssoil}.

En résumé, le choix de ces trois groupes d'espèces est pertinent pour étudier l'impact des traitements sylvicoles. 
Leur sensibilité aux perturbations environnementales et leurs relations trophiques en font des candidat idéal pour analyser l'effet des coupes forestières sur la dynamique de la faune du sol.
Plusieurs recherches se sont déjà intéressées aux impacts des traitements sylvicoles sur la faune du sol dans le passé. 
Toutefois, la plupart discutent des effets directs des perturbations sur un ou plusieurs groupes d'espèces, 
sans tenir compte des relations existantes entre les variables environnementales et les différents groupes d'espèces, 
négligeant ainsi les effets indirects des coupes sur la faune du sol \citep{josephIntegratingOccupancyModels2016,Kudrin2023metaanalysiseffects,Pollierer2021Diversityfunctional}. 
Il est cependant essentiel de comprendre de manière intégrée les effets des traitements sylvicoles sur la faune du sol, 
comment ceux-ci se propagent à l'intérieur du réseau écologique forestier et de connaitre l'ensemble des répercussions sur la structure 
et la dynamique des communautés du sol. 
Ultimement, ce gain de connaissances fournira des outils précieux pour faciliter la gestion durable des forêts.

\section*{Objectifs et hypothèses}
\label{sec:objectifs}
\phantomsection\addcontentsline{toc}{section}{\nameref{sec:objectifs}}

Le but de notre étude est de comprendre comment les traitements sylvicoles, effectués dans un contexte de migration assistée, affecte la dynamique des écosystèmes du sol forestier.
Les objectifs qui s'y rattachent sont : 

\begin{enumerate}
    \item Mesurer l'impact des coupes forestières sur l'utilisation de l'habitat par la faune du sol.
    \item Comparer les traitements de coupes forestières et les variables environnementales qui influencent l'utilisation de l'habitat par la faune du sol. 
\end{enumerate}

L'hypothèse 1.1 attaché à notre premier objectif stipule que les traitements de coupe forestière entraînent une modification de l'utilisation de l'habitat 
par la faune du sol et se propage à travers le réseau trophique. Plus précisément, les coupes affectent l'utilisation de l'habitat par les salamandres et 
les grands carabes, ce qui modifie ensuite la sélection d'habitat des petits carabes, puis enfin des collemboles.

L'hypothèse 2.1 liée à notre deuxième objectif soutient que les variables environnementales favorables à l'utilisation de l'habitat par les espèces fluctuent 
en fonction de l'intensité des coupes forestières. Ainsi, les traitements de coupes forestières peuvent être utilisés comme une variable englobant 
les changements de conditions environnementales.

\cleardoublepage

\bibliography{References.bib}
\bibliographystyle{ecologyNewFR.bst}
