\chapter*{Introduction générale}         % ne pas numéroter
\label{chap-introduction}       % étiquette pour renvois
\phantomsection\addcontentsline{toc}{chapter}{\nameref{chap-introduction}} % inclure dans TdM

\section*{Mise en contexte}
\label{sec:contexte}
\phantomsection\addcontentsline{toc}{section}{\nameref{sec:contexte}}

Les forêts jouent un rôle prédominant à travers le monde par leurs valeurs économiques et les services écosystémiques qu'elles fournissent \citep{Balvanera2006Quantifyingevidence}.
Ces milieux permettent le maintien d’une biodiversité importante ainsi qu'une régulation de facteurs biogéochimiques dans les écosystèmes terrestres \citep{Pawson2013Plantationforests}. 
Cependant, les changements climatiques représentent aujourd'hui un des défis les plus préoccupant pour le maintient des forêts tel qu'on les connais \citep{McKenney2009Climatechange,Messier2022Warningnatural,Seidl2017Forestdisturbances,Trumbore2015Foresthealth}.  
Malgré les engagements internationaux pris au courant des dernière décennie pour diminuer la hausse de température lié aux activités anthropiques, 
les projections climatique actuelle nous indique une hausse future de température globale dépassant 1.5\up{o}C par rapport à l'air préindustrielle \citep{Matthews2022Currentglobal}.
En raison de ses latitudes nordiques, le Canada est particulièrement vulnérable à l'augmentation des températures et aux perturbations environnementales qui en découlent \citep{Alo2008Potentialfuture,Bush2019Canadachanging}. 

Les forêts de l'Est de l'Amérique du Nord sont et seront donc ainsi particulièrement touché par cette hausse de température \citep{Park2014Canboreal,Mahony2017closerlook,Messier2022Warningnatural,Sittaro2017Treerange}.
Parmis les conséquences attendu, on prévoit un allongement et une intensifications des périodes de sècheresse, une augmentation du nombre de feu de forêt et une hausse des perturbation biotique \citep{Gatti2021Amazoniacarbon,Heidari2021Effectsclimate,Joyce2013Climatechange,Parmesan2007Influencesspecies}. 
À cela s'ajoute un changement dans la phenologie \citep{Chuine2010Whydoes} et la distribution des arbres \citep{Gray2013Trackingsuitable,Zhu2012Failuremigrate} due au manque d'adaptation des plantes aux nouvelles conditions de leur milieu \citep{Aitken2008Adaptationmigration}.
Cependant, les changements climatiques se produise plus rapidement que la capacité des arbres à s'adapter ou à se déplacer \citep{Aitken2008Adaptationmigration,Harrison2020Plantcommunity,Loarie2009velocityclimate,Messier2022Warningnatural,Williams2013Preparingclimate,Vitt2010Assistedmigration}, 
mettant ainsi à risque la capacité des arbres à se maintenir dans des milieu favorable à leur survie et leur croissance \citep{Sittaro2017Treerange,Woodall2018Decadalchanges,Zhu2012Failuremigrate}.
Il serait ainsi possible d'observer une modification dans la composition des forêts menant à un changements dans l'amenagement forestier et la conservation \citep{Chmura2011Forestresponses,Lo2011Linkingclimate,McKenney2009Climatechange}.

\section*{Migration assitée et préparation de sites}
\label{sec:fam}
\phantomsection\addcontentsline{toc}{section}{\nameref{sec:fam}}

Plusieurs appel à adapter l'amenagement ont été proposer afin de préserver les milieux forestiers et leurs bienfaits \citep{Messier2021sakeresilience,Nagel2017Adaptivesilviculture}
Différentes mesures d'atténuation ont été proposées pour prévenir la perte des forêts et améliorer la résilience de celle-ci comme l'augmentation de la diversité fonctionnel et l'amélioration de la connectivité à travers le paysage forestier \citep{Messier2019functionalcomplex}.
Parmi toutes les solutions proposées, la migration assisté d'arbres est envisagé comme une mesure d'atténuation permettant le déplacement d'individus ou de matériel génétique depuis un territoire climatique originel vers une zone de climat futur plus propices à la croissances des arbres \citep{Dumroese2015Considerationsrestoring,Palik2022Operationalizingforestassisted,Park2023Provenancetrials,Park2018Informationunderload,Pedlar2011implementationassisted,Vitt2010Assistedmigration,Williams2013Preparingclimate}. 
La migration assistée d'arbre permettrait de modifier rapidement la composition des communautés pour mieux convenir au climat future d'un milieu \citep{Pedlar2011implementationassisted} 
en plus de répondre à différents besoins tel que la préservation des espèces, le maintient de la valeur économique et la préservation de la biodiversité et des services écosystémiques \citep{Ste-Marie2011Assistedmigration,Winder2011Ecologicalimplications}.

Cependant, il subsiste encore un manque de connaissances et un degré d'incertitude entourant la migration assistée \citep{Park2018Informationunderload,Klenk2015assistedmigration}, 
notamment en ce qui concerne le dilemme entre les avantages de la préservation d'une espèce et les risques potentiels pour les espèces du milieu d'accueil \citep{Hewitt2011Takingstock,McLachlan2007frameworkdebate,Vitt2010Assistedmigration}.
Afin de reduire cette incertittude, divers scénarios sylvicoles sont actuellement étudiés dans le but d'améliorer nos connaissancesde et de limiter les risques associés à la migration assistée \citep{royoDesiredREgenerationAssisted2023}.
Les traitements sylvicoles sont généralement utilisées pour influencer la croissance, la santé et la composition des peuplements forestiers.
Les opérations de coupe totale et de coupe partielle font partie de de trairements sylvicoles couramment appliqués et étudier en foresterie \citep{Ameray2021Forestcarbon,Chaudhary2016Impactforest,Man2008Elevenyearresponses,MontoroGirona2018ConiferRegeneration,PamerleauCouture2015Effectthree}.

Les coupes totale implique l'abattage de tous les arbres dans une zone déterminée.
Elles sont généralement employées dans le cadre d'un plan d'aménagement intensif visant à accroître la productivité et la qualité du bois 
sur une courte période de temps dans afin de répondre aux besoins de l'industrie et d'optimiser les bénéfices \citep{Ameray2021Forestcarbon}.

Les coupes partielle se définisse par la suppression sélective d'arbres tout en laissant une partie du peuplement intacte. 
Ce type de coupe est habituellement utilisé dans le cadre d'un plan d'aménagement extensif qui favorise la régénération natuelle et imite les perturbation naturelle.
Cette méthode est souvent utilisée pour favoriser la croissance des arbres les plus vigoureux, pour encourager la diversité spécifique ou pour maintenir une canopée ouverte \citep{Ameray2021Forestcarbon,Irland2011Timberproductivity}.
La conservation d'arbres  en coupe partielle et l'allongement de la période de rotation offrent d'autres avantages sur le plan économique et écosystèmique. 
Cela favorise la séquéstration du carbone, préserve l'apport de matière organique ainsi que le cycle de nutriment, 
tout en maintenant différentes niches écologiques pour la faune \citep{Ameray2021Forestcarbon,Barg1999Influencepartial,Tong2020Forestmanagement}.

Toutefois, l'application de coupes forestières peux avoir des répercutions sur les conditions environnementales des forêts, 
telles que l'augmentation de l'exposition au rayonnement solaire , la hausse de vitesse des vents et la réception accrue de précipitations au sol suite à l'élimination de la canopée, 
menant à une augmentation de l'ensoleillement, de la température et de l'humidité du sol \citep{Keenan1993ecologicaleffects,Lindo2003Microbialbiomass,Heithecker2007Edgerelatedgradients}.
De plus, la disponibilité de nutriment au niveau sol ainsi que la compaction de celui-ci peuvent être affecté par l'application de coupe forestières \citep{Battigelli2004Shorttermimpact,Covington1981Changesforest,Lindo2003Microbialbiomass,rousseauLongtermEffectsBiomass2018}. 
Ultimement, les changements environnementaux résultant d'une coupe forestière auront un impact sur la biodiversité forestière \citep{Chaudhary2016Impactforest,Fedrowitz2014Canretention,Paillet2010Biodiversitydifferences}, 
tout particulièrement la faune vivant au niveau des sols \citep{Chaudhary2016Impactforest,Lindo2003Microbialbiomass,Kudrin2023metaanalysiseffects}.

La faune du sol possède une importance primordiale dans les écosystèmes forestiers, en contribuant notamment à la circulation de la matière et de l'énergie à travers la chaîne alimentaire, ainsi qu'au recyclage des nutriments \citep{Kudrin2023metaanalysiseffects}.
Ces communautés animales sont ainsi couramment étudié en tant qu'indicateurs de l'état des forêts et du niveau de perturbation \citep{birdChangesSoilLitter2004,Maleque2009Arthropodsbioindicators,pongeVerticalDistributionCollembola2000}.

De nombreuses recherches se sont interréssé dans le passé aux effets des coupes forestières sur la faune du sol. 
Cependant, la plupart discutent des effets directs des perturbations sur un ou plusieurs groupes d'espèces, 
sans tenir compte des relations existantes entre les variables environnementales et les différents groupes d'espèces, 
négligeant ainsi les effets indirectes des coupes sur la faune du sol \citep{graceStructuralEquationModeling2008,josephIntegratingOccupancyModels2016,Kudrin2023metaanalysiseffects,Pollierer2021Diversityfunctional}. 

Il est cependant indispensable de comprendre quels sont les effets des coupes forestières sur la faune du sol, 
comment ces effets se propages à l'intérieur du réseau écologique forestier et connaitre l'ensemble des répercussions sur la structure 
et la dynamique des communautés d'organismes du sol. 
Ultimement, cette acquisition de connaissances fournira des outils précieux pour faciliter la gestion durable des forêts.


\section*{Espèces à l'étude}
\label{sec:species}
\phantomsection\addcontentsline{toc}{section}{\nameref{sec:species}}

Les groupes d'espèces choisit pour étudier l'effet des coupes sur la dynamique de la faune du sol sont la salamandre cendrée de l'Est (\textit{Plethodon cinereus}), 
les carabes (Carabidae) et les collemboles (Collembola).


\section*{Objectifs et hypothèses}
\label{sec:objectifs}
\phantomsection\addcontentsline{toc}{section}{\nameref{sec:objectifs}}

Le but de notre étude est de comprendre l'impact des traitements sylvicoles, effectuée en vue d'une migration assistée d'arbres en forêt mixte, sur la dynamique des écosystèmes du sol forestier.
L'objectif est de quantifier les impacts des traitements de coupes forestières sur la faune du sol et son habitat. 
Le moyen d'atteindre cet objectif est de construire un modèle d'équations structurelles pour mesurer les effets directs et indirects des traitements de coupes coupe totale et de coupe partielle sur : 

\begin{enumerate}
    \item Le relations de cooccurences entre la salamandres cendrée de l'Est (\textit{Plethodon cinereus}), les carabes (Carabidae) et les collemboles (Collembola).
    \item Les attributs forestiers propices à la sélection d'habitats par la faune du sol tel que le volume de débris ligneux, la profondeur de litière ainsi que l'ouverture de la canopée. 
\end{enumerate}

% Normalement avant de formuler des hypothèses tu fais une genre de revue de littérature (courte) qui fait un peu le tour de la question 
% et qui identifie les besoins de connaissances. J’imagine que ca va venir. (Mathieu)

L'hypothèse 1.1 liée à notre premier sous-objectif atteste que les traitements de coupes forestières entraînent un changement dans l'utilisation de l'habitat par la faune du sol en se propageant via le réseau trophique. 
Spécifiquement, les coupes affectent l'utilisation d'habitat des salamandres et des carabes de grandes tailles et modifie par la suite la sélection d'habitat de carabes de petite taille puis enfin des collemboles. 

L'hypothèse 1.2 associé à notre deuxième sous-objectif soutient que les traitements de coupes forestières influencent directement les attributs environnementaux des forêts important pour la faune du sol, 
tel qu'un apport en débris de coupe issus des opérations forestières ou un changement de luminosité associé à l'ouverture du couvert.
% hypothèse sur les changements


\cleardoublepage

\bibliography{References.bib}
\bibliographystyle{ecologyNewFR.bst}
