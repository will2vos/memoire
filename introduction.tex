\chapter*{Introduction générale}         % ne pas numéroter
\label{chap-introduction}       % étiquette pour renvois
\phantomsection\addcontentsline{toc}{chapter}{\nameref{chap-introduction}} % inclure dans TdM

% \usepackage[french]{babel}

\section*{Mise en contexte}
\label{sec:contexte}
\phantomsection\addcontentsline{toc}{section}{\nameref{sec:contexte}}

% 1. **Contexte général : Importance des forêts et de la biodiversité**
%    - Présentation des écosystèmes forestiers comme des réservoirs de biodiversité essentiels à la stabilité environnementale.
%    - Rôle des forêts dans la fourniture de services écosystémiques : séquestration du carbone, régulation du climat, conservation de l'eau, et soutien à la biodiversité.

Les écosystème forestiers jouent un rôle essentiel dans la biosphère à travers leur rôle économique et leur valeur écosystèmiques en fournissant une multitude de produit et services \citep{Balvanera2006Quantifyingevidence}. 
Leur présence permet de réguler les flux de nutriments et d'énergie, notamment à travers la séquestration du carbone, la régulation du climat, la rétention de l'eau et la conservation de la biodiversité \citep{Balvanera2006Quantifyingevidence,Diaz2006BiodiversityLoss,Canadell2008Managingforests,Pawson2013Plantationforests}. 

% 3. **Impacts potentiels des coupes forestières sur la biodiversité**
%    - Dérèglement des structures des communautés animales et végétales après une coupe.
%    - Effets sur la fragmentation des habitats, la modification de la composition et la structure du peuplement forestier.
%    - Rôle clé des coupes dans la réduction de la diversité spécifique et fonctionnelle à différents niveaux trophiques.

La diversité des espèces forestières est désormais reconnue comme essentielle pour le bon fonctionnement des écosystèmes terrestres, cependant les activités humaines et l'utilisation des terres entraînent des changements majeurs qui affectent cette biodiversité \citep{Newbold2015Globaleffects}.
Par ailleurs, les engagements internationaux ont souligné l'urgence de freiner la perte de biodiversité tout en promouvant une gestion durable des forêts \citep{Scherer-Lorenzen2005ForestDiversity,Parviainen2007Maintenanceconservation}. 
En effet, la croissance démographique mondiale et l'accroissement des besoins en produits forestiers et autres services ont conduit à une intensification des pratiques d'exploitation forestière au cours des dernières décennies \citep{Foley2005GlobalConsequences}. 
Plus précisement, les modifications de l'utilisation des terres, telles que les récoltes forestières, l'agriculture et l'urbanisation, sont des facteurs majeurs associées à la perte de biodiversité et à la réduction des écosystèmes forestiers \citep{Sala2000Globalbiodiversity,Naeem2012functionsbiological,Bichet2016Maintaininganimal}.
Un défi essentiel en biologie de la conservation est aujourd'hui de préserver la biodiversité face aux impacts croissants des activités humaines sur les habitats fauniques. 
L'exploitation de bois permettant de répondre aux besoins industriels est ainsi reconnue pour participer au déclin de nombreuses espèces qui dépendent des habitats forestiers \citep{Bengtsson2000Biodiversitydisturbances}. 
Des recherches ont notamment démontré les effets des coupes forestières sur divers groupes taxonomiques, tels que les oiseaux, chauves-souris, papillons, tortues, petits mammifères et insectes \citep{Summerville2011Managingforest,Currylow2012ShortTermForest,Kaminski2013EffectsForest,Kellner2013Shorttermresponses,Caldwell2019ComparisonBat}.

% 2. **Pressions exercées par les activités humaines sur les forêts**
%    - Mention des pratiques d’exploitation forestière et leur intensification au fil du temps.
%    - Présentation des types de coupes forestières : coupes à blanc, coupes partielles, sélectives, etc.
%    - Implication croissante des coupes forestières dans la gestion durable des forêts et les objectifs économiques.

L'exploitation des ressources naturelles par les humains peut entraîner la mortalité des animaux et des plantes, forcer ou limiter le déplacement des espèces, intensifier les interactions biotiques, modifier leurs traits de vie, altérer leur morphologie et leur physiologie, et même affecter leurs formes polymorphiques \citep{Sergio2018Animalresponses}. 
De plus, l'exploitation forestière entraîne des pertes d'habitats et une fragmentation des milieux naturels, ce qui limite l'accès à la nourriture, aux refuges et aux zones de reproduction ainsi que la taille et la diversité génétique des populations \citep{Coelho2020Effectsanthropogenic}.

La gestion forestière peut entraîner une réduction de la connectivité entre les habitats, les communautés et les processus écologiques \citep{Lindenmayer2006Generalmanagement}.
Cette diminution peut avoir un impacts majeur sur la conservation de la biodiversité en compromettant la capacité des populations à se rétablir et à se maintenir après une perturbation \citep{Lamberson1994ReserveDesign}. 
Le manque de connectivité limite les déplacements des individus au sein de leur habitat et des corridors écologique, réduisant ainsi le flux génétique entre populations, ce qui accroît le risque d'extinction locale \citep{Saccheri1998Inbreedingextinction}.

En ramenant les peuplements à un stade de succession précoce, l'exploitation forestière homogénéise le paysage forestier, ce qui conduit à une surreprésentation des forêts en début de succession au détriment des stades plus avancés \citep{Cyr2009Forestmanagement,Boucher2017Cumulativepatterns}. 
Plus précisément, l'exploitation forestière diminue la complexité structurelle, notamment au niveau de la composition des espèces d'arbres, de la stratification verticale, de la structure d'age, de dynamique de succession et de la fréquence de perturbation \citep{Commarmot2005Structurevirgin}. 
L'hétérogénéité structurelle d'une forêt joue un rôle important pour la biodiversité puisqu'elle fournit une plus grande variété d'habitat et participe à la connectivité en favoridant la dispersition de certaine espèces. 
Lors d'une perturbation naturelle ou artificiel, un peuplement avec une structure plus hétérogène se régénérera plus rapidement, car il favorise la survie des espèces présentes et facilite le retour des espèces qui ont disparu. 
Ainsi, la récolte forestière peut menacer les espèces associées aux attributs des vieilles forêts, ainsi que les organismes ayant une faible capacité de dispersion, pour lesquels le risque d'extinction augmente rapidement lorsque la continuité écologique est rompue \citep{Norden2001Conceptualproblems,Martin2021indicatorspecies}.  
Ces altérations sont prévues pour être particulièrement néfastes pour la biodiversité, notamment pour les espèces associées aux forêts mixtes et conifériennes plus anciennes \citep{Tremblay2018Harvestinginteracts,Cadieux2020Projectedeffects}.

Ces changement dans les attribut forestiers participe une perte de la diversité spécifique et fonctionnelle à différents niveaux trophiques. 
Sur le long terme, cela peut entrainer une diminution de la résilience des forêt à l'échelle locale et engendrer une diminution des services écosystémiques et avoir un impacte négatif pour les sociétés \citep{Hooper2012globalsynthesis,Edwards2014Maintainingecosystem}. 

Parmis les traitements sylvicoles pouvant avoir un effets majeur sur les milieux naturels on retrouve les traitements de coupes forestières.
Cependant, l'objectif économique et l'impacts écologique des coupes forestières varie selon le type traitement, allant d'un niveau de pertubation naturel ou semi-naturelle dans le cas de la gestion forestière extensive à artificiel dans le cas gestion forestière intensive \citep{Ameray2021Forestcarbon}. 

Historiquement, les coupes totales font partie des pratiques sylvicoles les plus courantes dans les forêts tempérés et boréaux \citep{Fedrowitz2014Canretention,Chaudhary2016Impactforest}. 
Elles font partie d'une gestion forestière intensive largement utilisée pour accroître la productivité et la qualité du bois à court terme, afin de répondre aux besoins croissants de l'industrie et d'augmenter la rentabilité \citep{Irland2011TimberProductivitya}.
Ces coupes implique l'abattage de tous les arbres dans une zone définis.
Elles sont techniquement facile à exécuter, car l'ensemble du couvert supérieur de la parcelle est retiré en une seule récolte, entraînant une déforestation temporaire d'une zone auparavant boisée. 
Les coupes totale utilise une structure équienne avec une seule espèce, ce qui simplifie la structure forestière et réduit la diversité biologique entrainant une homogénéisation des peuplements \citep{Rosenvald2008whatwhen}. 
Cette simplification du milieu perturbe des processus écologiques et évolutifs importants et entraine un déclin de la résilience des forêts \citep{Holling2001UnderstandingComplexity}. 
De plus, la période de rotation est plus courte pour ce type de traitement ce qui amène un fréquence des perturbation plus élevé. 
Le maintient de pratiques de gestion forestière transformant radicalement les structures des écosystèmes par rapport à celles observées naturellement augmente de façon importante les probabilités d'observer un déclin progressif de la biodiversité ainsi que l'extinction locale d'espèces \citep{Hanski2000Extinctiondebt}.  
Cependant, certains auteurs souligne que les coupes totales peuvent être utiliser pour imitation des perturbations naturelles majeures \citep{Greenberg1995comparisonbird}. 

Ces dernières décennies, des pratiques sylvicoles combinant la récolte de bois et la préservation de la biodiversité ont été promues pour atténuer les impacts des coupes totales \citep{Gustafsson2012Retentionforestry}.
Une gestion écosystemique caractérisé par l'émulation des perturbations naturelles a été proposée comme une stratégie prometteuse pour une gestion durable des forêts \citep{Perry1998scientificbasis,Kuuluvainen2002Naturalvariabilitya}. 
Selon cette stratégie, les actions de gestion sont planifiées de manière à émuler les perturbations naturelles et leurs résultats, y compris les structures des peuplements et les successions. 
La gestion écosytémique a pour but de préserver la biodiversité et de maintenir la résilience des peuplements, tout en garantissant la disponibilité d'une grande variété de services écosystémiques \citep{Szaro1998emergenceecosystem,MacDicken2015Globalprogress}.

Les coupes partielles font partie d'une gestion écosystémique visant à préserver la composition et la structure forestières \citep{Bergeron1999Forestmanagementa}.
Elles sont habituellement utilisées dans le cadre d'un plan d'aménagement extensif qui favorise la régénération naturelle et reproduit les perturbations naturelles \citep{Irland2011Timberproductivity}. 
Les coupes partielles consistuent une suppression sélective d'arbres tout en laissant une partie du peuplement intacte \citep{Ameray2021Forestcarbon}. 
La rétention d'arbres dans les coupes partielles offre des structures de succession tardive, favorisant ainsi la préservation d'une plus grande biodiversité \citep{Ameray2021Forestcarbon}.
Elles sont souvent employées pour stimuler la croissance des arbres les plus vigoureux, encourager la diversité des espèces ou préserver une canopée ouverte \citep{Irland2011Timberproductivity}.
Ce type de traitement repose sur une structure multi-âge, avec ou sans mélanges d'espèces, et est caractérisée par des rotations plus longues \citep{Kuuluvainen2009Forestmanagement}. 
La conservation d'arbres en coupe partielle et l'allongement de la période de rotation offrent d'autres avantages sur le plan économique et écosystémique. 
Cela favorise la séquestration du carbone, préserve l'apport de matière organique ainsi que le cycle de nutriment, tout en maintenant un structure hétérogène et différentes niches écologiques pour la faune \citep{Barg1999Influencepartial,Tong2020Forestmanagement,Ameray2021Forestcarbon}.

Comme mentionné précédemment, l'impact de la gestion forestière sur la biodiversité dépend du type de traitement sylvicole utilisé. Cependant, la réaction de la faune à un traitement peut varier selon le type de forêt et les groupes d'espèces concernés \citep{Paillet2010Biodiversitydifferences,Kudrin2023metaanalysiseffects}.

% réalité des changement climatiques

Le défis de préserver la biodiversité et les écosystèmes forestiers tout en répondant aux exigences économiques est d'autant plus grand dans le contexte actuelle des changements climatiques. 
La hausse globale de température représente une menace de plus pour la pérennité de la faune et de la flore en modifiant de façon importantes les conditions environnementales \citep{McKenney2009Climatechange,Trumbore2015Foresthealth,Seidl2017Forestdisturbances,Messier2022Warningnatural}. 
Parmi ces modifications, on prévoit un allongement et une intensification des périodes de sècheresse, une augmentation du nombre de feux de forêt, une altération des régimes de précipitation et une hausse des perturbations biotiques \citep{Parmesan2007Influencesspecies,Joyce2013Climatechange,Gatti2021Amazoniacarbon,Heidari2021Effectsclimate}. 
À cela s'ajoutent des changements dans la phénologie ainsi que dans la distribution des arbres due au manque d'adaptation des végétaux aux nouvelles conditions de leur milieu \citep{Aitken2008Adaptationmigration,Chuine2010Whydoes,Zhu2012Failuremigrate,Gray2013Trackingsuitable}.
De plus, les stresseurs climatiques agissent de façon additive ou synergique avec les activités forestières et l'interactions entre ces facteurs exerce un impact beaucoup plus grand sur la biodiversité et l'environnement \citep{Brook2008Synergiesextinctiona,Tremblay2018Harvestinginteracts,Ochs2022Responseterrestrial,Bouderbala2023Longtermeffect}. 
La composition des forêts pourrait ainsi être altérée, entraînant des ajustements dans les pratiques d'aménagement forestier et les stratégies de conservation \citep{McKenney2009Climatechange,Chmura2011Forestresponses,Lo2011Linkingclimate}.
Différentes mesures d'atténuation ont été proposées pour prévenir la perte des forêts et améliorer la résilience de celle-ci comme l'augmentation de la diversité fonctionnelle et l'amélioration de la connectivité à travers le paysage forestier \citep{Messier2019functionalcomplex}.
D'autres solutions tel que la migration assistée d'arbres ont été proposée comme une mesure d'atténuation permettant le déplacement d'individus ou de matériel génétique depuis un territoire climatique originel vers une zone de climat futur plus propice à la croissance des arbres \citep{Vitt2010Assistedmigration,Dumroese2015Considerationsrestoring,Park2018Informationunderload,Park2023Provenancetrials}. 
La migration assistée d'arbre permettrait de modifier rapidement la composition des peuplements pour mieux convenir au climat futur de celui-ci répondant ainsi aux besoins de conservation, maintenant les services écosystémiques et préservant la valeur économique \citep{Pedlar2011implementationassisted,Ste-Marie2011Assistedmigration,Winder2011Ecologicalimplications}.
Cependant un manque de connaissances et un degré d'incertitude subsistent toutefois autour des nouvelles mesures d'adaptation \citep{Klenk2015assistedmigration,Park2018Informationunderload}. 
Il est donc essentiel de comprendre l'impact des traitements sylvicoles sur la biodiversité, dans un contexte où l'aménagement forestier doit adapter sa pratique aux changements climatiques et aux déclin de la biodiversité, tout en répondant aux besoin économiques.



%restant
Les coupe totales et les coupes partielle font partie des coupes les plus courrament utiliser dans l'amenagement forestiers et répondent  à différents besoins en terme de gestion \citep{Man2008Elevenyearresponses,Chaudhary2016Impactforest,MontoroGirona2018ConiferRegeneration,Ameray2021Forestcarbon}.
Dans l'ensemble, il apparaît clairement que pour imiter les perturbations partielles et à petite échelle, une gestion visant à maintenir un couvert forestier continu et une hétérogénéité structurelle sur une grande partie du paysage est nécessaire \citep{Kuuluvainen2009Forestmanagement}.  
Le maintien de l'hétérogénéité des paysages.
Les écosystèmes sont naturellement hétérogènes, et l'hétérogénéité des paysages est une caractéristique des forêts naturelles à travers le monde. 
Les régimes de perturbation peuvent créer une couverture terrestre hétérogène, avec par exemple différentes étapes de succession suivant un incendie de forêt (Whelan et al., 2002). 
En outre, les paysages sont caractérisés par des gradients environnementaux naturels (comme la topographie, le climat, ou le type et la profondeur du sol ; voir Austin et Smith, 1989). 
L'hétérogénéité des paysages correspond à une mosaïque de parcelles représentant différentes compositions et classes d'âge forestières, où des conditions structurelles variées existent (Forman, 1995). 
Différentes espèces habitent dans diverses conditions environnementales des paysages naturels, et la diversité, la taille et la disposition spatiale des parcelles d'habitat sont importantes pour de nombreuses taxons (e.g., Hanski, 1994 ; Saab, 1999 ; Debinski et al., 2001).
L'utilisation des connaissances sur les régimes de perturbation naturelle dans les forêts naturelles pour guider les pratiques de gestion forestière hors réserve. 
Les stratégies de conservation de la biodiversité sont plus susceptibles de réussir lorsque les régimes de perturbation humaine (comme l'exploitation forestière) ont des effets similaires à ceux des perturbations naturelles (Hunter, 1993 ; Korpilahti et Kuuluvainen, 2002), par exemple, les types et nombres de legs biologiques (sensu Franklin et al., 2000) et les schémas spatiaux des conditions environnementales (par exemple, types de parcelles) qui restent après la perturbation (Delong et Kessler, 2000). 
Les organismes sont probablement mieux adaptés aux régimes de perturbation dans lesquels ils ont évolué (Bergeron et al., 1999 ; Hobson et Schieck, 1999), mais peuvent être vulnérables à des formes de perturbation nouvelles (ou des combinaisons de perturbations) qui sont plus ou moins fréquentes et/ou plus ou moins intensives que celles qui se produiraient normalement (Lindenmayer et McCarthy, 2002). 
Par conséquent, les régimes de perturbation naturelle peuvent être des bases de référence appropriées et des plages de variabilité contre lesquelles les régimes de perturbation humaine peuvent être comparés (Hunter, 1993 ; Angelstam et al., 1995).
De même, la création et/ou le maintien de la complexité structurelle des peuplements est essentielle pour la conservation de la biodiversité dans toutes les forêts, y compris celles ayant une longue histoire de gestion (par exemple, Linder et Ostlund, 1998). 
En outre, l'hétérogénéité des paysages est préférable à une gestion forestière intensive qui résulte en une homogénéité des paysages (Lindenmayer et Fischer, en presse).

Aujourd'hui, le besoin de récolter les ressources naturelles pour répondre au besoin indutriel et la volonté de préserver une certaine qualité environnementale excèdent ce que les terres disponibles peuvent offrir avec les méthodes d'exploitation traditionnelles \citep{Sala2000Globalbiodiversity,Newbold2015Globaleffects}. 

% a ajouter

La coupe totale, qui est historiquement la pratique sylvicole la plus commune dans les biomes tempérés et boréaux [11], peut entraîner des changements importants dans les conditions environnementales. 
Cela inclut des modifications de la lumière, de l'humidité, de la vitesse du vent et d'autres conditions qui peuvent limiter la biote forestière, par exemple [16].

La coupe partielle est un autre type de traitement dans laquelle certaines parties des arbres sont laissées sur place pour maintenir un apport en matière organique et préserver le cycles des nutriments, tout en offrant un refuge aux organismes souterrains \citep{Dahlgren1994effectswholetree,Barg1999Influencepartial}. 
Plusieurs étude in observer un diminution du niveau de perturbations des coupes partielle  comparativement au coupe total dans la biodiversité, notamment au niveau de l'abondance, de la richesse spécifique et de la compositon spécifique \citep{Kudrin2023metaanalysiseffects}. 
La coupe partielle est considérée comme ayant un impact négatif moins marqué sur le milieu par rapport à la coupe total \citep{Fedrowitz2014Canretention,Chaudhary2016Impactforest}, 
car elle entraîne moins de modifications des conditions hydrothermales du sol, préserve la biomasse microbienne, 
et maintient des refuges qui conservent la structure, la composition et les caractéristiques fonctionnelles des forêts non perturbées \citep{Londo1999Forestharvesting,Gustafsson2012Retentionforestry,Holden2013metaanalysissoil}. 
Toutefois, l'effet du type de récolte sur la faune du sol varie considérablement selon les groupes taxonomiques.

Nos résultats suggèrent également que la coupe partielle pourrait avoir des effets moins dramatiques sur la faune du sol que la coupe à blanc, 
ce qui confirme le potentiel de cette pratique forestière pour réduire les perturbations dans les écosystèmes forestiers \citep{Kudrin2023metaanalysiseffects},



% 4. **Focus sur la faune du sol : une composante souvent négligée de la biodiversité**
%    - Présentation des espèces de la faune du sol (invertébrés, micro-organismes, etc.) et de leur rôle fondamental dans le fonctionnement des écosystèmes forestiers : décomposition de la matière organique, recyclage des nutriments, régulation des populations.
%    - Importance de la faune du sol comme bio-indicateurs sensibles aux perturbations environnementales.

\section*{Espèces à l'étude}
\label{sec:species}
\phantomsection\addcontentsline{toc}{section}{\nameref{sec:species}}

% faune du sol
Au sein de la biodiversité forestières, la faune du sol possède une importance primordiale dans les écosystèmes forestiers, en contribuant entre autres à la circulation de la matière et de l'énergie à travers la chaîne alimentaire, ainsi qu'au recyclage des nutriments \citep{Seibold2021contributioninsects,Kudrin2023metaanalysiseffects}.
Cependant cette communauté fait partie des groupes d'espèces les plus affecté par les pertubations environnementales au sein de la biodiveristé forestières \citep{Marshall2000Impactsforest,Coyle2017Soilfauna}. 
En raison de leur petite taille, la faune du sol possède une capacité de dispersion plus restreinte, ce qui les rend plus vulnérables face aux perturbations de leur environnement \citep{Kudrin2023metaanalysiseffects}.

% impacte des traitement sur la faune du sol

Les pratiques sylvicoles comme les coupes forestières peuvent induisent des modifications brusque et drastique dans les propriétés de l'habitat forestiers. 
Ces altérations peuvent affecter négativement la biodiversité, particulièrement la faune vivant à la hauteur du sol \citep{Lindo2003Microbialbiomass,Paillet2010Biodiversitydifferences,Fedrowitz2014Canretention,Chaudhary2016Impactforest}. 
Le retrait de la canopée augmente l'exposition de la surface du sol au rayonnement solaire, ce qui entraîne une élévation de la température et une modification de l'humidité du sol \citep{Lindo2003Microbialbiomass,Brook2008Synergiesextinction,Zhang2022Intensiveforest}. 
Ces changements sont exacerbés par une augmentation de la vitesse du vent et une intensification des précipitations atteignant le sol \citep{Keenan1993ecologicaleffects,Heithecker2007Edgerelatedgradients}. 

La structure du sol subit également des altérations, notamment une compaction accrue due au passage des machines forestières, ce qui peut affecter la porosité du sol et, par conséquent, la faune qui y vit \citep{Battigelli2004Shorttermimpact,Mazerolle2021Woodlandsalamander}. 
De plus, les opérations forestières perturbent la disponibilité des nutriments en modifiant la qualité et la quantité de litière, en altérant les sécrétions racinaires, en provoquant un lessivage, et en changeant les propriétés chimiques du sol \citep{Covington1981Changesforest,Marshall2000Impactsforest,Lindo2003Microbialbiomass,Battigelli2004Shorttermimpact}. 
Ce bouleversement des conditions microclimatiques et physico-chimiques du sol peut entraîner des répercussions directes sur la biodiversité, 
en particulier sur les organismes qui dépendent des microhabitats comme le bois mort, les cavités dans les arbres matures ou les plaques racinaires \citep{Berg1994ThreatenedPlant,Spies1999Dynamicforest,Bouget2005Shorttermeffect,Christensen2005Deadwood,Brassard2008EffectsForest}.
Ces éléments structurels, souvent éliminés ou réduits lors des opérations de coupe sont généralement remplacés par des pistes de débardage et les chemins d'exploitation \citep{Hansen1991ConservingBiodiversity}. 

Les microhabitats jouent pourtant un rôle crucial pour une grande partie de la faune du sol, comme les arthropodes, les amphibiens et les microorganismes \citep{Paillet2010Biodiversitydifferences,Fedrowitz2014Canretention,Chaudhary2016Impactforest}. 
La disparition de ces refuges accroît la vulnérabilité de nombreuses espèces, tout en favorisant l’émergence de conditions environnementales défavorables à leur survie. 
Ainsi, les espèces forestières qui dépendent des conditions de fraîcheur et d’humidité typiques des sols forestiers non perturbés peuvent voir leur population diminuer, voire disparaître localement \citep{Kudrin2023metaanalysiseffects}. 
En revanche, des espèces associées aux zones ouvertes et plus sèches peuvent coloniser les parcelles récemment coupées, modifiant ainsi la composition spécifique des communautés fauniques.

Plusieurs recherches se sont intéressés aux changements des caractéristiques forestières comme le débris ligneux grossiers, la profondeur de la litière, l'ouverture de la canopée 
et leur influence sur la faune du sol après la récolte forestière afin de guider la gestion forestière \citep{Semlitsch2002CriticalElements,McKenny2006Effectsstructural}. 

La faune du sol regroupe cependant une grande diversité de taxons, caractérisés par des différences significatives sur le plan biologique et écologique. 
Par conséquent, leur réponse à l'exploitation forestière varier selon le type de traitement appliqué et le groupe d'espèce étudié \citep{Malmstrom2009Dynamicssoil,Paillet2010Biodiversitydifferences}.

% amphibien et arthopodes

Malgré leur abondance dans les écosystèmes forestiers de l'est de l'Amérique du Nord, les amphibiens et les arthropodes ont longtemps été négligés dans les stratégies de gestion forestière \citep{deMaynadier1995relationshipforest}. 
Or, des recherches ont souligné leur importance écologique, tant pour le fonctionnement des réseaux trophiques que pour la dynamique du carbone au sein des sols forestiers. 
Leur sensibilité aux changements environnementaux en font des modèles d’étude pertinent pour mieux comprendre les effets des pratiques forestières sur la biodiversité et l'intégrité écologique des forêts \citep{pongeVerticalDistributionCollembola2000,birdChangesSoilLitter2004,Maleque2009Arthropodsbioindicators}.
De plus, les amphibiens et les arthropodes ont subi un déclin majeur au cours des dernières décennies, principalement dû aux pratiques d'aménagement forestier et aux changements climatiques \citep{Houlahan2000Quantitativeevidence,Stuart2004Statustrends,Warren2018projectedeffect,Wagner2021Insectdecline}. 

Parmi les groupes d'espèces fréquemment étudiés chez ces taxons, on retrouve la salamandre cendrée de l'Est (\textit{Plethodon cinereus} (Green, 1818)), les carabes (Carabidae) et les collemboles (Collembola).


\subsection*{Salamandre cendrée}

La salamandre cendrée, un membre des Plethodontidae, représente l'une des biomasses les plus importantes chez les vertébrés des forêts nord-américaines \citep{Burton1975Salamanderpopulations,Petranka1993Effectstimber,semlitschAbundanceBiomassProduction2014a}. 
En tant qu'espèce exclusivement terrestre etdépourvue de poumons, elle dépend de la respiration cutanée et donc des conditions d'humidité pour assurer ses échanges gazeux \citep{Heatwole1961Relationsubstrate}. 
Ce mode de respiration la contraint à occuper des microhabitats spécifiques, en surface lorsque la température et l'humidité sont favorables, ou en profondeur dans le sol durant des périodes moins propices \citep{Grizzell1949HibernationSite,FraserEmpiricalEvaluation1976,Jaeger1980MicrohabitatsTerrestrial}. 
Prédateur généraliste, la salamandre cendrée contribue de manière significative à la régulation des invertébrés détritivores, influençant directement les processus de décomposition, la circulation des nutriments dans les solset la dynamique du carbone \citep{Burton1975Energyflow,Wyman1998Experimentalassessment,Walton2013Topdownregulation,Hickerson2017Easternredbacked}. 
Par ailleurs, elle constitue une proie de haute valeur nutritive pour divers prédateurs tels que les oiseaux, mammifères et reptiles, renforçant ainsi son rôle dans les dynamiques trophiques forestières \citep{Burton1975Energyflow,Pough1987abundancesalamanders,Petranka1998SalamandersUnited}. 
Compte tenu de sa sensibilité aux perturbations environnementales, notamment en raison de sa respiration cutanée, la salamandre cendrée est souvent utilisée comme indicateur de la qualité des sols forestiers \citep{Welsh2001caseusing}. 
Son abondance et sa présence sont ainsi des paramètres couramment évalués pour mesurer l'impact des pratiques sylvicoles sur la biodiversité \citep{Harpole1999Effectsseven,Grialou2000effectsforest,Homyack2009Longtermeffects,Hocking2013Effectsexperimental,Mazerolle2021Woodlandsalamander}. 



En raison de contraintes physiologiques, notamment leur peau mince, les salamandres cendrée sont davantage à risque de dessèchement lorsqu'elles sont exposées à des environnements plus secs au niveau du sol forestier. 
Pour faire face à ce stress environnemental, elles restent souvent sous terre, où les sols sont plus frais et l'humidité moins limitée. 

De plus, le retrait des refuges de surface tels que les branches, les cimes d'arbres et autres débris ligneux grossiers après la récolte peut réduire la qualité de l'habitat pour les salamandres 
et diminuer le temps qu'elles passent à la surface du sol forestier \citep{Achat2015Quantifyingconsequences,Peele2017EffectsWoody}. 

En conséquence, étant donné que les salamandres terrestres se nourrissent et se reproduisent à la surface du sol forestier, des conditions de surface sévères, 
un compactage des sols et de faibles niveaux de CWD peuvent affecter la dynamique des populations \citep{Peterman2014Spatialvariation}. 

Des études antérieures dans les forêts de feuillus de l'est de l'Amérique du Nord et de la région des Appalaches ont montré qu'une plus grande rétention de la couverture de la canopée grâce à des approches de 
récolte partielle, telles que la gestion inéquienne, pouvait entraîner des densités relatives plus élevées de salamandres par rapport aux traitements équiens qui enlèvent une plus grande partie de la canopée \citep{Hocking2013Effectsexperimental,Harper2015Impactforestry,Mahoney2016Woodlandsalamander}. 
Une plus grande rétention de la canopée limite la quantité de rayonnement solaire atteignant le sol forestier, ce qui conserve des microclimats plus humides et plus frais favorable pour la physiologie des salamandres \citep{Homyack2011Energeticssurfaceactive}

Une plus grande quantité de CWD peut également atténuer les effets négatifs de la gestion équienne des forêts en fournissant des refuges de surface qui maintiennent des conditions environnementales favorables 
sous les bûches et les rochers, permettant aux individus de rester plus longtemps à la surface \citep{Grover1998Influencecover,Moseley2009Locallandscape}. 

Un résultat commun à toutes les formes de gestion forestière est la réduction de la canopée due à l'enlèvement des arbres, et le pourcentage de couverture de la canopée et 
la densité des salamandres des bois sont souvent fortement positivement corrélés \citep{Tilghman2012Metaanalysiseffects}.



\subsection*{Carabes}

Les carabes sont des coléoptères particulièrement actifs, vivant au niveau du sol forestier \citep{loveiEcologyBehaviorGround1996,Rochefort2006GroundBeetle}.
Cette famille rassemble la plus forte diversité spécifique parmi les coléoptères avec 40 000 espèces identifiées \citep{Erwin1985taxonpulse} 
et représente une des plus grandes abondances au sein des arthropodes vivant au sol \citep{loveiEcologyBehaviorGround1996,Rochefort2006GroundBeetle}.
La majorité des carabes sont carnivores et polyphages en plus d’être des prédateurs voraces \citep{loveiEcologyBehaviorGround1996}. 
Ils agissent comme des régulateurs des populations d’invertébrés dans la chaîne trophique et consomment surtout des aphides, des collemboles et des escargots \citep{loveiEcologyBehaviorGround1996}. 
Les carabes représentent aussi des proies pour plusieurs espèces d’amphibiens, de reptiles, d’oiseaux et de mammifères \citep{loveiEcologyBehaviorGround1996}. 
Ce groupe est largement répandu dans les écosystèmes terrestres et se retrouve dans une grande variété de milieux naturels, tels que les forêts, les cultures, les marais ou encore les sablières \citep{Larochelle2003naturalhistory}. 
Toutefois, la sélection d'habitats varie selon l'espèce, ce qui entraîne une différence de communauté en fonction du type de milieu.
En milieu forestier, les carabes sont le plus souvent classés selon trois types de communautés : les espèces des milieux forestiers matures et fermés, les espèces des milieux ouverts et les espèces généralistes \citep{Niemela2007effectsforestry}. 
Cette variation dans la sélection d'habitats fait des carabes un des taxons les plus intéressants à étudier lors de perturbations environnementales.
Ce groupe a été utilisé dans des travaux portant sur les traitements sylvicoles, notamment les coupes totales \citep{Niemela1993Effectsclearcut,Heliola2001Distributioncarabid,koivulaBorealCarabidbeetleColeoptera2002a}, 
les coupes partielles et l'élagage \citep{Lemieux2004Groundbeetle,mooreEffectsTwoSilvicultural2004,Peck2004Longertermeffects}.
D'autres études se sont interresées à la fragmentation forestière et à la pollution environnementale pour documenter l'utilisation de l'habitats par les carabes et pour évaluer la valeurs nutritives des sols forestiers propices à ces invertébrés \citep{bouchardBeetleCommunityResponse2016b,Luff1992Classificationprediction,Rainio2003Groundbeetles,Work2008Evaluationcarabid}.

\subsection*{Collemboles}

Les collemboles sont un regroupement polyphylétique d'arthropodes faisant partie de la mésofaune établie dans les sols forestiers.
Ces invertébrés comportent une grande richesse spécifique et sont très abondants \citep{rusekBiodiversityCollembolaTheir1998}, 
malgré leur capacité limitée de dispersion due à leur petite taille, leur absence d’ailes et leur niche écologique \citep{Ojala2001Dispersalmicroarthropods}.
Différentes communautés de collemboles occupent un ensemble de niches écologiques allant de la litière aux différents horizons du sol \citep{pongeVerticalDistributionCollembola2000}.
La répartition verticale de ces communautés dépend essentiellement des conditions abiotiques du sol telles que la luminosité, le taux d’humidité ou encore la porosité.
Les collemboles peuvent ainsi servir à caractériser un substrat en fonction de la communauté qu’on y retrouve \citep{rusekBiodiversityCollembolaTheir1998}.
De plus, les collemboles contribuent à la formation de microstructures dans les sols. 
Ce taxon joue un rôle prédominant dans de nombreux processus écologiques. 
Principalement fongivores et détritivores, ces organismes se nourrissent en grande partie de champignons, de bactéries, d'actinomycètes et d'algues. 
Chaque espèce est cependant spécialisée à un type spécifique de ressource alimentaire \citep{Chen1995Foodpreference,rusekBiodiversityCollembolaTheir1998}.
Ces arthropodes contribuent de façon importante à la décomposition de la matière organique, à la transformation de nutriments et 
au transfert d’énergie dans les écosystèmes terrestres \citep{rusekBiodiversityCollembolaTheir1998,Hattenschwiler2005Biodiversitylitter,Cuchta2019importantrole,Marsden2020Howagroforestry}.
Avec les acariens, les collemboles accélèrent de 10 à 20\% la décomposition de la matière organique \citep{Hattenschwiler2005Biodiversitylitter}.
Ils participent également à la chaîne trophique en tant que proie pour plusieurs espèces d’arachnides, de coléoptères, d’amphibiens, de reptiles et d’oiseaux. 
D’autre part, ils servent d’hôtes à des parasites incluant les nématodes, les trématodes, les protozoaires, les bactéries et les champignons \citep{rusekBiodiversityCollembolaTheir1998}.
L'étude des collemboles est pertinente pour ce projet en raison de la relation proie-prédateur existante entre ce taxon, les salamandres et les carabes. 
De plus, le groupe des collemboles est couramment utilisé dans le cadre de recherches portant sur les répercussions des traitements sylvicoles sur la mésofaune \citep{Salmon2008Relationshipssoil,farskaManagementIntensityAffects2014,rousseauWoodyBiomassRemoval2019}.


En résumé, le choix de ces trois groupes d'espèces est pertinent pour étudier l'impact des traitements sylvicoles. 
Leur sensibilité aux perturbations environnementales et leurs relations trophiques en font des candidats idéaux pour analyser l'effet des coupes forestières sur la dynamique de la faune du sol.


Ainsi, différents groupes taxonomiques peuvent réagir de manière différente aux opérations forestières en fonction de leur taille, mobilité ou régime alimentaire \citep{Barlow2007Quantifyingbiodiversity,Stork2009Vulnerabilityresilience}.


La faune constitue une composante essentielle de la biota du sol et joue un rôle clé dans les écosystèmes forestiers en assurant les flux de matière et d'énergie à travers le réseau trophique et en recyclant les nutriments. 
Il est donc impératif de comprendre comment l'exploitation forestière influence la faune du sol.

Leur sensibilité aux changements environnementaux en fait un modèle d’étude pertinent pour mieux comprendre les effets des pratiques forestières sur la biodiversité et l'intégrité écologique des forêts.

% 5. **Effets spécifiques des coupes forestières sur la faune du sol**
%    - Impacts des modifications microclimatiques (température, humidité, lumière) résultant de la perte de la canopée sur les organismes du sol.
%    - Réduction des ressources comme le bois mort et la litière après des coupes intensives.
%    - Perturbation des chaînes trophiques qui influence la dynamique des populations et des interactions dans les sols forestiers.


 \citep{Kudrin2023metaanalysiseffects}

Cependant, les effets des pratiques sylvicoles varient également selon la biologie et l'écologie des espèces. 
Certains groupes d’invertébrés du sol, comme les collemboles et les acariens, sont plus sensibles aux changements d'humidité et de température, tandis que d'autres, comme certains coléoptères, peuvent mieux tolérer ces fluctuations. 
Cette hétérogénéité des réponses complique l'évaluation précise des impacts à long terme sur la biodiversité des sols. 
Par conséquent, une meilleure compréhension des mécanismes écologiques sous-jacents à ces réponses différentielles est nécessaire, afin d'informer les pratiques de gestion forestière qui concilient à la fois les objectifs économiques et la conservation de la biodiversité.


% 6. **Justification de l’étude**
%    - Nécessité d’études plus approfondies sur l’impact des coupes forestières à différentes échelles (spatiales et temporelles) sur la faune du sol.
%    - Lacunes dans la littérature scientifique concernant les réponses des communautés de la faune du sol face aux pratiques sylvicoles.
%    - Importance de mieux comprendre ces interactions pour proposer des recommandations en matière de gestion forestière durable.


Malgré l'importance de la faune du sol et la multitude d'études ayant évalué la réponse de différents groupes d'invertébrés du sol à l'exploitation forestière, il existe encore peu de revues sur cette problématique. 
Marshall a publié une revue traditionnelle sur l'influence de l'exploitation forestière sur les processus biologiques il y a plus de 20 ans, et depuis, de nombreuses nouvelles recherches ont été menées. 
Bien que les méta-analyses évaluant les changements de biodiversité liés à la gestion forestière incluent les invertébrés du sol, elles ne prennent pas en compte les données sur l'abondance et la réaction des différents groupes de faune du sol.


There are urgent needs to identify the most appropriate forest harvesting scenarios to achieve management objectives, such as species recovery [25] or habitat restoration [26].

Ce contexte souligne l'importance de mieux comprendre et minimiser les impacts des pratiques forestières sur la biodiversité, notamment en ce qui concerne la faune du sol, qui joue un rôle essentiel dans le maintien des écosystèmes forestiers.


There is a need to better understand the underlying mechanisms driving responses of soil fauna cooccurence to forest management in order to provide managers with harvesting strategies that are compatible with soil fauna occupancy.
A greater understanding of the mechanistic drivers from forestry prescriptions will allow managers to directly incorporate these measures into their forest planning to help mitigate negative effects from forest management.

Such single-species approaches, however, have often been criticized for their poor efficiency in maintaining biodiversity in managed landscapes (Roberge and Angelstam 2004, Branton and Richardson 2011)

peu d'étude se sont interréssé au interaction entre plusieurs groupe tel que les amphibien et les arthropodes, il sont souvent étudié séparément.

La plupart des travaux qui se sont intéressés aux impacts des traitements sylvicoles sur la faune discutent généralement des effets directs des perturbations sur un ou plusieurs groupes d'espèces, 
sans tenir compte des relations existantes entre les variables environnementales et les différents groupes d'espèces, 
négligeant ainsi les effets indirects des coupes sur la faune du sol \citep{josephIntegratingOccupancyModels2016,Pollierer2021Diversityfunctional,Kudrin2023metaanalysiseffects}. 
Mon projet comblait justement cette lacune en essayant de comprendre comment les effets des traitements sylvicoles se propagent à l’intérieur du réseau écologique forestier et influence la dynamique de la faune du sol.  
Ultimement, ce gain de connaissances fournira des outils précieux pour faciliter la gestion durable des forêts.

% 7. **Objectifs de l’étude**
%    - Identifier et quantifier les effets directs et indirects des différents types de coupes forestières sur les communautés de la faune du sol.
%    - Explorer les changements dans l’utilisation de l’habitat et la structure des populations d’espèces clés du sol après différentes intensités de coupes.
%    - Proposer des pistes pour intégrer ces connaissances dans les pratiques de gestion forestière, tout en maintenant les services écosystémiques essentiels.


\section*{Objectifs et hypothèses}
\label{sec:objectifs}
\phantomsection\addcontentsline{toc}{section}{\nameref{sec:objectifs}}

Le but de mon étude était de comprendre comment les traitements sylvicoles, effectués dans un contexte de migration assistée, 
affecte la dynamique des écosystèmes du sol forestier. Les objectifs qui s’y rattachaient étaient :

\begin{enumerate}
    \item De quantifier l'effet des traitements de coupes forestières sur les variables environnementales qui influencent l'utilisation de l'habitat par la faune du sol.
    \item De mesurer l'impact des coupes forestières sur l'utilisation de l'habitat par la faune du sol.
\end{enumerate}

L'hypothèse 1.1 liée à notre deuxième objectif soutenait que les variables environnementales favorables à l'utilisation de l'habitat par les espèces fluctuent 
en fonction de l'intensité des coupes forestières. Ainsi, les traitements de coupes forestières constituent une variable englobant 
les changements de conditions environnementales.

L'hypothèse 2.1 attachée à notre premier objectif stipulait que les traitements de coupe forestière entraînent une modification de l'utilisation de l'habitat 
par la faune du sol et se propage à travers le réseau trophique. Spécifiquement, les coupes affectent l'utilisation de l'habitat par les salamandres et 
les grands carabes (compétiteurs de salamandres), ce qui modifie ensuite la sélection d'habitat des petits carabes (proies de salamandres), puis enfin des collemboles (proies de grands carabes et salamandres).



\cleardoublepage

\bibliography{References.bib}
\bibliographystyle{ecologyNewFR.bst}
