\chapter*{Remerciements}        % ne pas numéroter
\label{chap-remerciements}      % étiquette pour renvois
\phantomsection\addcontentsline{toc}{chapter}{\nameref{chap-remerciements}} % inclure dans TdM

%\usepackage[french]{babel}

 Ce travail n'aurait pas pris sa forme actuelle sans le soutien que j'ai reçu tout au long de mon parcours de maîtrise.

 Mes remerciements vont d'abord à mon directeur, Marc J. Mazerolle, pour m'avoir accueilli au sein de son laboratoire de recherche et pour son soutien constant tout au long de ce projet. 
 Sa confiance, ses précieux conseils en écologie, en statistique et en gestion de projet, ainsi que sa disponibilité ont été des atouts inestimables.
 Je souhaite également exprimer ma gratitude à mon co-directeur, Mathieu Bouchard, dont le soutien et les conseils dans un domaine que je maîtrisais peu, la foresterie, ont été d'une grande aide. 
 Sa vision du projet et ses recommandations m'ont permis de prendre des décisions plus judicieuses et éclairées.

 Je suis également reconnaissant envers Patricia Raymond et Émilie Champagne du Ministère des Ressources Naturelles et de Forêts, Direction de la recherche forestière, pour leur aide et 
 pour m'avoir permis de participer au projet enrichissant qu'est DREAM-Qc.
 Un grand merci également aux techniciennes Karine Thériault et Marie-Claude Martin pour leur soutien logistique et leurs conseils avisés.
 Je suis reconnaissant pour l'aide que j'ai reçu de la part de mon auxiliaire de terrain, Rebecca Dubé Messier. Sa motivation et son travail assidu ont largement contribué à la qualité de ce mémoire.

 Le parcours d'un projet de maîtrise et toujours plus facile quand on est bien entouré. 
 Pour cela je remercie les membres de mon laboratoire : 
 Aurore Fayard, Laura Millard, Mariano Feldman, Anais Baillet, Lucas Voirin, Félicia Beaulieu, Jeanne Dudemaine, Laurianne Plante et Naomie Herpin Saunier. 
 Votre soutien, nos activités extérieures ainsi que nos discussions ont été des moments précieux pour moi.
 Un merci spécial à Joëlle Spooner, qui a été une amie, une colocataire et une collègue de laboratoire exceptionnelle tout au long de ce parcours. 
 Sa présence a été un véritable soutien.

 Je souhaite également remercier mes parents, Isabelle et Patrick, mes beaux-parents, Michel et Anne-Chantal, ainsi que mes frères, Pierre et Guillaume, pour leur soutien depuis 
 mon plus jeune âge. Votre support dans ma passion pour la biologie est inestimable.
 Je conclus ces remerciements en exprimant ma reconnaissance envers la personne qui m'accompagne au quotidien, Léonie St-Onge. 
 Merci pour ton écoute, ta patience, tes conseils et pour être simplement toi.

