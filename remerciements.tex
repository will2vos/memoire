\chapter*{Remerciements}        % ne pas numéroter
\label{chap-remerciements}      % étiquette pour renvois
\phantomsection\addcontentsline{toc}{chapter}{\nameref{chap-remerciements}} % inclure dans TdM

% \usepackage[french]{babel}

 Ce travail n'aurait pas atteint son niveau actuel sans le soutien que j'ai reçu tout au long de mon parcours de maîtrise. 
 Je tiens à exprimer ma profonde gratitude envers plusieurs personnes qui ont contribué à cette réalisation.

 Tout d'abord, je souhaite remercier mon directeur de thèse, Marc J. Mazerolle, pour avoir cru en moi et m'avoir accueilli dans son laboratoire de recherche. 
 Son accompagnement, sa confiance, ses précieux conseils en écologie, en statistique et en gestion de projet, ainsi que sa disponibilité ont été des atouts inestimables. 
 Je suis également reconnaissant envers mon co-directeur, Mathieu Bouchard, dont le soutien et les conseils dans un domaine que je maîtrisais peu, la foresterie, ont été d'une grande aide. 
 Sa vision du projet et ses recommandations m'ont permis de prendre des décisions éclairées et judicieuses.

 Je tiens à adresser mes remerciements à Patricia Raymond et Émilie Champagne du Ministère des Ressources Naturelles et de Forêts, Direction de la recherche forestière, 
 pour leur appui et pour m'avoir permis de participer au projet enrichissant qu'est DREAM-Qc. 
 Un grand merci également à Karine Thériault et Marie-Claude Martin pour leur soutien logistique et leurs conseils avisés. 
 Je suis reconnaissant envers mon aide de terrain, Rebecca Dubé Messier, dont la motivation et le travail assidu ont largement contribué à la qualité de ce travail.

 Le parcours d'un projet de maîtrise est toujours plus agréable lorsqu'on est bien entouré. 
 Je souhaite donc exprimer ma gratitude envers l'ensemble des membres de mon laboratoire. 
 Votre soutien, nos activités et nos échanges ont été une source d'inspiration et de plaisir quotidien. 
 Je tiens à adresser un merci particulier à Joëlle Spooner, qui a été une amie, une colocataire et une collègue de laboratoire exceptionnelle tout au long de ce parcours. 
 Sa présence a été d'un grand réconfort dans les bons moments comme dans les plus difficiles.

 Je souhaite également remercier mes parents, Isabelle et Patrick, mes beaux-parents, Michel et Anne-Chantal, ainsi que mes frères, Pierre et Guillaume, pour leur soutien indéfectible depuis mes débuts et à travers ma passion pour la biologie et l'écologie. 
 Enfin, je conclus ces remerciements en exprimant ma reconnaissance envers la personne qui m'accompagne au quotidien, Léonie St-Onge. 
 Merci pour ton amour, ton soutien, ta patience, tes conseils et pour être simplement toi.


