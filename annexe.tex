\chapter{Supplementary material for the article}     % numérotée
\label{chap:supp}                   % étiquette pour renvois (à compléter!)

\pagebreak
\section{Carabids and collembola identifications}

\begin{table}[ht]
    \centering
    \caption[List of carabid species captured during summer 2021 in Portneuf Wildlife Reserve and classification between salamanders competitor and prey groups.]
    {List of carabids captured during summer 2021 in Portneuf Wildlife Reserve}
    \label{tab:carabid}
    \begin{tabular}{lll} 
        \hline
        Group & Specie & Count \\ [0.5ex] 
        \hline      
        Competitor          & \textit{Chlaenius sericeus}               & 2 \\  
                            & \textit{Harpalus erythropus}              & 1 \\
                            & \textit{Harpalus faunus}                  & 1 \\
                            & \textit{Harpalus rufipes}                 & 1 \\
                            & \textit{Patrobus longicornis}             & 1 \\
                            & \textit{Platynus decentis}                & 2 \\
                            & \textit{Poecilus lucublandus}             & 3 \\
                            & \textit{Pterostichus adstrictus}          & 27 \\
                            & \textit{Pterostichus coracinus}           & 32 \\
                            & \textit{Pterostichus diligendus}          & 3 \\
                            & \textit{Pterostichus mutus}               & 10 \\
                            & \textit{Pterostichus pensylvanicus}       & 18 \\
                            & \textit{Pterostichus punctatissimus}      & 1 \\
                            & \textit{Pterostichus tristis}             & 25 \\
                            & \textit{Sphaeroderus canadensis}          & 6 \\
                            & \textit{Sphaeroderus nitidicollis}        & 1 \\
        Prey                & \textit{Agonum affine}                    & 8 \\ 
                            & \textit{Agonum palustre}                  & 3 \\
                            & \textit{Agonum punctiforme}               & 2 \\ 
                            & \textit{Agonum retractum}                 & 3 \\ 
                            & \textit{Bradycellus lugubris}             & 1 \\
                            & \textit{Bradycellus semipubescens}        & 1 \\
                            & \textit{Calathus gregarius}               & 1 \\
                            & \textit{Harpalus providens}               & 1 \\
                            & \textit{Loricera pilicornis}              & 1 \\
                            & \textit{Notiobia terminata}               & 5 \\
                            & \textit{Pseudamara arenaria}              & 7 \\
                            & \textit{Pterostichus commutabilis}        & 1 \\
                            & \textit{Synuchus impunctatus}             & 20 \\
                            & \textit{Trechus apicalis}                 & 1 \\
        \textbf{Total}      &                                           & 189 \\
        \hline
    \end{tabular}
  \end{table}

  \begin{table}[ht]
    \centering
    \caption[List of sringtails order and families]
    {List of sringtails order and families captured during summer 2021 in Portneuf Wildlife Reserve.}
    \label{tab:springtail}
    \begin{tabular}{lll} 
        \hline
        Order & Family & Count \\ [0.5ex] 
        \hline      
        Entomobryomorpha    & Entomobryidae     & 62 \\  
                            & Isotomidae        & 182 \\
                            & Tomoceridae       & 8 \\
        Poduromorpha        & Hypogastruridae   & 92 \\
                            & Neanuridae        & 11 \\
                            & Onychiuridae      & 33 \\
                            & Tullbergidae      & 24 \\
        Symphypleona        & Dicyrtomidae      & 6 \\
                            & Katiannidae       & 6 \\
                            & Neelidae          & 23 \\
                            & Sminthuridae      & 10 \\
                            & Sminthurididae    & 11 \\
        \textbf{Total}      &                   & 468 \\
        \hline
    \end{tabular}
  \end{table}

\chapter*{Bibliographie}         
\label{chap:biblio}         
\phantomsection\addcontentsline{toc}{chapter}{\nameref{chap:biblio}} 

\nocite{*}
\renewcommand{\bibsection}{}
\begin{otherlanguage}{english}
\bibliography{References}
\bibliographystyle{ecologyNewEN}
\end{otherlanguage}
