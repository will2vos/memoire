\chapter*{Abstract}             % ne pas numéroter
\label{chap-abstract}           % étiquette pour renvois
\phantomsection\addcontentsline{toc}{chapter}{\nameref{chap-abstract}} % inclure dans TdM

Forest environments play a major role on a global scale through their ecosystem functions and the economic benefits they provide. 
However, human population growth and increasing demand for wood products are leading to increased logging. 
These changes can lead to habitat loss essential for biodiversity and disrupt ecological dynamics between species inhabiting these ecosystems. 
This study aims to quantify the effects of silvicultural treatments, such as clearcutting and partial cutting, on soil fauna dynamics in temperate mixed forests. 
The objectives were, first, to assess the impact of logging on environmental variables influencing soil fauna habitat use and, second, 
to evaluate the direct and indirect effects of these treatments on soil amphibians and arthropods through trophic networks. 
We hypothesized that environmental variables associated with small fauna habitats (volume of coarse woody debris, canopy openness, and litter depth) vary depending on the intensity of logging. 
Thus, harvesting treatments would act as an overarching variable encompassing changes in environmental conditions. 
According to our second hypothesis, forest harvesting treatments would modify habitat use by Eastern red-backed salamanders (\textit{Plethodon cinereus}), ground beetles (\textit{Carabidae}), and springtails (\textit{Collembola}), 
either directly or through trophic interactions from predators to prey. 
To test these hypotheses, we developed a structural equation model (SEM) integrating occupancy models and linear mixed models. 
This approach allowed us to simultaneously measure, in a network framework, the effects of forest logging on environmental variables and the cooccurrence between different taxa. 
Our results show that harvesting treatments significantly influence environmental variables. 
Clearcutting led to greater canopy openness, reduced litter depth, and decreased coarse woody debris volume,  
whereas partial cutting allowed for better retention of these environmental attributes. 
Overall, silvicultural treatments had no significant direct effects on the occupancy probabilities of salamanders and ground beetles or the biomass of springtails. 
However, we measured a slightly lower occupancy probability for salamanders in clearcuts compared to partial cuts, although this effect was marginal (90\% CI). 
Moreover, no changes in cooccurrence between these groups were observed, suggesting that the effects of logging do not propagate through trophic networks from predators to prey. 
Our study highlights the relevance of structural equation models for analyzing complex trophic networks of forest ecosystems. 
This research deepens our understanding of the relationships between silvicultural practices and the dynamics of forest soil fauna. 
\hl{The environmental variables influencing the habitat of the studied species (coarse woody debris volume, litter depth, and canopy openness) vary depending on the intensity of logging. 
We therefore recommend promoting harvesting methods that minimize habitat disturbance, such as partial cuts. 
These practices help preserve key environmental attributes, benefiting soil fauna.} 
The knowledge gained from this study provides valuable tools for guiding management plans and adapting silvicultural practices. 

\begin{otherlanguage*}{english}
  \textbf{Keywords:} \textit{Plethodon cinereus}, ground beetle, springtail, structural equation modelling, cooccurrence, logging.
  
\end{otherlanguage*}
