\chapter*{Conclusion générale}           % ne pas numéroter
\label{chap-conclusion}         % étiquette pour renvois
\phantomsection\addcontentsline{toc}{chapter}{\nameref{chap-conclusion}} % inclure dans TDM

Le but de cette étude était de comprendre comment les traitements sylvicoles, effectués dans un contexte de migration assistée, affectent la dynamique des écosystèmes du sol forestier.
Les objectifs qui s’y rattachaient étaient, d'une part, de quantifier l'effet des traitements de coupes forestières sur les variables environnementales qui influencent 
l'utilisation de l'habitat par la faune du sol et, d'autre part, de mesurer l'impact des coupes forestières sur l'occupation de cet habitat par la faune du sol.
Pour répondre à ces deux objectifs, nous avons développé un modèle d'équations structurelles pour mesurer l'effet des coupes sur le volume de débris ligneux, 
la profondeur de litière et l'ouverture de la canopée, ainsi que sur la probabilité d'occupation des salamandres cendrées, des carabes et sur la biomasse de collemboles.
Notre première hypothèse stipulait que les variables environnementales favorables à l'utilisation de l'habitat par les espèces fluctuent 
en fonction de l'intensité des coupes forestières et qu'ainsi, les traitements de coupes forestières constituent une variable englobant les changements de conditions environnementales. 
Notre seconde hypothèse soutenait que les traitements de coupe forestière entraînent une modification de l'utilisation de l'habitat par la faune du sol et se propagent à travers 
le réseau trophique des salamandres vers les collemboles, en passant par les carabes. 
Les principales conclusions tirées de cette étude sont :

\begin{enumerate}
  \item Les coupes forestières ont un effet significatif sur les variables environnementales influençant l'utilisation de l'habitat par la faune du sol. Ces changements suivent de manière générale le niveau d'intensité du traitement appliqué, avec des perturbations plus importantes dans les coupes totales, suivi des coupes partielles.
  \item La probabilité d'occupation des salamandres et des carabes ne semble globalement pas affectée par les différents traitements de coupes forestières comparativement aux sites témoins. Toutefois, un faible effet a été mesuré chez les salamandres où la probabilité d'occupation de celles-ci était plus faible dans les coupes totales par rapport aux coupes partielles.
  \item La biomasse des collemboles ne variait pas de façon significative entre les différents traitements sylvicoles.
  \item La relation de cooccurence entre les trois taxons ne semble pas subir de changement en fonction du type de coupe. Ainsi, la biomasse des collemboles n'est pas affectée par le changement de probabilité d'occupation des salamandres et des carabes, et la présence des salamandres ne modifie pas la probabilité d'occupation des carabes de petite taille.
\end{enumerate}


\subsection{Résultat, limitations et ouvertures}

Notre étude a révélé que les coupes forestières peuvent avoir un impact négatif sur certaines variables environnementales importantes pour la faune du sol. 
Ces effets sont particulièrement prononcés dans les traitements de coupe totale, où l'on observe une réduction du volume de débris ligneux, une augmentation significative de l'ouverture de la canopée et une diminution de la profondeur de la litière. 
L'étude a également permis d'observer que les traitements de coupe partielle permettent une meilleure rétention des attributs environnementaux, puisque les effets pour les trois variables mesurées étaient moindres dans ce type de traitement comparativement aux témoins.
Malgré les changements environnementaux observés, cette recherche ne permet pas de conclure que les traitements de coupe forestière influencent directement ou indirectement l'occupation de l'habitat par la salamandre cendrée et les carabes ou la biomasse des collemboles. 
Notre hypothèse concernant la relation entre les taxons et le modèle d'équation structurelle qui en découle suggérait que l'effet des perturbations causé par les coupes forestières se propage à travers le réseau trophique des prédateurs vers les proies. 
La récolte forestière ne semble toutefois pas affecter la cooccurrence entre ces trois groupes d'espèces, selon ce type de relation. 

Une autre approche serait d'envisager que les perturbations environnementales affectent en premier lieu le bas de la chaîne alimentaire, avant de se propager à travers le réseau trophique jusqu'aux prédateurs \citep{Laigle2021Directindirect}. 
En effet, les espèces les plus sensibles aux perturbations sont généralement celles ayant une grande taille corporelle et un niveau trophique élevé \citep{Seibold2015Associationextinction,Nolte2019Habitatspecialization,Hagge2021Whatdoes}. 
En tant que consommateurs au sommet de la chaîne alimentaire au niveau du sol, les salamandres seraient ainsi particulièrement vulnérables aux changements dans les niveaux trophiques inférieurs. 
Un modèle alternatif d'équations structurelles aurait pu être élaboré pour examiner si les traitements sylvicoles peuvent influencer les collemboles, avec des effets se propageant ensuite vers les carabes et les salamandres \citep{Laigle2021Directindirect}. 
Nous ne pensons cependant pas que ce type de dynamique puisse avoir lieu dans notre cas, étant donné que nous n'avons mesuré aucun effet direct des traitements de coupes forestières sur la biomasse des collemboles. 

Les SEM sont des outils puissants pour étudier la dynamique des milieux naturels, analyser les relations entre diverses variables sous forme de réseau et observer la propagation des effets dans des systèmes complexes, 
tels que les écosystèmes forestiers \citep{graceSpecificationStructuralEquation2010}.  
De plus, plusieurs études ont souligné l’importance d’inclure les interactions biotiques, notamment entre espèces appartenant à différents niveaux trophiques, pour mieux comprendre les impacts des perturbations 
sur le fonctionnement des écosystèmes \citep{Thebault2003Foodwebconstraints,Seibold2018necessitymultitrophic,Laigle2021Directindirect}.  
Dans cette étude, l’intégration de modèles d’occupation et de modèles linéaires mixtes au SEM a ainsi permis d’adopter une approche multi-trophique sous forme de réseau, 
afin d’analyser les interactions entre différents taxons en réponse aux traitements de coupe forestière \citep{josephIntegratingOccupancyModels2016}.  
Cette méthodologie nous a permis de quantifier les effets directs et indirects des coupes forestières sur la faune du sol, tout en évaluant les modifications de plusieurs variables environnementales. 
Cependant, la complexité des SEM dépend fortement de la quantité de données disponibles, car l’utilisation d’un modèle complexe avec un jeu de données limité peut entraîner des problèmes d’estimation lors des analyses.  
Dans cette étude, les données recueillies se limitent à une seule année, ce qui restreint le niveau de complexité possible dans l'élaboration du SEM.
De plus, le fait que notre étude repose sur une seule année de données nous a empêchés de considérer la variabilité interannuelle.  
Nous estimons donc qu’il serait essentiel de reproduire cette recherche pour intégrer cette dimension dans l’analyse. 

Concernant les autres limitations de ce projet, nous aurions aimé initialement utiliser des modèles N-mélange pour quantifier l'impact des traitements sylvicoles sur l'abondance des différents taxons \citep{Royle2004Nmixturemodels,Mazerolle2021Woodlandsalamander}. 
Intégré au SEM, cette approche aurait permis d'acquérir une compréhension plus fine de l'impact des traitements sylvicoles sur nos espèces. 
Le nombre de salamandres observées n'a cependant pas été suffisant pour utiliser ce type d'analyse. 
Il est possible que le manque de vieillissement des planches servant de refuge artificiel explique en partie le faible descompte des salamandres. 
En effet, la salamandre cendrée favorise habituellement les débris ligneux ayant un stade de décomposition avancé pour s'abriter \citep{Otto2011ComparingCover,hedrickEffectsCoverboardAge2021}. 
Dans cette étude, les planches ont été installées en milieu naturel seulement trois mois avant le début de l'échantillonnage, ce qui pourrait limiter leur utilisation par les salamandres. 
Concernant les carabes et les collemboles, les étapes d’identification et de pesée requièrent la collecte des individus. 
Cependant, l’une des suppositions des modèles N-mélange est que la population étudiée doit rester fermée \citep{Royle2004Nmixturemodels}. 
Cette condition n’est pas respectée dans notre cas, puisque des individus sont retirés au cours de la collecte.

Le dispositif développé dans le cadre du projet DREAM (Desired Regeneration through Assisted Migration), et utilisé dans notre étude, a été conçu à l'origine pour examiner la migration assistée en tant que solution pour préserver les forêts face aux changements climatiques \citep{royoDesiredREgenerationAssisted2023}. 
Cette méthode consiste à déplacer des individus ou du matériel génétique d’une région climatique originelle vers une zone mieux adaptée aux conditions futures \citep{Vitt2010Assistedmigration}. 
Cette approche permettrait de modifier rapidement la composition des peuplements pour les adapter aux climats futurs, répondant ainsi à des besoins de conservation \citep{Dumroese2015Considerationsrestoring,Park2018Informationunderload,Park2023Provenancetrials}. 
Cependant, un manque de connaissances et un degré d’incertitude subsistent autour de cette mesure d’adaptation \citep{Klenk2015assistedmigration,Park2018Informationunderload}. 
Il serait donc important d’évaluer si la migration assistée peut influencer la faune présente dans le milieu hôte.
Lors de notre étude, les semis issus de diverses origines géographiques étaient encore trop jeunes pour potentiellement modifier les conditions environnementales et influencer les groupes d’espèces étudiés. 
Il serait néanmoins pertinent de répéter cette expérience lorsque les arbres auront atteint une maturité suffisante pour modifier les conditions au sol. 
De plus, le dispositif que nous avons utiliser dans le cadre de cette étude se trouve en forêt mixte tempérée.
Toutefois, il est possible que la réponse des espèces aux perturbations puisse varier selon le type d'habitat dans lequel elles se trouvent \citep{Kudrin2023metaanalysiseffects}. 
Il serait intéressant de reproduire ce type d'expérience dans d'autres environnements, tels que les forêts feuillues ou conifériennes.

Enfin, une meilleure compréhension des relations entre les perturbations, les variables environnementales et la biodiversité est cruciale pour préserver la pérennité des écosystèmes forestiers. 
L'ensemble de cette recherche a pour but d'approfondir la compréhension des relations existantes entre les pratiques sylvicoles et la dynamique des communautés du sol forestier. 
L'amélioration des connaissances permettra, à long terme, de fournir de meilleurs outils pour orienter les plans de gestion et adapter les pratiques sylvicoles.


\cleardoublepage

\bibliography{References.bib}
\bibliographystyle{ecologyNewFR.bst}