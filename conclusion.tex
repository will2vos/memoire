\chapter*{Conclusion générale}           % ne pas numéroter
\label{chap-conclusion}         % étiquette pour renvois
\phantomsection\addcontentsline{toc}{chapter}{\nameref{chap-conclusion}} % inclure dans TDM

Le but de cette étude était de comprendre comment les traitements sylvicoles, effectués dans un contexte de migration assistée, affectent la dynamique des écosystèmes du sol forestier.
Les objectifs qui s’y rattachaient étaient, d'une part, de quantifier l'effet des traitements de coupes forestières sur les variables environnementales qui influencent 
l'utilisation de l'habitat par la faune du sol et, d'autre part, de mesurer l'impact des coupes forestières sur l'utilisation de cet habitat par la faune du sol.
Pour répondre à ces deux objectifs, nous avons développé un modèle d'équations structurelles pour tester et mesurer l'effet des coupes sur le volume de débris ligneux, 
la profondeur de litière et l'ouverture de la canopée, ainsi que sur la probabilité d'occupation des salamandres cendrées, des carabes et sur la biomasse de collemboles.
Notre première hypothèse stipulait que les variables environnementales favorables à l'utilisation de l'habitat par les espèces fluctuent 
en fonction de l'intensité des coupes forestières et qu'ainsi, les traitements de coupes forestières constituent une variable englobant les changements de conditions environnementales. 
Notre seconde hypothèse soutenait que les traitements de coupe forestière entraînent une modification de l'utilisation de l'habitat par la faune du sol et se propagent à travers 
le réseau trophique des salamandres vers les collemboles, en passant par les carabes. 
Les principales conclusions tirées de cette étude sont :

\begin{enumerate}
  \item Les coupes forestières ont un effet significatif sur les variables environnementales influençant l'utilisation de l'habitat par la faune du sol. Ces changements suivent de manière générale le niveau d'intensité du traitement appliqué, avec des perturbations plus importantes dans les coupes totales, suivi des coupes partielles.
  \item La probabilité d'occupation des salamandres et des carabes ne semble globalement pas affectée par les différents traitements de coupes forestières comparativement aux sites témoins. Toutefois, un effet marginal a été mesuré chez les salamandres où la probabilité d'occupation de celles-ci était plus faible dans les coupes totales par rapport aux coupes partielles.
  \item La biomasse des collemboles ne variait pas de façon significative entre les différents traitements sylvicoles.
  \item La relation entre les trois groupes d'espèces ne semble pas subir de changement en fonction du type de coupe. Ainsi, la biomasse des collemboles n'est pas affectée par le changement de probabilité d'occupation des salamandres et des carabes, et la présence des salamandres ne modifie pas la probabilité d'occupation des carabes de petite taille.
\end{enumerate}



\subsection{Résultat et perspective de recherche}

présentation des résultats
interprétations de résultats
inclure des amélioration dans la recherche, des chose a améliorer ou a investiguer

importance de l'étude pour combler le manque de connaissances
avantage de l'étude
SEM
projet dreamQC étude a long terme sur l'impacte de la migration assistée sur petite faune 

% complexité du SEM

L'élaboration d'un modèle d'équation structurelle a permis de tenir compte de la complexité des relation entre les variable environnementale et les différent groupe d'espèce en tenant compte de effet directe et indirecte des perturbations, pouvant modifier la cooccurence entre les différent groupes d'espèces.

Toutefois, ce type d'analyse peut devenir et complexe et nécéssité un nombre important de donner afin de réduire le problème d'estimations. 

problème  de donner : 
- faible nombre de salamandres
- une année de récolte
- methode de récolte pour empéchant l'estimation de l'abondance pour les carabes et les collemboles
- complexité demander par le SEM

amélioration possible : 
- augmentation du nombre d'année de récolte
- bottom-up
- changement de type de forêts
- utilisation de trait fonctionnel
- étude a d'autre période de temps

Ouverture :
- migration assistée

% trait

kudrin The various reactions of soil invertebrates to disturbances may be attributed to their functional traits [34,35]. 
Recent works suggest that focusing on functional traits can provide greater insights into the mechanisms driving ecosystem change and recovery [36–38].

% top -down vs bottom-up

Notre hypothèse et le modèle d'équation structurelle qui en découle suggère que l'effet des perturbations causé par les coupes forestières se propage a travers le réseau trophique des prédateurs vers les prois. 
Toutefois les perturbations environnementales peuevent parfois impacter en premier lieu le bas de la chaine alimentaire pour esuite remonter a travers le réseau trophique vers les prédateurs. 
Nous ne pensons cependant pas que ce type de dynamique puisse avoir lieu dans notre cas étant donné que nous n'avons mesuré aucun effet directe des traitements de coupes forestières sur la biomasse des collemeboles. 

% type de foret

Le type de milieu dans lequel se déroule la recherche peut avoir un 
La réponse des espèces à une perturbation peux varier selon le type de milieu dans lequel elle se trouve \citep{Kudrin2023metaanalysiseffects}. 
Bien que notre étude n'est mesurer que peu de changement chez les différent groupe d'espèces en forêt mixte, 
il serait interressant de reproduire cette étude dans d'autre type de milieu tel que les forêt feuillu et conniférienne afin de voir si une réponse différente est obtenue. 


 % Quels sont les defaults ou amelioration du projet, ainsi que ses limitations.
  % une plus longue prériode d'échantillonnage
  limited number of sampling periods
  limitation dans la complexité du des analyses SEM

Le fait que notre étude ne contiennent qu'une seule année de données représente une limitation puisque cel a restraint la quantité de donnée disponible pour les analayse statistique  et que 
nos donner ne tiennent pas compte de la variation inter-annuel. 

Le vieillissement des planches utiliser pour le descomptes des salamandre pourrait également représenter un facteur slimitant dans cette études. Une fois découpé, nos planche ont été 

  otto 2011:
  Natural cover object searches provided greater power for detecting a similar change in occupancy, largely because of high sampling size

  % limitation dans le nombre d'individus pour les salamandres
    intervalle de confiance
  % essaie d,autre methodes
  reproduire l'étude avec de l'abondance fournira une compréhention approfondit mais ne pouvez pas etre utilisé ici en raison de la recolte des individus
  contrainte limitant de population fermé
  abondance au lieu d'occupation nécéssite ( carabes et collemboles), CMR pour les salamandres
  s'interresser au trait fonctionnel 

  % choix de classification des groupes
  trait fonctionnel a la place de taxonomique

  % degrée de décomposition des planche , voir otto pour info sur importance des planches dégradée

  % ouverture

Laigle2021Directindirect
Numerous studies highlighted the need to incorporate biotic interactions, in particular between organisms of different trophic levels (Lavorel 2013; Brousseau and others 2018a; Seibold and others 2018), to better understand the effects of disturbance on ecosystem functioning (Werner and Peacor 2003; The bault and Loreau 2003; Moretti and others 2013)

Studies of biodiversity responses to disturbance have traditionally focused on direct effects on a single taxon and trophic level at a given point in time, neglecting potential interactions between species of other trophic levels (Werner and Peacor 2003; Barnes and others 2016; Brose and Hillebrand 2016).
Forest disturbance intensity was found to have a bottom-up effect on species community composition, from lower trophic levels (for example, detritivorous springtails) up to soil fauna top predators (for example, running spiders).


  % Faire de étude à plus court terme mais surtout a plus long terme (temporel)
  we have limited knowledge of how quickly soil animals react to harvesting and what the rate of their recovery is.
  Despite the relatively large datasets on the abundance of Collembola and Coleoptera, no temporal dependence of the effect size over twenty years was found. kudrin


  Several mechanisms may explain the effect of management on forest biodiversity: changes in tree age structure, vertical stratification, and composition of tree species, which affect light, temperature, moisture, litter, and topsoil conditions (Sebastia et al. 2005; Standovar et al. 2006); presence of microhabitats (e.g., dead wood, veteran trees, cavities, root plates) specific to unmanaged (Berg et al. 1994; Bouget 2005a; Christensen et al. 2005; Gibb et al. 2005) or managed forests (e.g., skid trails and haul roads) (Hansen et al. 1991; Gosselin 2004); and forest cover continuity and features resulting from extensive management in the past (Hjalten et al. 2007). The pattern of response may therefore depend on which of the above mechanisms, or which combinations of them, have the strongest effects on different taxonomic or functional groups. paillet 

  kudrin :
Les forêts de conifères et les forêts de feuillus, par exemple, sont très différentes en termes de conditions pédologiques et microclimatiques, ce qui se reflète dans la dissimilarité de la composition et de la structure de la faune du sol [14,19,20].

% Force de l'utilisation des SEM

% migration assistée

% Possibilité de mesuré l'abondance  et limitation