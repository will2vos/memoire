\chapter{Direct and indirect effects on soil fauna of silvicultural treatments in the context of assisted forest migration}     % numéroté
\label{chapitre1-articles}    

William Devos$^1$, Mathieu Bouchard$^1$, Marc J. Mazerolle$^1$

%href{mailto:william.devos.1@ulaval.ca}
$^1$ Centre d'étude de la forêt, Département des sciences du bois \\ 
et de la forêt, Université Laval, Québec, QC G1V 0A6, Canada. \\ 

\clearpage

\section*{Résumé}
\label{sec:resume1}
\phantomsection\addcontentsline{toc}{section}{\nameref{sec:resume1}}

\begin{otherlanguage*}{french}
  <Résumé de l'article en français. Obligatoire.>

  \textbf{Mots-clés} : <ajouter des mots clés>
\end{otherlanguage*}

\clearpage

\section*{Abstract}
\label{sec:abstract1}
\phantomsection\addcontentsline{toc}{section}{\nameref{sec:abstract1}}

\begin{otherlanguage*}{english}
  <English abstract of the paper. Optional, but recommended.>

\textbf{Keywords}: <add some keywords> 
\end{otherlanguage*}

\cleardoublepage

\section*{Introduction}
\label{sec:intro1}
\phantomsection\addcontentsline{toc}{section}{\nameref{sec:intro1}}

%\defcitealias{keylist}{alias}

Due to their ecosystem services and economic values, forests play a predominant role on a global scale \citep{Balvanera2006Quantifyingevidence}. 
Within terrestrial ecosystems, they maintain significant biodiversity and act as regulators of biogeochemical factors \citep{Pawson2013Plantationforests}. 
However, the reality of climate change poses urgent challenges to the sustainability of current forests \citep{McKenney2009Climatechange,Messier2022Warningnatural,Seidl2017Forestdisturbances,Trumbore2015Foresthealth}. 
Despite international commitments to reduce greenhouse gas emissions, climate projections predict a global temperature exceeding 1.5 \up{o}C above preindustrial temperatures \citep{Matthews2022Currentglobal}. 
Canada is particularly vulnerable to this warming, due to its northern latitudes \citep{Alo2008Potentialfuture,Bush2019Canadachanging} and forests in Eastern North America will be particularly affected by this disruption \citep{Park2014Canboreal,Mahony2017closerlook,Messier2022Warningnatural,Sittaro2017Treerange}. 
Various studies have predicted lengthening and intensification of drought periods, an increase in wildfires and a higher presence of biotic disturbances \citep{Gatti2021Amazoniacarbon,Heidari2021Effectsclimate,Joyce2013Climatechange,Parmesan2007Influencesspecies}. 
Additionally, shifts in phenology and plant distribution are expected \citep{Aitken2008Adaptationmigration,Chuine2010Whydoes,Gray2013Trackingsuitable,Zhu2012Failuremigrate}. 
However, climate changes are occurring more rapidly than the ability of trees to adapt or migrate \citep{Aitken2008Adaptationmigration,Harrison2020Plantcommunity,Loarie2009velocityclimate,Messier2022Warningnatural,Williams2013Preparingclimate,Vitt2010Assistedmigration}, 
consequently threatening the growth and survival of these species \citep{Sittaro2017Treerange,Woodall2018Decadalchanges,Zhu2012Failuremigrate}.
This would ultimately result in a shift in forest composition, impacting forest management and conservation efforts \citep{Chmura2011Forestresponses,Lo2011Linkingclimate,McKenney2009Climatechange}.

Several calls for adaptation in forest management have been made to preserve forest ecosystems and their benefits \citep{Messier2021sakeresilience,Nagel2017Adaptivesilviculture}. 
Among the proposed solutions, assisted tree migration is suggested as a mitigation measure involving the movement of individuals or genetic material from their original climatic range to a more suitable area for species survival and growth in the future \citep{Dumroese2015Considerationsrestoring,Palik2022Operationalizingforestassisted,Park2023Provenancetrials,Park2018Informationunderload,Pedlar2011implementationassisted,Vitt2010Assistedmigration,Williams2013Preparingclimate}. 
This movement would rapidly change stand composition, meeting conservation needs, maintaining ecosystem services and preserving economic value \citep{Pedlar2011implementationassisted,Ste-Marie2011Assistedmigration,Winder2011Ecologicalimplications}. 
However, there remains a lack of knowledge and uncertainty surrounding assisted tree migration \citep{Park2018Informationunderload,Klenk2015assistedmigration}, particularly regarding the trade-offs between preserving one species and the risks to the ecosystem of the host territory \citep{Hewitt2011Takingstock,McLachlan2007frameworkdebate,Vitt2010Assistedmigration}.

To address this knowledge gap and reduce uncertainty, various forestry scenarios are currently under examination \citep{royoDesiredREgenerationAssisted2023}. 
Among the different silvicultural interventions considered, overstory treatments have been often used because they influence stand growth, health, and composition \citep{Ameray2021Forestcarbon,Chaudhary2016Impactforest,Man2008Elevenyearresponses,MontoroGirona2018ConiferRegeneration,PamerleauCouture2015Effectthree}. 
For example, clear-cut treatments involve the removal of all trees within a designated area and are commonly used in intensive forest management plans focused on increasing wood productivity and quality over a short period to meet industry needs \citep{Ameray2021Forestcarbon}. 
On the other hand, partial-cut treatments entail selective tree removal, maintaining a portion of the stand and are usually applied in extensive management plans that favor natural regeneration, mimic natural disturbances and preserve the ecological value of habitats \citep{Ameray2021Forestcarbon,Barg1999Influencepartial,Irland2011Timberproductivity,Tong2020Forestmanagement}. 
However, forest harvesting changes environmental conditions, leading to soil compaction as well as increased solar exposure, greater exposure to winds and increased precipitation reaching the forest floor \citep{Keenan1993ecologicaleffects,Lindo2003Microbialbiomass,Heithecker2007Edgerelatedgradients}. 
Ultimately, these changes can affect nutrient availability and impact soil biodiversity \citep{Battigelli2004Shorttermimpact,Chaudhary2016Impactforest,Covington1981Changesforest,Fedrowitz2014Canretention,Kudrin2023metaanalysiseffects,Lindo2003Microbialbiomass,Paillet2010Biodiversitydifferences,rousseauLongtermEffectsBiomass2018}.
Soil fauna plays a crucial role in forest ecosystems by contributing to the circulation of matter and energy \citep{Kudrin2023metaanalysiseffects,Seibold2021contributioninsects}. 
Amphibians and arthropods are among the groups most often affected by environmental disturbances such as forestry practices \citep{Hartshorn2021reviewforest,Semlitsch2009Effectstimber,Stuart2004Statustrends} or climate change \citep{Alford1999Globalamphibian,Houlahan2000Quantitativeevidence,Milanovich2010Projectedloss,Parmesan2006EcologicalEvolutionary,Pounds2006Widespreadamphibian,Warren2018projectedeffect}.
In the present study, we used an integrative approach and focused on three groups: the Eastern Red-backed Salamander (Plethodon cinereus (Green, 1818)), ground beetles (Carabidae) and springtails (Collembola).

The Eastern Red-backed Salamander has one of the highest biomass among vertebrates in North American forests \citep{Burton1975Salamanderpopulations,Petranka1993Effectstimber,semlitschAbundanceBiomassProduction2014a}. 
Like other Plethodontidae, this salamander is strictly terrestrial and relies on skin respiration due to the absence of lungs \citep{Heatwole1961Relationsubstrate}. 
It occupies forest soils when temperature and humidity levels are optimal for cutaneous respiration. 
Outside these periods, it will move vertically in the soil to maintain favorable conditions for its survival \citep{Grizzell1949HibernationSite}. 
This species has a small home range and typically shows philopatric behavior \citep{Yurewicz2004ResourceAvailability}. 
Its role is significant in forest ecosystems as it acts as a generalist predator and regulates detritivore invertebrate populations \citep{Burton1975Energyflow,Hickerson2017Easternredbacked,Walton2013Topdownregulation}. 
It also serves as prey in trophic networks and constitutes a rich energy food source \citep{Burton1975Energyflow,Pough1987abundancesalamanders}. 
Due to its cutaneous respiration, this salamander is highly sensitive to environmental disturbances \citep{Welsh2001caseusing} and is commonly used as a bioindicator \citep{Baecher2018Environmentalgradients,gibbsDistributionWoodlandAmphibians1998,Heatwole1962EnvironmentalFactors,Harpole1999Effectsseven,Hocking2013Effectsexperimental,Mazerolle2021Woodlandsalamander}.

On their part, ground beetles gather the highest specific diversity among beetles with 40,000 identified species and represent one of the most abundant groups among soil arthropods \citep{Erwin1985taxonpulse,loveiEcologyBehaviorGround1996,Rochefort2006GroundBeetle}. 
As voracious carnivores and polyphagous predators, they act as regulators of invertebrate populations \citep{loveiEcologyBehaviorGround1996}. 
Ground beetles are also prey for several species of amphibians, reptiles, birds and mammals \citep{loveiEcologyBehaviorGround1996}. 
While widely distributed in terrestrial ecosystems, habitat selection varies among species \citep{Larochelle2003naturalhistory}. 
Ground beetles are often classified into three species communities : mature and closed forest species, open habitat species and generalist species \citep{Niemela2007effectsforestry}. 
This variation in habitat selection makes ground beetles an interesting taxon to study during environmental disturbances \citep{bouchardBeetleCommunityResponse2016b,Halme1993Carabidbeetles,Heliola2001Distributioncarabid,koivulaBorealCarabidbeetleColeoptera2002a,Luff1992Classificationprediction,Niemela2001Carabidbeetles,Rainio2003Groundbeetles,Work2008Evaluationcarabid}.

As for springtails, they are polyphyletic species of arthropods belonging to the mesofauna established in forest soils. 
These invertebrates have a high species richness and represent a significant abundance \citep{rusekBiodiversityCollembolaTheir1998}. 
Different springtail communities occupy a range of ecological niches from litter to various soil horizons \citep{pongeVerticalDistributionCollembola2000}.
The vertical distribution of these communities depends mainly on abiotic conditions such as light, humidity, or porosity. 
Therefore, springtails can be used to characterize a substrate based on the community found there \citep{rusekBiodiversityCollembolaTheir1998}. 
Primarily fungivore and detritivore, these organisms play a predominant ecological role by feeding largely on fungi, bacteria, actinomycetes and algae. 
They contribute to the decomposition of organic matter, nutrient transformation and energy transfer in terrestrial ecosystems \citep{Cuchta2019importantrole,Hattenschwiler2005Biodiversitylitter,Marsden2020Howagroforestry,Petersen2000Collembolapopulations,rusekBiodiversityCollembolaTheir1998,Wolters1991SoilInvertebrates}. 
Springtails also represent a food source for several species of arachnids, beetles, amphibians, reptiles and birds. 
This group of species is commonly used in studies focusing on the effects of environmental changes on forest soils and mesofauna \citep{farskaManagementIntensityAffects2014,rousseauWoodyBiomassRemoval2019,Salmon2008Relationshipssoil}.

These three taxa are relevant to quantifying the impact of silvicultural treatments as their sensitivity to environmental changes and trophic relationships enable the analysis of disturbance effects on soil fauna dynamics.
Several papers have already studied the effects of those treatments on soil fauna.
However, most of them have only focused on the direct impacts of disturbances for one or more species groups, neglecting the potential relationships between environmental variables and species groups \citep{josephIntegratingOccupancyModels2016,Kudrin2023metaanalysiseffects,Pollierer2021Diversityfunctional}. 
Quantifying treatments effects and their propagation within the ecological network, namely on soil fauna, will provide useful tools to improve sustainable forest management.

Our study aimed to understand how silvicultural practices, conducted in an assisted tree migration context, affect the dynamics of forest soil ecosystems. 
The specific objectives were to evaluate the impact of overstory treatments on habitat use by soil fauna and to compare overstory 
treatments and environmental variables that influence habitat use by soil fauna.
We hypothesized that overstory treatments modify habitat use by soil fauna and propagate through the trophic network. 
We predicted that tree harvest affects habitat use by salamanders and large ground beetles, 
followed by modifications in habitat selection for small ground beetles and ultimately affecting springtail biomass. 
We also hypothesized that environmental variables, favorable to habitat use by taxa, fluctuate according to the intensity of forest harvests. 
We predicted that more intense treatments would decrease litter depth and CWD, as a consequence of reduced accumulation of leaves and woody debris on the forest floor and increase canopy openness. 

\section*{Material and methods}
\label{sec:matmet1}
\phantomsection\addcontentsline{toc}{section}{\nameref{sec:matmet1}}

\subsection*{Study area}
\label{subsec:area}
\phantomsection\addcontentsline{toc}{subsection}{\nameref{subsec:area}}

\begin{otherlanguage*}{english}

  Our study was conducted within the Portneuf Wildlife Reserve in the Capitale-Nationale administrative region near Lac des Amanites (47°07’N, 72°24’W, Figure \ref{fig:area}). 
  This area is located within the balsam fir (\textit{Abies balsamea})-yellow birch (\textit{Betula alleghaniensis}) bioclimatic domain, according to the ecological classification used in Québec \citep{saucierChapitreEcologieForestiere2009}. 
  Other tree species in this bioclimatic domain included sugar maple (\textit{Acer saccharum}), red maple (\textit{Acer rubrum}), white spruce (\textit{Picea glauca}), black spruce (\textit{Picea mariana}), red spruce (\textit{Picea rubens}), white birch (\textit{Betula papyrifera}), and quaking aspen (\textit{Populus tremuloides})\citep{olaBelowgroundCarbonStocks2024}. 
  The experimental sites are located on deep glacial tills with moderately well-drained sandy loams soil \citep{CanadianSystemSoil1998}. 
  The mean daily temperature is 4\up{o}C based on the 1981-2010 period at the nearest weather station (Lac aux sables, \citealp{environmentcanadaCanadianClimateNormals2019}). 
  Based on the same report, the mean annual precipitation and snowfall are 1133.2 mm and 230.3 cm, respectively. 

\end{otherlanguage*}

\begin{figure}[ht!]
	\centering
	\includegraphics[scale=0.60]{fig_area4.png}
	\caption[Localization of the Capitale-Nationale administrative region in Quebec, Canada and position of the study area near Lac des Amanites in Portneuf Wildlife Reserve, Quebec, Canada.]
  {Localization of the Captial-Nationale administrative region in Quebec, Canada (A) and position of the study area near Lac des Amanites in Portneuf Wildlife Reserve, Quebec, Canada (B) where the assisted migration experimental system was implemented in 2018 (47°07'N, 72°24'W).}
	\label{fig:area}
	\end{figure}  


\subsection*{Sampling design}
\label{subsec:sampling}
\phantomsection\addcontentsline{toc}{subsection}{\nameref{subsec:sampling}}

We conducted our study within the assisted migration experimental system established in 2018 by the Ministère des Ressources naturelles et des Forêts (\citealp{royoDesiredREgenerationAssisted2023}). 
This experimental system uses a factorial experimental design with split-plots replicated in four blocks. 
Each whole block (200 m x 140 m) is split in two overstory treatment : clear-cut and partial-cut. 

We selected a total of 60 sampling units measuring 10 m by 7.5 m to collect our data: we used four blocks serving as replicates, 
with each block containing six sampling units for both the clear-cut and partial-cut overstory treatments, 
whereas three additional sampling units were positioned outside the block, serving as uncut controls (Figure \ref*{fig:blockSU}). 
The uncut controls were separated from blocks by at least 10 m. 
In each sampling unit, we used three sampling methods to collect species data : artificial coverboards, pitfall traps, and soil cores.

We used artificial coverboards to sample eastern red-backed salamanders \textit{Plethodon cinereus} at our study sites \citep{hesedUncoveringSalamanderEcology2012,hydeSamplingPlethodontidSalamanders2001,Mazerolle2021Woodlandsalamander,mooreComparisonPopulationEastern2009c}. 
Coverboards consisted of 25 cm x 30 cm x 5 cm pieces of untreated spruce wood. Each board was in direct contact with the soil after we had cleared the litter underneath \citep{Mazerolle2021Woodlandsalamander}. 
Six coverboards spaces by at least 2.5 m were arranged in a rectangular array in each sampling unit, resulting in a total of 360 coverboards across our 60 sampling stations (Figure \ref{fig:blockSU}). 
All coverboards were placed outdoors in March 2022 to allow for natural aging \citep{hedrickEffectsCoverboardAge2021,Grasser2014Effectscover}. 
We conducted four visits to coverboards, namely in mid-May, mid-June, mid-July and mid-August, during the summer 2022. 
Blocks and sampling units within blocks  were visited in a random sequence to reduce the potential effects of time of day and observer fatigue. 
During a given visit, the 360 coverboards were inspected on the same day and we counted the number of red-backed salamanders underneath.  

We used pitfall traps were used to capture ground beetles \citep{baarsCatchesPitfallTraps1979,knappEffectPitfallTrap2012,kotzeFortyYearsCarabid2011a,loveiEcologyBehaviorGround1996,spenceSamplingCarabidAssemblages1994a}. 
Trap design was based on Multipher\up{\textregistered{}} traps and included a container with a diameter of 12.5 cm, a depth of 25 cm and a cover raised 4.5 cm above the trap 
to prevent debris and rain from filling the container \citep{bouchardBeetleCommunityResponse2016b,Jobin1988MultiPherinsect,mooreEffectsTwoSilvicultural2004}. 
We covered each trap with a protective mesh size of 15 mm, allowing trap access to carabid-sized individuals and limiting access by predators.  
We did not add preserving liquid in traps, but we added wet sponges to avoid harming vertebrates small enough to pass through the mesh. 
We centered a pitfall in each sampling unit (Figure \ref{fig:blockSU}). 
Traps were inserted in the soil at a depth allowing the container’s opening to be level with the soil surface. 
Trapping occurred during four sampling periods during 2022 (mid-May, mid-June, mid-July and mid-August). 
Traps were opened on the first day of each survey, and we collected captured ground beetles daily during five days. 
Outside of the trapping periods, pitfall traps were sealed with adhesive tape to prevent captures. 
Individuals were preserved in 70\% alcohol and identified at the species level afterwards. 
Identification was conducted with a ZEISS SteREO Discovery.V12 bionocular microscope using the \cite{larochelleManuelIdentificationCarabidae1976} taxonomic keys. 
We categorized ground beetles into two groups, either as salamander prey or salamander competitors, based on the red-backed salamander gape size (Table \ref{tab:carabid}, \citealp{jaegerFoodLimitedResource1972,magliaModulationPreycaptureBehavior1995,magliaOntogenyFeedingEcology1996}).

We extracted soil cores to sample springtails \citep{chauvatChangesSoilFaunal2011a,farskaManagementIntensityAffects2014,pongeVerticalDistributionCollembola2000,salamonEffectsPlantDiversity2004,wuCompositionSpatiotemporalVariation2014}. 
Core sampling occurred during the same four sampling periods as for the salamanders and ground beetles (mid-May, mid-June, mid-July and mid-August). 
During a sampling period, we collected two cores in each sampling unit using a pedological probe (5 cm diameter x 5 cm depth). 
We also collected a 15 cm x 15 cm litter quadrat each soil sample \citep{raymond-leonardSpringtailCommunityStructure2018a,rousseauForestFloorMesofauna2018}.
We extracted springtail communities directly related to the ecology of salamanders and ground beetles \citep{chauvatChangesSoilFaunal2011a,edwardsAssessmentPopulationsSoilinhabiting1991,raymond-leonardSpringtailCommunityStructure2018a,rousseauForestFloorMesofauna2018}.
Soil and litter from the same unit were pooled in Ziploc\up{\texttrademark{}} bags and stored in a cooler at ca. 4 °C \citep{chauvatChangesSoilFaunal2011a,rousseauForestFloorMesofauna2018}, providing 60 samples during each of the four sampling periods.
We placed each sample in an individual Tullgren dry-funnel for springtail extraction within 48 h after collection \citep{rousseauForestFloorMesofauna2018,rusekBiodiversityCollembolaTheir1998,wuCompositionSpatiotemporalVariation2014}. 
The extraction process lasted six days with a gradual temperature increase (25 °C to 50 °C) \citep{raymond-leonardSpringtailCommunityStructure2018a}.
Springtails were preserved in 75\% alcohol \citep{wuCompositionSpatiotemporalVariation2014}.
Identification at the family level was done with a ZEISS SteREO Discovery.V12 binocular microscope and a Leitz orthoplan phase-contrast fluorescent trinocular microscope using \cite{bellingerChecklistCollembolaWorld1996} identification keys. 
Following identification, springtails within a sample were pooled and dried in a freeze dryer (Labconco FreeZone Bulk tray dryer 78060 series) for 24 hours. 
We determined the total dry biomass of springtails in each sample with a micro balance (Sartorius Cubis\up{\texttrademark{}} MSA3.6P-000-DM).

\pagebreak

\begin{figure}[ht]
	\centering
	\includegraphics[scale=0.50]{fig_blockSU2.png}
	\caption[Design of one block and one sampling unit with three sampling methods.]{
  Design of a block (left) and a sampling unit (right). 
  The block contains two overstory treatments : clear-cut (grey background), partial-cut (white background). 
  Fifteen sampling units were used per block : six per overstory treatment and three controls (\textbf{c}) outside each block.
  Each sampling unit contained six artificial coverboards (rhombus) and one pitfall trap (triangles). Two soil cores (circles) were collected per survey.
  }
	\label{fig:blockSU}
	\end{figure}  

\vspace{0.5cm}


\subsection*{Environmental variables}
\label{subsec:EnvVar}
\phantomsection\addcontentsline{toc}{subsection}{\nameref{subsec:EnvVar}}

In each sampling unit, we measured several environmental variables that could affect occupancy probability of salamanders, ground beetles, and springtails.
CWD and litter depth play a crucial role in habitat use for salamanders, ground beetles and springtails as
they serve for feeding and protection  \citep{birdChangesSoilLitter2004,groverInfluenceCoverMoisture1998a,harmonEcologyCoarseWoody1986,koivula.LeafLitterSmallscale1999,mckennyEffectsStructuralComplexity2006,patrickEffectsExperimentalForestry2006a}. 
Salamanders also utilize CWD as shelter to maintain suitable temperature and moisture levels during dry periods \citep{groverInfluenceCoverMoisture1998a,Jaeger1980MicrohabitatsTerrestrial,patrickEffectsExperimentalForestry2006a}.
We used 400 m\up2 plots centered on every sampling unit to estimate CWD (20 m $\times$  20 m) (\citealp{methotGuideInventaireEchantillonnage2014}). 
We only considered CWD with a basal diameter $\geq$ 9 cm and a length $\geq$ 1 m.
For each CWD, we measured the basal diameter, the apical diameter and length with a tree caliper.
Segments of CWD outside the plot boundaries were not included.
We employed the conic–paraboloid formula to estimate the volume of each CWD \citep{fraverRefiningVolumeEstimates2007} :

\begin{equation}
  \text{Volume} = L/12 \times (5A_b + 5A_u + 2\sqrt{A_b \times A_u})
\end{equation}

\vspace{0.5cm}

Where $L$ is the length of log (cm), $A_b$ the basal area (cm\up{2}) and $A_u$ the apical area (cm\up{2}).
We measured litter depth next to each coverboard and computed the average litter depth for each sampling unit \citep{Mazerolle2021Woodlandsalamander}. \\

We assessed canopy openness, as it may influence species occupancy \citep{henneronForestPlantCommunity2017,koivulaBorealCarabidbeetleColeoptera2002a,kotzeFortyYearsCarabid2011a,messereForestFloorDistribution1998,tilghmanMetaanalysisEffectsCanopy2012}.
Canopy openness was measured 130 cm above the ground at the center of each sampling unit using a spherical densiometer \citep{lemmonSphericalDensiometerEstimating1956}. 
We averaged four measurements per sampling unit, oriented toward each of the four cardinal points as an estimate of canopy openness.

We collected data locally for air temperature, air humidity and precipitation levels during the summer 2022. 
Two weather stations (Em50 Digital Decagon Data Logger, Part \#40800, Meter Group Inc., USA) were installed inside both overstory treatments. 
Each weather station measured temperature, air humidity, and atmospheric pressure, 130 cm above the ground (VP-4 Sensor (Temp/RH/Barometer), Part \#40023). 
Rain gauges were installed in the clear-cut treatments to monitor precipitation levels. 
The temperature and humidity sensors were programmed to record data every 15 minutes. 
We averaged the measurements across both weather stations to get daily average measurements. 
These variables fluctuate on a daily basis, affecting species activity and, consequently, the probability of detecting individuals \citep{butterfieldCarabidLifeCycle1996,kotzeFortyYearsCarabid2011a,loveiEcologyBehaviorGround1996,odonnellPredictingVariationMicrohabitat2014a,spotilaRoleTemperatureWater1972}.

\subsection*{Statistical analyses}
\label{subsec:analyses}
\phantomsection\addcontentsline{toc}{subsection}{\nameref{subsec:analyses}} 


\subsubsection{Structural equations models} 

To assess the effects of overstory treatments on habitat use by soil fauna (hypothesis 1.1) and the relationship between overstory treatments 
and environmental variables (hypothesis 2.1), we employed a structural equation model (SEM) combining occupancy models and linear mixed models (LMM) \citep{graceSpecificationStructuralEquation2010,josephIntegratingOccupancyModels2016,mackenzieOccupancyEstimationModeling2006a}.
This approach enabled us to test both hypotheses within one analysis. 
One part of the SEM was designed to evaluate the direct and indirect effects of overstory treatments on taxa (Figure \ref*{fig:SEM}), 
while other components focused on the variations in environmental variables across different overstory treatments. 

Some components of our SEM consisted of occupancy models to quantify the impact of overstory treatments on the occupancy (presence) probabilities of salamanders and ground beetles. 
Occupancy models estimate the presence of species difficult to detect after accounting for imperfect detection probability \citep{mackenzieEstimatingSiteOccupancy2002,baileyEstimatingSiteOccupancy2004,mazerolleMakingGreatLeaps2007,spiersEstimatingSpeciesMisclassification2022}. 
The data type required to distinguish between occupancy and detection probabilities consists of repeated visits at a series of sites 
(here, 4 visits at the sampling units). Specifically, a detection history is constructed for each site, 
using 1 to denote the detection of the species on a given visit and 0 to non-detection. 
For example, the detection history 0010 at a site would indicate that the species was detected at the site on the third visit, but not detected during the first, 
second, and fourth visits. 

Other components of the SEM consisted of linear mixed models to estimate the effect overstory treatments on springtail biomass. 
The component of the model predicting springtail biomass included the latent occupancy state of salamanders and ground beetles to measure the impact of these groups on springtails. 
We formulated the SEM using a Bayesian framework, which we describe in Table S1. 
Parameters were estimated using Markov chain Monte Carlo (MCMC) with JAGS 4.3.0 included in the jagsUI package in R 4.3.1 \citep{lunnBUGSProjectEvolution2009,kellnerJagsUIWrapperRjags2024,rcoreteamLanguageEnvironmentStatistical2020}. 
We ran the model with five chains and 200,000 iterations each \citep{gelmanUnderstandingPredictiveInformation2014}. 
The first 75,000 iterations were used as burn-in and we used a thinning rate of 5. 
We assessed convergence of MCMC chains by examining trace plots, posterior density plots, and using the Brooks-Gelman-Rubin statistic. 
The JAGS model code is available in Table \ref{ann:SEM_script}.


\begin{figure}[ht!]
	\centering
	\includegraphics[scale=0.55]{fig_sem.png}
	\caption[Theoretical model illustrating the anticipated relationships between overstory treatments, environmental variables and species groups.]
  {Theoretical model illustrating the anticipated relationships between overstory treatments, coarse woody debris volume, canopy openness, litter depth,
   salamander occupancy, ground beetle occupancy and springtail biomass in the Portneuf Wildlife Reserve, Quebec, Canada. 
   Each arrow indicates the direction of a potential effect, from an explanatory variable to a response variable. 
   Note that the large and small carabid categories are based on salamander gape size.}
	\label{fig:SEM}
\end{figure} 

\subsubsection{Occupancy model component} 


The occupancy models of salamanders and ground beetles used the observed detections and non-detections at site $i$ during survey $j$, 
represented by $\text{Salamander}_{ij}$, $\text{Carabid.comp}_{ij}$ and $\text{Carabid.prey}_{ij}$. These detections and non-detections followed a Bernoulli distribution with $z_{i} \times p_{ij}$ as the success parameter, 
where $z_{i}$ represents the latent occupancy state of a given species group at site $i$ and $p_{ij}$ represents the probability of detecting the same species group at site $i$ during survey $j$ :


\begin{align}
  \text{Salamander}_{ij} &\sim \text{Bernoulli}(z_{\text{Salamander}_i} \times p_{\text{Salamander}_{ij}}) \nonumber \\
  \text{Carabid.comp}_{ij} &\sim \text{Bernoulli}(z_{\text{Carabid.comp}_i} \times p_{\text{Carabid.comp}_{ij}})  \\
  \text{Carabid.prey}_{ij} &\sim \text{Bernoulli}(z_{\text{Carabid.prey}_i} \times p_{\text{Carabid.prey}_{ij}}) \nonumber
\end{align}


The latent occupancy state of salamanders and large ground beetles at a site ($z_{i}$) followed a Bernoulli distribution 
with probability of occupancy ($\psi$) of the given species for a given harvest treatment (control, partial cut, clearcut):


\begin{align}
  z_{\text{Salamander}_i} &\sim \text{Bernoulli}(\psi_{\text{Salamander}_{\text{Treat}_i}}) \nonumber \\
  z_{\text{Carabid.comp}_i} &\sim \text{Bernoulli}(\psi_{\text{Carabid.comp}_{\text{Treat}_i}})
\end{align}


We used uniform distribution priors ($\text{U}(0, 1)$) for the occupancy probabilities of salamanders and large ground beetles in a given treatment. 

For small ground beetles, the latent occupancy state was drawn from a Bernoulli distribution, where the occupancy probability ($\psi_{Carabid.prey_{i}}$) 
depended on salamander latent occupancy ($z_{\text{Salamander}_i}$) and overstory treatments (control as reference):


\begin{align}
  \text{logit}(\psi_{\text{Carabid.prey}_i}) &= \beta_{0[\text{Carabid.prey}]} + \beta_{z_{\text{Salamander}}[\text{Carabid.prey}]} \times z_{\text{Salamander}_i} + \nonumber \\
  &\beta_{\text{Cutpartial}[\text{Carabid.prey}]} \times \text{Cutpartial}_i + \\
  &\beta_{\text{Cutclear}[\text{Carabid.prey}]} \times \text{Cutclear}_i \nonumber
\end{align}

We assumed vague normal priors for the $\beta$ parameters, $\text{N}(0, \sigma = \sqrt{10})$. 
We allowed the detection probability of a given species groups ($\text{logit}(p_{ij})$) with the volume of coarse woody debris ($\text{CWD}_i$) and the precipitation levels 
($\text{Precipitation}_{ij}$) as explanatory variables, and a block random effect ($\alpha_{Block}$) to reflect the experimental design:


\begin{align}
  \text{logit}(p_{\text{Salamander}_{ij}}) &= \alpha_{0[\text{Salamander}]} + \alpha_{\text{CWD}[\text{Salamander}]} \times \text{CWD}_i + \nonumber \\
  &\alpha_{\text{Precipitation}[\text{Salamander}]} \times \text{Precipitation}_{ij} + \alpha_{\text{Block}[\text{Salamander}]_{\text{Block}_i}} \nonumber
\end{align}

\begin{align}
  \text{logit}(p_{\text{Carabid.comp}_{ij}}) &= \alpha_{0[\text{Carabid.comp}]} + \alpha_{\text{CWD}[\text{Carabid.comp}]} \times \text{CWD}_i + \\
  &\alpha_{\text{Precipitation}[\text{Carabid.comp}]} \times \text{Precipitation}_{ij} + \alpha_{\text{Block}[\text{Carabid.comp}]_{\text{Block}_i}} \nonumber 
\end{align}

\begin{align}
  \text{logit}(p_{\text{Carabid.prey}_{ij}}) &= \alpha_{0[\text{Carabid.prey}]} + \alpha_{\text{CWD}[\text{Carabid.prey}]} \times \text{CWD}_i + \nonumber \\
  &\alpha_{\text{Precipitation}[\text{Carabid.prey}]} \times \text{Precipitation}_{ij} + \alpha_{\text{Block}[\text{Carabid.prey}]_{\text{Block}_i}} \nonumber 
\end{align}


We used vague normal priors for CWD and precipitation levels $\text{N}(0, \sigma = \sqrt{10})$. 
The priors for block random effects were $\text{N}(0, \alpha_{Block})$, where $\alpha_{Block}$ was drawn from a uniform distribution $\text{U}(0, 10)$. 
Precipitation was highly correlated with air temperature and relative humidity. 
For this reason, we did not include these two variables in our models.


\subsubsection{Linear mixed model component} 

We used linear mixed models to assess how overstory treatments affect springtail biomass ($\text{Springtail}_{i}$) and 
environmental variables ($\text{CWD}_{i}$, $\text{Canopy}_{i}$, $\text{Litter}_{i}$) at site $i$ :

\begin{align}
  \text{Springtail}_{i} &\sim \text{N} (\mu_{\text{Springtail}_i}, \sigma_{\text{Springtail}_{\text{Group}_i}}) \nonumber \\
  \text{CWD}_{i} &\sim \text{N} (\mu_{\text{CWD}_i}, \sigma_{\text{CWD}_{\text{Group}_i}}) \\
  \text{Canopy}_{i} &\sim \text{N} (\mu_{\text{Canopy}_i}, \sigma_{\text{Canopy}_{\text{Group}_i}}) \nonumber \\
  \text{Litter}_{i} &\sim \text{N} (\mu_{\text{Litter}_i}, \sigma_{\text{Litter}_{i}}) \nonumber 
\end{align}

Due to heteroscedasticity in the Springtail biomass, we allowed each treatment group $j$ to have their own variances ($\sigma_j \sim \text{U}(0,150)$). 
Springtail biomass was drawn from a normal distribution $\text{N} (\mu_{\text{Springtail}_i}, \sigma_{\text{Springtail}_{\text{Group}_i}})$, where $\sigma_{\text{Springtail}_{\text{Group}_i}}$ denotes the 
residual variance of a given treatment group and ($\mu_{i}$) corresponds to the linear predictor including overstory treatments ($\text{Cutpartial}_i$, $\text{Cutclear}_i$), 
a block random effect ($\alpha_{\text{Block}}$), as well as the latent occupancy state of salamanders and of each ground beetle group 
($z_{Salamander}$, $z_{Carabid.comp}$, $z_{Carabid.prey}$):


\begin{align}
  \mu_{\text{Springtail}_i} &= \beta_{0[\text{Springtail}]} + \beta_{\text{Cutpartial}[\text{Springtail}]} \times \text{Cutpartial}_i + \nonumber\\
  &\beta_{\text{Cutclear}[Springtail]} \times \text{Cutclear}_i + \beta_{z_{\text{Salamander}}[\text{Springtail}]} \times z_{Salamander} +  \nonumber\\
  &\beta_{z_{\text{Carabid.prey}}[\text{Springtail}]} \times z_{Carabid.prey} + \beta_{z_{\text{Carabid.comp}}[\text{Springtail}]} \times z_{Carabid.comp} + \nonumber\\
  &\alpha_{\text{Block}[\text{Springtail}]_{\text{Block}_i}} \nonumber
\end{align}


Again, we assumed vague normal priors for the coefficients $\text{N}(0, \sigma = \sqrt{10})$. 
We used $\text{N}(0, \alpha_{Block})$ priors for the block random effects, where $\alpha_{Block})$ is drawn from a uniform distribution $\text{U}(0, 50)$. 


\begin{align}
  \mu_{\text{CWD}_i} &= \beta_{0[\text{CWD}]} + \beta_{\text{Cutpartial}[\text{CWD}]} \times \text{Cutpartial}_{i} + \nonumber\\
  & \beta_{\text{Cutclear}[\text{CWD}]} \times \text{Cutclear}_{i} + \alpha_{\text{Block}[\text{CWD}]_{\text{block}_i}} 
\end{align}

\begin{align}
  \mu_{\text{Canopy}_i} &= \beta_{0[\text{Canopy}]} + \beta_{\text{Cutpartial}[\text{Canopy}]} \times \text{Cutpartial}_{i} + \nonumber \\
  & \beta_{\text{Cutclear}[\text{Canopy}]} \times \text{Cutclear}_{i} + \alpha_{\text{Block}[\text{Canopy}]_{\text{block}_i}} \nonumber
\end{align}

\begin{align}
  \mu_{\text{Litter}_i} &= \beta_{0[\text{Litter}]} + \beta_{\text{Cutpartial}[\text{Litter}]} \times \text{Cutpartial}_{i} + \nonumber\\
  & \beta_{\text{Cutclear}[\text{Litter}]} \times \text{Cutclear}_{i} + \alpha_{\text{Block}[\text{Litter}]_{\text{block}_i}} \nonumber
\end{align}


\clearpage

\section*{Results}
\label{sec:results1}
\phantomsection\addcontentsline{toc}{section}{\nameref{sec:results1}}


\subsection*{Soil fauna}
\label{subsec:taxa}
\phantomsection\addcontentsline{toc}{subsection}{\nameref{subsec:taxa}} 

Model diagnostics indicated that the chains were of sufficient length, as the Brooks-Gelman-Rubin statistic was below 1.04. 
Trace plot analysis revealed that all chains had converged towards similar values, and none of the ratios of MCMC error to posterior standard deviation exceeded 5\%.

\vspace{10pt}

\begin{figure}[ht]
	\centering
	\includegraphics[scale=0.60]{fig_sem_res3.png}
	\caption[Results from structural equation modeling analysis revealing effects of overstory treatments on coarse woody debris volume,
  canopy openness, litter depth, salamander occupancy, ground beetle occupancy, and springtail biomass.]
  {Results from SEM analysis showing effects of overstory treatments on CWD, 
  canopy openness, litter depth, salamander occupancy, ground beetle occupancy, and springtail biomass in the Portneuf Wildlife Reserve, 
  Quebec, Canada. Bold arrows represent significant effects, while gray arrows indicate no discernible effects. 
  Values above bold arrows represent average differences between posterior distributions of two overstory groups: 
  partial-cut (PC), clear-cut (CC), and control (C). Estimates marked with one asterisk (*) 
  indicate a 90\% credible interval (CI) excluding 0, while estimates marked with two asterisks (**) indicate a 95\% CI excluding 0. 
  Note that the large and small carabid categories are based on salamander gape size.}
	\label{fig:SEMres}
\end{figure}  

\vspace{10pt}

The average salamander observation per survey was 8 individuals with a range of 17 salamanders, during the summer 2022.
We captured 189 ground beetles belonging to 30 species, with 49 ground beetles found in the partial-cut treatments, 105 in the clear-cuts, and 35 in the control sites (Table \ref{tab:carabid}). 
We collected 468 springtails representing 12 families, with 219 springtails collected from partial-cut treatments, 131 from clear-cuts, and 118 from the control areas (Table \ref{tab:springtail}). 
The average springtail biomass collected per overstory treatments was 24.3 $\mu$g (SD = 18.2 $\mu$g), 56.8 $\mu$g (SD = 78.0 $\mu$g), and 31.1 $\mu$g (SD = 52.8 $\mu$g) in the partial-cut treatments, clear-cuts and control sites, respectively.

Habitat use by the different species groups generally did not vary across the overstory treatments. 
Salamander occupancy probability was marginally lower in sites subjected to clear-cutting compared to those with partial-cutting (90\% CI : [-0.74, -0.07], Figure \ref{fig:pcin}, Table \ref{tab:overstorysp}). 
However, these two groups did not differ from the control sites. 
Occupancy probability for each carabid group and the springtail biomass did not vary between the overstory treatments and control sites (Table \ref{tab:overstorysp}). 
Furthermore, the occupancy probabilities of small ground beetles (salamander prey) did not vary with the presence of salamanders. 
Similarly, the biomass of springtails did not vary with the presence of salamanders, large ground beetles or small ground beetles (Table \ref{tab:overstorysp}).

\vspace{10pt}

\begin{table}[ht]
  \centering
  \caption[Contrasts between overstory treatments for salamander occupancy, ground beetle occupancy, and springtail biomass.]
  {Contrasts between overstory treatments for salamander occupancy, ground beetle occupancy, and springtail biomass, during the summer 2022 in the Portneuf Wildlife Reserve, Quebec, Canada. 
  This table also shows the estimated effect of interactions between different groups: salamander presence on small and large ground beetles and the effects of the presence of salamanders and both ground beetle groups on springtail biomass.}
  \label{tab:overstorysp}
  \begin{tabular}{lllll} 
      \hline
      &&&&95\% Bayesian \\
      Variable&Unit& Comparison & Estimate &  credible interval \\ [0.5ex] 
      \hline     
      Salamander           &probability& Partial vs control & \hspace{1mm}0.07 & [-0.29, 0.45] \\ 
      occupancy       && Clear vs control  & -0.38 & [-0.75, 0.11] \\ 
                          && Clear vs partial  & -0.45 & [-0.74, -0.07]$^{a}$ \\       
      Carabid$_{competitor}$ &probability& Partial vs control & -0.12 & [-0.35, 0.15] \\
      occupancy       && Clear vs control  & -0.06 & [-0.29, 0.20] \\ 
                          && Clear vs partial  & \hspace{1mm}0.06 & [-0.19, 0.30] \\ 
      Carabid$_{prey}$    &logit& Partial vs control & \hspace{1mm}3.31 & [-10.12, 17.72] \\
      occupancy             && Clear vs control  & \hspace{1mm}10.19 & [-4.15, 24.45] \\ 
                          && Clear vs partial  & \hspace{1mm}6.88 & [-12.81, 23.42] \\  
                          && Salamander        & -2.20 & [-17.15, 16.59] \\  
      Springtail          &$\mu$g& Partial vs control & \hspace{1mm}8.11 & [-9.38, 25.40] \\
      biomass             && Clear vs control  & \hspace{1mm}2.11 & [-13.98, 18.11] \\ 
                          && Clear vs partial  & -6.00 & [-29.09, 17.26] \\  
                          && Salamander        & \hspace{1mm}6.80 & [-10.43, 23.26] \\ 
                          && Carabid$_{competitor}$      & \hspace{1mm}0.56 & [-16.75, 17.66] \\ 
                          && Carabid$_{prey}$      & \hspace{1mm}7.62 & [-8.93, 24.09] \\ 
      \hline
      \multicolumn{5}{l}{\textbf{Note:} Estimates from Bayesian SEM are presented in terms of posterior mean with 95\%} \\
      \multicolumn{5}{l}{credible intervals, where an interval excluding 0 indicates a difference between groups.} \\
      \multicolumn{5}{l}{$^{a}$Marginal difference based on 90\% Bayesian credible interval excluding 0}
  \end{tabular}
\end{table}


\clearpage

\begin{figure}[ht]
  \centering
  \includegraphics[scale=0.55]{fig_pcin.png}
  \caption[Occupancy probability of salamanders under overstory treatments]
  {Occupancy probability of salamanders within two overstory treatments and controls during the summer 2022 in the Portneuf Wildlife Reserve device, Quebec, Canada. 
  Error bars denote 95\% credible intervals around estimates.}
  \label{fig:pcin}
\end{figure}

\vspace{10pt}

We did not observe significant impacts of CWD volume and precipitation level on salamander detection probabilities. 
However, the precipitation level had a positive effect on detection probability for small ground beetles (95\% CI : [0.59, 1.77]) and large ground beetles (95\% CI : [0.70, 3.23]) (Table \ref{tab:detection}). 
Detection probabilities of ground beetles did not vary with the volume of CWD.

\begin{table}[ht]
  \centering
  \caption[Estimated effects of coarse woody debris and precipitation level on detection probabilities of salamanders and both ground beetles.]
  {Estimated effects of coarse woody debris and precipitation level on detection probabilities of salamanders and both ground beetles, during the summer 2022 in the Portneuf Wildlife Reserve,  Quebec, Canada.}
  \label{tab:detection}
  \begin{tabular}{lllll} 
      \hline
      &&&95\% Bayesian \\
      Variable & Taxa & Estimate &  credible interval \\ [0.5ex] 
      \hline      
      Precipitation       & Salamander              & \hspace{1mm}0.11 & [-0.83, 1.06] \\ 
                          & Carabid$_{competitor}$  & \hspace{1mm}1.17 & [0.59, 1.77] \\ 
                          & Carabid$_{prey}$        & \hspace{1mm}1.87 & [0.70, 3.23] \\  
      \hline      
      Coarse woody debris & Salamander              & -0.59 & [-1.39, 0.12] \\ 
                          & Carabid$_{competitor}$  & \hspace{1mm}0.06 & [-0.26, 0.38] \\ 
                          & Carabid$_{prey}$        & \hspace{1mm}0.27 & [-0.74, 1.37] \\   

      \hline
      \multicolumn{4}{l}{\textbf{Note:} Estimates from Bayesian SEM are presented in terms of posterior mean} \\
      \multicolumn{4}{l}{with 95\% credible intervals, where an interval excluding 0 indicates} \\
      \multicolumn{4}{l}{a difference between groups.} \\
  \end{tabular}
\end{table}


\subsection*{Environmental variables}
\label{subsec:ResEnv}
\phantomsection\addcontentsline{toc}{subsection}{\nameref{subsec:ResEnv}} 

Environmental variables usually differed between overstory treatments and control conditions.
We found that clear-cutting treatments had significantly less CWD compared to partial-cutting
(95\% CI : [-1.15, -0.43]) (Figure \ref{fig:envar} A, Table \ref{tab:overstoryenvar}). 
However, these treatments did not differ from the control sites. 
Canopy openness was significantly higher in both the partial-cut (95\% CI : [1.97, 11.02]) and clear-cut treatments (95\% CI : [51.39, 77.06]) when compared 
to the control sites, with the clear-cuts being more open than the partial-cuts (95\% CI : [44.61, 70.76], Figure \ref{fig:envar} B, Table \ref{tab:overstoryenvar}). 
In contrast, litter depth was lower in both the partial-cut (95\% CI : [-2.44, -0.65]) and clear-cut treatments (95\% CI : [-4.28, -2.50]) compared to the controls, 
with the litter being shallower in the clear-cuts than in the partial-cuts (95\% CI : [-2.57, -1.12], Figure \ref{fig:envar} C, Table \ref{tab:overstoryenvar}).


\vspace{10pt}

\begin{figure}[ht]
  \centering
  \includegraphics[scale=0.23]{fig_envar2.png}
  \caption[Environmental variables with a potential effect on soil species within two different overstory treatments and control.]
  {Environmental variables with a potential effect on soil species estimations within two different overstory treatments and control 
  during the summer 2022 in the Portneuf Wildlife Reserve, Quebec, Canada. Error bars denote 95\% credible intervals around estimates.}
  \label{fig:envar}
\end{figure}

\begin{table}[ht]
  \centering
  \caption[Contrasts between overstory treatments for environmental variables that could affect habitat selection of fauna on the forest soil.]
  {Contrasts between overstory treatments for environmental variables that could affect habitat use of fauna on the forest soil during the summer 2022 in the Portneuf Wildlife Reserve,
  Quebec, Canada.}
  \label{tab:overstoryenvar}
  \begin{tabular}{lllll} 
      \hline
      &&&&95\% Bayesian \\
      Variable&Unit& Comparison & Estimate &  credible interval \\ [0.5ex] 
      \hline
      Coarse woody debris &m\up{3}& Partial vs control & \hspace{1mm}0.02 & [-1.01, 1.06] \\ 
                 && Clear vs control  & -0.77 & [-1.79, 0.23] \\ 
                          && Clear vs partial  & -0.79 & [-1.15, -0.43] \\
      Canopy openness     &\%& Partial vs control & \hspace{1mm}6.49 & [1.97, 11.02] \\ 
                      && Clear vs control  & \hspace{1mm}64.19 & [51.39, 77.06] \\ 
                          && Clear vs partial  & \hspace{1mm}57.69 & [44.61, 70.76] \\ 
      Litter depth        &cm& Partial vs control & -1.54 & [-2.44, -0.65] \\ 
                      && Clear vs control  & -3.39 & [-4.28, -2.50] \\ 
                          && Clear vs partial  & -1.85 & [-2.57, -1.12] \\       
      \hline
      \multicolumn{5}{l}{\textbf{Note:} Estimates from Bayesian SEM are presented in terms of posterior mean with 95\%} \\
      \multicolumn{5}{l}{credible intervals, where an interval excluding 0 indicates a difference between groups.} \\
  \end{tabular}
\end{table}




\clearpage

\section*{Discussion}
\label{sec:discu1}
\phantomsection\addcontentsline{toc}{section}{\nameref{sec:discu1}}

\section*{Conclusion}
\label{sec:conclu1}
\phantomsection\addcontentsline{toc}{section}{\nameref{sec:conclu1}}

\section*{Acknowledgements}
\label{sec:acknowl1}
\phantomsection\addcontentsline{toc}{section}{\nameref{sec:acknowl1}}

\section*{Conflict of interest}
\label{sec:conflict1}
\phantomsection\addcontentsline{toc}{section}{\nameref{sec:conflict1}}

None declared
\section*{Author contributions}
\label{sec:author1}
\phantomsection\addcontentsline{toc}{section}{\nameref{sec:author1}}

\cleardoublepage

\begin{otherlanguage}{english}
\bibliographystyle{ecologyNewEN} % Style de citation en français
\bibliography{References}
\addcontentsline{toc}{section}{References}
\end{otherlanguage}
