\chapter{Soil fauna occupancy responses to cutting treatments in assisted tree migration context}     % numéroté
\label{chap:SEM}    

William Devos$^1$, Mathieu Bouchard$^1$, Marc Mazerolle$^1$

%href{mailto:william.devos.1@ulaval.ca}
$^1$ Centre d'étude de la forêt, Département des sciences du bois \\ 
et de la forêt, Université Laval, Québec, QC G1V 0A6, Canada. \\ 

\clearpage



\section*{Résumé}
\label{sec:resume1}
\phantomsection\addcontentsline{toc}{section}{\nameref{sec:resume1}}

\begin{otherlanguage*}{french}
  <Résumé de l'article en français. Obligatoire.>

  \textbf{Mots-clés} : <ajouter des mots clés>
\end{otherlanguage*}

\clearpage

\section*{Abstract}
\label{sec:abstract1}
\phantomsection\addcontentsline{toc}{section}{\nameref{sec:abstract1}}

\begin{otherlanguage*}{english}
  <English abstract of the paper. Optional, but recommended.>

\textbf{Keywords}: <add some keywords> 
\end{otherlanguage*}

\cleardoublepage

\section*{Introduction}
\label{sec:intro1}
\phantomsection\addcontentsline{toc}{section}{\nameref{sec:intro1}}

%\defcitealias{keylist}{alias}

\section*{Material and methods}
\label{sec:matmet1}
\phantomsection\addcontentsline{toc}{section}{\nameref{sec:matmet1}}

\subsection*{Study area}
\label{subsec:area}
\phantomsection\addcontentsline{toc}{subsection}{\nameref{subsec:area}}

\begin{otherlanguage*}{english}
  Our study was conduct within the Portneuf Wildlife Reserve in the Captial-Nationale administrative region (47°07'N, 72°24'W, Figure \ref{fig:cartoblock}) near Rivière-à-Pierre and Lac Amanites. 
  This area is located within the balsam fir (\textit{Abies balsamea})-yellow birch (\textit{Betula alleghaniensis}) bioclimatic domain (Saucier et al., 2009).
  Tree cover also contain sugar maple (\textit{Acer saccharum}), red maple (\textit{Acer rubrum}), white spruce (\textit{Picea glauca}), black spruce (\textit{Picea mariana}), red spruce (\textit{Picea rubens}),
  white birch (\textit{Betula papyrifera}) and quaking aspen (\textit{Populus tremuloides}) (foret ouverte, rapport d'inventaire). 
  The site lie on a deep glacial till as surface deposit with a moderately well-drained sandy loams soil (Gosselin, 1998).
  The mean daily temperature is 4 ◦C for the 1981-2010 period of the nearest weather station (Lac aux sables)(Environment Canada, 2019). 
  Based on the same report, the mean annual precipitation and snowfall are respectively 1133.2 mm and 230.3 cm.
  We used the assisted migration experimental system establish in 2018 by the Minister of Natural Resources and Forests to collect our data (MNRF)(Royo, 2023).
  This system is a factorial experimental design with a split-split-split-plots using 4 replications (complete random blocks). 
  Each whole block (200 m x 140 m) is occupied by an overstory treatment (clear-cut and partial-cut). 
  A cervid exclusion treatments (excluded and non-excluded) is used as a split plot and a competing vegetation treatments as a split-split plot. 
  Climatic analogs associated with three climate projections (current climate, mid-century 2050, and end of century 2080) 
  are used as a split-split-split plots with the seedlings of 9 species in mixed planting: black cherry (\textit{Prunus serotina}), northern red oak (\textit{Quercus rubra}), 
  northern white-cedar (\textit{Thuja occidentalis}), shagbark hickory, (\textit{Carya ovata}), sugar maple (\textit{Acer saccharum}), red pine (\textit{Pinus resinosa}), 
  red spruce (\textit{Picea rubens}), white spruce (\textit{Picea glauca}), and white pine (\textit{Pinus strobus}).

\end{otherlanguage*}

\begin{figure}[ht!]
	\centering
	\includegraphics[scale=0.70]{MapBlocks.png}
	\caption[Localisation of study sites in Portneuf, Canada.]{Localisation the assisted migration experimental system with four blocks (A, B, C, D) in Portneuf Wildlife Reserve (47°07'N, 72°24'W) and two overstory treatments : clear-cut (orange) and partial-cut (grey).}
	\label{fig:cartoblock}
	\end{figure}  



\subsection*{Sampling design}
\label{subsec:sampling}
\phantomsection\addcontentsline{toc}{subsection}{\nameref{subsec:sampling}}

We selected 60 sampling units (10 m $\times$ 7,5 m) to collect our data : six units per overstory treatment (clear-cut and partial-cut) with three control outside the blocks for four blocks.
Controls were separated from the blocks per at least 10 meters to remove treatments effects.
Each sampling unit contained three methods to collect species data. 
Artificial cover boards were used to count red-backed salamanders (Hesed, 2012; Mazerolle et al., 2021; Moore, 2009), 
pitfall traps for the carabid monitoring (sources), and soil cores with litter collection for springtails assessment (sources) (figure).
We conducted four sampling visits of five consecutive days each, namely in mid-May, mid-June, mid-July, and mid-August, during the summer 2022, to assesse the temporal effect. 
Blocks were randomly visited to reduce time of day and sampler's fatigue level effects.

Artificial coverboards is commonly employed for salamander sampling and yield similar or superior results compared to other methods such as active searching (Hyde, Simons, 2001; Moore, 2009). 
This method helps reduce variability in the number and size of sampled ground objects (Hyde and Simons 2001). 
Coverboards were made of spruce wood, measured 25 cm x 30 cm x 5 cm and were placed directly on the ground without litter underneath to maintain higher humidity level under the boards (Mazerolle et al., 2021). 
Six boards were set per sample and control units, resulting in a total of 360 coverboards.
They were positioned in two rows of three boards each, centered and oriented along the length of the sampling units.
Boards were spaced 2.5 m apart along the length of the sampling unit and 5 m apart along the width.
All boards were put outside during march 2022 to provide aging of the boards under natural conditions, thereby increasing their likelihood of being used by salamanders (article).
Throughout the sampling visits, cover boards were inspected once on the same day, and salamanders were counted without any manipulation.

Pitfall trapping is a passive sampling method used to assess the species richness of ground-dwelling invertebrates (Knapp, Růžička, 2012; Kotze et al., 2011; Lövei,  Sunderland, 1996). 
This capture method works particularly well for active arthropods moving on the ground, such as carabids (Baars, 1979; Lövei, Sunderland, 1996). 
Pitfall traps were produced by Bio.Contrôle services and included a container with a diameter of 12.5 cm, a depth of 25 cm and a cover raised 4.5 cm above the trap 
to prevent debris and rain from filling the container (figure ?).
They were equipped with a protective grid with a mesh size of 15 mm, limiting trap access to carabid-sized individuals and reducing the chances of predation by small mammals. 
Typically, a preserving liquid (such as propylene glycol or alcohol) is placed in the bottom of the container to preserve captured individuals. 
However, salamanders can have a width comparable to that of certain carabids, and therefore get captured and drown in the traps. 
For ethical reasons and considering the difficulties to limit access to salamanders without influencing the sampling of carabids, we decided to use dry pitfall traps without any preserving liquid (Luff, 1975). 
We also add wet sponges to the bottom of each container to maintain a suitable level of humidity for salamanders.
We centered one pitfall in each sampling and control units (figure), resulting in a total of 60 pitfall traps. 
Traps were inserted in the soil at a depth allowing the container's opening to be juxtaposed with the soil surface. 
Outside the sampling visits, all pitfall traps were closed with adhesive tape around the opening to prevent individuals capture. 
On the first day of each visit, traps were opened and carabids captured were collected for each remaining day. 
Individuals were placed in a container with 20\% alcohol and identified at the species level afterwards.
Identification was conduct with a (modèle du binoculaires) using André Larochelle identification keys (source)(annexe).
Carabids were categorized in two groups as salamander prey or competitors based on the salamander gape size (table)(source).

Soil cores is commonly used to sample mesofauna in litter and different soil horizons (Chauvat et al., 2011; Farská et al., 2014; Ponge, 2000; Salamon et al., 2004; Wu et al., 2014). 
Two soil cores with litter collection were harvested per sampling unit per visit using a soil sampling pedological probe. 
Cores were harvested inside the coverboards area to associate the presence of springtails with salamanders and carabids detection (figure). 
Cores had a diameter of 5 cm, a depth of 5 cm and a 15 cm x 15 cm litter quadrat was collected above each soil sample (sources).
Both substrates were used to target mesofauna and obtain springtail communities directly related to the ecology of salamanders and carabids (Chauvat et al., 2011; Edwards, 1991; Raymond-Leonard et al., 2018, Rousseau ).
Soil and litter from the same unit were pooled in Ziploc\up{\uppercase{tm}} bags and stored in a cooler at $\pm$ 4°C (Chauvat et al., 2011, Rousseau), providing 60 unit samples per visit.
Samples were placed in a Tullgren dry-funnel for springtails extraction within a maximum of 48 hours after the harvest (Figure )(Rusek, 1998; Wu et al., 2014, Laigle). 
The extraction process lasted six days with a gradual temperature escalation (25°C to 50°C)(Raymond-Leonard).
Springtails were preserved in 75\% alcohol (Wu et al., 2014) prior to isolation from surrounding organisms, with subsequent identification at the family level.
Identification was done with a (modèle du bino et du microscope) by using (nom de l'auteur de la clé d'identification) identification keys (source).
Springtails dry biomass per samples units was quantified with a Sartorius Cubis\up{\uppercase{tm}} MSA3.6P-000-DM micro balance after being lyophilized with a Labconco FreeZone Bulk tray dryers 78060 series.



\subsection*{Environmental variables}
\label{subsec:EnvVar}
\phantomsection\addcontentsline{toc}{subsection}{\nameref{subsec:EnvVar}}

In each sample units, we measured several environmental variables that could affect occupancy, as it has an impact on habitat use by species.
Coarse woody debris (CWD) and litter depth play a crucial role in habitat use for salamanders, carabids and springtails as
they serve for protection, feeding and to maintain suitable temperature and humidity (Bird et al., 2004; Grover, 1998; Harmon et al., 1986; Koivula et al., 1999; McKenny et al., 2006; Patrick et al., 2006). 
We used a 400 m\up2 plots centered inside every sampling units to estimate CWD per units (20 m $\times$  20 m)(Figure 7)(Méthot et al., 2014). 
We only selected CWD with a basal diameter greater than or equal to nine centimeters and a length greater than or equal to one meter.
Subsequently, we measured basal diameters, apical diameters and CWD's length with a tree caliper. 
Segments of CWD outside the plots boundaries wasn't considered.
We employed (Fraver et al., 2007) conic–paraboloid formula to estimate the volume of each CWD :
\[Volume = L/12 \times (5A_b + 5A_u + 2\sqrt{A_b \times A_u})\]

Where $L$ is the length of logs, $A_b$ the basal area, and $A_u$ the apical area.
We measured litter depth next to each coverboards and estimated the mean depth per sampling units (Mazerolle et al., 2021). 
We assessed canopy openness, as it may influence species activity (Koivula et al., 2002 ; Kotze et al., 2011, Messere 1998, Henneron 2016, Tilghman 2012 ).
Measurements were conducted at the center of all sampling units using a spherical densiometer (Lemmon 1956) and were taken at 130 cm above the ground. 
We took four measurements per units, oriented toward each of the four cardinal points, and used the mean results to estimate the canopy openness per sampling units.

We used air temperature, air humidity and precipitation levels as detection variables.
These variables fluctuate on a daily basis, affecting species activity and, consequently, the probability of detecting individuals. 
(annexe)(Kotze et al., 2011; Lövei , Sunderland, 1996; Spotila, 1972). (O’Donnell et al., 2014) (Butterfield, 1996; Kotze). % vérifier source
Two compact weather stations (Em50 Digital Decagon Data Logger, Part \#40800, Meter Group Inc., USA), placed on the clear-cut area, were available during the summer 2022.
Those station were equipped with a rain gauge and a probe measuring temperature, air humidity, and atmospheric pressure (1.30 m above the ground)(VP-4 Sensor (Temp/RH/Barometer), Part \#40023)
The temperature and humidity sensors were programmed for readings every 15 minutes. 
We used daily average measurements from both station since informations form previous years showed no differences between blocks (Annexe).



\subsection*{Statistical analyses}
\label{subsec:analyses}
\phantomsection\addcontentsline{toc}{subsection}{\nameref{subsec:analyses}} 

To assess the effects of overstory treatments on the site occupancy of taxa, we combined structural equation (SEM) with occupancy model (Grace 2010,  MacKenzie et al., 2006, Joseph 2016).
SEM allow us to measure direct and indirect processes that influence species occurrence by considering interrelationships between treatments, taxa and environmental variables (figure SEM) (Grace 2010).
Furthermore, occupancy models enable us to account for the imperfect detection of species since salamanders and carabids are known to be cryptic species and therefore may be present but unobserved (Bailey 2004, source carabes )
Occupancy modeling uses repeated surveys to compute detection probabilities and estimate the species true occupancy (MacKenzie et al. 2002).

SEM analysis was performed with a Bayesian approche. 
Parameters were estimated using Markov chain Monte Carlo method with JAGS 4.3.0 include in the jagsUI package in R 4.3.1. 

Structural equation modeling (SEM) provides one means by which to formally represent causal assumptions in the form of direct and indirect causal pathways, 
helping to close the gap between biological mechanisms and statistical methodology (Bollen 1989, Grace 2006).

Using multiple equations to represent complex path relationships (grace 2008)

structural equation models are better suited to study the multiple processes that control the behavior of systems. (grace 2008)
Researchers commonly want to know the interplay between processes, their relative importance, and how effects of perturbations cascade through systems (grace 2008)


% We used vague prior distributions for the means (N(m =0,s2 = 1000)) and standard deviations (Uniform(0, 100)) of each group (Supplementary Table S11
% We tested differences among groups with Bayesian multiple comparisons and reported differences between means using 95% credible intervals (95% CRI; Kruschke 2011; Gelman et al. 2012)
% We ran five chains with 250 000 iterations, using 125 000 iterations as burn-in, and a thinning rate of 10
% We used trace plots, posterior density plots, and the Brooks–Gelman–Rubin statistic to assess convergence, whereas we evaluated potential departures from model assumptions with standardized residual plots.
% choix des variables environnementales






Parameters were estimated in a Bayesian framework using Markov chain Monte Carlo implemented in JAGS 4.3.0 using the jagsUI package in R 4.0.3 
(Lunn et al. 2009; Kellner 2018; RCoreTeam2020). We used vague prior distributions for the means (N(m =0,s2 = 1000)) and standard deviations 
(Uniform(0, 100)) of each group (Supplementary Table S11)

%%% qu'est ce qu'on a fait avec les données ?
%% espèces
% salamandre
- presence/absence de salamandres pour un site pour une journée 
- modèle d'occupation % quel variable envionnementales ?
- detection imparfaites

% carabes
- présence/absence de carabes pour un site, pour une journée 
- séparation des carabes en deux groupes (proie/compétiteur)
- modèle d'occupation

% collemboles
- biomasse par site
- GLM

%% données ENV
% Température, humidité
- moyenne quotidienne des deux station apliquer a tous les blocs
% précitation
- somme des précipitation pour trois journée avant la visites
% CWD
- somme des cwd par sites
% Ouverture de la canopé
- moyenne des quatre mesures par sites
% Profondeur de litière
- moyenne des six mesure par sites

% Corrélation entre les variables

% SEM
- comprend quel variables
- découdre pour avoir le file d'idée




% selection de modèles
% modèles d'occupation % detection imparfaites
However, it would be possible to account for imperfect and variable species detection in our models to enable comparison (Mazerolle et al., 2007).
% Modèle d'équations structurelles

\clearpage

\section*{Results}
\label{sec:results1}
\phantomsection\addcontentsline{toc}{section}{\nameref{sec:results1}}

\clearpage

\section*{Discussion}
\label{sec:discu1}
\phantomsection\addcontentsline{toc}{section}{\nameref{sec:discu1}}

\section*{Conclusion}
\label{sec:conclu1}
\phantomsection\addcontentsline{toc}{section}{\nameref{sec:conclu1}}

\section*{Acknowledgements}
\label{sec:acknowl1}
\phantomsection\addcontentsline{toc}{section}{\nameref{sec:acknowl1}}

\section*{Conflict of interest}
\label{sec:conflict1}
\phantomsection\addcontentsline{toc}{section}{\nameref{sec:conflict1}}

None declared

\section*{Author contributions}
\label{sec:author1}
\phantomsection\addcontentsline{toc}{section}{\nameref{sec:author1}}

\cleardoublepage


\begin{otherlanguage}{english}
\bibliography{references.bib}
\bibliographystyle{ecologyNewEN.bst}
\addcontentsline{toc}{section}{References}
\end{otherlanguage}
