\chapter{Direct and indirect effects on soil fauna of silvicultural treatments in the context of forest assisted migration}     % numéroté
\label{chapitre1-articles}    

William Devos$^1$, Mathieu Bouchard$^1$, Marc Mazerolle$^1$

%href{mailto:william.devos.1@ulaval.ca}
$^1$ Centre d'étude de la forêt, Département des sciences du bois \\ 
et de la forêt, Université Laval, Québec, QC G1V 0A6, Canada. \\ 

\clearpage

\section*{Résumé}
\label{sec:resume1}
\phantomsection\addcontentsline{toc}{section}{\nameref{sec:resume1}}

\begin{otherlanguage*}{french}
  <Résumé de l'article en français. Obligatoire.>

  \textbf{Mots-clés} : <ajouter des mots clés>
\end{otherlanguage*}

\clearpage

\section*{Abstract}
\label{sec:abstract1}
\phantomsection\addcontentsline{toc}{section}{\nameref{sec:abstract1}}

\begin{otherlanguage*}{english}
  <English abstract of the paper. Optional, but recommended.>

\textbf{Keywords}: <add some keywords> 
\end{otherlanguage*}

\cleardoublepage

\section*{Introduction}
\label{sec:intro1}
\phantomsection\addcontentsline{toc}{section}{\nameref{sec:intro1}}

%\defcitealias{keylist}{alias}

Due to their ecosystem services and economic values, forests play a predominant role on a global scale \citep{Balvanera2006Quantifyingevidence}. 
Within terrestrial ecosystems, they maintain significant biodiversity and act as regulators of biogeochemical factors \citep{Pawson2013Plantationforests}. 
However, the reality of climate change poses urgent challenges for the sustainability of current forests \citep{McKenney2009Climatechange,Messier2022Warningnatural,Seidl2017Forestdisturbances,Trumbore2015Foresthealth}. 
Despite international commitments to reduce greenhouse gas emissions, climate projections predict a global temperature exceeding 1.5 \up{o}C above preindustrial temperatures \citep{Matthews2022Currentglobal}. 
Canada is particularly vulnerable to this warming, due to its northern latitudes \citep{Alo2008Potentialfuture,Bush2019Canadachanging} and forests in Eastern North America will be particularly affected by this disruption \citep{Park2014Canboreal,Mahony2017closerlook,Messier2022Warningnatural,Sittaro2017Treerange}. 
Various studies have predicted lengthening and intensification of drought periods, an increase in wildfires and a higher presence of biotic disturbances \citep{Gatti2021Amazoniacarbon,Heidari2021Effectsclimate,Joyce2013Climatechange,Parmesan2007Influencesspecies}. 
Additionally, shifts in phenology and plant distribution are expected \citep{Aitken2008Adaptationmigration,Chuine2010Whydoes,Gray2013Trackingsuitable,Zhu2012Failuremigrate}. 
However, climate changes are occurring more rapidly than trees ability to adapt or migrate \citep{Aitken2008Adaptationmigration,Harrison2020Plantcommunity,Loarie2009velocityclimate,Messier2022Warningnatural,Williams2013Preparingclimate,Vitt2010Assistedmigration}, 
consequently threatening the growth and survival of these species \citep{Sittaro2017Treerange,Woodall2018Decadalchanges,Zhu2012Failuremigrate}.
This would ultimately result in a shift in forest composition, impacting forest management and conservation efforts \citep{Chmura2011Forestresponses,Lo2011Linkingclimate,McKenney2009Climatechange}.

Several calls for adaptation in forest management have been made to preserve forest ecosystems and their benefits \citep{Messier2021sakeresilience,Nagel2017Adaptivesilviculture}. 
Among the proposed solutions, assisted tree migration is suggested as a mitigation measure involving the movement of individuals or genetic material from their original climatic range to a more suitable area for species survival and growth in the future \citep{Dumroese2015Considerationsrestoring,Palik2022Operationalizingforestassisted,Park2023Provenancetrials,Park2018Informationunderload,Pedlar2011implementationassisted,Vitt2010Assistedmigration,Williams2013Preparingclimate}. 
This movement would rapidly change stand composition, meeting conservation needs, maintaining ecosystem services and preserving economic value \citep{Pedlar2011implementationassisted,Ste-Marie2011Assistedmigration,Winder2011Ecologicalimplications}. 
However, there remains a lack of knowledge and uncertainty surrounding assisted tree migration \citep{Park2018Informationunderload,Klenk2015assistedmigration}, particularly regarding the trade-offs between preserving one species and the risks to the ecosystem of the host territory \citep{Hewitt2011Takingstock,McLachlan2007frameworkdebate,Vitt2010Assistedmigration}.

To address this knowledge gap and reduce uncertainty, various forestry scenarios are currently under examination to mitigate risks \citep{royoDesiredREgenerationAssisted2023}. 
Therefore, the study of different silvicultural interventions such as overstory treatments is considered since they influence stand growth, health and composition \citep{Ameray2021Forestcarbon,Chaudhary2016Impactforest,Man2008Elevenyearresponses,MontoroGirona2018ConiferRegeneration,PamerleauCouture2015Effectthree}. 
For example, clear-cut treatments involve the removal of all trees within a designated area and is commonly used in intensive forest management plans focused on increasing wood productivity and quality over a short period to meet industry needs \citep{Ameray2021Forestcarbon}. 
On the other hand, partial-cut treatments entail selective tree removal, maintaining a portion of the stand and are usually applied in extensive management plans that favor natural regeneration, mimic natural disturbances and preserve the ecological value habitats \citep{Ameray2021Forestcarbon,Barg1999Influencepartial,Irland2011Timberproductivity,Tong2020Forestmanagement}. 
However, forest harvesting leads to changes in environmental conditions, such as soil compaction, increased solar exposure, higher winds and increased precipitation reaching the forest floor \citep{Keenan1993ecologicaleffects,Lindo2003Microbialbiomass,Heithecker2007Edgerelatedgradients}. 
Ultimately, these changes can affect nutrient availability and impact soil biodiversity \citep{Battigelli2004Shorttermimpact,Chaudhary2016Impactforest,Covington1981Changesforest,Fedrowitz2014Canretention,Kudrin2023metaanalysiseffects,Lindo2003Microbialbiomass,Paillet2010Biodiversitydifferences,rousseauLongtermEffectsBiomass2018}.
Soil fauna plays a crucial role in forest ecosystems by contributing to the circulation of matter and energy \citep{Kudrin2023metaanalysiseffects,Seibold2021contributioninsects}. 
Amphibians and arthropods are among the groups most often affected by environmental disturbances such as forestry practices \citep{Hartshorn2021reviewforest,Semlitsch2009Effectstimber,Stuart2004Statustrends} or climate change \citep{Alford1999Globalamphibian,Houlahan2000Quantitativeevidence,Milanovich2010Projectedloss,Parmesan2006EcologicalEvolutionary,Pounds2006Widespreadamphibian,Warren2018projectedeffect}.
Consequently, to better understand the impact of  treatments on soil fauna, we chose to study the Eastern Red-backed Salamander (\textit{Plethodon cinereus} (Green, 1818)), ground beetles (Carabidae) and springtails (Collembola).

The Eastern Red-backed Salamander is one of the most significant biomass among vertebrates in North American forests \citep{Burton1975Salamanderpopulations,Petranka1993Effectstimber,semlitschAbundanceBiomassProduction2014a}. 
Like other Plethodontidae, this salamander is strictly terrestrial and relies on skin respiration due to the absence of lungs \citep{Heatwole1961Relationsubstrate}. 
It occupies forest soils when temperature and humidity levels are optimal for cutaneous respiration. 
Outside these periods, it will move vertically in the soil to maintain favorable conditions to its survival \citep{Grizzell1949HibernationSite}. 
This species has a small home range and typically shows philopatric behavior \citep{Yurewicz2004ResourceAvailability}. 
Its role is significant in forest ecosystems as it acts as a generalist predator and regulates detritivore invertebrate populations \citep{Burton1975Energyflow,Hickerson2017Easternredbacked,Walton2013Topdownregulation}. 
It also serves as prey in trophic networks and constitutes a rich energy food source \citep{Burton1975Energyflow,Pough1987abundancesalamanders}. 
Due to its cutaneous respiration, this salamander is highly sensitive to environmental disturbances \citep{Welsh2001caseusing} and is commonly used as a bioindicator \citep{Baecher2018Environmentalgradients,gibbsDistributionWoodlandAmphibians1998,Heatwole1962EnvironmentalFactors,Harpole1999Effectsseven,Hocking2013Effectsexperimental,Mazerolle2021Woodlandsalamander}.

On their part, ground beetles gather the highest specific diversity among beetles with 40,000 identified species and represent one of the most abundant groups among soil arthropods \citep{Erwin1985taxonpulse,loveiEcologyBehaviorGround1996,Rochefort2006GroundBeetle}. 
As voracious carnivores and polyphagous predators, they act as regulators of invertebrate populations \citep{loveiEcologyBehaviorGround1996}. 
Ground beetles are also prey for several species of amphibians, reptiles, birds and mammals \citep{loveiEcologyBehaviorGround1996}. 
While widely distributed in terrestrial ecosystems, habitat selection varies among species \citep{Larochelle2003naturalhistory}. 
Ground beetles are often classified into three species communities : mature and closed forest species, open habitat species and generalist species \citep{Niemela2007effectsforestry}. 
This variation in habitat selection makes ground beetles an interesting taxon to study during environmental disturbances \citep{bouchardBeetleCommunityResponse2016b,Halme1993Carabidbeetles,Heliola2001Distributioncarabid,koivulaBorealCarabidbeetleColeoptera2002a,Luff1992Classificationprediction,Niemela2001Carabidbeetles,Rainio2003Groundbeetles,Work2008Evaluationcarabid}.

As for springtails, they are polyphyletic species of arthropods belonging to the mesofauna established in forest soils. 
These invertebrates have a high species richness and represent a significant abundance \citep{rusekBiodiversityCollembolaTheir1998}. 
Different springtail communities occupy a range of ecological niches from litter to various soil horizons \citep{pongeVerticalDistributionCollembola2000}.
The vertical distribution of these communities depends mainly on abiotic conditions such as light, humidity, or porosity. 
Therefore, springtails can be used to characterize a substrate based on the community found there \citep{rusekBiodiversityCollembolaTheir1998}. 
Primarily fungivore and detritivore, these organisms play a predominant ecological role by feeding largely on fungi, bacteria, actinomycetes and algae. 
They contribute to the decomposition of organic matter, nutrient transformation and energy transfer in terrestrial ecosystems \citep{Cuchta2019importantrole,Hattenschwiler2005Biodiversitylitter,Marsden2020Howagroforestry,Petersen2000Collembolapopulations,rusekBiodiversityCollembolaTheir1998,Wolters1991SoilInvertebrates}. 
Springtails also represent a food source for several species of arachnids, beetles, amphibians, reptiles and birds. 
This group of species is commonly used in studies focusing on the effects of environmental changes on forest soils and mesofauna \citep{farskaManagementIntensityAffects2014,rousseauWoodyBiomassRemoval2019,Salmon2008Relationshipssoil}.

These three taxa are relevant to study the impact of silvicultural treatments as their sensitivity to environmental changes and trophic relationships enable the analysis of disturbance effects on soil fauna dynamics.
Several papers have already studied the effects of those treatments on soil fauna.
However, most of them have only focused on the direct impacts of disturbances for one or more species group, neglecting the existing relationships between environmental variables and species groups \citep{josephIntegratingOccupancyModels2016,Kudrin2023metaanalysiseffects,Pollierer2021Diversityfunctional}. 
It is essential to have a better comprehension of those treatments effects, their propagation within the ecological network and their repercussions on soil organism communities.
Ultimately, this knowledge acquisition will provide useful tools to facilitate sustainable forest management.

Our study aims to understand how silvicultural practices, conducted in an assisted tree migration context, affect the dynamics of forest soil ecosystems. 
The objectives associated with this aim are firstly to measure the impact of overstory treatments on habitat use by soil fauna, and secondly to compare overstory 
treatments and environmental variables that influence habitat use by soil fauna.
Regarding our first objective, we hypothesized that overstory treatments result in a modification of habitat use by soil fauna and propagate 
through the trophic network. Specifically, the harvests initially affect habitat use by salamanders and large ground beetles, 
followed by modifications in habitat selection for small ground beetles, and ultimately affecting springtail biomass.
The hypothesis associated with our second objective suggested that environmental variables, favorable to habitat use by taxa, fluctuate according to 
the intensity of forest harvests. Therefore, overstory treatments can be used as a variable encompassing changes in environmental conditions.
As a result, more intense treatments would decrease litter depth and CWD, as a consequence of reduced accumulation of leaves and woody debris on the forest floor, 
and increase canopy openness. 

%% est ce que je parles de CWD, litter et canopy dans l'intro

\section*{Material and methods}
\label{sec:matmet1}
\phantomsection\addcontentsline{toc}{section}{\nameref{sec:matmet1}}

\subsection*{Study area}
\label{subsec:area}
\phantomsection\addcontentsline{toc}{subsection}{\nameref{subsec:area}}

\begin{otherlanguage*}{english}
  Our study was conducted within the Portneuf Wildlife Reserve in the Captiale-Nationale administrative region near Lac des Amanites and Rivière-à-Pierre (47°07'N, 72°24'W, Figure \ref{fig:area}). 
  This area is located within the balsam fir (\textit{Abies balsamea})-yellow birch (\textit{Betula alleghaniensis}) bioclimatic domain \citep{saucierChapitreEcologieForestiere2009}.
  Other tree species include sugar maple (\textit{Acer saccharum}), red maple (\textit{Acer rubrum}), white spruce (\textit{Picea glauca}), black spruce (\textit{Picea mariana}), red spruce (\textit{Picea rubens}), white birch (\textit{Betula papyrifera}) and quaking aspen (\textit{Populus tremuloides}) \citep{olaBelowgroundCarbonStocks2024}. 
  Study sites rest on a deep glacial till as surface deposit with a moderately well-drained sandy loams soil \citep{CanadianSystemSoil1998}.
  The mean daily temperature is 4\up{o}C based on the 1981-2010 period at the nearest weather station (Lac aux sables, \citealp{environmentcanadaCanadianClimateNormals2019}). 
  Based on the same report, the mean annual precipitation and snowfall are 1133.2 mm and 230.3 cm, respectively.
  We used the assisted migration experimental system established in 2018 by the Ministère des Ressources naturelles et des Forêts to collect our data (\citealp{royoDesiredREgenerationAssisted2023}).
  This system is a factorial experimental design with a split-plots replicated in four blocks. 
  Each whole block (200 m x 140 m) is split in two overstory treatment : clear-cut and partial-cut. 
  See \cite{royoDesiredREgenerationAssisted2023} for more details about the assisted migration experiment.

\end{otherlanguage*}

\begin{figure}[ht!]
	\centering
	\includegraphics[scale=0.60]{fig_area4.png}
	\caption[Localization of the Captiale-Nationale administrative region in Quebec, Canada and position of the study area near Lac des Amanites in Portneuf Wildlife Reserve, Quebec, Canada.]
  {Localization of the Captial-Nationale administrative region in Quebec, Canada (A) and position of the study area near Lac des Amanites in Portneuf Wildlife Reserve, Quebec, Canada (B) where the assisted migration experimental system was implemented in 2018 (47°07'N, 72°24'W).}
	\label{fig:area}
	\end{figure}  



\subsection*{Sampling design}
\label{subsec:sampling}
\phantomsection\addcontentsline{toc}{subsection}{\nameref{subsec:sampling}}


We selected a total of 60 sampling units measuring 10 meters by 7.5 meters each to collect our data: we used four blocks serving as replicates, 
with each block containing six sampling units for both the clear-cut and partial-cut overstory treatments, 
while three additional sampling units were positioned outside the block, serving as controls (Figure \ref*{fig:blockSU}).
Sampling units serving as controls were separated from blocks by at least 10 meters to remove treatments effects.
In each sampling unit, we used three sampling methods to collect species data: artificial coverboards, pitfall traps, and soil cores. 

Artificial coverboards is commonly employed to count salamanders \citep{hesedUncoveringSalamanderEcology2012,Mazerolle2021Woodlandsalamander,mooreComparisonPopulationEastern2009c}. 
This method helps to standardize the number and size of sampled ground objects under which animals will hide, thereby reducing variability \citep{hydeSamplingPlethodontidSalamanders2001}. 
Coverboards were made of untreated spruce wood, measured 25 cm x 30 cm x 5 cm and were placed directly on the ground without litter underneath \citep{Mazerolle2021Woodlandsalamander}. 
Six coverboards were set per sample unit and spaced by at least 2.5 m, resulting in a total of 360 coverboards.
Two rows of three boards each were aligned along the length of the sampling units and centrally positioned (Figure \ref{fig:blockSU}).
All coverboards were placed outdoors in March 2022 to allow for natural aging, increasing the likelihood of salamander usage \citep{hedrickEffectsCoverboardAge2021,smithEffectsCoverBoard2014a}.
Throughout the surveys, coverboards were inspected once on the same day, and salamanders were counted without any manipulation.

Pitfall traps were used to capture ground beetles.
This method is commonly employed to assess the abundance and species richness of soil invertebrates, such as ground beetles \citep{baarsCatchesPitfallTraps1979,knappEffectPitfallTrap2012,kotzeFortyYearsCarabid2011a,loveiEcologyBehaviorGround1996,spenceSamplingCarabidAssemblages1994a}. 
Pitfall traps were Multipher\up{\textregistered{}} traps and included a container with a diameter of 12.5 cm, a depth of 25 cm and a cover raised 4.5 cm above the trap 
to prevent debris and rain from filling the container \citep{bouchardBeetleCommunityResponse2016b,Jobin1988MultiPherinsect,mooreEffectsTwoSilvicultural2004}.
They were equipped with a protective grid with a mesh size of 15 mm, limiting trap access to carabid-sized individuals and reducing the chances of predation by small mammals. 
Typically, a preserving liquid (such as propylene glycol or alcohol) is placed in the bottom of the container to preserve captured individuals. 
In this study, to avoid killing salamanders that are small enough to pass through the grid, we decided to use dry pitfall traps without any preserving liquid \citep{luffFeaturesInfluencingEfficiency1975}. 
We added wet sponges to the bottom of each container to maintain a suitable level of humidity for salamanders.
We centered one pitfall in each sampling and control units, resulting in a total of 60 pitfall traps (Figure \ref{fig:blockSU}). 
Traps were inserted in the soil at a depth allowing the container's opening to be juxtaposed with the soil surface. 
During periods when the traps were not visited daily, all pitfall traps were closed with adhesive tape around the opening to prevent individuals capture. 
On the first day of each survey, the traps were opened and captured ground beetles were collected for each remaining day for a five day period. 
Individuals were preserved in 70\% alcohol and identified at the species level afterwards.
Identification was conducted with a ZEISS SteREO Discovery.V12 bionocular microscope using \cite{larochelleManuelIdentificationCarabidae1976} identification keys.
Ground beetles were categorized in two groups as salamander prey or competitors based on the salamander gape size (Table \ref{tab:carabid}, \citealp{jaegerFoodLimitedResource1972,magliaModulationPreycaptureBehavior1995,magliaOntogenyFeedingEcology1996}).

Soil cores was used to sample springtails and are commonly employed to assess mesofauna in litter and different soil horizons \citep{chauvatChangesSoilFaunal2011a,farskaManagementIntensityAffects2014,pongeVerticalDistributionCollembola2000,salamonEffectsPlantDiversity2004,wuCompositionSpatiotemporalVariation2014}. 
Two soil cores with litter collection were harvested per sampling unit per survey using a soil sampling pedological probe. 
Cores had a diameter of 5 cm, a depth of 5 cm and a 15 cm x 15 cm litter quadrat was collected above each soil sample \citep{raymond-leonardSpringtailCommunityStructure2018a,rousseauForestFloorMesofauna2018}.
Both substrates were used to target mesofauna and obtain springtail communities directly related to the ecology of salamanders and ground beetles \citep{chauvatChangesSoilFaunal2011a,edwardsAssessmentPopulationsSoilinhabiting1991,raymond-leonardSpringtailCommunityStructure2018a,rousseauForestFloorMesofauna2018}.
Soil and litter from the same unit were pooled in Ziploc\up{\texttrademark{}} bags and stored in a cooler at $\pm$ 4 °C \citep{chauvatChangesSoilFaunal2011a,rousseauForestFloorMesofauna2018}, providing 60 samples per survey.
Each sample was placed in an individual Tullgren dry-funnel for springtail extraction within 48 h after they were collected \citep{rousseauForestFloorMesofauna2018,rusekBiodiversityCollembolaTheir1998,wuCompositionSpatiotemporalVariation2014}. 
The extraction process lasted six days with a gradual temperature increase (25 °C to 50 °C) \citep{raymond-leonardSpringtailCommunityStructure2018a}.
Springtails were preserved in 75\% alcohol \citep{wuCompositionSpatiotemporalVariation2014} prior to isolation from surrounding organisms, with subsequent identification at the family level.
Identification was done with a ZEISS SteREO Discovery.V12 binocular microscope and a Leitz orthoplan phase-contrast fluorescent trinocular microscope using \cite{bellingerChecklistCollembolaWorld1996} identification keys.
Springtail dry biomass of each sampled was measured with micro balance (Sartorius Cubis\up{\texttrademark{}} MSA3.6P-000-DM, city, state, country) after being lyophilized (Labconco FreeZone Bulk tray dryer 78060 series, city, state, country).

We conducted four surveys of five consecutive days each, namely in mid-May, mid-June, mid-July, and mid-August, during the summer 2022.
Blocks were visited in a random sequence to reduce effects of time of day and observer fatigue.

\pagebreak

\begin{figure}[ht]
	\centering
	\includegraphics[scale=0.50]{fig_blockSU2.png}
	\caption[Design of one block and one sampling unit with three sampling methods.]{
  Design of a block (left) and a sampling unit (right). 
  The block contains two overstory treatments : clear-cut (grey background), partial-cut (white background). 
  Fifteen sampling units were used per block : six per overstory treatment and three controls (\textbf{c}) outside each block.
  Each sampling unit contained six artificial coverboards (rhombus) and one pitfall trap (triangles). Two soil cores (circles) were collected per survey.
  }
	\label{fig:blockSU}
	\end{figure}  

\vspace{0.5cm}


\subsection*{Environmental variables}
\label{subsec:EnvVar}
\phantomsection\addcontentsline{toc}{subsection}{\nameref{subsec:EnvVar}}

In each sampling unit, we measured several environmental variables that could affect occupancy probability.
CWD and litter depth play a crucial role in habitat use for salamanders, ground beetles and springtails as
they serve for feeding and protection  \citep{birdChangesSoilLitter2004,groverInfluenceCoverMoisture1998a,harmonEcologyCoarseWoody1986,koivula.LeafLitterSmallscale1999,mckennyEffectsStructuralComplexity2006,patrickEffectsExperimentalForestry2006a}. \\
Salamanders also utilize CWD as shelter to maintain suitable temperature and moisture levels during dry periods \citep{groverInfluenceCoverMoisture1998a,Jaeger1980MicrohabitatsTerrestrial,patrickEffectsExperimentalForestry2006a}
We used 400 m\up2 plots centered inside every sampling unit to estimate CWD (20 m $\times$  20 m)(\citealp{methotGuideInventaireEchantillonnage2014}). 
We only selected CWD with a basal diameter greater than or equal to 9 cm and a length greater than or equal to 1 m.
Subsequently, we measured the basal diameter, the apical diameter, and length of CWD with a tree caliper.
Segments of CWD outside the plot boundaries were not considered.
We employed the conic–paraboloid formula to estimate the volume of each CWD \citep{fraverRefiningVolumeEstimates2007} :

\begin{equation}
  \text{Volume} = L/12 \times (5A_b + 5A_u + 2\sqrt{A_b \times A_u})
\end{equation}

\vspace{0.5cm}

Where $L$ is the length of log (cm), $A_b$ the basal area (cm\up{2}), and $A_u$ the apical area (cm\up{2}).
We measured litter depth next to each coverboard and estimated the mean depth per sampling unit \citep{Mazerolle2021Woodlandsalamander}. \\
We also assessed canopy openness, as it may influence species occupancy \citep{henneronForestPlantCommunity2017,koivulaBorealCarabidbeetleColeoptera2002a,kotzeFortyYearsCarabid2011a,messereForestFloorDistribution1998,tilghmanMetaanalysisEffectsCanopy2012}.
Measurements were conducted at the center of all sampling units using a spherical densiometer \citep{lemmonSphericalDensiometerEstimating1956} and were taken 130 cm above the ground. 
We took four measurements per sampling unit, oriented toward each of the four cardinal points, and computed the mean as an estimate of canopy openness in each sampling unit.

We collected data for air temperature, air humidity and precipitation levels during the summer 2022.
Two compact weather stations (Em50 Digital Decagon Data Logger, Part \#40800, Meter Group Inc., USA), were used inside both overstory treatments.
Each weather station was equipped with a probe measuring temperature, air humidity, and atmospheric pressure, 1.30 m above the ground (VP-4 Sensor (Temp/RH/Barometer), Part \#40023). 
Rain gauges were installed in the clear-cut treatments to monitor precipitation levels.
The temperature and humidity sensors were programmed for record data every 15 minutes. 
We used the means of both weather stations to get daily average measurements.
These variables fluctuate on a daily basis, affecting species activity and, consequently, the probability of detecting individuals. 
\citep{butterfieldCarabidLifeCycle1996,kotzeFortyYearsCarabid2011a,loveiEcologyBehaviorGround1996,odonnellPredictingVariationMicrohabitat2014a,spotilaRoleTemperatureWater1972}.

\subsection*{Statistical analyses}
\label{subsec:analyses}
\phantomsection\addcontentsline{toc}{subsection}{\nameref{subsec:analyses}} 

% expliquer dans les methodes les analyses de comparaison entre coupe forestière et variables environnementales

\subsubsection{Structural equations models} 

To assess the effects of overstory treatments on the habitat use (hypothesis 1.1) and the relationship between overstory treatments and environmental variables (hypothesis 2.1), 
we employed a structural equation models (SEM) combining occupancy models and linear mixed models (LMM) \citep{graceSpecificationStructuralEquation2010,josephIntegratingOccupancyModels2016,mackenzieOccupancyEstimationModeling2006a}.
This approach enables us to test both hypotheses within one analysis using the same data and method, thereby reducing estimation variations.
One part of the SEM was designed to evaluate the direct and indirect effects of overstory treatments on taxa (Figure \ref*{fig:SEM}), 
while another section focused on studying the variations in environmental variables across different overstory treatments. 
SEM are employed to analyze complex relationships among observed and latent variables (unobserved but inferred from observed data) to investigate multicausal ecological processes \citep{graceStructuralEquationModeling2008}.

We combined occupancy models with SEM to measure the impact of overstory treatments on the presence probabilities of salamanders and ground beetles. 
Occupancy models allow accounting for imperfect detection since salamanders and ground beetles are cryptic species \citep{baileyEstimatingSiteOccupancy2004,spiersEstimatingSpeciesMisclassification2022}.
This method use repeated surveys to estimate the presence probabilities of species and the probability of detecting them if they are present, 
leading to an estimation of the species true occupancy \citep{mackenzieEstimatingSiteOccupancy2002,mazerolleMakingGreatLeaps2007}.
By combining SEM and occupancy models, the occupancy probability estimated from the occupancy models becomes a latent variable in the SEM. 
The SEM then explores how the explanatory variables directly and indirectly affect this latent occupancy variable.
This approach enables us to investigate complex relationships influencing the presence of salamanders and ground beetles 
while addressing the issue of imperfect detection.

LMM were combined with SEM to estimate the effect overstory treatments on springtail biomass and environmental variables.
For springtails, incorporating LMM into a SEM enabled us to use the estimated occupancy probabilities 
to measure the direct and indirect effects of the presence of salamanders or ground beetles on springtail biomass. 
Also, LMM allows us to account for random variations of blocks, enhancing model accuracy. 
Those models can handle data with hierarchical structures and can estimate the amount of variation due to fixed effects while also accounting for 
random effects \citep{Bolker2009Generalizedlinear}.


\begin{figure}[ht!]
	\centering
	\includegraphics[scale=0.55]{fig_sem.png}
	\caption[Theoretical model illustrating the anticipated relationships between overstory treatments, environmental variables and species groups.]
  {Theoretical model illustrating the anticipated relationships between overstory treatments, coarse woody debris volume, canopy openness, litter depth,
   salamander occupancy, ground beetle occupancy, and springtail biomass in the Portneuf Wildlife Reserve, Quebec, Canada. 
   Each arrow indicates the direction of a potential effect, from an explanatory variable to a response variable.}
	\label{fig:SEM}
\end{figure}  

\subsubsection{Occupancy models} 


To estimate the occupancy probabilities of salamanders and ground beetles, we used the observed detection at site $i$ during survey $j$, 
represented by $\text{Salamander}_{ij}$, $\text{Carabid.comp}_{ij}$ and $\text{Carabid.prey}_{ij}$. Observed detection followed a Bernoulli distribution with $z_{i} \times p_{ij}$ as parameter, 
where $z_{i}$ represents the latent occupancy state of species groups at a site $i$ and $p_{ij}$ represents the probability of detecting species groups at a site $i$ during a survey $j$ :


\begin{align}
  \text{Salamander}_{ij} &\sim \text{Bernoulli}(z_{\text{Salamander}_i} \times p_{\text{Salamander}_{ij}}) \nonumber \\
  \text{Carabid.comp}_{ij} &\sim \text{Bernoulli}(z_{\text{Carabid.comp}_i} \times p_{\text{Carabid.comp}_{ij}})  \\
  \text{Carabid.prey}_{ij} &\sim \text{Bernoulli}(z_{\text{Carabid.prey}_i} \times p_{\text{Carabid.prey}_{ij}}) \nonumber
\end{align}


$z_{i}$ follows a Bernoulli distribution with $\psi$ as parameter, 
which represent the occupancy probability of each species group. 
The latent occupancy state $z$ indicates whether a site is occupied by a species group ($z = 1$) or not ($z = 0$) :


\begin{align}
  z_{\text{Salamander}_i} &\sim \text{Bernoulli}(\psi_{\text{Salamander}_{\text{Group}_i}}) \nonumber \\
  z_{\text{Carabid.comp}_i} &\sim \text{Bernoulli}(\psi_{\text{Carabid.comp}_{\text{Group}_i}}) \\
  z_{\text{Carabid.prey}_i} &\sim \text{Bernoulli}(\psi_{\text{Carabid.prey}_{i}}) \nonumber
\end{align}


Salamanders and large ground beetle occupancy probabilities ($\psi_{\text{Salamander}_{\text{Group}_i}}$, $\psi_{\text{Carabid.comp}_{\text{Group}_i}}$ ) are drawn from uniform distribution $\text{U}(0, 1)$ for each overstory group $i$, 
whereas small ground beetle occupancy probabilities ($\psi_{\text{Carabid.prey}_{i}}$) were estimated from a linear predictor with salamander latent occupancy state ($z_{\text{Salamander}_i}$) and overstory treatments 
($\text{Cutpartial}_i$, $\text{Cutclear}_i$) as explanatory variables on a logit scale. 
$\beta$s represent the coefficients associated with each parameter affecting occupancy in the linear predictor. 
We assumed vague normal priors, $\text{N}(0, \sigma = 10)$ for those $\beta$ parameters :


\begin{align}
  \text{logit}(\psi_{\text{Carabid.prey}_i}) &= \beta_{0[\text{Carabid.prey}]} + \beta_{z_{\text{Salamander}}[\text{Carabid.prey}]} \times z_{\text{Salamander}_i} + \nonumber \\
  &\beta_{\text{Cutpartial}[\text{Carabid.prey}]} \times \text{Cutpartial}_i + \\
  &\beta_{\text{Cutclear}[\text{Carabid.prey}]} \times \text{Cutclear}_i \nonumber
\end{align}

We applied a logit scale on parameters influencing the probability of detecting species groups ($\text{logit}(p_{ij})$), allowing us to use 
the coarse woody debris ($\text{CWD}_i$), the precipitation levels ($\text{Precipitation}_{ij}$) as explanatory variables with a block random effect ($\alpha_{Block}$). 
$\alpha$s represent the coefficients associated with each explanatory variable affecting detection in the linear predictor. 
We assumed vague normal priors for CWD, precipitation levels $\text{N}(0, \sigma = 10)$ and block random effects $\text{N}(0, \sigma_{Block})$ 
where $\sigma_{Block}$ is drawn form uniform distribution $\text{U}(1, 10)$. 
Air temperature and relative humidity were excluded from detection variables due to a high correlation with the precipitation level.


\begin{align}
  \text{logit}(p_{\text{Salamander}_{ij}}) &= \alpha_{0[\text{Salamander}]} + \alpha_{\text{CWD}[\text{Salamander}]} \times \text{CWD}_i + \nonumber \\
  &\alpha_{\text{Precipitation}[\text{Salamander}]} \times \text{Precipitation}_{ij} + \alpha_{\text{Block}[\text{Salamander}]_{\text{Block}_i}} \nonumber
\end{align}

\begin{align}
  \text{logit}(p_{\text{Carabid.comp}_{ij}}) &= \alpha_{0[\text{Carabid.comp}]} + \alpha_{\text{CWD}[\text{Carabid.comp}]} \times \text{CWD}_i + \\
  &\alpha_{\text{Precipitation}[\text{Carabid.comp}]} \times \text{Precipitation}_{ij} + \alpha_{\text{Block}[\text{Carabid.comp}]_{\text{Block}_i}} \nonumber 
\end{align}

\begin{align}
  \text{logit}(p_{\text{Carabid.prey}_{ij}}) &= \alpha_{0[\text{Carabid.prey}]} + \alpha_{\text{CWD}[\text{Carabid.prey}]} \times \text{CWD}_i + \nonumber \\
  &\alpha_{\text{Precipitation}[\text{Carabid.prey}]} \times \text{Precipitation}_{ij} + \alpha_{\text{Block}[\text{Carabid.prey}]_{\text{Block}_i}} \nonumber 
\end{align}

%%%%%%%%%%%%%%%%%%%
%%%%%%%%%%%%%%%%%%% fin


% \begin{align}
%   z_{\text{Salamander}_i} &\sim \text{Bernoulli}(\psi_{\text{Salamander}_{\text{Group}_i}}) \nonumber \\
%   \text{logit}(p_{\text{Salamander}_{ij}}) &= 
%   \alpha_{0[\text{Salamander}]} + \alpha_{\text{CWD}[\text{Salamander}]} \times \text{CWD}_i + \\
%   &\alpha_{\text{Precipitation}[\text{Salamander}]} \times \text{Precipitation}_{ij} + \alpha_{\text{Block}[\text{Salamander}]_{\text{Block}_i}} \nonumber \\
%   \text{Salamander}_{ij} &\sim \text{Bernoulli}(z_{\text{Salamander}_i} \times p_{\text{Salamander}_{ij}}) \nonumber
% \end{align} 

% \begin{align}
% z_{\text{Salamander}_i} &\sim 
% \text{Bernoulli}(\psi_{\text{Salamander}_{\text{Group}_i}}) \nonumber \\
% \text{logit}(p_{\text{Salamander}_{ij}}) &= 
% \alpha_{0[\text{Salamander}]} + \alpha_{\text{CWD}[\text{Salamander}]} \times \text{CWD}_i + \\
% &\alpha_{\text{Precipitation}[\text{Salamander}]} \times \text{Precipitation}_{ij} + \alpha_{\text{Block}[\text{Salamander}]_{\text{Block}_i}} \nonumber \\
% \text{Salamander}_{ij} &\sim 
% \text{Bernoulli}(z_{\text{Salamander}_i} \times p_{\text{Salamander}_{ij}}) \nonumber
% \end{align} 

% \begin{align}
%   z_{\text{Carabid.comp}_i} &\sim 
%   \text{Bernoulli}(\psi_{\text{Carabid.comp}_{\text{Group}_i}}) \nonumber \\
%   \text{logit}(p_{\text{Carabid.comp}_{ij}}) &= 
%   \alpha_{0[\text{Carabid.comp}]} + \alpha_{\text{CWD}[\text{Carabid.comp}]} \times \text{CWD}_i + \\
%   &\alpha_{\text{Precipitation}[\text{Carabid.comp}]} \times \text{Precipitation}_{ij} + \alpha_{\text{Block}[\text{Carabid.comp}]_{\text{Block}_i}} \nonumber \\
%   \text{Carabid.comp}_{ij} &\sim \text{Bernoulli}(z_{\text{Carabid.comp}_i} \times p_{\text{Carabid.comp}_{ij}}) \nonumber
%   \end{align}

% \begin{align}
%   \text{logit}(\psi_{\text{Carabid.prey}_i}) &= 
%   \beta_{0[\text{Carabid.prey}]} + \beta_{z_{\text{Salamander}}[\text{Carabid.prey}]} \times z_{\text{Salamander}_i} + \nonumber \\
%   &\beta_{\text{Cutpartial}[\text{Carabid.prey}]} \times \text{Cutpartial}_i + \beta_{\text{Cutclear}[\text{Carabid.prey}]} \times \text{Cutclear}_i \nonumber\\
%   z_{\text{Carabid.prey}_i} &\sim 
%   \text{Bernoulli}(\psi_{\text{Carabid.prey}_i}) \nonumber \\
%   \text{logit}(p_{\text{Carabid.prey}_{ij}}) &= \alpha_{0[\text{Carabid.prey}]} + \alpha_{\text{CWD}[\text{Carabid.prey}]} \times \text{CWD}_i +  \\
%   &\alpha_{\text{Precipitation}[\text{Carabid.prey}]} \times \text{Precipitation}_{ij} + \alpha_{\text{Block}[\text{Carabid.prey}]_{\text{Block}_i}} \nonumber \\
%   \text{Carabid.prey}_{ij} &\sim \text{Bernoulli}(z_{\text{Carabid.prey}_i} \times p_{\text{Carabid.prey}_{ij}}) \nonumber
% \end{align}

\subsubsection{Linear mixed models} 

We used linear mixed models to assess how overstory treatments affect springtail biomass ($\text{Springtail}_{i}$) and 
environmental variables ($\text{CWD}_{i}$, $\text{Canopy}_{i}$, $\text{Litter}_{i}$) at site $i$ :

\begin{align}
  \text{Springtail}_{i} &\sim \text{N} (\mu_{\text{Springtail}_i}, \sigma_{\text{Springtail}_{\text{Group}_i}}) \nonumber \\
  \text{CWD}_{i} &\sim \text{N} (\mu_{\text{CWD}_i}, \sigma_{\text{CWD}_{\text{Group}_i}}) \\
  \text{Canopy}_{i} &\sim \text{N} (\mu_{\text{Canopy}_i}, \sigma_{\text{Canopy}_{\text{Group}_i}}) \nonumber \\
  \text{Litter}_{i} &\sim \text{N} (\mu_{\text{Litter}_i}, \sigma_{\text{Litter}_{i}}) \nonumber 
\end{align}

Springtail biomass and environmental variables estimations are drawn from Normal distributions where means estimations at site $i$ ($\mu_{i}$) included overstory treatments ($\text{Cutpartial}_i$, $\text{Cutclear}_i$) and block random effect ($\alpha_{\text{Block}}$) as linear predictors. 
Estimations of $\mu_{\text{Springtail}_i}$ also accounted for latent occupancy state of salamanders and both ground beetles groups ($z_{Salamander}$, $z_{SCarabid.comp}$, $z_{Carabid.prey}$ ). 
We assumed vague normal priors for overstory treatments, latent occupancy states, $\text{N}(0, \sigma = 10)$ and block random effects $\text{N}(0, \sigma_{Block})$ 
where $\sigma_{Block}$ is drawn form uniform distribution $\text{U}(0, 50)$. 
The $\beta$ coefficients represent the weights assigned to each explanatory in the linear predictor.
Due to heteroscedasticity with $\text{Springtail}$, $\text{CWD}$ and $\text{Canopy}$, we allowed each overstory groups $j$ to have their own variances ($\sigma_j \sim \text{U}(0,150)$) :

\begin{align}
  \mu_{\text{Springtail}_i} &= \beta_{0[\text{Springtail}]} + \beta_{\text{Cutpartial}[\text{Springtail}]} \times \text{Cutpartial}_i + \nonumber\\
  &\beta_{\text{Cutclear}[Springtail]} \times \text{Cutclear}_i + \beta_{z_{\text{Salamander}}[\text{Springtail}]} \times z_{Salamander} +  \nonumber\\
  &\beta_{z_{\text{Carabid.prey}}[\text{Springtail}]} \times z_{Carabid.prey} + \beta_{z_{\text{Carabid.comp}}[\text{Springtail}]} \times z_{Carabid.comp} + \nonumber\\
  &\alpha_{\text{Block}[\text{Springtail}]_{\text{Block}_i}} \nonumber
\end{align}

\begin{align}
  \mu_{\text{CWD}_i} &= \beta_{0[\text{CWD}]} + \beta_{\text{Cutpartial}[\text{CWD}]} \times \text{Cutpartial}_{i} + \nonumber\\
  & \beta_{\text{Cutclear}[\text{CWD}]} \times \text{Cutclear}_{i} + \alpha_{\text{Block}[\text{CWD}]_{\text{block}_i}} 
\end{align}


\begin{align}
  \mu_{\text{Canopy}_i} &= \beta_{0[\text{Canopy}]} + \beta_{\text{Cutpartial}[\text{Canopy}]} \times \text{Cutpartial}_{i} + \nonumber \\
  & \beta_{\text{Cutclear}[\text{Canopy}]} \times \text{Cutclear}_{i} + \alpha_{\text{Block}[\text{Canopy}]_{\text{block}_i}} \nonumber
\end{align}

\begin{align}
  \mu_{\text{Litter}_i} &= \beta_{0[\text{Litter}]} + \beta_{\text{Cutpartial}[\text{Litter}]} \times \text{Cutpartial}_{i} + \nonumber\\
  & \beta_{\text{Cutclear}[\text{Litter}]} \times \text{Cutclear}_{i} + \alpha_{\text{Block}[\text{Litter}]_{\text{block}_i}} \nonumber
\end{align}


Analysis was performed with a Bayesian approach and parameters were estimated using Markov chain Monte Carlo (MCMC) with JAGS 4.3.0 include in the jagsUI package in R 4.3.1 \citep{lunnBUGSProjectEvolution2009,kellnerJagsUIWrapperRjags2024,rcoreteamLanguageEnvironmentStatistical2020}. 
We ran the model with five chains, 200,000 iterations each \citep{gelmanUnderstandingPredictiveInformation2014}. A burn-in period of 75,000 iterations was used and a thinning rate of 5 was applied. 
We verified the convergence of MCMC chains by examining trace plots, posterior density plots, and applying the Brooks-Gelman-Rubin statistic. 
The JAGS model code is available in Table

\clearpage

\section*{Results}
\label{sec:results1}
\phantomsection\addcontentsline{toc}{section}{\nameref{sec:results1}}


\subsection*{Soil fauna}
\label{subsec:taxa}
\phantomsection\addcontentsline{toc}{subsection}{\nameref{subsec:taxa}} 

Model diagnostics indicated that the chain lengths were sufficient, as the Brooks-Gelman-Rubin statistic was below 1.1. 
Trace plot analysis revealed that all chains had converged towards similar values, and none of the ratios of MCMC error to posterior standard deviation exceeded 5\%.

\vspace{10pt}

\begin{figure}[ht]
	\centering
	\includegraphics[scale=0.60]{fig_sem_res3.png}
	\caption[Results from structural equation modeling analysis revealing effects of overstory treatments on coarse woody debris volume,
  canopy openness, litter depth, salamander occupancy, ground beetle occupancy, and springtail biomass.]
  {Results from SEM analysis showing effects of overstory treatments on CWD, 
  canopy openness, litter depth, salamander occupancy, ground beetle occupancy, and springtail biomass in the Portneuf Wildlife Reserve, 
  Quebec, Canada. Bold arrows represent significant effects, while gray arrows indicate no discernible effects. 
  Values above bold arrows represent average differences between posterior distributions of two overstory groups: 
  partial-cut (PC), clear-cut (CC), and control (C). Estimates marked with one asterisk (*) 
  indicate a 90\% credible interval (CI) excluding 0, while estimates marked with two asterisks (**) indicate a 95\% CI excluding 0.}
	\label{fig:SEMres}
\end{figure}  

\vspace{10pt}

The average salamander observation per survey was 8 individuals with a range of 17 salamanders, during the summer 2022.
We captured 189 ground beetles belonging to 30 species, with 49 ground beetles found in the partial-cut treatments, 105 in the clear-cuts, and 35 in the control sites (Table \ref{tab:carabid}). 
We collected 468 springtails representing 12 families, with 219 springtails collected from partial-cut treatments, 131 from clear-cuts, and 118 from the control areas (Table \ref{tab:springtail}). 
The average springtail biomass collected per overstory treatments was 24.3 $\mu$g (SD = 18.2 $\mu$g), 56.8 $\mu$g (SD = 78.0 $\mu$g), and 31.1 $\mu$g (SD = 52.8 $\mu$g) in the partial-cut treatments, clear-cuts and control sites, respectively.

The effects of overstory treatments on habitat selection by species groups were generally not significant. 
Salamander occupancy probability was marginally lower in sites subjected to clear-cutting compared to those with partial-cutting (90\% CI : [-0.74, -0.07], Figure \ref{fig:pcin}, Table \ref{tab:overstorysp}). 
However, these two groups did not differ from the control sites. 
Occupancy probability for both carabid groups and the springtail biomass, did not vary significantly between the overstory treatments and control sites (Table \ref{tab:overstorysp}). 
Overall, the presence of salamanders had no significant impact on the occupancy probabilities of ground beetles representing prey for salamanders, and we did not observe significant effects of either salamanders or ground beetles on springtail biomass (Table \ref{tab:overstorysp}).

\vspace{10pt}

\begin{table}[ht]
  \centering
  \caption[Differences between overstory treatments on the forest soil fauna.]
  {Differences between overstory treatments on salamander occupancy, both ground beetle groups occupancy and springtail biomass. 
  This table also show estimated effect of salamanders presence on ground beetles form the salamanders prey group occupancy 
  and the effects of salamanders and both ground beetle groups presence on springtail biomass, during the summer 2022 in the Portneuf Wildlife Reserve, Quebec, Canada.}
  \label{tab:overstorysp}
  \begin{tabular}{lllll} 
      \hline
      &&&&95\% Bayesian \\
      Variable&Unit& Comparison & Estimate &  credible interval \\ [0.5ex] 
      \hline     
      Salamander           &probability& Partial vs control & \hspace{1mm}0.07 & [-0.29, 0.45] \\ 
      occupancy       && Clear vs control  & -0.38 & [-0.75, 0.11] \\ 
                          && Clear vs partial  & -0.45 & [-0.74, -0.07]$^{a}$ \\       
      Carabid$_{competitor}$ &probability& Partial vs control & -0.12 & [-0.35, 0.15] \\
      occupancy       && Clear vs control  & -0.06 & [-0.29, 0.20] \\ 
                          && Clear vs partial  & \hspace{1mm}0.06 & [-0.19, 0.30] \\ 
      Carabid$_{prey}$    &logit& Partial vs control & \hspace{1mm}3.31 & [-10.12, 17.72] \\
      occupancy             && Clear vs control  & \hspace{1mm}10.19 & [-4.15, 24.45] \\ 
                          && Clear vs partial  & \hspace{1mm}6.88 & [-12.81, 23.42] \\  
                          && Salamander        & -2.20 & [-17.15, 16.59] \\  
      Springtail          &$\mu$g& Partial vs control & \hspace{1mm}8.11 & [-9.38, 25.40] \\
      biomass             && Clear vs control  & \hspace{1mm}2.11 & [-13.98, 18.11] \\ 
                          && Clear vs partial  & -6.00 & [-29.09, 17.26] \\  
                          && Salamander        & \hspace{1mm}6.80 & [-10.43, 23.26] \\ 
                          && Carabid$_{competitor}$      & \hspace{1mm}0.56 & [-16.75, 17.66] \\ 
                          && Carabid$_{prey}$      & \hspace{1mm}7.62 & [-8.93, 24.09] \\ 
      \hline
      \multicolumn{5}{l}{\textbf{Note:} Estimates from Bayesian SEM are presented in terms of mean number with 95\%} \\
      \multicolumn{5}{l}{credible intervals, where an interval excluding 0 indicates a difference between groups.} \\
      \multicolumn{5}{l}{$^{a}$Marginal difference based on 90\% Bayesian credible interval excluding 0}
  \end{tabular}
\end{table}


\clearpage

\begin{figure}[ht]
  \centering
  \includegraphics[scale=0.55]{fig_pcin.png}
  \caption[Occupancy probability of salamanders under overstory treatments]
  {Occupancy probability of salamanders within two overstory treatments and controls during the summer 2022 in the Portneuf Wildlife Reserve device, Quebec, Canada. 
  Error bars denote 95\% credible intervals around estimates.}
  \label{fig:pcin}
\end{figure}

\vspace{10pt}

We did not observe significant impacts of CWD and precipitation level on salamander detection probabilities. 
However, the precipitation level had a positive effect on detection probability for both small ground beetles (95\% CI : [0.59, 1.77]) and large carabid (95\% CI : [0.70, 3.23]) (Table \ref{tab:detection}). 
CWD had no significant impact on ground beetle detection probabilities.


\begin{table}[ht]
  \centering
  \caption[Estimated effects of coarse woody debris and precipitation level on detection probabilities of salamanders and both ground beetle groups.]
  {Estimated effects of coarse woody debris and precipitation level on detection probabilities of salamanders and both ground beetle groups, during the summer 2022 in the Portneuf Wildlife Reserve,  Quebec, Canada.}
  \label{tab:detection}
  \begin{tabular}{lllll} 
      \hline
      &&&95\% Bayesian \\
      Variable & Taxa & Estimate &  credible interval \\ [0.5ex] 
      \hline      
      Precipitation       & Salamander              & \hspace{1mm}0.11 & [-0.83, 1.06] \\ 
                          & Carabid$_{competitor}$  & \hspace{1mm}1.17 & [0.59, 1.77] \\ 
                          & Carabid$_{prey}$        & \hspace{1mm}1.87 & [0.70, 3.23] \\  
      \hline      
      Coarse woody debris & Salamander              & -0.59 & [-1.39, 0.12] \\ 
                          & Carabid$_{competitor}$  & \hspace{1mm}0.06 & [-0.26, 0.38] \\ 
                          & Carabid$_{prey}$        & \hspace{1mm}0.27 & [-0.74, 1.37] \\   

      \hline
      \multicolumn{4}{l}{\textbf{Note:} Estimates from Bayesian SEM are presented in terms of mean} \\
      \multicolumn{4}{l}{number with 95\% credible intervals, where an interval excluding 0 indicates} \\
      \multicolumn{4}{l}{a difference between groups.} \\
  \end{tabular}
\end{table}


\subsection*{Environmental variables}
\label{subsec:ResEnv}
\phantomsection\addcontentsline{toc}{subsection}{\nameref{subsec:ResEnv}} 

Environmental variables usually differed between forest cutting treatments and control conditions. 
We found that clear-cutting treatments had significantly less CWD compared to partial-cutting (95\% CI : [-1.15, -0.43]) (Figure \ref{fig:envar} A, Table \ref{tab:overstoryenvar}). 
However, these both treatments did not differ from the control sites. 
Canopy openness was significantly higher in both the partial-cut (95\% CI : [1.97, 11.02]) and clear-cut treatments (95\% CI : [51.39, 77.06]) when compared 
to the control sites, with the clear-cuts having a greater percentage of openness than the partial-cuts (95\% CI : [44.61, 70.76], Figure \ref{fig:envar} B, Table \ref{tab:overstoryenvar}). 
Conversely, litter depth was lower in both the partial-cut (95\% CI : [-2.44, -0.65]) and clear-cut treatments (95\% CI : [-4.28, -2.50]) compared to the controls, 
with the depth being lower in the clear-cuts than in the partial-cuts (95\% CI : [-2.57, -1.12], Figure \ref{fig:envar} C, Table \ref{tab:overstoryenvar}).

\vspace{10pt}

\begin{figure}[ht]
  \centering
  \includegraphics[scale=0.23]{fig_envar2.png}
  \caption[Environmental variables estimations with a potential effect soil species within two different overstory treatments and control.]
  {Environmental variables with a potential effect soil species estimations within two different overstory treatments and control 
  during the summer 2022 in the Portneuf Wildlife Reserve, Quebec, Canada. Error bars denote 95\% credible intervals around estimates.}
  \label{fig:envar}
\end{figure}

\begin{table}[ht]
  \centering
  \caption[Differences between overstory treatments on environmental variables that could affect habitat selection of fauna on the forest soil.]
  {Differences between overstory treatments on environmental variables that could affect habitat selection of fauna on the forest soil during the summer 2022 in the Portneuf Wildlife Reserve,
  Quebec, Canada.}
  \label{tab:overstoryenvar}
  \begin{tabular}{lllll} 
      \hline
      &&&&95\% Bayesian \\
      Variable&Unit& Comparison & Estimate &  credible interval \\ [0.5ex] 
      \hline
      Coarse woody debris &m\up{3}& Partial vs control & \hspace{1mm}0.02 & [-1.01, 1.06] \\ 
                 && Clear vs control  & -0.77 & [-1.79, 0.23] \\ 
                          && Clear vs partial  & -0.79 & [-1.15, -0.43] \\
      Canopy openness     &\%& Partial vs control & \hspace{1mm}6.49 & [1.97, 11.02] \\ 
                      && Clear vs control  & \hspace{1mm}64.19 & [51.39, 77.06] \\ 
                          && Clear vs partial  & \hspace{1mm}57.69 & [44.61, 70.76] \\ 
      Litter depth        &cm& Partial vs control & -1.54 & [-2.44, -0.65] \\ 
                      && Clear vs control  & -3.39 & [-4.28, -2.50] \\ 
                          && Clear vs partial  & -1.85 & [-2.57, -1.12] \\       
      \hline
      \multicolumn{5}{l}{\textbf{Note:} Estimates from Bayesian SEM are presented in terms of mean number with 95\%} \\
      \multicolumn{5}{l}{credible intervals, where an interval excluding 0 indicates a difference between groups.} \\
  \end{tabular}
\end{table}




\clearpage

\section*{Discussion}
\label{sec:discu1}
\phantomsection\addcontentsline{toc}{section}{\nameref{sec:discu1}}

\section*{Conclusion}
\label{sec:conclu1}
\phantomsection\addcontentsline{toc}{section}{\nameref{sec:conclu1}}

\section*{Acknowledgements}
\label{sec:acknowl1}
\phantomsection\addcontentsline{toc}{section}{\nameref{sec:acknowl1}}

\section*{Conflict of interest}
\label{sec:conflict1}
\phantomsection\addcontentsline{toc}{section}{\nameref{sec:conflict1}}

None declared
\section*{Author contributions}
\label{sec:author1}
\phantomsection\addcontentsline{toc}{section}{\nameref{sec:author1}}

\cleardoublepage

\begin{otherlanguage}{english}
\bibliographystyle{ecologyNewEN} % Style de citation en français
\bibliography{References}
\addcontentsline{toc}{section}{References}
\end{otherlanguage}
