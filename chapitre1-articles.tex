\chapter{Direct and indirect effects on soil fauna of silvicultural treatments in the context of assisted forest migration}     % numéroté
\label{chapitre1-articles}    

William Devos$^1$, Mathieu Bouchard$^1$, Marc J. Mazerolle$^1$

%href{mailto:william.devos.1@ulaval.ca}
$^1$ Centre d'étude de la forêt, Département des sciences du bois \\ 
et de la forêt, Université Laval, Québec, QC G1V 0A6, Canada. \\ 

\clearpage

\section*{Résumé}
\label{sec:resume1}
\phantomsection\addcontentsline{toc}{section}{\nameref{sec:resume1}}

\begin{otherlanguage*}{french}
  <Résumé de l'article en français. Obligatoire.>

  \textbf{Mots-clés} : <ajouter des mots clés>
\end{otherlanguage*}

\clearpage

\section*{Abstract}
\label{sec:abstract1}
\phantomsection\addcontentsline{toc}{section}{\nameref{sec:abstract1}}

\begin{otherlanguage*}{english}
  <English abstract of the paper. Optional, but recommended.>

\textbf{Keywords}: <add some keywords> 
\end{otherlanguage*}

\cleardoublepage

\section*{Introduction}
\label{sec:intro1}
\phantomsection\addcontentsline{toc}{section}{\nameref{sec:intro1}}

%\defcitealias{keylist}{alias}

Forest ecosystems play a vital role in the biosphere, both economically and ecologically, by providing essential services such as carbon sequestration, climate regulation, water retention, and biodiversity conservation \citep{Balvanera2006Quantifyingevidence,Diaz2006BiodiversityLoss,Canadell2008Managingforests,Pawson2013Plantationforests}. 
However, the growing demand for forest products and other services has intensified harvesting practices, leading to ecosystem degradation and loss of biodiversity\citep{Bengtsson2000Biodiversitydisturbances,Sala2000Globalbiodiversity,Foley2005GlobalConsequences,Naeem2012functionsbiological}. 

Numerous studies have highlighted the effects of forest management on various species, including birds, bats, butterflies, turtles, small mammals, and insects \citep{Summerville2011Managingforest,Currylow2012ShortTermForest,Kaminski2013EffectsForest,Kellner2013Shorttermresponses,Caldwell2019ComparisonBat}. .
Logging activities cause habitat loss and fragmentation, reducing access to food, shelter, and breeding sites for certain species \citep{Bouderbala2023Longtermeffect}. 
It also reduces connectivity between habitats, restricting individual movement, which reduces genetic flow between populations and increases the risk of local extinction \citep{Saccheri1998Inbreedingextinction}. 
At the stand level, logging often results in an overrepresentation of early-successional forests, reducing the presence of later successional stages \citep{Cyr2009Forestmanagement,Boucher2017Cumulativepatterns}. 
This shift reduces the structural complexity of stands, such as tree species composition, vertical stratification, age structure, successional dynamics, and disturbance frequency \citep{Commarmot2005Structurevirgin}. 
Moreover, stands with more homogeneous structures are less resilient to disturbances, as it slows down the regeneration of existing tree species and the return of those that have disappeared \citep{Kuuluvainen2009Forestmanagement}. 

Overall, modifications of forest attributes contribute to a loss of species and functional diversity. 
In the long term, this erodes the resilience of forests at a local scale and can lead to a decline in ecosystem services \citep{Hooper2012globalsynthesis,Edwards2014Maintainingecosystem}.

%

Among the silvicultural treatments that significantly affect natural environments, logging plays a major role. 
However, the level of disturbance caused by these operations depends on the type of treatment employed \citep{Ameray2021Forestcarbon}. 

Clear-cutting is one of the most widely used practices in temperate and boreal forests \citep{Fedrowitz2014Canretention,Chaudhary2016Impactforest}. 
It is part of intensive forest management used to increase short-term wood productivity and quality \citep{Irland2011TimberProductivitya}.
This method involves the complete removal of commercial trees and results in a single-species, even-aged structure, leading to a drastic homogenization of stands \citep{Rosenvald2008whatwhen}. 
Additionally, shorter rotation periods increase the frequency of disturbances. 
These practices substantially alter ecosystem structures compared to natural conditions, increasing the risk of biodiversity loss and local extinctions \citep{Hanski2000Extinctiondebt}. 
However, some authors suggest that clear-cutting can mimic large-scale natural disturbances, such as wildfires or storms \citep{Greenberg1995comparisonbird}. 

Over the past few decades, ecosystem-based management has been proposed as a more sustainable approach to forest ecosystem \citep{Perry1998scientificbasis,Kuuluvainen2002Naturalvariabilitya}. 
This strategy aims to emulate natural disturbances and the resulting stand structures and successional processes.
The goal is to preserve biodiversity and maintain the resilience of forest ecosystems while ensuring the availability of a variety of ecosystem services \citep{Szaro1998emergenceecosystem,MacDicken2015Globalprogress}. 

Partial cuts are thus integrated into extensive management plans that prioritize regeneration and mimic natural disturbances \citep{Irland2011Timberproductivity}. 
These treatments rely on multi-aged structures, potentially including species mixes, and are characterized by longer rotations \citep{Kuuluvainen2009Forestmanagement}. 
They are used to stimulate the growth of the most vigorous trees, promote species diversity, or maintain an open canopy \citep{Irland2011Timberproductivity}.
Tree retention associated with these types of cuts promotes late-successional structures, which support richer biodiversity \citep{Ameray2021Forestcarbon}. 
Partial cuts also also help carbon sequestration and the supply of organic matter, while maintaining a heterogeneous structure and diverse ecological niches for wildlife \citep{Dahlgren1994effectswholetree,Barg1999Influencepartial,Tong2020Forestmanagement,Ameray2021Forestcarbon}. 

In addition to the challenge of conserving biodiversity and forest ecosystems while meeting economic demands, climate change adds further complexity to forest management. 
Rising global temperatures pose an additional threat to the sustainability of flora and fauna by significantly altering environmental conditions \citep{McKenney2009Climatechange,Trumbore2015Foresthealth,Seidl2017Forestdisturbances,Messier2022Warningnatural}.  
Climate stresses often act additively or synergistically with forestry activities, amplifying their impact on biodiversity and ecosystems \citep{Brook2008Synergiesextinction,Tremblay2018Harvestinginteracts,Ochs2022Responseterrestrial,Bouderbala2023Longtermeffect}. 
Forest composition could thus be altered, requiring adjustments in management practices and conservation strategies \citep{McKenney2009Climatechange,Chmura2011Forestresponses,Lo2011Linkingclimate}. 

For example, assisted tree migration, which involves relocating species or genetic material from their original climatic region to areas better suited to future conditions, 
has been proposed as a potential mitigation strategy to maintain ecosystem services and the economic value of forests \citep{Vitt2010Assistedmigration,Pedlar2011implementationassisted,Ste-Marie2011Assistedmigration,Winder2011Ecologicalimplications}. 
However, there remains a lack of knowledge and uncertainties surrounding these new adaptation methods \citep{Klenk2015assistedmigration,Park2018Informationunderload}. 
Therefore, it is essential to better understand the impact of silvicultural practices on biodiversity, especially in a context where forest management must adapt to climate change, biodiversity loss, and economic pressures.

%

Soil fauna plays a major role in forest ecosystems by facilitating the flow of matter and energy. 
However, this community is among the most impacted by disturbances due to their sensitivity to environmental changes and limited dispersal abilities.

Canopy removal significantly alters soil conditions by increasing its exposure to sunlight, leading to increased temperatures, changes in humidity, higher wind speeds, and intensified precipitation. 
The movement of machinery also increases soil compaction, reducing porosity and consequently affecting the organisms that live there.
These changes at the surface level affect nutrient availability by altering litter, root secretions, leaching, and soil chemical properties. 
Logging also reduces microhabitats, such as deadwood, cavities in mature trees, and root plates, which provide shelter for soil fauna. 
These modifications create unfavorable conditions for species that rely on cool, moist environments, such as amphibians and arthropods, leading to population declines or local extinctions.

Due to their sensitivity to environmental shifts and limited dispersal capacity, amphibians and arthropods serve as important indicators to understand the effects of forestry practices on biodiversity and the ecological integrity of forests. 
Moreover, both groups have experienced significant declines in recent decades, primarily due to forest management practices and climate change. 
Species frequently studied in this context include the Eastern Red-backed Salamander (\textit{Plethodon cinereus} (Green, 1818)), ground beetles (Carabidae), and springtails (Collembola).

The Eastern Red backed Salamander represents one of the largest vertebrate biomasses in North American forests \citep{Burton1975Salamanderpopulations,Petranka1993Effectstimber,semlitschAbundanceBiomassProduction2014a}. 
As a generalist predator, this plethodontid species plays a key role in regulating detritivorous invertebrates, influencing decomposition processes, nutrient cycling, and carbon dynamics \citep{Burton1975Energyflow,Wyman1998Experimentalassessment,Walton2013Topdownregulation,Hickerson2017Easternredbacked}. 
Moreover, it serves as a high nutritional prey for many predators, including birds, mammals, and reptiles \citep{Burton1975Energyflow,Pough1987abundancesalamanders}. 
Without lungs, the Eastern Red backed Salamander relies on cutaneous respiration for gas exchange \citep{Heatwole1961Relationsubstrate}. 
This type of respiration requires salamanders to occupy specific microhabitats, staying on the surface when temperature and humidity are favorable, or retreating into the soil during less suitable conditions \citep{Grizzell1949HibernationSite,FraserEmpiricalEvaluation1976,Jaeger1980MicrohabitatsTerrestrial}. 
The loss of refuges due to logging decreases habitat quality, limiting the time it spends on the forest floor to feed and reproduce \citep{Achat2015Quantifyingconsequences,Peele2017EffectsWoody}. 
Consequently, poor surface conditions, soil compaction, and low levels of coarse woody debris can negatively impact population dynamics \citep{Peterman2014Spatialvariation}. 

On the other hand, ground beetles are a well documented taxon both in terms of taxonomy and ecology \citep{loveiEcologyBehaviorGround1996}. 
These insects have short lifespans, occupy a high position in the soil food web, and respond rapidly and in complex ways to environmental changes \citep{loveiEcologyBehaviorGround1996}. 
They play an ecological role by regulating invertebrate populations, while also serving as prey for various amphibians, reptiles, birds, and mammals \citep{loveiEcologyBehaviorGround1996}. 
With around 40,000 known species, ground beetles are one of the most diverse beetle families and among the most abundant soil dwelling arthropods \citep{Erwin1985taxonpulse,loveiEcologyBehaviorGround1996,Rochefort2006GroundBeetle}. 
They are widely distributed across nearly all terrestrial ecosystems, although species vary in their habitat preferences  \citep{loveiEcologyBehaviorGround1996,kotzeFortyYearsCarabid2011a,Larochelle2003naturalhistory}. 
Environmental changes can benefit some species at the expense of others, making their diversity and sensitivity valuable for studying the effects of environmental disturbances \citep{Rainio2003Groundbeetles}. 

Springtails represent one of the most abundant and diverse groups within the mesofauna \citep{rusekBiodiversityCollembolaTheir1998}. 
Different springtail communities occupy a variety of ecological niches within the soil \citep{pongeVerticalDistributionCollembola2000}.
Their vertical distribution is largely influenced by abiotic factors such as light, humidity levels, and soil porosity. 
Primarily fungivores and detritivores, springtails play a key role in the decomposition of wood. 
They also impact the physical structure and mineralization rates of litter, influence the nutrient absorption, 
regulate microbial communities, and contribute to soil microstructure formation \citep{Petersen1982comparativeanalysis,Neher2012Linkinginvertebrate,Maass2015Functionalrole,Potapov2016Connectingtaxonomy}. 
Additionally, springtails serve as an important food source for a range of organisms, including amphibians, beetles, arachnids, birds, and reptiles.

Through their trophic relationships, their sensitivity to environmental condition changes, and their dependence on forest features such as woody debris and litter, 
these three species groups are relevant models for studying the impact of forestry practices on soil fauna.

Most studies addressing the impacts of forestry treatments on fauna generally discuss the direct effects of disturbances on one or more species groups, 
without considering the relationships between environmental variables and different species groups, thereby neglecting the indirect effects of logging on soil fauna. 
My project aimed to fill this gap by trying to understand how the effects of forestry treatments propagate through the forest ecological network and influence soil fauna dynamics. 
Ultimately, this knowledge gain will provide valuable tools to facilitate sustainable forest management.

%% re voire la fin et regarder Laigle si on peux améliorer

These three taxa are relevant to quantifying the impact of silvicultural treatments as their sensitivity to environmental changes and trophic relationships enable the analysis of disturbance effects on soil fauna dynamics \citep{Salmon2008Relationshipssoil}.
Several papers have already studied the effects of overstory treatments on soil fauna.
However, most of those researches have only focused on the direct impacts of disturbances for one or more species groups, neglecting the potential relationships between environmental variables and species groups \citep{josephIntegratingOccupancyModels2016,Pollierer2021Diversityfunctional,Kudrin2023metaanalysiseffects}. 
Quantifying treatments effects and their propagation within the ecological network, namely on soil fauna, will provide useful tools to improve sustainable forest management.

Our study aimed to understand how silvicultural practices, conducted in an assisted tree migration context, affect the dynamics of forest soil ecosystems. 
The specific objectives were to quantify the effect of overstory treatments on environmental variables that influence habitat use by soil fauna
and to evaluate the impact of overstory treatments on habitat use by soil fauna.
We hypothesized that environmental variables, favorable to habitat use by taxa, fluctuate according to the intensity of forest harvests. 
We predicted that more intense treatments would decrease litter depth and CWD, as a consequence of reduced accumulation of leaves and woody debris on the forest floor and increase canopy openness. 
We also hypothesized that overstory treatments modify habitat use by soil fauna and propagate through the trophic network. 
We predicted that tree harvest impacts occupancy of salamanders and large ground beetles, 
followed by small ground beetles and ultimately affecting springtail biomass. 


\section*{Material and methods}
\label{sec:matmet1}
\phantomsection\addcontentsline{toc}{section}{\nameref{sec:matmet1}}

\subsection*{Study area}
\label{subsec:area}
\phantomsection\addcontentsline{toc}{subsection}{\nameref{subsec:area}}

\begin{otherlanguage*}{english}

  Our study was conducted within the Portneuf Wildlife Reserve in the Capitale-Nationale administrative region near Lac des Amanites (47°07’N, 72°24’W, Figure \ref{fig:area}). 
  This area is located within the balsam fir (\textit{Abies balsamea})-yellow birch (\textit{Betula alleghaniensis}) bioclimatic domain, according to the ecological classification used in Québec \citep{saucierChapitreEcologieForestiere2009}. 
  Other tree species in this bioclimatic domain included sugar maple (\textit{Acer saccharum}), red maple (\textit{Acer rubrum}), white spruce (\textit{Picea glauca}), black spruce (\textit{Picea mariana}), red spruce (\textit{Picea rubens}), white birch (\textit{Betula papyrifera}), and quaking aspen (\textit{Populus tremuloides})\citep{olaBelowgroundCarbonStocks2024}. 
  The experimental sites are located on deep glacial tills with moderately well-drained sandy loams soil \citep{CanadianSystemSoil1998}. 
  The mean daily temperature is 4\up{o}C based on the 1981-2010 period at the nearest weather station (Lac aux sables, \citealp{environmentcanadaCanadianClimateNormals2019}). 
  Based on the same report, the average annual precipitation, including snow-covered months, is 1133.2 mm, with snowfall averaging 230.3 cm.

  We conducted our study within the assisted migration experimental system established in 2018 by the Ministère des Ressources naturelles et des Forêts (\citealp{royoDesiredREgenerationAssisted2023}). 
  This experimental system uses a factorial experimental design with split-plots replicated in four blocks. 
  Each whole block (200 m x 140 m) is split in two overstory treatment : clear-cut and regular shelterwood cut at 50\% of the merchantable basal area (partial-cut). 
  The harvest took place during the summer 2017, followed by trenching in June 2018 within the clear-cut areas to create more uniform site conditions and facilitate planting. 
  No site preparation was conducted in the shelterwood cuts.

\end{otherlanguage*}

\begin{figure}[ht!]
	\centering
	\includegraphics[scale=0.31]{fig_area6.png}
	\caption[Localization of the Capitale-Nationale administrative region in Quebec, Canada and position of the study area near Lac des Amanites in Portneuf Wildlife Reserve, Quebec, Canada.]
  {Localization of the Captial-Nationale administrative region in Quebec, Canada (A) and position of the study area in Portneuf Wildlife Reserve, Quebec, Canada (B) where the clear-cut (orange) and the partial-cut (green) associated with the assisted migration experimental system were applied in 2017 (47°07'N, 72°24'W)(C).}
	\label{fig:area}
	\end{figure}  


\subsection*{Environmental variables}
\label{subsec:EnvVar}
\phantomsection\addcontentsline{toc}{subsection}{\nameref{subsec:EnvVar}}

We selected a total of 60 sampling units measuring 10 m by 7.5 m to collect our data: we used four blocks serving as replicates, 
with each block containing six sampling units for both the clear-cut and partial-cut overstory treatments, 
whereas three additional sampling units were positioned outside the block, serving as uncut controls (Figure \ref*{fig:blockSU}). 
The uncut controls were separated from blocks by at least 10 m. 

In each sampling unit, we measured several environmental variables that could affect occupancy probability of red-backed salamanders, ground beetles, and springtails.
CWD and litter depth play a crucial role in habitat use for salamanders, ground beetles and springtails as
they serve for feeding and protection  \citep{harmonEcologyCoarseWoody1986,koivula.LeafLitterSmallscale1999,birdChangesSoilLitter2004,mckennyEffectsStructuralComplexity2006}. 
Salamanders also utilize CWD as shelter to maintain suitable temperature and moisture levels during dry periods \citep{Jaeger1980MicrohabitatsTerrestrial,groverInfluenceCoverMoisture1998a,patrickEffectsExperimentalForestry2006a}.
We used 400 m\up2 plots centered on every sampling unit to estimate CWD (20 m $\times$  20 m) (\citealp{methotGuideInventaireEchantillonnage2014}). 
We only considered CWD with a basal diameter $\geq$ 9 cm and a length $\geq$ 1 m.
For each CWD, we measured the basal diameter, the apical diameter and length with a tree caliper.
Segments of CWD outside the plot boundaries were not included.
We employed the conic–paraboloid formula to estimate the volume of each CWD \citep{fraverRefiningVolumeEstimates2007} :

\begin{equation}
  \text{Volume} = L/12 \times (5A_b + 5A_u + 2\sqrt{A_b \times A_u})
\end{equation}

\vspace{0.5cm}

Where $L$ is the length of log (cm), $A_b$ the basal area (cm\up{2}) and $A_u$ the apical area (cm\up{2}).
We measured litter depth next to each coverboard and computed the average litter depth for each sampling unit \citep{Mazerolle2021Woodlandsalamander}. \\

We assessed canopy openness, as it may influence species occupancy \citep{messereForestFloorDistribution1998,koivulaBorealCarabidbeetleColeoptera2002a,tilghmanMetaanalysisEffectsCanopy2012,henneronForestPlantCommunity2017}.
Canopy openness was measured 130 cm above the ground at the center of each sampling unit using a spherical densiometer \citep{lemmonSphericalDensiometerEstimating1956}. 
We averaged four measurements per sampling unit, oriented toward each of the four cardinal points as an estimate of canopy openness.

We collected data locally for air temperature, air humidity and precipitation levels during the summer 2022. 
Two weather stations (Em50 Digital Decagon Data Logger, Part \#40800, Meter Group Inc., USA) were installed inside both overstory treatments. 
Each weather station measured temperature, air humidity, and atmospheric pressure, 130 cm above the ground (VP-4 Sensor (Temp/RH/Barometer), Part \#40023). 
Rain gauges were installed in the clear-cut treatments to monitor precipitation levels. 
The temperature and humidity sensors were programmed to record data every 15 minutes. 
We averaged the measurements across both weather stations to get daily average measurements. 
These variables fluctuate on a daily basis, affecting species activity and, consequently, the probability of detecting individuals \citep{spotilaRoleTemperatureWater1972,butterfieldCarabidLifeCycle1996,loveiEcologyBehaviorGround1996,odonnellPredictingVariationMicrohabitat2014a}.


\subsection*{Sampling design}
\label{subsec:sampling}
\phantomsection\addcontentsline{toc}{subsection}{\nameref{subsec:sampling}}

In each sampling unit, we used three sampling methods to collect species data : artificial coverboards, pitfall traps, and soil cores.
We used artificial coverboards to sample eastern red-backed salamanders \textit{Plethodon cinereus} at our study sites \citep{hydeSamplingPlethodontidSalamanders2001,mooreComparisonPopulationEastern2009c,hesedUncoveringSalamanderEcology2012,Mazerolle2021Woodlandsalamander}. 
Coverboards consisted of 25 cm x 30 cm x 5 cm pieces of untreated spruce wood. Each board was in direct contact with the soil after we had cleared the litter underneath \citep{Mazerolle2021Woodlandsalamander}. 
Six coverboards spaces by at least 2.5 m were arranged in a rectangular array in each sampling unit, resulting in a total of 360 coverboards across our 60 sampling stations (Figure \ref{fig:blockSU}). 
All coverboards were placed outdoors in March 2022 to allow for natural aging \citep{hedrickEffectsCoverboardAge2021,Grasser2014Effectscover}. 
We conducted four visits to coverboards during the summer 2022, with each visit spaced one month apart, namely in mid-May, mid-June, mid-July and mid-August, during the summer 2022. 
Blocks were visited in a random sequence to reduce the potential effects of time of day and observer fatigue. 
During a given visit, the 360 coverboards were inspected on the same day and we counted the number of red-backed salamanders underneath.  

We used pitfall traps were used to capture ground beetles \citep{baarsCatchesPitfallTraps1979,spenceSamplingCarabidAssemblages1994a,loveiEcologyBehaviorGround1996,kotzeFortyYearsCarabid2011a,knappEffectPitfallTrap2012}. 
Trap design was based on Multipher\up{\textregistered{}} traps and included a container with a diameter of 12.5 cm, a depth of 25 cm and a cover raised 4.5 cm above the trap 
to prevent debris and rain from filling the container \citep{Jobin1988MultiPherinsect,mooreEffectsTwoSilvicultural2004,bouchardBeetleCommunityResponse2016b}. 
We covered each trap with a protective stainless steel mesh size of 15 mm, allowing trap access to carabid-sized individuals and limiting access by predators.  
We did not add preserving liquid in traps, but we added wet sponges to avoid harming vertebrates small enough to pass through the mesh. 
We centered a pitfall in each sampling unit (Figure \ref{fig:blockSU}). 
Traps were inserted in the soil at a depth allowing the container’s opening to be level with the soil surface. 
Trapping occurred during four sampling periods during 2022 (mid-May, mid-June, mid-July and mid-August). 
Traps were opened on the first day of each survey, and we collected captured ground beetles daily during five days. 
Outside of the trapping periods, pitfall traps were sealed with adhesive tape to prevent captures. 
Individuals were preserved in 70\% alcohol and identified at the species level afterwards. 
Identification was conducted with a ZEISS SteREO Discovery.V12 bionocular microscope using the \cite{larochelleManuelIdentificationCarabidae1976} taxonomic keys. 
We categorized ground beetles into two groups, either as small carabids (salamander prey )or large carabids (salamander competitors), based on the red-backed salamander gape size (Table \ref{tab:carabid}, \citealp{jaegerFoodLimitedResource1972,magliaModulationPreycaptureBehavior1995,magliaOntogenyFeedingEcology1996}).

We extracted soil cores to sample springtails \citep{pongeVerticalDistributionCollembola2000,salamonEffectsPlantDiversity2004,chauvatChangesSoilFaunal2011a,farskaManagementIntensityAffects2014}. 
Core sampling occurred during the same four sampling periods as for the salamanders and ground beetles (mid-May, mid-June, mid-July and mid-August). 
During a sampling period, we collected two cores in each sampling unit using a pedological probe (5 cm diameter x 5 cm depth). 
We also collected a 15 cm x 15 cm litter quadrat each soil sample \citep{raymond-leonardSpringtailCommunityStructure2018a,rousseauForestFloorMesofauna2018}.
We extracted springtail communities directly related to the ecology of salamanders and ground beetles \citep{edwardsAssessmentPopulationsSoilinhabiting1991,chauvatChangesSoilFaunal2011a,raymond-leonardSpringtailCommunityStructure2018a,rousseauForestFloorMesofauna2018}.
Soil and litter from the same unit were pooled in Ziploc\up{\texttrademark{}} bags and stored in a cooler at ca. 4 °C \citep{chauvatChangesSoilFaunal2011a,rousseauForestFloorMesofauna2018}, providing 60 samples during each of the four sampling periods.
We placed each sample in an individual Tullgren dry-funnel for springtail extraction within 48 h after collection \citep{rusekBiodiversityCollembolaTheir1998,wuCompositionSpatiotemporalVariation2014,rousseauForestFloorMesofauna2018}. 
The extraction process lasted six days with a gradual temperature increase (25 °C to 50 °C) \citep{raymond-leonardSpringtailCommunityStructure2018a}.
Springtails were separated from other invertebrates, pooled by sampling unit and preserved in 75\% alcohol \citep{wuCompositionSpatiotemporalVariation2014}.
Identification at the family level for all individuals was done with a ZEISS SteREO Discovery.V12 binocular microscope and a Leitz orthoplan phase-contrast fluorescent trinocular microscope using \cite{bellingerChecklistCollembolaWorld1996} identification keys. 
Following identification, springtails within each sample were dried in a freeze dryer (Labconco FreeZone Bulk tray dryer 78060 series) for 24 hours. 
We determined the total dry biomass of springtails in each sample with a micro balance (Sartorius Cubis\up{\texttrademark{}} MSA3.6P-000-DM).

\pagebreak

\begin{figure}[ht]
	\centering
	\includegraphics[scale=0.50]{fig_blockSU.png}
	\caption[Design of one block and one sampling unit with three sampling methods.]{
  Design of a block (left) and a sampling unit (right). 
  The block contains two overstory treatments : clear-cut (grey background), partial-cut (white background). 
  Fifteen sampling units were used per block : six per overstory treatment and three controls (\textbf{c}) outside each block.
  Each sampling unit contained six artificial coverboards (squares) and one pitfall trap (circle). Two soil cores (stars) were collected per survey.
  }
	\label{fig:blockSU}
	\end{figure}  

\vspace{0.5cm}

\subsection*{Statistical analyses}
\label{subsec:analyses}
\phantomsection\addcontentsline{toc}{subsection}{\nameref{subsec:analyses}} 


\subsubsection{Structural equations models} 

To assess the effects of overstory treatments on habitat use by soil fauna (hypothesis 1.1) and the relationship between overstory treatments 
and environmental variables (hypothesis 2.1), we employed a structural equation model (SEM) combining occupancy models and linear mixed models \citep{mackenzieOccupancyEstimationModeling2006a,graceSpecificationStructuralEquation2010,josephIntegratingOccupancyModels2016}.
This approach enabled us to test both hypotheses within one analysis. 

One part of the SEM focused on the variations in environmental variables across different overstory treatments. 
Specifically, we estimated the effects of partial-cut and clear-cut on coarse woody debris volume, canopy openness and litter depth. 
The other part of the SEM was designed to evaluate the direct and indirect effects of overstory treatments on taxa (Figure \ref{fig:SEM}). 
We presumed that partial-cut and clear-cut will directly impact each taxon, while salamander occupancy will influence small ground beetle occupancy and springtail biomass. 
We aslo suggested that both ground beetle groups will affect springtails biomass.

Some components of the SEM consisted of linear mixed models to estimate the effect overstory treatments on environmental variables (Table \ref{ann:SEM_Env_eq}) and springtail biomass. 
The component of the model predicting springtail biomass also included the latent occupancy state of salamanders and ground beetles to measure the impact of these groups on springtails. 
Other components of our SEM consisted of occupancy models to quantify the impact of overstory treatments on the occupancy (presence) probabilities of salamanders and ground beetles (Table \ref{ann:SEM_Sp_eq}). 
Occupancy models estimate the presence of species difficult to detect after accounting for imperfect detection probability \citep{mackenzieEstimatingSiteOccupancy2002,baileyEstimatingSiteOccupancy2004,mazerolleMakingGreatLeaps2007,spiersEstimatingSpeciesMisclassification2022}. 
The data type required to distinguish between occupancy and detection probabilities consists of repeated visits at a series of sites 
(here, 4 visits at the sampling units). Specifically, a detection history is constructed for each site, 
using 1 to denote the detection of the species on a given visit and 0 to non-detection. 
For example, the detection history 0010 at a site would indicate that the species was detected at the site on the third visit, but not detected during the first, 
second, and fourth visits. 

We formulated the SEM using a Bayesian framework, which we describe in Table \ref{ann:SEM_script}. 
Parameters were estimated using Markov chain Monte Carlo (MCMC) with JAGS 4.3.0 included in the jagsUI package in R 4.3.1 \citep{lunnBUGSProjectEvolution2009,rcoreteamLanguageEnvironmentStatistical2020,kellnerJagsUIWrapperRjags2024}. 
We ran the model with five chains and 200,000 iterations each \citep{gelmanUnderstandingPredictiveInformation2014}. 
The first 75,000 iterations were used as burn-in and we used a thinning rate of 5. 
We assessed convergence of MCMC chains by examining trace plots, posterior density plots, and using the Brooks-Gelman-Rubin statistic. 
The JAGS model code is available in Table \ref{ann:SEM_script}.

\vspace{10pt}

\begin{figure}[h!]
	\centering
	\includegraphics[scale=0.55]{fig_sem.png}
	\caption[Theoretical model illustrating the anticipated relationships between overstory treatments, environmental variables and species groups.]
  {Theoretical model illustrating the anticipated relationships between overstory treatments, coarse woody debris volume, canopy openness, litter depth,
   salamander occupancy, ground beetle occupancy and springtail biomass in the Portneuf Wildlife Reserve, Quebec, Canada. 
   Each arrow indicates the direction of a potential effect, from an explanatory variable to a response variable. 
   Note that the large and small carabid categories are based on salamander gape size.}
	\label{fig:SEM}
\end{figure} 

\clearpage

\section*{Results}
\label{sec:results1}
\phantomsection\addcontentsline{toc}{section}{\nameref{sec:results1}}


\subsection*{Environmental variables}
\label{subsec:ResEnv}
\phantomsection\addcontentsline{toc}{subsection}{\nameref{subsec:ResEnv}} 

Model diagnostics indicated that the chains were of sufficient length, as the Brooks-Gelman-Rubin statistic was below 1.04. 
Trace plot analysis revealed that all chains had converged towards similar values, and none of the ratios of MCMC error to posterior standard deviation exceeded 5\%.

Environmental variables usually differed between overstory treatments and control conditions.
We found that clear-cutting treatments had significantly less CWD compared to partial-cutting
(95\% CI : [-1.15, -0.43]) (Figure \ref{fig:envar} A, Table \ref{tab:overstoryenvar}). 
However, these treatments did not differ from the control sites. 
Canopy openness was significantly higher in both the partial-cut (95\% CI : [1.97, 11.02]) and clear-cut treatments (95\% CI : [51.39, 77.06]) when compared 
to the control sites, with the clear-cuts being more open than the partial-cuts (95\% CI : [44.61, 70.76], Figure \ref{fig:envar} B, Table \ref{tab:overstoryenvar}). 
In contrast, litter depth was lower in both the partial-cut (95\% CI : [-2.44, -0.65]) and clear-cut treatments (95\% CI : [-4.28, -2.50]) compared to the controls, 
with the litter being shallower in the clear-cuts than in the partial-cuts (95\% CI : [-2.57, -1.12], Figure \ref{fig:envar} C, Table \ref{tab:overstoryenvar}).


\vspace{10pt}

\begin{figure}[ht]
  \centering
  \includegraphics[scale=0.23]{fig_envar2.png}
  \caption[Environmental variables with a potential effect on soil species within two different overstory treatments and control.]
  {Environmental variables with a potential effect on soil species estimations within two different overstory treatments and control 
  during the summer 2022 in the Portneuf Wildlife Reserve, Quebec, Canada. Error bars denote 95\% credible intervals around estimates.}
  \label{fig:envar}
\end{figure}

\begin{table}[ht]
  \centering
  \caption[Contrasts between overstory treatments for environmental variables that could affect habitat selection of fauna on the forest soil.]
  {Contrasts between overstory treatments for environmental variables that could affect habitat use of fauna on the forest soil during the summer 2022 in the Portneuf Wildlife Reserve,
  Quebec, Canada.}
  \label{tab:overstoryenvar}
  \begin{tabular}{lllll} 
      \hline
      &&&&95\% Bayesian \\
      Variable&Unit& Comparison & Estimate &  credible interval \\ [0.5ex] 
      \hline
      Coarse woody debris &m\up{3}& Partial vs control & \hspace{1mm}0.02 & [-1.01, 1.06] \\ 
                 && Clear vs control  & -0.77 & [-1.79, 0.23] \\ 
                          && Clear vs partial  & -0.79 & [-1.15, -0.43] \\
      Canopy openness     &\%& Partial vs control & \hspace{1mm}6.49 & [1.97, 11.02] \\ 
                      && Clear vs control  & \hspace{1mm}64.19 & [51.39, 77.06] \\ 
                          && Clear vs partial  & \hspace{1mm}57.69 & [44.61, 70.76] \\ 
      Litter depth        &cm& Partial vs control & -1.54 & [-2.44, -0.65] \\ 
                      && Clear vs control  & -3.39 & [-4.28, -2.50] \\ 
                          && Clear vs partial  & -1.85 & [-2.57, -1.12] \\       
      \hline
      \multicolumn{5}{l}{\textbf{Note:} Estimates from Bayesian SEM are presented in terms of posterior mean with 95\%} \\
      \multicolumn{5}{l}{credible intervals, where an interval excluding 0 indicates a difference between groups.} \\
  \end{tabular}
\end{table}

\clearpage


\subsection*{Soil fauna}
\label{subsec:taxa}
\phantomsection\addcontentsline{toc}{subsection}{\nameref{subsec:taxa}} 

\vspace{10pt}

Across the sixty sampling units, Red-backed salamanders were detected in 0, 1, 11, and 11 sites during the May, June, July, and August 2022 surveys, respectively. 
Over the same periods, small ground beetles were detected in 2, 5, 12, and 2 sites, while large ground beetles were detected in 16, 30, 35, and 21 sites.
A total of 30 ground beetle species were identified from harvest in the pitfall traps and under the coverboards (Table \ref{tab:carabid}). 
We collected 468 springtails representing 12 families, with 219 springtails collected from partial-cut treatments, 131 from clear-cuts, and 118 from the control areas (Table \ref{tab:springtail}). 
The average springtail biomass collected per overstory treatments was 24.3 $\mu$g (SD = 18.2 $\mu$g), 56.8 $\mu$g (SD = 78.0 $\mu$g), and 31.1 $\mu$g (SD = 52.8 $\mu$g) in the partial-cut treatments, clear-cuts and control sites, respectively.

Occupancy and biomass generally did not vary significantly across the cutting treatments. 
Salamander occupancy probability was marginally lower in sites subjected to clear-cutting compared to those with partial-cutting (90\% CI : [-0.74, -0.07], Figure \ref{fig:pcin}, Table \ref{tab:overstorysp}). 
However, these two groups did not differ from the control sites. 
Occupancy probability for each carabid group and the springtail biomass did not vary between the overstory treatments and control sites (Table \ref{tab:overstorysp}). 
Furthermore, the occupancy probabilities of small ground beetles (salamander prey) did not vary with the presence of salamanders. 
Similarly, the biomass of springtails did not vary with the presence of salamanders, large ground beetles or small ground beetles (Table \ref{tab:overstorysp}).

\begin{figure}[h!]
	\centering
	\includegraphics[scale=0.60]{fig_sem_res.png}
	\caption[Results from structural equation modeling analysis revealing effects of overstory treatments on coarse woody debris volume,
  canopy openness, litter depth, salamander occupancy, ground beetle occupancy, and springtail biomass.]
  {Results from SEM analysis showing effects of overstory treatments on CWD, 
  canopy openness, litter depth, salamander occupancy, ground beetle occupancy, and springtail biomass in the Portneuf Wildlife Reserve, 
  Quebec, Canada. Bold arrows represent significant effects, while dotted line indicate no discernible effects. 
  Estimates marked with one asterisk (*) indicate a 90\% credible interval (CI) excluding 0, while estimates marked with two asterisks (**) indicate a 95\% CI excluding 0. 
  Note that the large and small carabid categories are based on salamander gape size.}
	\label{fig:SEMres}
\end{figure}  

\vspace{10pt}

\begin{table}[h!]
  \centering
  \caption[Contrasts between overstory treatments for salamander occupancy, ground beetle occupancy, and springtail biomass.]
  {Contrasts between overstory treatments for salamander occupancy, ground beetle occupancy, and springtail biomass, during the summer 2022 in the Portneuf Wildlife Reserve, Quebec, Canada. 
  This table also shows the estimated effect of interactions between different groups: salamander presence on small and large ground beetles and the effects of the presence of salamanders and both ground beetle groups on springtail biomass.}
  \label{tab:overstorysp}
  \begin{tabular}{lllll} 
      \hline
      &&&&95\% Bayesian \\
      Variable&Unit& Comparison & Estimate &  credible interval \\ [0.5ex] 
      \hline     
      Salamander           &probability& Partial vs control & \hspace{1mm}0.07 & [-0.29, 0.45] \\ 
      occupancy       && Clear vs control  & -0.38 & [-0.75, 0.11] \\ 
                          && Clear vs partial  & -0.45 & [-0.74, -0.07]$^{a}$ \\       
      Carabid$_{large}$ &probability& Partial vs control & -0.12 & [-0.35, 0.15] \\
      occupancy       && Clear vs control  & -0.06 & [-0.29, 0.20] \\ 
                          && Clear vs partial  & \hspace{1mm}0.06 & [-0.19, 0.30] \\ 
      Carabid$_{small}$    &logit& Partial vs control & \hspace{1mm}3.31 & [-10.12, 17.72] \\
      occupancy             && Clear vs control  & \hspace{1mm}10.19 & [-4.15, 24.45] \\ 
                          && Clear vs partial  & \hspace{1mm}6.88 & [-12.81, 23.42] \\  
                          && Salamander        & -2.20 & [-17.15, 16.59] \\  
      Springtail          &$\mu$g& Partial vs control & \hspace{1mm}8.11 & [-9.38, 25.40] \\
      biomass             && Clear vs control  & \hspace{1mm}2.11 & [-13.98, 18.11] \\ 
                          && Clear vs partial  & -6.00 & [-29.09, 17.26] \\  
                          && Salamander        & \hspace{1mm}6.80 & [-10.43, 23.26] \\ 
                          && Carabid$_{large}$      & \hspace{1mm}0.56 & [-16.75, 17.66] \\ 
                          && Carabid$_{small}$      & \hspace{1mm}7.62 & [-8.93, 24.09] \\ 
      \hline
      \multicolumn{5}{l}{\textbf{Note:} Estimates from Bayesian SEM are presented in terms of posterior mean with 95\%} \\
      \multicolumn{5}{l}{credible intervals, where an interval excluding 0 indicates a difference between groups.} \\
      \multicolumn{5}{l}{$^{a}$Marginal difference based on 90\% Bayesian credible interval excluding 0}
  \end{tabular}
\end{table}

\clearpage

\begin{figure}[h!]
  \centering
  \includegraphics[scale=0.55]{fig_pcin.png}
  \caption[Occupancy probability of salamanders under overstory treatments]
  {Occupancy probability of salamanders within two overstory treatments and controls during the summer 2022 in the Portneuf Wildlife Reserve device, Quebec, Canada. 
  Error bars denote 95\% credible intervals around estimates.}
  \label{fig:pcin}
\end{figure}

\vspace{10pt}

\clearpage

We did not observe significant impacts of CWD volume and precipitation level on salamander detection probabilities. 
However, the precipitation level had a positive effect on detection probability for small ground beetles (95\% CI : [0.59, 1.77]) and large ground beetles (95\% CI : [0.70, 3.23]) (Table \ref{tab:detection}). 
Detection probabilities of ground beetles did not vary with the volume of CWD.

\begin{table}[ht]
  \centering
  \caption[Estimated effects of coarse woody debris and precipitation level on detection probabilities of salamanders and both ground beetles.]
  {Estimated effects of coarse woody debris and precipitation level on detection probabilities of salamanders and both ground beetles, during the summer 2022 in the Portneuf Wildlife Reserve,  Quebec, Canada.}
  \label{tab:detection}
  \begin{tabular}{lllll} 
      \hline
      &&&95\% Bayesian \\
      Variable & Taxa & Estimate &  credible interval \\ [0.5ex] 
      \hline      
      Precipitation       & Salamander              & \hspace{1mm}0.11 & [-0.83, 1.06] \\ 
                          & Carabid$_{large}$  & \hspace{1mm}1.17 & [0.59, 1.77] \\ 
                          & Carabid$_{small}$        & \hspace{1mm}1.87 & [0.70, 3.23] \\  
      \hline      
      Coarse woody debris & Salamander              & -0.59 & [-1.39, 0.12] \\ 
                          & Carabid$_{large}$  & \hspace{1mm}0.06 & [-0.26, 0.38] \\ 
                          & Carabid$_{small}$        & \hspace{1mm}0.27 & [-0.74, 1.37] \\   

      \hline
      \multicolumn{4}{l}{\textbf{Note:} Estimates from Bayesian SEM are presented in terms of posterior mean} \\
      \multicolumn{4}{l}{with 95\% credible intervals, where an interval excluding 0 indicates} \\
      \multicolumn{4}{l}{a difference between groups.} \\
  \end{tabular}
\end{table}

\clearpage

\section*{Discussion}
\label{sec:discu1}
\phantomsection\addcontentsline{toc}{section}{\nameref{sec:discu1}}

Le but de notre étude était de vérifier comment différents traitements de coupe forestières influence les variable environnementales importante pour la faune du sol et la relations .cologique de la petite faune faune du sol.
 Globalement nous avons observé un effet important des traitement de coupe forestière sur les varaible environnemnt

% petit paragraphe sur ce qu'a montrer  note étude
% Revenir sur les objectifs et hypothèeses
% dire ce qu'a amené le projet
% revenir sur les hypothèese, sont elles confirmé ou non
% ce que suggèere les analyses, qeuls sont les effets
% ce que ca suggèere

% transfer de perturbation a travers le réseau trophique 

Le but de l'étude est de comprendre comment les traitements sylvicoles, effectués dans un contexte de migration assistée, 
affecte la dynamique des écosystèmes du sol forestier. 

Les objectifs qui s’y rattachaient étaient de quantifier l'effet des traitements de coupes forestières sur les variables environnementales qui influencent l'utilisation de l'habitat par la faune du sol 
et de mesurer l'impact des coupes forestières sur l'utilisation de l'habitat par la faune du sol.

Notre hypothèse pour le premier objectif soutenait que les variables environnementales favorables à l'utilisation de l'habitat par les espèces fluctuent 
en fonction de l'intensité des coupes forestières. Ainsi, les traitements de coupes forestières constituent une variable englobant 
les changements de conditions environnementales.

Notre hypothèse lier à notre deuxième objectif stipulait que les traitements de coupe forestière entraînent une modification de l'utilisation de l'habitat 
par la faune du sol et se propage à travers le réseau trophique. Spécifiquement, les coupes affectent l'utilisation de l'habitat par les salamandres et 
les grands carabes (compétiteurs de salamandres), ce qui modifie ensuite la sélection d'habitat des petits carabes (proies de salamandres), 
puis enfin des collemboles (proies de grands carabes et salamandres).

mesure des effets 5 ans après l'application des traitements de coupes



\subsection*{Environnementala variables}
\label{disc:env_var}
\phantomsection\addcontentsline{toc}{subsection}{\nameref{disc:env_var}} 

Overall, control treatments had greater basal area and litter depth as well as a lower degree of canopy openness and of soil compaction than irregular shelterwoods. 
Although irregular shelterwood cutting often maintains higher CWD volumes than other silvicultural treatments (Nolet et al. 2018),

% Hypothèse :

Notre hypothèse pour le premier objectif soutenait que les variables environnementales favorables à l'utilisation de l'habitat par les espèces fluctuent 
en fonction de l'intensité des coupes forestières. 
Ainsi, les traitements de coupes forestières constituent une variable englobant les changements de conditions environnementales.

% résultat :

Dans l'ensemble notre hypothèse lié au variable environnementale est confirmé puisque les variables favorables à l'utilisation de l'habitat par les espèces du sol 
diffère des témoins et suive dans l'ensemble l,intensité du traitements de coupe.

Dans l'ensemble une intensification des traitements de coupe amène une diminution du volume de débris ligneux et de la profondeur de litière ainsi qu'une augmentation de l'ouverture de la canopé.

La variable des traitements de coupes peux ainsi être utilisé comme variable englobant les changement de conditions environnemental utile pour la faune du sol. 


% résultats : CWD

  % effets mesuré
On observe que les traitement de coupe partielle on un meilleurs capacité de rétention de débris ligneux comparativement au coupe totale. 
Le volume de bris ligneux est ainsi signifactivement plus faible dans les traitement de coupe totale par rapport au coupe partielle. 
Cependant il nous n'avons pas observé de différence significative entre les témoins par rapport aux deux traitements de coupes. 

Nos résultats concorde partiellement avec ce que l'on retrouve dans la littérature. 
Plusieurs études ont observé un plus grand volume de débris ligneux dans les coupes partielle comparativement aux témoins
Ochs2022Responseterrestrial a observé que le volume de débri ligneux était plus grand dans les coupe totale 0-3 ans après la récolte par apport au témoins 
mais qu'il n'y avait pas de différence entre témoins et coupe totale de 4-6 ans  ou 7-11 ans après la récolte ce qui concorde avec notre résultat.

Ochs2022Responseterrestrial a observé que le volume de débris ligneux était plus élevé dans les traitements de coupes partielles que dans les témoins

Mazerolle2021Woodlandsalamander a observé que les traitement de coupe totale possédé un plus grand volume de débris ligneux que les témoins 
mais que pour la même charactéristique il n'y avait pas de différence entre le témoins et les traitement de coupe partielle ou entre coupe partielle et coupe totale.
En revanche pour les débris ligneux avec une décompositiona avancée, le volume était plus faible dans les traitement de coupe partielle et totale comparativement au témoins. 

Les milieux ayant subit un traitement de coupe partielle sont souvent ceux possédant le grand volume de débris ligneux 
Les traitements de coupe partielle permettent souvent le maintient d'un pluis grand volume de débris ligneux que d'utre type de traitement sylvicoles \citep{Nolet2018Comparingeffects}


la perte de débris ligneux engendre une diminution de la disponibilité en refuge et en nutriment pour la faune du sol et augmente ainsi le risque de déclin ou d'extinction local. 

% résultats : Canopy
On observe que les traitements de coupes totales ont une ouverture de la canopé largement supérieur aux témoins et aux traitement de coupes partielles.
En revanche les traitement de coupe partielles à 50\% n'augmente que légèrement l'ouverture de la canopée.

Nos résultat concorde avec ceux observé par Mazerolle2021Woodlandsalamander qui a également observé une ffaible différence entre les témoins et 
les coupe partielle mais une importante différence entre les coupes total et les deux autres type de traitements.


Malgré une coupe partielle à 50\% les coupes partielle permettrait de préserver une canopé relativement fermé et ainsi de favoriser le maintient 
des conditions environnementales comme la température, l'humidité la vitesse des vents, l'interception des précipitation. 
ce qui diminureait l'impacte des coupes sur la faune du sol.

L'ouverture importante de la canopé qu'entraine les coupes totale sont souvent associé a des changements majeur de l'environnement du sol en augmentant l'augmentation des rayonnement solaire ce qui amène un assèchement de la litière ainsi qu'une ausse de température \citep{Brooks2008Forestfloor} 


% résultats : Litter

les traitements de coupes forestières amène une diminution significatif de la profondeur de la litières. 
L'intensité du traitement de coupes concorde avec l'importance de la diminution. 
en effet les traitements de coupe partielles et de coupes totale démontre une diminution significative de la profondeur de litière par rapport au témoins 
avec une diminution plus important dans les coupes totales par rapport au coupes partielles.

Mazerolle2021Woodlandsalamander a observé que la profondeur de litière plus grand dans les témoins, puis dans les coupes partielle 
et enfin dans les traitement de coupe total ce qui concordes parfaitement avec ce que nous observons dans notre étude.

Ochs2022Responseterrestrial a observé que la profondeur de litière était significativement plus faible dans les traitements de coupe totale dans les période de 0-3 ans et 4-6 ans après la coupe, par rapport au témoins mais cete différence n'était plus significative pour la période 7-11 ans. 
pour les traitement de coupe partielle les auteur n'ont observé qu'une diminution marginale entre les témoins et les coupes partielle lors de la preparatory cut alors que la diminution était significative lors de la establishement cut.


% Effet des coupes sur CWD, impact sur le milieu et comment cela peu affecter la faune

Plusieurs études se sont interressé è l'effets des charactéristique forestières comme le volume de débris ligneux, l'ouverture de la canopé et la profondeur de litière sur la faune du sol afin de diminuer l'impact de gestion forestière sur les communauté du sols \citep{Semlitsch2002Criticalelements,McKenny2006Effectsstructural}

Otto2014ComparingPopulation a mesuré que la survie apparente des salamandre cendrée augmenté avec une fermture de la canopé et un volume de débris ligneux plus important mais que ces deux varibles ne semblaient en revanche pas affecter l'abondance. 

Le retrait de 
Further, the removal of surface refugia following harvesting can reduce habitat quality pour la faune du sol (Achat et al., 2015; Peele et al., 2017)
Consequently, because woodland salamanders forage and breed at the forest floor surface, harsh surface conditions, soil compaction, and low levels of CWD can affect population dynamics (Peterman and Semlitsch, 2014).
Greater amounts of CWD can also mitigate the negative effects of even-aged forest management by providing surface refugia that maintains suitable environmental conditions underneath logs and rocks and allows individuals to remain at the surface for longer periods of time (Grover, 1998; Moseley et al., 2009; Strojny and Hunter, 2009; Carusco, 2016).

Mazerolle et al. (2021)  et margenau 2023 also found that BCI was higher in strip-cutting treatments, where there were greater amounts of early-stage decay, indicating that woodland salamander density and BCI responses can vary in relation to the amount of downed CWD present.
Ecological forestry practices include retaining some preharvest downed CWD levels post-harvest because it provides important structure and habitat for an array of taxa, including woodland salamanders (Franklin et al., 2007).

otto :
Coarse woody debris provides ideal microhabitat conditions for P. cinereus and is an important structural component for salamander populations in harvested forests (e.g., McKenny et al., 2006; Owens et al., 2008).

Staab :
Species richness correlations were further negatively related to change in deadwood volume (-0.073 ± 0.036, p = 0.044) pour plusieurs groupe d'insectes

La présence de débris ligneux 
% Effet des coupes sur canope, impact sur le milieu et comment cela peu affecter la faune
changement des condition en forêt
Mazerolle 2021 : Overall, control treatments had greater basal area and litter depth as well as a lower degree of canopy openness and of soil compaction than irregular shelterwoods.
Mazerolle 2021 : The control sites in our study had greater canopy cover, whereas shrub and herbaceous cover were highest in gap and strip treatments, consistent with previous studies (Nolet et al. 2018; Raymond and Bédard 2017).
(Otto et al. 2014). For sites of 3600 m2,the authors observed a positive relationship between P. cinereus abundance and canopy cover 
Previous research has focused on certain forest characteristics (e.g., overstory cover, coarse woody debris, leaf litter depth) and their influence on salamanders post-harvest in order to guide forest management (Semlitsch, 2002; McKenny et al., 2006).

staab
Species richness correlations were further negatively related to change in deadwood volume (-0.073 ± 0.036, p = 0.044), change in the proportion of non-native trees (-0.115 ± 0.042, p = 0.008), change in canopy openness (-0.130 ± 0.040, p = 0.002)
abundance to change in canopy openness (-0.104 ± 0.042, p = 0.014) but positively related to tree diversity (0.103 ± 0.043, p = 0.017)
For myceto-detritivores, species richness and biomass correlations were negatively related to change in the proportion of non-native trees. Myceto-detritivore richness correlations were, furthermore, negatively related to the change in canopy openness,
Lastly, carnivore species richness correlations were negatively related to change in canopy openness. (insecte seulement)

% Effet des coupes sur litter, impact sur le milieu et comment cela peu affecter la faune
effet sur la disponibilité en nutriment
Mazerolle 2021 : Overall, control treatments had greater basal area and litter depth as well as a lower degree of canopy openness and of soil compaction than irregular shelterwoods.
Mazerolle 2021 : Clearcuts are typically conducted at large scales but are similar in nature to strip cuttings of a 10 m width in our study, which still had a lower litter depth than control sites 5–6 years after treatment.
Previous research has focused on certain forest characteristics (e.g., overstory cover, coarse woody debris, leaf litter depth) and their influence on salamanders post-harvest in order to guide forest management (Semlitsch, 2002; McKenny et al., 2006).

voit t-on des différence entre les coupe partielle et les coupe total , une coupe est t'elle meilleur que l'autre pour les gourpe taxonomique ou les variable environnementale?

Previous research has focused on certain forest characteristics (e.g., overstory cover, coarse woody debris, leaf litter depth) and their influence on salamanders post-harvest in order to guide forest management (Semlitsch, 2002; McKenny et al., 2006).



\subsection*{Soil fauna}
\label{disc:soil_fauna}
\phantomsection\addcontentsline{toc}{subsection}{\nameref{disc:soil_fauna}} 


\subsubsection*{Logging effects}
\label{disc:logging_effects}


La seconde hypothèse liée à notre objectif postulait que les traitements de coupe forestière modifieraient l'utilisation de l'habitat par la faune du sol, avec des effets se propageant à travers le réseau trophique, des prédateurs vers les proies. 
Suite à notre analyse, nous n'avons globalement pas mesuré d'effet significatifs des coupes sur les différent groupe d'espèces, que ce soit de façon directe ou indirecte. 
Les résultats de notre analyse ne permettent donc pas de valider notre hypothèse. 

Dans le cas des salamandres, nous avons toutefois observé un effet marginal des traitements de coupes sur la probabilité d'occupation.
La probabilité d'occupation des salamandres était significativement plus élevée dans les zones de coupe partielle comparativement aux zones de coupe totale, 
bien que les résultats des coupes partielles et des coupes totales ne diffèrent pas significativement de ceux obtenus pour les sites témoins. 

Les coupes totales soient généralement perçues comme néfastes pour les populations de salamandres, cependant leur impact varie considérablement selon les études \citep{Hocking2013Effectsexperimental,Chaudhary2016Impactforest}. 
Certaines recherches ont observé une diminution de l'abondance des salamandres 4 à 6 ans après la récolte de bois \citep{Petranka1993Effectstimber,Herbeck1999PlethodontidSalamander,Grialou2000effectsforest,Macneil2014Effectstimber}. 
Toutefois, d'autres études, n'ont relevé aucune différence significative entre les sites ayant subit une coupe majeure, comme la coupe totale, et les témoins \citep{Renken2004EffectsForest,Mazerolle2021Woodlandsalamander}. 
L'incertitude entourant l'impact des coupes totales sur les populations de salamandres correspondrait, selon nous, à l'absence d'effet observée sur la probabilité d'occupation des salamandres.
Notre étude s'aligne également avec le consensus général quant à l'absence d'effet des coupes partielles sur l'abondance des salamandres, pour une période de temps similaire \citep{McKenny2006Effectsstructural,Mazerolle2021Woodlandsalamander,Ochs2022Responseterrestrial}.
En revanche, plusieurs études ont observé une diminution de l'abondance des salamandres dans les coupes partielles au court des premières années après la récolte \citep{Harpole1999Effectsseven,Knapp2003Initialeffects,Morneault2004effectshelterwood}. 
Toutefois, ces effets ne semblent pas dépasser les cinq premières années \citep{Morneault2004effectshelterwood}. 


L'abscence de différence entre les coupe partielle et les témoins aini que l'ambivalence sur l'effets des coupes totales sur l'abondance des salamandre 
expliquerais la différence marginal observé entre nos traitement de coupe partielle et nos traitement de coupe totales ou la probabilité d'occupation 
était plus faible dans ce dernier.
Cette effet même si marginale suggère que les traitements de coupe partielle permette un meilleur maintient des salamandres dans l'habitat, ce que valide plusieur étude.

La différence dans la probabilité d'occupation des salamandre entre les deux traitements de coupes s'expliquerait essentiellement 
par la capacité des coupes partielle a préserver une qualité d'habitat supérieur à celle retrouvé dans les coupes totales, grace au maintient des attributs forestiers favorable à la salamandre.

Nous suggérons que la probabilité d'occupation des salamandre est fortement associé au volume de débris ligneux dans notre étude, puisque nous obtenons le même type d'effet chez ces deux variables réponse, 
soit l'absence d'effet entre les témoins par rapport au deux type de coupe, mais une différence entre les traitement de coupe totale et les traitement de coupe partielle avec une diminution de la probabilité d'occupation et de volume de débris ligneux dans les traitement de coupe totale. 
Plusieurs étude ont par ailleurs identifier que l'abondance des salamandres cendrée salamandre cendrée étant fortement associé au volume de débris ligneux \citep{McKenny2006Effectsstructural,Nolet2018Comparingeffects,Mazerolle2021Woodlandsalamander}. 
En raison de leur respiration cutanée le salamandre cendrée sont plus a risque de déssication quand elle sont exposé a un environnement plus chaud et sec. 
La présence de refuge temporaire comme le débris ligneux et autre microstrucutre joue ainsi un rôle prédominant dans leur écologie puisuq'il permette au salamandre de se maintenir a la surface pour se nourrir et se reproduire \citep{Peterman2014Spatialvariation,Achat2015Quantifyingconsequences,Peele2017Effectswoody}. 


La différence de probabilité d'occupation entre les deux type de coupe peux aussi être expliqué par le maintient du couvert forestier dans les coupes partielle et le fait que ce type de traitement favorise un retour rapide de la végétation de sous-étage \citep{Raybuck2015silviculturalpractices}.
Une plus grande rétention du couvert forestier limite la quantité de rayonnement solaire atteignant le sol et réduit la présence de vents extrême, ce qui permet de maintenir des microclimats humide et frais à la surface du sol, compatibles avec la physiologie des salamandres \citep{Homyack2011Energeticssurfaceactive}
Les travaux précédents ont montré que la rétention accrue de la couverture du couvert forestier comme dans les coupes partielle peut entraîner des densités relatives plus élevées de salamandres 
comparativement aux traitements équiennes comme les coupe totale qui éliminent une plus grande quantité de couvert forestier \citep{Hocking2013Effectsexperimental,Harper2015Impactforestry,Mahoney2016Woodlandsalamander}. 

Les effets des traitements sur les salamandres semble varier à l'echelle temporelle en effet.
Dans leur étude, \cite{Ochs2022Responseterrestrial} n'ont pas trouvé d'effet des coupe total sur l'abondance relative des salamandres durant les trois première années après la recolte.
En revanche, les auteurs ont observé une diminution significative de l'abondance relative des salamandres durant la période de 4-6 ans après la récoltes sans récupérer durant la période de 7-11 ans post-récolte.
Toutefois la réponse a court terme semble varier en fonction de l'étude puisque papier relate une diminution de l'abondance des salamandres peu de temps durant les trois première années suivant une coupe \citep{deMaynadier1995relationshipforest,Macneil2014Effectstimber}
\cite{Ochs2022Responseterrestrial} suggère que la réponse tardive à l'exploitation forestière par les salamandres pourraient être expliqué par la maintient de débris ligneux, après coupe, durant les première années, ce qui permettrait aux salamandres de subsister à court terme le temps que le débris ligneux se décompose.


À l'inverse \cite{Ash1997DisappearanceReturn} ont observé un retour à l'abondance d'avant coupe, chez les salamandre cendrée, 4-6 ans après l'applications d'une coupe total et ont associé ce retour a la réapparition de la litière. 
La quantité de litière pourrait ainsi contribuer à expliquer la différence de probabilité d,occupation que nous observons dans les traitements de coupe partielle par rapprot au traitement de coupe total puisque dans notre étude la profondeur de litière est significativement plus élevé dans les traitement de coupe partielle que dans ceux de coupe totale.

D'autre étude on aussi observer que le rétablissement des populations de salamandres se faisait suit à la régénération de la forêts \citep{Tilghman2012Metaanalysiseffects}. 
Toutefois le temps de récupération peux grandement varier selon les milieu et le type de traitement et prendre parfois plusieurs décénnies \citep{Homyack2013Effectsrepeatedstand,Ochs2022Responseterrestrial}



% carabes et collemboles

Contrairement au salamandre, nos résultats n'ont indiqué aucun effet directe des coupes forestières sur les deux groupes carabes et les collemboles.

Nos résultats sont partiellement supporté par la littérature concernant les carabes. 

\cite{Kudrin2023metaanalysiseffects} ont en autre observé comme nous une absence de différence entre les témoins et les coupes partielles pour les coléoptères et les collemboles.
Toutefois les auteurs on observé une diminution de l'abondance pour ces deux même groupe d'espèces.
Cette diminution d'abondance ne semble toutefois pas impacter la probabilité d'occupation chez les carabes. 
De l'ouverture du couvert forestier est reconnue pour avoir un effet positif sur la richesse spécifique et la diversité taxonomique des carabes \citep{Kudrin2023metaanalysiseffects}.
Ce constat est principalemnt explicable par la création de nouveaux habitats lors des coupes totales et la capacité des carabes a se déplacer rapidement \citep{Niemela2007effectsforestry}. 
Nous supponsons que l'absence de changement dans la probabilite d'occupations serait du au fait qu'il n'y ai pas de changement dans la présence de carabes en générale mais plutôt dans la composition taxonomique.
Par exemple, l'application de coupe total amène la formation de nouveau habitat et entraine une diminution des espèces associer au milieu fermer et une colonisation par les espèces favorisant les milieu ouvert \citep{Niemela2007effectsforestry,Pohl2007Rovebeetles}

L'étude observe aussi une augmentation de la richesse spécifique des coleoptère dans les traitement de coupe totale mais pas dans les traitement de coupes partielle. 


Plusieurs étude ont observer que la réponse des groupe d'espèce aux coupes forestières varié en fonction leur taille.
\cite{Nolte2019Habitatspecialization} affirme ainsi que les coupes forestières augmente le risque de déclin chez les carabes de plus grande taille comparativement au carabes de ptite taille.
Nous n'avons cependant pas observé de différence dans de probabilité d'occupation des carabes entre les différents traitement de coupe, que ce soit pour le groupe des petits carabes ou pour le groupe des gros carabes.


Les différentes espèces de carabes sont courramment classé par espèces associé au habitat forestier, ouvert ou généraliste.
Les habitat forestier indique ici un milieu possédant une canopé fermée d'une plantation vieille de quelque décénnie.
les espèces spécialisé dans le milieu forestier sont souvent dépendante de la présence d'élément strucuraux tel que de large débris de bois, une végétation dense ainsi que des micro-conditions créer par la fermeture de la canopé \citep{Niemela2007effectsforestry}
À l'inverse les espèces favorisant les milieux ouvert sont réticante à l'utilisationdes milieu forestier \citep{Heliola2001Distributioncarabid}

La richesse spécifique et l'abondance des espèces généraliste et associé au habitat ouvert augmente généralement suite a une coupe forestière \citep{Halme1993Carabidbeetles,Heliola2001Distributioncarabid,Koivula2002Alternativeharvesting}
Carabid species richness and the abundances of generalist and open-habitat species usually increase following forest cutting (Niemela  et al. 1988, 1993a, b; Halme and Niemela  1993; Haila et al. 1994; Spence et al. 1996; Beaudry et al. 1997; Helio  la  et al. 2001; Koivula 2002a, b; Sippola et al. 2002; Pearce et al. 2003; de Warnaffe and Lebrun 2004).

Les habitat ouvert, plus sec et chaud, comme les coupes totales sont favorisées par de nombreuses espèces de carabes habituellement présentes dans les prairies et autres habitats similaires, tandis que seules quelques espèces sont associées aux forêts d'épicéas sombres et fraîches \citep{Niemela1993Effectsclearcut}

Les coupe totale on un effet négatif sur l'abondance des espèces forestières \citep{Werner2000Effectsforest,Niemela2007effectsforestry,Kudrin2023metaanalysiseffects}
Cependant ces espèces peuvent être retrouvé dans un habitat ouvert suite a une coupe \citep{Koivula2002Alternativeharvesting}.
Il peut s'agir par exemple d'individus provenent d'une forest mature collé mitoyen au milieu ouvert \citep{Spence1996Northernforestry,Koivula2002Alternativeharvesting}.
Il est également possible, pour certaine espèce forestière plus tolérante au perturbation naturel, de survire dans des zone de coupe totale \citep{Niemela2007effectsforestry}
En revanche les espèces associé au milieu ouvert disparaisse rapidement au moment ou la canopé se referme, soit 20 a 30 ans après l'application d'une coupes totale, et sont remplacé par les espèces forestière \citep{Niemela1996importancesmallscale,Koivula2002Alternativeharvesting}. 
Les forêt fermé représente des environenment hostile pour de nombresue espèce de carabes et seuelement un faible nombre d'espèces 

Les forêt fermé représente un environnement plutôt hostile pour de nombreuses espèces de carabes, car seules quelques espèces parviennent à y prospérer \citep{Koivula2002Borealcarabidbeetle}
On retrouve ainsi une plus faible richesse spécifique en milieu forestier mais pas nécéssairement une plus faible densité. 
For carabids, therefore, boreal forests may be an adversity or A-selected environment (Greenslade 1983) with low species richness (but not necessarily low densities). 


La plupart des espèces de carabes favorisant les foret sont dépendent de la présence de micro-structure \citep{Niemela1996importancesmallscale,Heliola2001Distributioncarabid,Koivula2002Alternativeharvesting,Work2004Standcomposition}

La quantité de litière est un autre facteur important qui affecte la dirstribution des carabes pour les carabes \citep{Koivula.1999Leaflitter,Heliola2001Distributioncarabid,Magura2005ImpactsLeaflitter}

\cite{mooreEffectsTwoSilvicultural2004} There was no significant difference in the abundance of carabid beetle species between treated and control stands for the selective cutting and strip clearcuts treatments, exept for one species (Synuchus impunctatus) who was more abundant in strip cleracut treatmens.


les coupes forestière ne semble donc pas avoir d'effet sur la probabilité d'occupations des carabes, peut importe le groupe de tailles. 
En revanche basé sur la littérature, le type de traitement de coupe peux créer une différence dans la richesse spécifique et la composition spécifique entre les différent traitements de coupes 
Toutefois ce changement de communauté ne semble pas avoir de répercution sur le réseaux trophique puisqu'il n'y a pas a notre connaissance de préférence alimentaire des salamandres vers les espèces de carabes ou de différence dans le régime alimentaires des carabes basé sur leur type d'habitat.

Les traitement de coupes partielle ne semble pas avoir d'effet sur les carabes au générale contraireent au coupe totale qui entraine une diminution de l'abondance, une augmentation de la richesse spécifique et un changement dans la composition taxonomique \citep{Niemela2007effectsforestry,Kudrin2023metaanalysiseffects}
% est ce que cela peux avoir un impacte sur le réseau trophique

Niemela2007effectsforestry
open-habitat species do not colonise large forest stands (Helio  la  et al. 2001), and secondly, the majority of forest species are assumed to survive in the clear-cut sites adjacent to mature forest stands.
The retained old forest stands should include specific structural features (Werner and Raffa 2000). These include coarse woody debris (Work et al. 2004) that is an important habitat for many carabids (Pearce et al. 2003; Koivula et al. 2005).
In this review we showed that there are, broadly speaking, three types of carabid beetle responses to forestry practices. 
In strongly managed stands, such as clear-cuts, (1) open-habitat species appear and increase in abundance 
(but disappear when the canopy closes ca. 20–30 years later); (2) forest generalists persist throughout the clear-cut originated succession; 
and (3) species requiring mature closed-canopy forests are affected negatively by management and may not recover within several decades. 
The latter group further seems to consist of two recovery types: (a) the majority of forest-specialist species tend to recover following logging. 
This view is supported by the low forest carabid abundances during the first ca. 20–30 years of clear-cut originated succession followed by an increase later on (30–60 years; Koivula et al. 2002), but (b) several species show poor or no re-colonisation even after tens of years after the harvesting event (Niemela  et al. 1993a).
As there are more open-habitat species that colonise clear-cuts than there are forest species disappearing from them (Niemela  et al. 1988, 1993a, b; Koivula 2002a, b), species richness tends to increase following clear-cutting or other major forestry practices. Modern forestry creates clear-cuts suitable for open-habitat and disturbance-tolerant species, and consequently these are the ‘winners’ in intensively managed forest landscapes today. Also forest-habitat generalists (species found in different types and ages of forest) appear to be thriving. 
From a conservation point of view, species requiring mature closed-canopy forest are of concern.
% effet des coupes sur les collemboles

Koivula.1999Leaflitter
The significant infiuence of leaf litter on carabid abundance can be attributable to both abiotic factors (microenvironmental eonditions. especially humidity and temperature), and biotic ones (ehanges in niche structure, improved food supply).
la présence de littière augmente l'abondance des carabes mais ne semble pas avoir d'effet sur la richesse et la diversité spécifique.
Humidity, temperature, light, and physieal and ehemicai qualities of soil can have an effect on their distribution patterns (Thiele 1977. Lindroth 1985)
Also interspecific interactions such as competition, predation and parasitism can greatly influence the distribution of carabid beetles and food available to them (Sergeeva 1994)
However, abiotie factors are probably more important than biotic ones in determining distribution patterns (e.g, Thiele 1964).
The amount of food is one of the most important biotic factors affecting carabid distributions (Thiele 1977. Niemela 1993a. Sergeeva 1994).
Many species-specific characteristics, sueh as feeding behavior, may be related to species distribution in association to leaf litter.
In the soil samples taken from the humus layer oi our study plots, there were more springtails in the litter than in the control plots,
ith the numbers of springtails. but not with those of mites. A similar positive correlation between pitfall catches oi carabids and springtails was reported by Niemela et al. (1986).
Leaf-litter microhabitat seems to be favorable for some species (e,g. C. micropterus.. P. oblongopunctatus and C. carahoides). but not for some others (e,g. N. biguttatus). At a forest-stand level, this probably leads to different aggregation patterns of different carabid species.
The importance of microhabitiil differences, such as litter depth, was found lo be imporlant in explaining carabid abundance in a field experiment in Canada (Niemela et al. 1997).

Magura 2005
Koivula et al. (1999) showed that artificial leaf-litter addition increased the habitat heterogeneity and affected the carabid assemblage structure by increasing the total carabid abundance
Results of the repeated measures ANOVA showed that leaf-litter addition altered the microclimatic conditions of the plots. The ground temperature was significantly lower in the leaf litter plots compared to the control plots (Table 1). There was significantly more prey in the leaf-litter plots than in the control (Table 1; repeated measures ANOVA). Similarly, the carabid larvae were significantly more numerous in the leaf-litter plots than in the control plots (Table 1; repeated measures ANOVA).
The total number of individuals was higher in the leaf-litter plots than in the control plots in both years.

Heliola2001Distributioncarabidspecies 
richness was significantly higher in the clearcut than in the forest fragments, (2) clearcuts hosted many open-habitat species, which increased overall species richness in these sites, (3) carabid assemblages in the edges were more similar to forest assemblages than to those found in the clearcuts, (4) no edge specialists were found, and (5) open-habitat species did not penetrate into the forest fragments from the clearcut.
There were no statistically significant differences in carabid species abundance between the traditionally clearcut sites and the retention tree cutting.
According to Scheffe’s post hoc test, species richness was significantly higher in the clearcuts than in the forest interiors and edges, whereas species richness in edges and forest interiors did not differ
In the RDA ordination, almost every carabid species was located left of the origin, indicating that most of the species were positively associated 
with habitat variables in the clearcut (Fig. 5). 
For instance, Pterostichus adstrictus and P. oblongopunctatus correlated highly positively with logging residue and needle litter. 
Thus, it is possible that some carabid species benefit from the increased structural elements on the forest floor following clearcutting. 
In contrast, forest specialist species were located in the lower part of the ordination space. These species (e.g., C. hortensis and C. caraboides) 
did not correlate with variables associated with clearcuts, but they correlated positively with, for instance, the cover of forest mosses, 
Vaccinium dwarf shrubs, and the amount of large spruce trees.

\cite{Kudrin2023metaanalysiseffects} ont observé qu'il n'y avait pas de différence dans l'abondance des collemboles dans les traitement de coupe partielles mais que ce groupes d'espèces subissait un déclin de 24\% dans les traitements de coupes totales.
L'auteur a aussi observé augmentation de la richesse spécifique des collemboles dans les deux types de coupes, toutefois cette augmentation n'était pas significative dans les traitement de coupe totale.
l'auteur explique le déclin des collemboles dans le traitement de coupe totale par une disparition des ressource alimentaire tels que les débris de bois ou la litière, la biomass fongique \citep{Baath1995Microbialcommunity}.

La conservation d'une partie du volume de débris ligneux dans les coupes partielles peux expliquer le maintient des communauté de collemboles dans les traitements de coupe partielle \citep{Raymond-Leonard2020Deadwood}


\subsubsection*{Relations between taxa}
\label{disc:relations_between_taxa}
 impact des salamandre sur trophique:

 oui : 

 -Wyman (1998) found that the Eastern Redbacked Salamander, Plethodon cinereus, in field enclosures reduced densities of several macroinvertebrate taxa in comparison to control enclosures lacking salamanders.
 During the first 2 yr of the experiment reported upon here, I recorded reductions of several mesofaunal taxa within plots occupied by P. cinereus in comparison to plots not occupied by the salamanders, but I observed no effects of salamanders on macroinvertebrate taxa (Walton, 2005)
 Rooney et al. (2000), however, reported increases of Collembola in field enclosures with P. cinereus in contrast to control enclosures without salamanders, a result they attributed to salamander consumption of ants, a predator of Collembola.
 The presence of P. cinereus within microcosms was associated with reductions of macroinvertebrate detritivores (Walton and Steckler, 2005; Walton et al., 2006) and pseudoscorpions (Walton et al., 2006), but also substantial increases among oribatid mites and several taxa of Collembola.
 In these cases, increases in mesofauna were attributed to indirect effects of salamander predation that mediated outcomes of competitive and predator–prey interactions among invertebrates within the microcosms. 
 Homyack et al. (2010) Homyack et al. (2010), however, reported no effects of P. cinereus on invertebrate densities during a 2yr experiment in field enclosures.
 As with terrestrial salamanders, the causes of this variation are not fully understood, but have been associated with complex interactions among precipitation, drainage properties of forest soils, and changes in Collembola behavior in response to hydric conditions and availability of fungal resources (Lawrence and Wise, 2004; Lensing and Wise, 2006; Schultz et al., 2006).
 P. cinereus also defends its territory against arthropod predators that are likely to compete with salamanders for food resources (e.g., centipedes, carabid beetles, and spiders; Gall et al., 2003; Hickerson et al., 2004; Anthony et al., 2007; Figura, 2007)
 Hickerson et al. (2012) found that the abundance of spiders increased under ceramic-plate ACOs from which P. cinereus had been removed, and P. cinereus abundance in creased under ACOs from which centipedes had been removed.

 
 % est ce qu'il y a une différence dans les communauté consommer par les salamandre qui expliquerais une augmentation et une diminution en même temps ?
 non :




Walton 2013
Statistically significant effects of salamander plot-occupancy on invertebrate densities were found for several taxa of mesofauna, including several Collembola taxa, oribatid mites, pseudoscorpions, and psocoptera.
The magnitude and sign of salamander effects on invertebrate densities were predicted by seasonal and interannual variation in leaf litter mass and, to a lesser extent, litter moisture content.
Salamander effects decreased with increasing litter mass and were more often negative when litter mass was high,
whereas positive effects on invertebrate densities were more likely when litter mass was low.
For several taxa, the positive effect of P. cinereus also increased with litter moisture.

However, studies that have sought to quantify the effects of terrestrial plethodontids on lower trophic levels have produced equivocal findings, including negative, positive, and no effect on invertebrate density.


De plus, aucun effet indirects des perturbations forestières n'a pus être observé dans notre études.
Ainsi nous ne pouvons pas affirmé que l'impactes des coupes forestières se propage a travers le réseaux trophique puisque les salamandres 
n'ont pas eu d'effet sur la probabilité d'occupation des carabes de petites taille et qu'il n'y a pas eu de changement significatif 
dans la biomasse des collemboles en fonction de la probabilité d'occupation des salamandres ou des deux groupes de carabes. 


Staab:
population trends can be related to trophic level35, as top consumers are more sensitive towards environmental perturbations37,38, and as population sizes may be smaller at the top of the trophic pyramid39.
Species at highest risk are often characterized by small population size, large body size and high trophic level1,35,36,38,53,79.
Carnivores are typically more sensitive to environmental change38 and have  disproportionate local extinction risk60.

Hocking2013Effectsexperimental
Litter decomposition is also likely influenced by the feeding behavior of microarthropods and their behavioral response to predation risk [40,52]. 
For example, collembola, a primary prey item of red-backed salamanders, can decrease saprotrophic fungal biomass through direct grazing or increase fungal biomass by feeding preferentially on senescent fungal hyphae [53,54].
Additionally, red-backed salamanders are known to prey on spiders and other litter-dwelling invertebrate predators and spiders also prey on amphibians [18].


% effet des salamandres sur les petits carabes

Hocking2013Effectsexperimental: 
Across both experimental venues, we found no significant effect of red-backed salamanders on any of the ecosystem functions. We also found no effect of salamanders on intraguild predator abundance (carabid beetles, centipedes, spiders).
In addition to ecosystem functions, we did not observe an effect of salamander depletions on the abundances of spiders, centipedes, or carabid beetles. This is in contrast to Hickerson et al. [26] who showed that red-backed salamander depletion resulted increased spider counts and decreased carabid beetle counts.

Hocking2013Effectsexperimental:
Although researchers have predicted that woodland salamanders are important regulators of ecosystem functions [22,23], we found no evidence that red-backed salamanders affect litter or wood decomposition, 
nitrogen cycling, acorn germination, herbivory, or the abundance of other litter predators.
This is consistent with Homyack et al. [24] who did not find any effect of red-backed salamanders on oak or maple litter decomposition in a Virginia mixed-hardwood forest, 
but in contrast to Wyman [13] who showed that red-backed salamanders lowered beech-dominated leaf litter decomposition rates by 11–17\%. 
Homyack et al. [49] suggested the conflicting result with Wyman [13] may have been due to differences in litter type.
Given this complexity, the effect of amphibians on decomposition is likely influenced by the relative importance of top-down predatory effects on the invertebrate community and the bottom-up effects 
on available nutrients through ingestion and excretion [14]
For example, the top-down effects of red-backed salamanders on the invertebrate community are known to depend on the community composition and habitat heterogenity [50,51], which suggests that the effects of salamanders on litter decomposition depends on the structural complexity of the environment and the biotic community.
Given the complex dynamics of forest floor food webs and the variable effect of red-backed salamanders on invertebrates [32,51,55], the effects of salamanders on ecosystem functions should be expected to be highly variable even when top-down effects predominate.

% effet des salamandres sur les collemboles

% effet des carabes sur les collemboles

Overall le transfer de perturbation a travers la chaine trophique

The various reactions of soil invertebrates to disturbances may be attributed to their functional traits [34,35]. Recent works suggest that focusing on functional traits can provide greater insights into the mechanisms driving ecosystem change and recovery [36–38]. kudrin

Slight changes in the abundance and richness of some soil invertebrate groups in response to partial cutting suggest the benefits of this harvesting practice for the conservation of soil animal communities. However, even partial cutting can lead to disturbances and changes in soil fauna.kudrin


% Autre pertinenant :

% taille des groupes:

kudrin : 
l'impacte négatif des coupes forestières est plus important sur les groupe taxonomique de petite tailles.
Body size is a fundamental trait that determines numerous physiological and life history parameters of an organism [39–41]. The conducted meta-analysis shows that as the size of invertebrates decreases, the negative impact of harvesting on soil fauna becomes more pronounced (Figure 3). In our opinion, one of the reasons for this dependence may be the limited dispersal ability of small-bodied soil fauna [42–44]. Small soil invertebrates have a reduced ability to actively avoid unfavorable points and/or colonize disturbed plots from adjacent undisturbed areas or refuges compared to actively moving groups of macrofauna [45–47]. For instance, oribatid mites are highly limited in their dispersal ability, even over distances of only sever

% trait

kudrin The various reactions of soil invertebrates to disturbances may be attributed to their functional traits [34,35]. Recent works suggest that focusing on functional traits can provide greater insights into the mechanisms driving ecosystem change and recovery [36–38].


% type de foret

kudrin :
 l'abondance des colembole diminue dans les foret coniférienne mais ne change pas dans les foret mixte ou feuillu.
 la richesse spécifique des collemboles  ne change pas dans les foret conniférienne mais augmente dans les foret feuillu.
 l'abondance des coléoptère diminue dans les foret feuillu et mixte mais pa en foret connifère.
 le typede foret n'as pas d'efet sur la richesse spécifique des coléoptère.

%  autre menace

d'autre perturbation menace la biodiversité du sol et peuivent parfois avoir un impactye plus important que la récolte de bois

% feu
 Kudrin2023metaanalysiseffects
 It appears that harvesting is less destructive to soil fauna compared to forest fires, which can cause a decline in the abundance of soil macrofauna of up to 70\% [32] and a reduction in richness of up to 99\% [33].

% top -down vs bottom-up

Contrairement à notre approche top-down, certaine étude soutienne que l'effet des perturbations dans le réseau trophique se propage de façon bottom-up et impacterais premièrement le bas de la chaine alimentaire ce qui impacterais par la suite les prédateur.
Notre étude n'ayant pas mesuré d'effet directe des coupe forestières sur la biomasse des collemboles, il est peu probable que nous ayons observé ce genre d'effets. 

% Quels sont les defaults ou amelioration du projet, ainsi que ses limitations.
  % une plus longue prériode d'échantillonnage
  limited number of sampling periods
  limitation dans la complexité du des analyses SEM

  otto 2011:
  Natural cover object searches provided greater power for detecting a similar change in occupancy, largely because of high sampling size

  % limitation dans le nombre d'individus pour les salamandres
    intervalle de confiance
  % essaie d,autre methodes
  reproduire l,étude avec de l'abondance fournira une compréhention approfondit mais ne pouvez pas etre utilisé ici en raison de la recolte des individus
  contrainte limitant de population fermé
  abondance au lieu d'occupation nécéssite ( carabes et collemboles), CMR pour les salamandres
  s'interresser au trait fonctionnel 

  % choix de classification des groupes
  trait fonctionnel a la place de taxonomique


% ouverture
  % tester un effet bottom up a la place de top down
  % Comparer avec d'autres types de foret (spatiale)
  répéter le type d'étude dans d'autre type de foret, conifère, feuillu. (kudrin)
  Our analysis revealed a significantly weaker effect of harvesting on the abundance of Collembola and Oribatida in deciduous and mixed forests compared to coniferous forests (Figure 5). The negative response of soil fauna to harvesting may be due to significant changes in abiotic conditions [12]. Several studies have documented significant alterations in temperature, soil moisture, soil compaction, and the quality and thickness of forest litter resulting from harvesting coniferous forests [59–61]. kudrin.

  % Faire de étude à plus court terme mais surtout a plus long terme (temporel)
  we have limited knowledge of how quickly soil animals react to harvesting and what the rate of their recovery is.
  Despite the relatively large datasets on the abundance of Collembola and Coleoptera, no temporal dependence of the effect size over twenty years was found. kudrin


  Several mechanisms may explain the effect of management on forest biodiversity: changes in tree age structure, vertical stratification, and composition of tree species, which affect light, temperature, moisture, litter, and topsoil conditions (Sebastia et al. 2005; Standovar et al. 2006); presence of microhabitats (e.g., dead wood, veteran trees, cavities, root plates) specific to unmanaged (Berg et al. 1994; Bouget 2005a; Christensen et al. 2005; Gibb et al. 2005) or managed forests (e.g., skid trails and haul roads) (Hansen et al. 1991; Gosselin 2004); and forest cover continuity and features resulting from extensive management in the past (Hjalten et al. 2007). The pattern of response may therefore depend on which of the above mechanisms, or which combinations of them, have the strongest effects on different taxonomic or functional groups. paillet 

  kudrin :
Les forêts de conifères et les forêts de feuillus, par exemple, sont très différentes en termes de conditions pédologiques et microclimatiques, ce qui se reflète dans la dissimilarité de la composition et de la structure de la faune du sol [14,19,20].


\section*{Conclusion}
\label{sec:conclu1}
\phantomsection\addcontentsline{toc}{section}{\nameref{sec:conclu1}}

peu d'étude se sont interréssé au interaction entre plusieurs groupe tel que les amphibien et les arthropodes, il sont souvent étudié séparément.
importance des effet cumulé
importance de relation trophique lors d'étude de perturbation

\section*{Acknowledgements}
\label{sec:acknowl1}
\phantomsection\addcontentsline{toc}{section}{\nameref{sec:acknowl1}}

\section*{Conflict of interest}
\label{sec:conflict1}
\phantomsection\addcontentsline{toc}{section}{\nameref{sec:conflict1}}

None declared
\section*{Author contributions}
\label{sec:author1}
\phantomsection\addcontentsline{toc}{section}{\nameref{sec:author1}}

\cleardoublepage

\begin{otherlanguage}{english}
\bibliographystyle{ecologyNewEN} % Style de citation en français
\bibliography{References}
\addcontentsline{toc}{section}{References}
\end{otherlanguage}
