\chapter{Soil fauna occupancy responses to cutting treatments in assisted tree migration context}     % numéroté
\label{chap:SEM}    

William Devos$^1$, Mathieu Bouchard$^1$, Marc Mazerolle$^1$

%href{mailto:william.devos.1@ulaval.ca}
$^1$ Centre d'étude de la forêt, Département des sciences du bois \\ 
et de la forêt, Université Laval, Québec, QC G1V 0A6, Canada. \\ 

\clearpage



\section*{Résumé}
\label{sec:resume1}
\phantomsection\addcontentsline{toc}{section}{\nameref{sec:resume1}}

\begin{otherlanguage*}{french}
  <Résumé de l'article en français. Obligatoire.>

  \textbf{Mots-clés} : <ajouter des mots clés>
\end{otherlanguage*}

\clearpage

\section*{Abstract}
\label{sec:abstract1}
\phantomsection\addcontentsline{toc}{section}{\nameref{sec:abstract1}}

\begin{otherlanguage*}{english}
  <English abstract of the paper. Optional, but recommended.>

\textbf{Keywords}: <add some keywords> 
\end{otherlanguage*}

\cleardoublepage

\section*{Introduction}
\label{sec:intro1}
\phantomsection\addcontentsline{toc}{section}{\nameref{sec:intro1}}

%\defcitealias{keylist}{alias}

\section*{Material and methods}
\label{sec:matmet1}
\phantomsection\addcontentsline{toc}{section}{\nameref{sec:matmet1}}

\subsection*{Study area}
\label{subsec:area}
\phantomsection\addcontentsline{toc}{subsection}{\nameref{subsec:area}}

\begin{otherlanguage*}{english}
  Our study was conduct within the Portneuf Wildlife Reserve in the Captial-Nationale administrative region (72°24'W, 47°07'N) near Rivière-à-Pierre and Lac Amanites. 
  This area is located within the balsam fir-yellow birch bioclimatic domain (Saucier et al., 2009) and lie on a deep glacial till as surface deposit (Gosselin, 1998).
  The mean daily temperature is 4 ◦C for the 1981-2010 period of the nearest weather station (Lac aux sables)(Environment Canada, 2019). 
  Based on the same report, the mean annual precipitation and snowfall are respectively 1133.2 mm and 230.3 cm.
  We used the assisted migration experimental system establish in 2018 by the Minister of Natural Resources and Forests to collect our data (MNRF)(Royo, 2023).
  This system is a factorial experimental design with a split-split-split-plots using 4 replications (complete random blocks). 
  Each whole block (200 m x 140 m) is occupied by an overstory treatment (clearcut and 50\% shelterwood cut). 
  A cervid exclusion treatments (excluded and non-excluded) is used as a split plot and a competing vegetation treatments as a split-split plot. 
  Climatic analogs associated with three climate projections (current climate, mid-century 2050, and end of century 2080) 
  are used as a split-split-split plots with the seedlings of 9 species in mixed planting: black cherry (\textit{Prunus serotina}), northern red oak (\textit{Quercus rubra}), 
  northern white-cedar (\textit{Thuja occidentalis}), shagbark hickory, (\textit{Carya ovata}), sugar maple (\textit{Acer saccharum}), red pine (\textit{Pinus resinosa}), 
  red spruce (\textit{Picea rubens}), white spruce (\textit{Picea glauca}), and white pine (\textit{Pinus strobus}).

\end{otherlanguage*}


\subsection*{Sampling design}
\label{subsec:sampling}
\phantomsection\addcontentsline{toc}{subsection}{\nameref{subsec:sampling}}

We selected six sampling units (10 m x 7,5 m each) in both overstory treatment in four blocks and added three controls units outside every blocks for a total of 60 units in our system. 
Controls were separated from the blocks per at least 10 meters to remove treatments effects.
Each sampling unit contained three methods to collect species data. 
Artificial cover boards were used to count red-backed salamanders (Hesed, 2012; Mazerolle et al., 2021; Moore, 2009), 
pitfall traps for the carabid monitoring (sources), and soil cores with litter collection for springtails assessment (sources) (figure).
We conducted four sampling visits of five consecutive days each, namely in mid-May, mid-June, mid-July, and mid-August, during the summer 2022, to assesse the temporal effect. 
Blocks were randomly visited to reduce time of day and sampler's fatigue level effects.

Artificial coverboards is commonly employed for salamander sampling and yield similar or superior results compared to other methods such as active searching (Hyde, Simons, 2001; Moore, 2009). 
This method helps reduce variability in the number and size of sampled ground objects (Hyde and Simons 2001). 
Coverboards were made of spruce wood, measured 25 cm x 30 cm x 5 cm and were placed directly on the ground without litter underneath to maintain higher humidity level under the boards (Mazerolle et al., 2021). 
Six boards were set per sample and control units, resulting in a total of 360 coverboards.
They were positioned in two rows of three boards each, centered and oriented along the length of the sampling units.
Boards were spaced 2.5 m apart along the length of the sampling unit and 5 m apart along the width.
All boards were put outside during march 2022 to provide aging of the boards under natural conditions, thereby increasing their likelihood of being used by salamanders (article).
Throughout the sampling visits, cover boards were inspected once on the same day, and salamanders were counted without any manipulation.

Pitfall trapping is a passive sampling method used to assess the species richness of ground-dwelling invertebrates (Knapp, Růžička, 2012; Kotze et al., 2011; Lövei,  Sunderland, 1996). 
This capture method works particularly well for active arthropods moving on the ground, such as carabids (Baars, 1979; Lövei, Sunderland, 1996). 
Pitfall traps were produced by Bio.Contrôle services and included a container with a diameter of 12.5 cm, a depth of 25 cm and a cover raised 4.5 cm above the trap 
to prevent debris and rain from filling the container (figure ?).
They were equipped with a protective grid with a mesh size of 15 mm, limiting trap access to carabid-sized individuals and reducing the chances of predation by small mammals. 
Typically, a preserving liquid (such as propylene glycol or alcohol) is placed in the bottom of the container to preserve captured individuals. 
However, salamanders can have a width comparable to that of certain carabids, and therefore get captured and drown in the traps. 
For ethical reasons and considering the difficulties to limit access to salamanders without influencing the sampling of carabids, we decided to use dry pitfall traps without any preserving liquid (Luff, 1975). 
We also add wet sponges to the bottom of each container to maintain a suitable level of humidity for salamanders.
We centered one pitfall in each sampling and control units (figure), resulting in a total of 60 pitfall traps. 
Traps were inserted in the soil at a depth allowing the container's opening to be juxtaposed with the soil surface. 
Outside the sampling visits, all pitfall traps were closed with adhesive tape around the opening to prevent individuals capture. 
On the first day of each visit, traps were opened and carabids captured were collected for each remaining day. 
Individuals were placed in a container with 20\% alcohol and identified at the species level afterwards.
Identification was conduct with a (modèle du binoculaires) using André Larochelle identification keys (source)(annexe).
Carabids were categorized in two groups as salamander prey or competitors based on the salamander gape size (table)(source).

Soil cores is commonly used to sample mesofauna in litter and different soil horizons (Chauvat et al., 2011; Farská et al., 2014; Ponge, 2000; Salamon et al., 2004; Wu et al., 2014). 
Two soil cores with litter collection were harvested per sampling unit per visit using a soil sampling pedological probe. 
Cores were harvested inside the coverboards area to associate the presence of springtails with salamanders and carabids detection (figure). 
Cores had a diameter of 5 cm, a depth of 5 cm and a 15 cm x 15 cm litter quadrat was collected above each soil sample (sources).
Both substrates were used to target mesofauna and obtain springtail communities directly related to the ecology of salamanders and carabids (Chauvat et al., 2011; Edwards, 1991; Raymond-Leonard et al., 2018, Rousseau ).
Soil and litter from the same unit were pooled in Ziploc\up{\uppercase{tm}} bags and stored in a cooler at $\pm$ 4°C (Chauvat et al., 2011, Rousseau), providing 60 unit samples per visit.
Samples were placed in a Tullgren dry-funnel for springtails extraction within a maximum of 48 hours after the harvest (Figure )(Rusek, 1998; Wu et al., 2014, Laigle). 
The extraction process lasted six days with a gradual temperature escalation (25°C to 50°C)(Raymond-Leonard).
Springtails were preserved in 75\% alcohol (Wu et al., 2014) prior to isolation from surrounding organisms, with subsequent identification at the family level.
Identification was done with a (modèle du bino et du microscope) by using (nom de l'auteur de la clé d'identification) identification keys (source).
Springtails dry biomass per samples and control units was quantified with a Sartorius Cubis\up{\uppercase{tm}} MSA3.6P-000-DM micro balance after being lyophilized with a (modèle lyophilisation).



\subsection*{Sites habitat characteristics}
\label{subsec:habitat}
\phantomsection\addcontentsline{toc}{subsection}{\nameref{subsec:habitat}}

Several environmental variables that may affect the taxons's occupancy and detection were measure to understand more precisely overstory treatment effects.

detection, encompasses environmental factors affecting animal activity and, consequently, the probability of detecting individuals. 
These variables fluctuate on a daily basis. For instance, air temperature and humidity levels influence habitat use by salamanders and carabids (Kotze et al., 2011; Lövei , Sunderland, 1996; Spotila, 1972). 
Additionally, the date of the last precipitation can impact the activity level of salamanders (O’Donnell et al., 2014) and the distribution of carabid species (Butterfield, 1996; Kotze et al., 2011). 
Therefore, it is necessary to measure such variables as they influence individual detection. To achieve this, we will utilize eight meteorological stations set up by the MFFP. 
Each block will include two stations distributed in each harvesting treatment (CT, CPRU). 
The stations will facilitate the measurement of daily averages of temperature and relative humidity in the air (1.30 m above the ground) and on the ground (interface of humus and mineral soil). 
Moreover, each meteorological station will record daily precipitation levels.
The second category of variables relevant to our study concerns characteristics specific to the sampling site, which vary little throughout the season. 
This type of variable is termed occupation as it has an impact on habitat use by species. 
The volume of woody debris on the ground and litter depth are variables that play a crucial role in habitat use for the three taxa under study. 
These variables serve as protection, shelters to maintain suitable temperature and humidity, and are used for feeding 
(Bird et al., 2004; Grover, 1998; Harmon et al., 1986; Koivula et al., 1999; McKenny et al., 2006; Patrick et al., 2006). 
The volume of woody debris will be estimated by establishing 400 m2 plots (r = 11.28 m) (Méthot et al., 2014). 
The plots will be set up at the center of each sampling unit, totaling 60 plots (Figure 7). Woody debris enumeration will be conducted within the 400 m2 perimeter. 
Only woody debris with a basal diameter greater than or equal to 9 cm and a length greater than or equal to 1 m will be considered. 
If a piece of woody debris crosses the plot boundary, only the part inside the plot will be considered. Subsequently, we will measure the basal diameter and apical diameter of woody debris. 
The paraboloid-cone formula will be used to calculate the volume of woody debris (Fraver et al., 2007).

%formule

The woody debris will be grouped into four distinct classes based on their level of degradation, ranging from least degraded (Class 1) to most degraded (Class 4), following the classification by Fauteux et al. (2012). 
Debris will be categorized based on their texture, shape, the presence of bark, and the knife test. 
The litter depth will be measured in each artificial refuge (Figure 7) at the location where the board is placed, resulting in a total of six systematic measurements per sampling unit (Mazerolle et al., 2021).


We will also assess canopy openness, as it may influence the activity of carabids (Koivula et al., 2002; Kotze et al., 2011). 
This measurement will be conducted at the center of each 400 m2 plot in every sampling unit using a spherical densitometer (Figure 7) (Mazerolle et al., 2021). 
Canopy closure measurements will be taken from the ground to represent ground-level light conditions. 
Four measurements will be taken at each location, oriented toward each of the four cardinal points. 
The average of the results for the four points will represent the data for the sampling unit. 
Measurements of woody debris volume and litter depth will be taken in April 2022 during the installation of artificial refuges. 
The measurement for forest canopy cover will be conducted in June 2022 to capture the peak foliage representing the maximum vegetative cover in the environment.


% débris ligneux
% ouverture de la canopé
% pronfeur de litière
% température 
% précipitation humidité




\subsection*{Statistical analyses}
\label{subsec:analyses}
\phantomsection\addcontentsline{toc}{subsection}{\nameref{subsec:analyses}}
% selection de modèles
% modèles d'occupation % detection imparfaites
However, it would be possible to account for imperfect and variable species detection in our models to enable comparison (Mazerolle et al., 2007).
% Modèle d'équations structurelles

\clearpage

\section*{Results}
\label{sec:results1}
\phantomsection\addcontentsline{toc}{section}{\nameref{sec:results1}}

\clearpage

\section*{Discussion}
\label{sec:discu1}
\phantomsection\addcontentsline{toc}{section}{\nameref{sec:discu1}}

\section*{Conclusion}
\label{sec:conclu1}
\phantomsection\addcontentsline{toc}{section}{\nameref{sec:conclu1}}

\section*{Acknowledgements}
\label{sec:acknowl1}
\phantomsection\addcontentsline{toc}{section}{\nameref{sec:acknowl1}}

\section*{Conflict of interest}
\label{sec:conflict1}
\phantomsection\addcontentsline{toc}{section}{\nameref{sec:conflict1}}

None declared

\section*{Author contributions}
\label{sec:author1}
\phantomsection\addcontentsline{toc}{section}{\nameref{sec:author1}}

\cleardoublepage


\begin{otherlanguage}{english}
\bibliography{references.bib}
\bibliographystyle{ecologyNewEN.bst}
\addcontentsline{toc}{section}{References}
\end{otherlanguage}
