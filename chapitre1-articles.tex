\chapter{Direct and indirect effects of silvicultural treatments on soil fauna in mixed temperate forests}     % numéroté
\label{chapitre1-articles}    

William Devos$^1$, Mathieu Bouchard$^1$, Marc J. Mazerolle$^1$

%href{mailto:william.devos.1@ulaval.ca}
$^1$ Centre d'étude de la forêt, Département des sciences du bois \\ 
et de la forêt, Université Laval, Québec, QC G1V 0A6, Canada. \\ 

\clearpage

\section*{Résumé}
\label{sec:resume1}
\phantomsection\addcontentsline{toc}{section}{\nameref{sec:resume1}}

\begin{otherlanguage*}{french}

Les environnements forestiers jouent un rôle majeur à l’échelle planétaire à travers leurs fonctions écosystémiques et leur apport économique. 
Toutefois, l'intensification de l'exploitation des forêts peut amener une perte d'habitats essentiels à la biodiversité et perturber les dynamiques écologiques entre les espèces. 
Cette étude utilise un modèle d'équation structurelle, intégrant des modèles d’occupation et des modèles linéaires mixtes, pour évaluer l’impact des traitements sylvicoles sur la dynamique de la faune du sol en forêt mixte tempérée. 
Les objectifs étaient, dans un premier temps, de mesurer l’impact des traitements de coupe partielle et de coupe totale sur des variables environnementales influençant l’utilisation de l’habitat par la faune du sol. 
Dans un second temps, il s'agissait d’évaluer les effets directs et indirects de ces coupes sur la cooccurrence entre la salamandre cendrée, les carabes et les collemboles. 
Nos résultats montrent que les traitements de coupes forestières influencent significativement les variables environnementales. 
Les coupes totales ont entraîné une ouverture plus grande de la canopée, une diminution de la profondeur de la litière et une réduction du volume de débris ligneux, 
tandis que les coupes partielles permettent une meilleure rétention de ces attributs environnementaux. 
\hl{Les traitements sylvicoles ont, dans l’ensemble, eu peu d’effet sur la pro- babilité d’occupation des salamandres et des carabes, ou sur la biomasse des collemboles. 
Toutefois, la probabilité d’occupation des salamandres dans les traitements de coupe totale était marginalement plus faible que dans les coupes partielles} (IC à 90\%). 
De plus, nos résultats suggèrent que les effets des coupes ne se propagent pas à travers le réseau trophique des prédateurs vers les proies du sol forestier. 
Notre étude démontre l'importance de favoriser des méthodes de coupe limitant la perturbation du milieu, telles que les coupes partielles, lors des récoltes de bois. 
Ces pratiques permettent une meilleure préservation des attributs environnementaux, ce qui profite à la faune du sol. 
Notre approche met également en évidence la pertinence des modèles d’équations structurelles pour analyser les réseaux trophiques complexes des écosystèmes forestiers. 

\textbf{Mots-clés}: \textit{Plethodon cinereus}, carabe, collembole, modèle d'équations structurelles, cooccurrence, coupe forestière.
\end{otherlanguage*}

\clearpage

\section*{Abstract}
\label{sec:abstract1}
\phantomsection\addcontentsline{toc}{section}{\nameref{sec:abstract1}}

Forest ecosystems play a vital role in the biosphere through their ecosystem functions and economic contributions. 
However, intensified forest exploitation can lead to the loss of essential habitats for biodiversity and disrupt ecological dynamics among species. 
\hl{This study employed a structural equation model approach, integrating occu- pancy models and linear mixed models, 
to assess the impact of silvicultural treatments on soil fauna dynamics in temperate mixed forests. 
The objectives were 1) to evaluate the impact of partial and clear-cut treatments on environmental variables influencing habitat use by soil fauna, 
and 2) to evaluate the direct and indirect effects of these harvests on the cooccurrence of salamanders, ground beetles, and springtails.}
Our results show that forest harvesting treatments significantly influence environmental variables. 
Clear-cuts resulted in greater canopy opening, reduced litter depth, and decreased coarse woody debris volume, whereas partial cuts better preserved these environmental attributes. 
\hl{Silvicultural treatments had little to no effect on salamander and ground beetle occupancy probabilities or springtail biomass. However, the occupancy probability of salamanders in clearcut sites was marginally lower than in partial-cut sites }(90\% CI). 
Furthermore, our results suggest that the effects of harvesting do not propagate through the trophic network from predators to prey in forest soils. 
Our study highlights the importance of promoting harvesting methods that minimize habitat disturbance, such as partial cuts. 
These practices help preserve key environmental attributes, benefiting soil fauna. 
Our approach also underscores the relevance of structural equation models in analyzing the complex trophic networks of forest ecosystems.

\textbf{Keywords}: \textit{Plethodon cinereus}, ground beetle, springtail, structural equation modelling, cooccurrence, logging.


\cleardoublepage

\section*{Introduction}
\label{sec:intro1}
\phantomsection\addcontentsline{toc}{section}{\nameref{sec:intro1}}

%\defcitealias{keylist}{alias}

Forest ecosystems play a vital role in the biosphere, both economically and ecologically, by providing essential services such as carbon sequestration, climate regulation, water retention, and biodiversity conservation \citep{Balvanera2006Quantifyingevidence,Diaz2006BiodiversityLoss,Canadell2008Managingforests,Pawson2013Plantationforests}. 
However, the growing demand for forest products and other services has intensified harvesting practices, leading to ecosystem \hl{degradation and biodiversity loss} \citep{Bengtsson2000Biodiversitydisturbances,Sala2000Globalbiodiversity,Foley2005GlobalConsequences,Naeem2012functionsbiological}. 
\hl{At the landscape level}, logging often results in an overrepresentation of early-successional forests and a reduction of abundance of later successional stages \citep{Cyr2009Forestmanagement,Boucher2017Cumulativepatterns}. 
This shift may reduce the structural complexity of stands, such as tree species composition, vertical stratification, age structure, successional dynamics, and disturbance frequency \citep{Bergeron2000Speciesstand,Commarmot2005Structurevirgin,Varga2005Treesizediversity}. 
In the long term, this erodes the resilience of forests at a local scale and can lead to a decline in ecosystem services \citep{Hooper2012globalsynthesis,Edwards2014Maintainingecosystem}. 
Harvest activities can cause habitat loss and fragmentation, reducing access to food, shelter, and breeding sites for certain species \citep{Bouderbala2023Longtermeffect}. 
Overall, modifications of forest attributes can contribute to a loss of species and functional diversity \citep{Saccheri1998Inbreedingextinction}. 

Over the past few decades, ecosystem-based management has been proposed as a more sustainable approach to forest ecosystems \citep{Perry1998scientificbasis,Kuuluvainen2002Naturalvariabilitya}. 
This strategy aims to emulate natural disturbances and the resulting stand structure and successional processes \citep{Bergeron1999Forestmanagementa}. 
The goal is to preserve the resilience of forest ecosystems while ensuring the availability of a variety of services \citep{Szaro1998emergenceecosystem,MacDicken2015Globalprogress}. 
\hl{More precisely, silvicultural treatments at the stand level, such as clearcut and partial-cut, can replicate the ecological impacts associated with natural disturbances. }
However, the level of disturbance caused by logging depends on the type of treatment employed \citep{Harpole1999Effectsseven,Chaudhary2016Impactforest,Koivula2019Responsesboreal}. 

Clear-cutting is one of the most widely used practices in temperate and boreal forests \citep{Fedrowitz2014Canretention,Chaudhary2016Impactforest}. 
This method involves the complete removal of commercial trees and leads to the establishment of even-aged stands \citep{Brashears2004AssessmentCanopy,Martin2020Forestmanagement}. 
Clear-cutting substantially alters ecosystem structure compared to unharvested stands \citep{Hanski2000Extinctiondebt}. 
\hl{Consequently, this treatment can be used as an alternative for stands driven by large disturbances such as wildfire} \citep{Bergeron1999Forestmanagementa}. 
\hl{Moreover, some tree species regenerate more effectively after clear-cutting, depending on their shade tolerance and their need for soil disturbance} \citep{Bergeron1999Forestmanagementa}. 

\hl{Partial cuts can also be used to implement ecosystem-based management practices} \citep{Bergeron1999Forestmanagementa,Raymond2009irregularshelterwood}. 
This type of harvesting involves selective tree removal, retaining partial forest cover, and encouraging an irregular forest structure by maintaining multiple age classes or tree sizes \citep{Raymond2009irregularshelterwood}. 
This heterogeneity can include both vertical and horizontal diversity within forest stands. 
\hl{Therefore, partial cutting might be applied to emulates natural succession} \citep{Bergeron1999Forestmanagementa}.
Partial cuts \hl{can maintain} attributes often associated with old-growth forests, such as coarse woody debris, multiple canopy layers, varied tree sizes, and a favorable microclimate. 
This enhances the retention of species associated with mature or old stands, sensitive to climate variations or habitat loss \citep{Hansen1991Conservingbiodiversity,Ameray2021Forestcarbon}. 

Soil fauna is one of the groups most impacted by forest disturbances due to their sensitivity to environmental changes and limited dispersal abilities \citep{Marshall2000Impactsforest,Coyle2017Soilfauna}. 
Canopy removal alters soil conditions by increasing exposure to sunlight, leading to increased temperatures, changes in humidity, higher wind speeds, and intensified precipitation \citep{Keenan1993ecologicaleffects,Heithecker2007Edgerelatedgradients,Lindo2003Microbialbiomass,Brook2008Synergiesextinction,Zhang2022Intensiveforest}. 
The movement of machinery also increases soil compaction, reducing porosity and consequently affecting the organisms that live there \citep{Battigelli2004Shorttermimpact,Mazerolle2021Woodlandsalamander}. 
These changes at the surface level affect nutrient availability by altering litter, root secretions, leaching, and soil chemical properties. \citep{Covington1981Changesforest,Marshall2000Impactsforest,Lindo2003Microbialbiomass,Battigelli2004Shorttermimpact}. 
Logging also reduces the availability of microhabitats, such as deadwood, cavities in mature trees, and root plates, which provide shelter for soil fauna \citep{Spies1999Dynamicforest,Christensen2005Deadwood,Brassard2008EffectsForest}. 
These modifications create unfavorable conditions for species that rely on cool, moist environments, such as amphibians and arthropods, potentially leading to population declines or local extinctions \citep{Paillet2010Biodiversitydifferences,Fedrowitz2014Canretention,Chaudhary2016Impactforest}. 
Due to their sensitivity to environmental shifts and limited dispersal capacity, amphibians and arthropods serve as important indicators for understanding the effects of forestry practices on biodiversity and the ecological integrity of forests. 
Moreover, both groups have experienced substantial declines in recent decades, primarily due to habitat loss through different anthropic disturbances such as forest management practices \citep{Houlahan2000Quantitativeevidence,Stuart2004Statustrends,Wagner2021Insectdecline}. 
The eastern red-backed salamander (\textit{Plethodon cinereus} (Green, 1818)), ground beetles (Carabidae), and springtails (Collembola) are common \hl{groups} of species often used as indicators to assess the impact of disturbances on amphibians and arthropods. 

The eastern red-backed salamander accounts for one of the largest vertebrate biomass \hl{in many} North American forests \citep{Burton1975Salamanderpopulations,Petranka1993Effectstimber,semlitschAbundanceBiomassProduction2014a}. 
As a generalist predator, this plethodontid species plays a key role in regulating detritivorous invertebrates, subsequently influencing decomposition processes, nutrient cycling, and carbon dynamics \citep{Burton1975Energyflow,Wyman1998Experimentalassessment,Walton2013Topdownregulation,Hickerson2017Easternredbacked}. 
Moreover, red-back salamanders serve as high-nutritional prey for many predators, including birds, mammals, and reptiles \citep{Burton1975Energyflow,Pough1987abundancesalamanders}. 
The eastern red-backed salamander relies on cutaneous respiration for gas exchanges \citep{Heatwole1961Relationsubstrate}. 
This type of respiration requires salamanders to occupy specific microhabitats, foraging on the surface when temperature and humidity are favorable or \hl{retreating beneath the soil surface} during less suitable conditions \citep{Grizzell1949HibernationSite,FraserEmpiricalEvaluation1976,Jaeger1980MicrohabitatsTerrestrial}. 
The loss of refuges due to logging decreases habitat quality, limiting the time salamanders spend on the forest floor \citep{Achat2015Quantifyingconsequences,Peele2017Effectswoody}. 
Consequently, poor surface conditions, soil compaction, and low levels of coarse woody debris (CWD) can negatively impact population dynamics \citep{Peterman2014Spatialvariation}. 

Several studies suggested a complex relationship between red-backed salamanders and arthropod predators such as ground beetles \citep{Gall2003BehavioralInteractions,Walton2006Salamandersforestfloor,Hickerson2018Behavioralinteractions}. 
Ground beetles are a well-documented taxon both in terms of taxonomy and ecology \citep{loveiEcologyBehaviorGround1996}. 
These insects have short lifespans, occupy a high position in the soil food web, and respond rapidly and in complex ways to environmental change \citep{loveiEcologyBehaviorGround1996}. 
Ground beetles play an ecological role by regulating invertebrate populations while also serving as prey for various amphibians, reptiles, birds, and mammals \citep{loveiEcologyBehaviorGround1996}. 
With around 40,000 known species, ground beetles are one of the most diverse beetle families and among the most abundant soil-dwelling arthropods \citep{Erwin1985taxonpulse,loveiEcologyBehaviorGround1996,Rochefort2006GroundBeetle}. 
They are widely distributed across nearly all terrestrial ecosystems, although species vary in their habitat preferences \citep{loveiEcologyBehaviorGround1996,kotzeFortyYearsCarabid2011a,Larochelle2003naturalhistory}. 
The response of ground beetles to environmental change varies with species, making these organisms relevant for studying the effects of anthropic disturbances \citep{Rainio2003Groundbeetles}. 

Springtails represent one of the most abundant and diverse groups of mesofauna \citep{rusekBiodiversityCollembolaTheir1998}. 
Different springtail communities occupy a variety of ecological niches within the soil \citep{pongeVerticalDistributionCollembola2000}. 
Their vertical distribution is largely influenced by abiotic factors such as light, humidity levels, and soil porosity. 
Springtails are mainly fungivores and detritivores, playing a key role in wood decomposition. 
They also impact the physical structure and mineralization rates of litter, influence nutrient absorption, 
regulate microbial communities, and contribute to soil microstructure formation \citep{Petersen1982comparativeanalysis,Neher2012Linkinginvertebrate,Maass2015Functionalrole,Potapov2016Connectingtaxonomy}. 
Additionally, springtails serve as an important food source for a range of organisms, including amphibians, beetles, arachnids, birds, and reptiles \citep{Burton1975Energyflow,Bauer1982Predationcarabid,rusekBiodiversityCollembolaTheir1998}.

Through their trophic relationships, their sensitivity to environmental changes, and their dependence on forest features such as woody debris and litter, 
salamanders, ground beetles, and springtails are relevant target groups for studying the impact of forestry practices on soil fauna \citep{Salmon2008Relationshipssoil}. 
Most studies on the impacts of \hl{forest practices} on biodiversity focus on the direct effects of disturbances on one or more species groups, 
without considering the relationships between environmental variables and different species groups, thereby neglecting the indirect effects of logging on soil fauna \citep{josephIntegratingOccupancyModels2016,Pollierer2021Diversityfunctional,Kudrin2023metaanalysiseffects}. 

\hl{Our study aims to fill this gap by investigating how the short-term effects of two common types of logging, namely clear-cutting, and partial cutting, propagate through the forest ecological network and affect the dynamics of soil fauna. 
Ultimately, this knowledge will better inform sustainable forest management. }
The objectives were to quantify the effect of overstory treatments on environmental variables that influence habitat use by soil species 
and to evaluate the impact of overstory treatments on habitat use. 
We hypothesized that key environmental variables related to habitat vary with harvesting \hl{intensity}. 
Specifically, we predicted that the most intense treatments (clear-cutting followed by site preparation) would decrease litter depth and CWD compared to partial harvest or uncut controls, 
as a consequence of reduced accumulation of leaves and woody debris on the forest floor and increasing canopy openness. 
We also hypothesized that overstory treatments modify habitat use by soil fauna both directly and indirectly through the trophic network. 
This hypothesis is based on the fact that the eastern red-backed salamander, as a predator sensitive to disturbances, regulates soil invertebrate communities \citep{Wyman1998Experimentalassessment,MichaelWalton2005Salamandersforestfloor,Walton2006Salamandersforestfloor,Walton2013Topdownregulation,Hickerson2017Easternredbacked}. 
Additionally, ground beetles have a complex relationship with salamanders, \hl{acting both as competitors and prey, depending} on their size \citep{Jaeger1980MicrohabitatsTerrestrial,loveiEcologyBehaviorGround1996,Gall2003BehavioralInteractions}. 
We predicted that logging treatments would impact the occupancy of salamanders and large ground beetles, which subsequently alters small ground beetles occupancy and finally the springtail biomass. 


\section*{Material and methods}
\label{sec:matmet1}
\phantomsection\addcontentsline{toc}{section}{\nameref{sec:matmet1}}

\subsection*{Study area}
\label{subsec:area}
\phantomsection\addcontentsline{toc}{subsection}{\nameref{subsec:area}}

Our study was conducted within the Portneuf Wildlife Reserve in the Capitale-Nationale administrative region near Lac des Amanites (47°07’N, 72°24’W, Figure \ref{fig:area}). 
This area is located within the balsam fir (\textit{Abies balsamea})-yellow birch (\textit{Betula alleghaniensis}) bioclimatic domain, according to the ecological classification used in Québec \citep{saucierChapitreEcologieForestiere2009}. 
Other tree species in this bioclimatic domain included sugar maple (\textit{Acer saccharum}), red maple (\textit{Acer rubrum}), white spruce (\textit{Picea glauca}), black spruce (\textit{Picea mariana}), red spruce (\textit{Picea rubens}), white birch (\textit{Betula papyrifera}), and quaking aspen (\textit{Populus tremuloides}) \citep{olaBelowgroundCarbonStocks2024}. 
The experimental sites are located on deep glacial tills with sandy loams that are moderately well drained \citep{CanadianSystemSoil1998}. 
The mean daily temperature is 4\up{o}C based on the 1981-2010 period at the nearest weather station (Lac aux sables, \citealp{environmentcanadaCanadianClimateNormals2019}). 
Based on the same report, the average annual precipitation is 1133.2 mm (including water equivalent of snow), with snowfall averaging 2303 mm. 
We conducted our study within an assisted migration experimental system established in 2018 by the Ministère des Ressources naturelles et des Forêts du Québec \citep{Champagne2021Seedlingresponse}. 
This experiment uses a factorial experimental design with split-plots replicated in four blocks. 
Each block (200 m x 140 m) is split in two overstory treatments: clear-cut and regular shelterwood cut at 50\% of the merchantable basal area (partial cut). 
The harvest took place during the summer 2017, followed by a site preparation by trenching in June 2018 within the clear-cut areas to create more uniform site conditions and facilitate planting. 
No site preparation was conducted in the shelterwood cuts.

\pagebreak

\begin{figure}[ht!]
	\centering
	\includegraphics[scale=0.31]{fig_area7.png}
	\caption[Localization of the Capitale-Nationale administrative region in Quebec, Canada and position of the study area near Lac des Amanites in Portneuf Wildlife Reserve, Quebec, Canada.]
  {Localization of the Capitale-Nationale administrative region in Quebec, Canada (A) and position of the study area in Portneuf Wildlife Reserve, Quebec, Canada (B) where the clear-cut (orange) and the partial cut (green) associated with the assisted migration experimental system were applied in 2017 (47°07'N, 72°24'W)(C).}
	\label{fig:area}
	\end{figure}  


\subsection*{Salamander and arthropod sampling}
\label{subsec:sampling}
\phantomsection\addcontentsline{toc}{subsection}{\nameref{subsec:sampling}}

We selected a total of 60 sampling units measuring 10 m x 7.5 m to collect our data: the four blocks served as replicates, 
with each block containing six sampling units for both the clear-cut and partial cut overstory treatments, 
whereas three additional sampling units were positioned outside the block, serving as uncut controls (Figure \ref*{fig:blockSU}). 
\hl{Sampling units inside blocks were separated by at least 14.5 m and} the uncut controls were separated from the nearest harvesting by at least 10 m.

In each sampling unit, we used three sampling methods to collect species data: artificial coverboards, pitfall traps, and soil cores. 
We used artificial coverboards to sample eastern red-backed salamanders \textit{Plethodon cinereus} at our study sites \citep{hydeSamplingPlethodontidSalamanders2001,mooreComparisonPopulationEastern2009c,hesedUncoveringSalamanderEcology2012,Mazerolle2021Woodlandsalamander}. 
All coverboards were placed outdoors in March 2022 to allow for natural aging \citep{hedrickEffectsCoverboardAge2021,Grasser2014Effectscover}. 
Coverboards consisted of 25 cm x 30 cm x 5 cm pieces of untreated spruce wood. 
Six coverboards spaced by at least 2.5 m from their closest neighbor were arranged in a rectangular array in each sampling unit, resulting in a total of 360 coverboards across our 60 sampling stations (Figure \ref{fig:blockSU}). 
Each board was in direct contact with the soil after we had cleared the litter underneath \citep{Mazerolle2021Woodlandsalamander}. 
We conducted four visits to coverboards during the summer 2022, with each visit spaced one month apart, namely in mid-May, mid-June, mid-July, and mid-August. 
During each visit, blocks were visited in a random sequence to reduce the potential effects of time of day and observer fatigue. 
During a given visit, the 360 coverboards were inspected on the same day, and we counted the number of red-backed salamanders underneath. 

We used pitfall traps to capture ground beetles \citep{baarsCatchesPitfallTraps1979,spenceSamplingCarabidAssemblages1994a,loveiEcologyBehaviorGround1996,kotzeFortyYearsCarabid2011a,knappEffectPitfallTrap2012}. 
Trap design was based on Multi-Pher traps (Bio-Contrôle Services, Ste. Foy, Quebec, Canada). 
These traps consisted in a plastic cylinder (13 cm diameter x 21 cm height). 
Each pitfall trap was fitted with a cover (26 cm diameter) raised 4.5 cm above the opening to prevent debris and rain from filling the container \citep{Jobin1988MultiPherinsect,Moore2004Effectstwo,bouchardBeetleCommunityResponse2016b}. 
We covered each trap with a protective stainless steel mesh size of 15 mm, allowing trap access to carabid-sized individuals and limiting access by predators.  
We did not add preserving liquid in traps, but we added wet sponges to avoid harming vertebrates small enough to pass through the mesh. 
\hl{A single pitfall trap was installed at the center of each sampling unit} (Figure \ref{fig:blockSU}). 
Traps were inserted in the soil at a depth allowing the container’s opening to be level with the soil surface. 
Trapping occurred during four sampling periods during 2022 (mid-May, mid-June, mid-July and mid-August). 
Traps were opened on the first day of each survey, and we collected captured ground beetles daily during five consecutive days. 
Outside the trapping periods, pitfall traps were sealed with adhesive tape to prevent captures. 
The contents of the traps were preserved in 70\% alcohol and individuals were identified at the species level afterwards. 
Identification \hl{at the species} level was conducted with a ZEISS SteREO Discovery.V12 binocular microscope using the \cite{larochelleManuelIdentificationCarabidae1976} taxonomic keys. 
\hl{We categorized ground beetles into two groups, either as small ground beetles (salamander prey, $\leq$ 8,3 mm ) or large ground beetles (salamander competitors, $>$ 8.3 mm)}, based on the red-backed salamander gape size (Table \ref{tab:carabid}, \citealp{jaegerFoodLimitedResource1972,magliaModulationPreycaptureBehavior1995,magliaOntogenyFeedingEcology1996}).

We extracted soil cores to sample springtails \citep{pongeVerticalDistributionCollembola2000,salamonEffectsPlantDiversity2004,chauvatChangesSoilFaunal2011a,farskaManagementIntensityAffects2014}. 
Core sampling occurred during the same four sampling periods as for the salamanders and ground beetles (mid-May, mid-June, mid-July and mid-August). 
During a sampling period, we collected two cores in each sampling unit using a pedological probe (5 cm diameter x 5 cm depth). 
\hl{Each of the two soil samples was placed on either side of the pitfall trap, centered between four coverboards} (Figure \ref{fig:blockSU}). 
Directly above each core soil sample, we also collected leaf litter in 15 cm x 15 cm area, resulting in a total of two litter samples per sampling unit on each visit \citep{raymond-leonardSpringtailCommunityStructure2018a,rousseauForestFloorMesofauna2018}.  
We extracted springtail communities related to the ecology of salamanders and ground beetles \citep{edwardsAssessmentPopulationsSoilinhabiting1991,chauvatChangesSoilFaunal2011a,raymond-leonardSpringtailCommunityStructure2018a,rousseauForestFloorMesofauna2018}. 
Soil and litter from the same unit during the same sampling occasion (two core samples and two litter samples) were pooled in Ziploc\up{\texttrademark{}} bags and stored in a cooler at ca. 4 °C \citep{chauvatChangesSoilFaunal2011a,rousseauForestFloorMesofauna2018}, providing 60 samples during each of the four sampling periods. 
We placed each sample in an individual Tullgren dry-funnel for springtail extraction within 48 h after collection \citep{rusekBiodiversityCollembolaTheir1998,wuCompositionSpatiotemporalVariation2014,rousseauForestFloorMesofauna2018}. 
The extraction process lasted six days, with a gradual temperature increase: 2 days at 25 °C, followed by 2 days at 37 °C and 2 days at 50 °C \citep{raymond-leonardSpringtailCommunityStructure2018a}. 
Springtails were separated from other invertebrates and identified at the family level with a ZEISS SteREO Discovery.V12 binocular microscope and a Leitz orthoplan phase-contrast fluorescent trinocular microscope using \cite{bellingerChecklistCollembolaWorld1996} identification keys. 
Following identification, springtails were pooled by sampling unit across the four visits and preserved in 75\% alcohol \citep{wuCompositionSpatiotemporalVariation2014}. 
Springtails within each sampling unit were freeze dried for 24 hours (Labconco FreeZone Bulk tray dryer 78060 series) for 24 hours. 
We determined the total dry biomass of springtails in each sample to the nearest $\mu$g with a microbalance (Sartorius Cubis\up{\texttrademark{}} MSA3.6P-000-DM).

\pagebreak

\begin{figure}[ht]
	\centering
	\includegraphics[scale=0.50]{fig_blockSU.png}
	\caption[Design of one block and one sampling unit with three sampling methods.]{
  Design of a block (left) and a sampling unit within the block (right). 
  The block contains two overstory treatments: clear-cut (grey background), partial cut (white background). 
  Fifteen sampling units were established per block: six per overstory treatment and three controls (\textbf{c}) outside each block.
  Each sampling unit contained six artificial 25 cm x 30 cm x 5 cm spruce coverboards (squares) and one pitfall trap (13 cm diameter x 21 cm height, circle). 
  Two soil cores (5 cm diameter x 5 cm depth, stars) were collected per survey.
  }
	\label{fig:blockSU}
	\end{figure}  

\vspace{0.5cm}


\subsection*{Environmental variables}
\label{subsec:Envar}
\phantomsection\addcontentsline{toc}{subsection}{\nameref{subsec:Envar}}

In each sampling unit, we measured several environmental variables that could affect the occupancy probability of red-backed salamanders, ground beetles, and springtails biomass.
CWD and litter depth play a crucial role in habitat use for salamanders, ground beetles, and springtails as
they serve for feeding and protection \citep{harmonEcologyCoarseWoody1986,koivula.LeafLitterSmallscale1999,birdChangesSoilLitter2004,McKenny2006Effectsstructural}. 
Salamanders also utilize CWD as shelter to maintain suitable temperature and moisture levels during dry periods \citep{Jaeger1980MicrohabitatsTerrestrial,groverInfluenceCoverMoisture1998a,patrickEffectsExperimentalForestry2006a}. 
We used 400 m\up2 plots centered on every sampling unit to estimate CWD (20 m $\times$  20 m) \citep{methotGuideInventaireEchantillonnage2014}. 
\hl{Plots were separated at least 2 m apart and did not overlap with adjacent sampling units. }
We only considered CWD with a basal diameter $\geq$ 9 cm and a length $\geq$ 1 m. 
For each CWD, we measured the basal diameter, the apical diameter and length with a tree caliper. 
Segments of CWD outside the plot boundaries were not included. 
We employed the conic–paraboloid formula to estimate the volume of each CWD \citep{fraverRefiningVolumeEstimates2007} :

\begin{equation}
  \text{Volume} = L/12 \times (5A_b + 5A_u + 2\sqrt{A_b \times A_u})
\end{equation}

\vspace{0.5cm}

Where $L$ is the length of log (cm), $A_b$ the basal area (cm\up{2}) and $A_u$ the apical area (cm\up{2}).
We measured litter depth next to each coverboard and computed the average litter depth for each sampling unit \citep{Mazerolle2021Woodlandsalamander}. \\
We evaluated canopy openness as it may influence the habitat use of red-backed salamanders, ground beetles, and springtails \citep{messereForestFloorDistribution1998,koivulaBorealCarabidbeetleColeoptera2002a,tilghmanMetaanalysisEffectsCanopy2012,henneronForestPlantCommunity2017}.
Canopy openness was measured 130 cm above the ground at the center of each sampling unit using a spherical densiometer \citep{lemmonSphericalDensiometerEstimating1956}. 
We averaged four measurements per sampling unit, oriented toward each of the four cardinal points as an estimate of canopy openness.

We collected data locally for air temperature, air humidity, and precipitation levels during the entire sampling period. 
\hl{Eight weather stations (Em50 Digital Decagon Data Logger, Part \#40800, Meter Group Inc., USA) were installed across each of the four blocks and the two logging treatments (partial-cut and clearcut). }
Each weather station measured temperature, air humidity, and atmospheric pressure, 130 cm above the ground (VP-4 Sensor (Temp/RH/Barometer), Part \#40023). 
Rain gauges were installed in the clear-cut treatments to monitor precipitation levels. 
The temperature and humidity sensors were programmed to record data every 15 minutes. 
We averaged the measurements across both weather stations to get daily average measurements. 
\hl{Since red-backed salamanders tend to be more active a few days after rainfall, we combined the total precipitation over the three days preceding each field visit} \citep{odonnellPredictingVariationMicrohabitat2014a}. 
\hl{The level of precipitation was then classified into two categories: low ($\leq$ 1 mm) or high ($>$ 1 mm). }
These variables fluctuate on a daily basis, affecting species activity and, consequently, the probability of detecting individuals \citep{spotilaRoleTemperatureWater1972,butterfieldCarabidLifeCycle1996,loveiEcologyBehaviorGround1996,odonnellPredictingVariationMicrohabitat2014a}.


\subsection*{Statistical analyses}
\label{subsec:analyses}
\phantomsection\addcontentsline{toc}{subsection}{\nameref{subsec:analyses}} 

\subsubsection{Structural equations models} 

To assess the relationship between overstory treatments and environmental variables (hypothesis 1.1) and the effects of overstory treatments on habitat use by the species groups (hypothesis 2.1), 
we employed a structural equation model (SEM) approach \citep{graceSpecificationStructuralEquation2010}. 
Specifically, we combined occupancy models and linear mixed models in our SEM \citep{mackenzieOccupancyEstimationModeling2006a,graceSpecificationStructuralEquation2010,josephIntegratingOccupancyModels2016}.
This approach enabled us to test our hypotheses within a single model. 

Some components of the SEM predicted the effect of the different overstory treatments on three environmental variables: CWD volume, canopy openness, and litter depth (Figure \ref{fig:SEM}). 
Precipitation was highly correlated with air temperature and relative humidity. 
For this reason, we did not include \hl{air temperature and relative humidity} in our model. 
The other components of the model evaluated the direct and indirect effects of overstory treatments on taxa (Figure \ref{fig:SEM}). 
We hypothesized that partial cut and clear-cut impacted each taxon, whereas salamander occupancy influenced small ground beetle occupancy and springtail biomass, because both are potential prey for salamanders. 
We also predicted that both ground beetle groups affected springtail biomass.

Within our SEM, we used occupancy models to quantify the impact of overstory treatments on the occupancy (presence) probabilities of salamanders and ground beetles (Table \ref{ann:SEM_Sp_eq}). 
Occupancy models estimate the presence of species difficult to detect after accounting for imperfect detection probability \citep{mackenzieEstimatingSiteOccupancy2002,baileyEstimatingSiteOccupancy2004,mazerolleMakingGreatLeaps2007,spiersEstimatingSpeciesMisclassification2022}. 
Here, we considered the occupancy of three species groups: red-backed salamanders, large ground beetles, and small ground beetles. 
\hl{For each site and species group, we arranged the data in a detection history matrix, using 1 to denote the detection of the species on a given visit and 0 for non-detection. }
We included the occupancy of salamanders and large ground beetles as explanatory variables on the occupancy of small ground beetles (\textit{sensu} \cite{Feldman2023Beaveractivity}). 
\hl{Because precipitation and the presence of natural cover can influence species activity on the forest floor, 
we allowed the detection probability of a given species group to vary with the volume of coarse woody debris, and the precipitation levels. 
We included a block random effect to reflect the nested structure of the experimental design. }

Some components of the SEM consisted of linear mixed models to estimate the effect of overstory treatments on environmental variables (Table \ref{ann:SEM_Env_eq}). 
Because CWD and precipitation data were heteroscedastic across harvesting treatments, we included terms to model the variance of each \hl{treatment} group explicitly (Table \ref{ann:SEM_Env_eq}). 
We used a linear mixed model component to predict springtail biomass that also included the latent occupancy state of salamanders and ground beetles to measure the impact of these \hl{taxonomic} groups on springtails. 
Here, latent occupancy corresponded to the presence probability accounting for imperfect detection probability estimated from the occupancy model \citep{mackenzieOccupancyEstimationModeling2006a}.

We formulated the SEM using a Bayesian framework. 
We used vague prior distributions for the different parameters, namely Normal(0, $\sigma$=100) for fixed effects, Uniform(0, 1) for probabilities, and Uniform(0, 10) or Uniform(0, 150) for variance parameters. 
Parameters were estimated using Markov chain Monte Carlo (MCMC) with JAGS 4.3.0 using the jagsUI package in R 4.3.1 \citep{lunnBUGSProjectEvolution2009,rcoreteamLanguageEnvironmentStatistical2020,kellnerJagsUIWrapperRjags2024}. 
We ran the model with five chains and 200,000 iterations each \citep{gelmanUnderstandingPredictiveInformation2014}. 
The first 75,000 iterations were used as burn-in, and we used a thinning rate of 5. 
We assessed convergence of MCMC chains by examining trace plots, posterior density plots, and using the Brooks-Gelman-Rubin statistic. 
The JAGS model code is available in Table \ref{ann:SEM_script}.

\vspace{10pt}

\begin{figure}[h!]
	\centering
	\includegraphics[scale=0.55]{fig_sem.png}
	\caption[Theoretical model illustrating the anticipated relationships between overstory treatments, environmental variables and species groups.]
  {Theoretical model illustrating the anticipated relationships between overstory treatments, coarse woody debris volume, canopy openness, litter depth,
   salamander occupancy, ground beetle occupancy and springtail biomass in the Portneuf Wildlife Reserve, Quebec, Canada. 
   Each arrow indicates the direction of a potential effect, from an explanatory variable to a response variable in the structural equation model (SEM). 
   Note that the large and small carabid categories are based on salamander gape size.}
	\label{fig:SEM}
\end{figure} 

\clearpage


\section*{Results}
\label{sec:results1}
\phantomsection\addcontentsline{toc}{section}{\nameref{sec:results1}}

\subsection*{Effect of harvesting on environmental variables}
\label{subsec:ResEnv}
\phantomsection\addcontentsline{toc}{subsection}{\nameref{subsec:ResEnv}} 

Model diagnostics indicated that the chains were of sufficient length, as the Brooks-Gelman-Rubin statistic was below 1.04. 
Trace plots revealed that all chains had converged towards similar values, and none of the ratios of MCMC error to posterior standard deviation exceeded 5\%, suggesting that chain length was sufficient.

Environmental variables usually differed between overstory treatments and control conditions. 
Clear-cutting treatments had lower CWD volumes compared to partial cutting, \hl{as the confidence interval exclude 0} (95\% CI: [-1.15, -0.43]) (Figure \ref{fig:envar} A, Table \ref{tab:overstoryenvar}). 
However, neither logging treatment showed a significant difference from the control. 
Canopy openness was significantly higher in both the partial cut (95\% CI: [1.97, 11.02]) and clear-cut treatments (95\% CI: [51.39, 77.06]), when compared 
to the control sites, with the clear-cuts being more open than the partial cuts (95\% CI: [44.61, 70.76], Figure \ref{fig:envar} B, Table \ref{tab:overstoryenvar}). 
In contrast, litter depth was lower in both the partial cut (95\% CI: [-2.44, -0.65]) and clear-cut treatments (95\% CI: [-4.28, -2.50]) compared to the controls, 
with the litter being shallower in the clear-cuts than in the partial cuts (95\% CI: [-2.57, -1.12], Figure \ref{fig:envar} C, Table \ref{tab:overstoryenvar}).


\vspace{10pt}

\begin{figure}[ht]
  \centering
  \includegraphics[scale=0.23]{fig_envar2.png}
  \caption[Environmental variables with a potential effect on soil fauna within two different overstory treatments and control.]
  {Environmental variables with a potential effect on soil fauna within two different overstory treatments and control 
  during the summer 2022 in the Portneuf Wildlife Reserve, Quebec, Canada. Error bars denote 95\% credible intervals around estimates.}
  \label{fig:envar}
\end{figure}

\begin{table}[ht]
  \centering
  \caption[Contrasts between overstory treatments for environmental variables that could affect habitat selection of fauna on the forest soil.]
  {Contrasts between overstory treatments for environmental variables that could affect habitat use of fauna on the forest soil during the summer 2022 in the Portneuf Wildlife Reserve,
  Quebec, Canada.}
  \label{tab:overstoryenvar}
  \begin{tabular}{lllll} 
      \hline
      &&&&95\% Bayesian \\
      Variable&Unit& Comparison & Estimate &  credible interval \\ [0.5ex] 
      \hline
      Coarse woody debris &m\up{3}& Partial vs control & \hspace{1mm}0.02 & [-1.01, 1.06] \\ 
                 && Clear vs control  & -0.77 & [-1.79, 0.23] \\ 
                          && Clear vs partial  & -0.79 & [-1.15, -0.43] \\
      Canopy openness     &\%& Partial vs control & \hspace{1mm}6.49 & [1.97, 11.02] \\ 
                      && Clear vs control  & \hspace{1mm}64.19 & [51.39, 77.06] \\ 
                          && Clear vs partial  & \hspace{1mm}57.69 & [44.61, 70.76] \\ 
      Litter depth        &cm& Partial vs control & -1.54 & [-2.44, -0.65] \\ 
                      && Clear vs control  & -3.39 & [-4.28, -2.50] \\ 
                          && Clear vs partial  & -1.85 & [-2.57, -1.12] \\       
      \hline
      \multicolumn{5}{l}{\textbf{Note:} Estimates from Bayesian SEM are presented in terms of posterior mean with 95\%} \\
      \multicolumn{5}{l}{credible intervals, where an interval excluding 0 indicates a difference between groups.} \\
  \end{tabular}
\end{table}

\pagebreak


\subsection*{Soil fauna}
\label{subsec:taxa}
\phantomsection\addcontentsline{toc}{subsection}{\nameref{subsec:taxa}} 

\vspace{10pt}

Across the sixty sampling units, Red-backed salamanders were detected in 0, 1, 11, and 11 sites during the May, June, July, and August 2022 surveys, respectively. 
Over the same periods, small ground beetles were detected in 2, 5, 12, and 2 sites while large ground beetles were detected in 16, 30, 35, and 21 sites.
A total of 30 ground beetle species were identified in the pitfall traps and under the coverboards (Table \ref{tab:carabid}). 
We collected 468 springtails representing 12 families, with 219 springtails collected from partial cut treatments, 131 from clear-cuts, and 118 from the control areas (Table \ref{tab:springtail}). 
The average springtail biomass collected per overstory treatment was 24.3 $\mu$g (SD = 18.2 $\mu$g), 56.8 $\mu$g (SD = 78.0 $\mu$g), and 31.1 $\mu$g (SD = 52.8 $\mu$g) in the partial cut, clear-cut, and control sites, respectively. 

Occupancy and biomass generally did not vary significantly across the cutting treatments \hl{when using the 95\% CI criterion. 
However,} salamander occupancy probability was marginally lower in sites subjected to clear-cutting compared to those with partial cutting (90\% CI: [-0.74, -0.07], Figure \ref{fig:pcin}, Table \ref{tab:overstorysp}). 
The occupancy of salamanders in clearcut and partial cut sites did not differ from the control sites. 
Occupancy probability for each carabid group and the springtail biomass did not vary between the overstory treatments and control sites (Table \ref{tab:overstorysp}). 
Furthermore, the occupancy probabilities of small ground beetles (salamander prey) did not vary with the presence of salamanders. 
Similarly, the biomass of springtails did not vary with the presence of salamanders, large ground beetles, or small ground beetles (Table \ref{tab:overstorysp}).

\begin{figure}[h!]
	\centering
	\includegraphics[scale=0.60]{fig_SEM_res.png}
	\caption[Results from structural equation modeling analysis revealing effects of overstory treatments on coarse woody debris volume,
  canopy openness, litter depth, salamander occupancy, ground beetle occupancy, and springtail biomass.]
  {Results from SEM analysis showing effects of overstory treatments on CWD, 
  canopy openness, litter depth, salamander occupancy, ground beetle occupancy, and springtail biomass in the Portneuf Wildlife Reserve, 
  Quebec, Canada. Bold arrows represent significant effects while dotted lines indicate no discernible effects. 
  Estimates marked with one asterisk (*) indicate a 90\% credible interval (CI) excluding 0, while estimates marked with two asterisks (**) indicate a 95\% CI excluding 0. 
  Note that the large and small carabid categories are based on salamander gape size.}
	\label{fig:SEMres}
\end{figure}  

\vspace{10pt}

\begin{table}[h!]
  \centering
  \caption[Contrasts between overstory treatments for salamander occupancy, ground beetle occupancy, and springtail biomass.]
  {Contrasts between overstory treatments for salamander occupancy, ground beetle occupancy, and springtail biomass, during the summer 2022 in the Portneuf Wildlife Reserve, Quebec, Canada. 
  This table also shows the estimated effect of interactions between different groups: salamander presence on small and large ground beetles and the effects of the presence of salamanders and both ground beetle groups on springtail biomass.}
  \label{tab:overstorysp}
  \begin{tabular}{lllll} 
      \hline
      &&&&95\% Bayesian \\
      Variable&Unit& Comparison & Estimate &  credible interval \\ [0.5ex] 
      \hline     
      Salamander           &probability& Partial vs control & \hspace{1mm}0.07 & [-0.29, 0.45] \\ 
      occupancy       && Clear vs control  & -0.38 & [-0.75, 0.11] \\ 
                          && Clear vs partial  & -0.45 & [-0.78, 0.01]$^{a}$ \\       
      Carabid$_{large}$ &probability& Partial vs control & -0.12 & [-0.35, 0.15] \\
      occupancy       && Clear vs control  & -0.06 & [-0.29, 0.20] \\ 
                          && Clear vs partial  & \hspace{1mm}0.06 & [-0.19, 0.30] \\ 
      Carabid$_{small}$    &logit& Partial vs control & \hspace{1mm}3.31 & [-10.12, 17.72] \\
      occupancy             && Clear vs control  & \hspace{1mm}10.19 & [-4.15, 24.45] \\ 
                          && Clear vs partial  & \hspace{1mm}6.88 & [-12.81, 23.42] \\  
                          && Salamander        & -2.20 & [-17.15, 16.59] \\  
      Springtail          &$\mu$g& Partial vs control & \hspace{1mm}8.11 & [-9.38, 25.40] \\
      biomass             && Clear vs control  & \hspace{1mm}2.11 & [-13.98, 18.11] \\ 
                          && Clear vs partial  & -6.00 & [-29.09, 17.26] \\  
                          && Salamander        & \hspace{1mm}6.80 & [-10.43, 23.26] \\ 
                          && Carabid$_{large}$      & \hspace{1mm}0.56 & [-16.75, 17.66] \\ 
                          && Carabid$_{small}$      & \hspace{1mm}7.62 & [-8.93, 24.09] \\ 
      \hline
      \multicolumn{5}{l}{\textbf{Note:} Estimates from Bayesian SEM are presented in terms of posterior mean with 95\%} \\
      \multicolumn{5}{l}{credible intervals, where an interval excluding 0 indicates a difference between groups.} \\
      \multicolumn{5}{l}{$^{a}$ \hl{A marginal difference was detected based on the 90\% Bayesian credible interval} } \\
      \multicolumn{5}{l}{\hl{excluding 0 [-0.74, -0.07].}}
  \end{tabular}
\end{table}

\pagebreak

\begin{figure}[h!]
  \centering
  \includegraphics[scale=0.55]{fig_pcin.png}
  \caption[Occupancy probability of salamanders under overstory treatments]
  {Occupancy probability of salamanders within two overstory treatments and controls during the summer 2022 in the Portneuf Wildlife Reserve device, Quebec, Canada. 
  Error bars denote 95\% credible intervals around estimates.}
  \label{fig:pcin}
\end{figure}

\vspace{10pt}

\clearpage

We did not observe significant impacts of CWD volume and precipitation level on salamander detection probabilities. 
However, the precipitation level had a positive effect on detection probability for small ground beetles (95\% CI: [0.59, 1.77]) and large ground beetles (95\% CI: [0.70, 3.23]) (Table \ref{tab:detection}). 
Detection probabilities of ground beetles did not vary with the volume of CWD.

\begin{table}[ht]
  \centering
  \caption[Estimated effects of the volume of coarse woody debris and precipitation level on detection probabilities of salamanders, large ground beetles, and small ground beetles.]
  {Estimated effects of the volume of coarse woody debris and precipitation level on detection probabilities of salamanders, large ground beetles, and small ground beetles, during the summer 2022 in the Portneuf Wildlife Reserve,  Quebec, Canada.}
  \label{tab:detection}
  \begin{tabular}{lllll} 
      \hline
      &&&95\% Bayesian \\
      Variable & Taxa & Estimate &  credible interval \\ [0.5ex] 
      \hline      
      Precipitation       & Salamander              & \hspace{1mm}0.11 & [-0.83, 1.06] \\ 
                          & Carabid$_{large}$  & \hspace{1mm}1.17 & [0.59, 1.77] \\ 
                          & Carabid$_{small}$        & \hspace{1mm}1.87 & [0.70, 3.23] \\  
      \hline      
      Coarse woody debris & Salamander              & -0.59 & [-1.39, 0.12] \\ 
                          & Carabid$_{large}$  & \hspace{1mm}0.06 & [-0.26, 0.38] \\ 
                          & Carabid$_{small}$        & \hspace{1mm}0.27 & [-0.74, 1.37] \\   

      \hline
      \multicolumn{4}{l}{\textbf{Note:} Estimates from Bayesian SEM are presented in terms of posterior mean} \\
      \multicolumn{4}{l}{with 95\% credible intervals, where an interval excluding 0 indicates a difference} \\
      \multicolumn{4}{l}{between groups.} \\
  \end{tabular}
\end{table}

\clearpage


\section*{Discussion}
\label{sec:discu1}
\phantomsection\addcontentsline{toc}{section}{\nameref{sec:discu1}}

\subsection*{Effect of harvesting type on environmental variables}
\label{disc:env_var}
\phantomsection\addcontentsline{toc}{subsection}{\nameref{disc:env_var}} 

Our first hypothesis assumed that environmental variables key to soil fauna would vary depending on the type of logging, with a higher level of disturbance for the most intensive treatments. 
Overall, our results support this hypothesis, as changes in CWD, canopy openness, and litter depth were more pronounced in clearcut treatments, with partial cuts showing intermediate effects. 

We predicted that the most intense treatments (clear-cutting followed by site preparation) would decrease CWD volume. 
Our results showed the partial cuts maintain a higher volume of CWD compared to clear-cuts, aligning with similar studies \citep{Nolet2018Comparingeffects,Ochs2022Responseterrestrial}. 
However, the lack of a significant difference between control sites and clear-cuts contrasts with other research that reports a significant reduction in CWD volume following clear-cutting \citep{Farnell2020effectsvariable}. 
This difference may come from the variable amount of CWD left on the site after harvesting, which could temporarily increase the availability of dead wood in the habitat during the initial years post-harvest \citep{McCarthy1994Distributionabundance,Etcheverry2005Responsesmall}. 
For example, \cite{Ochs2022Responseterrestrial} found that the volume of CWD was higher in clear-cuts than in control sites within the first 0–3 years following harvesting. 
However, this difference disappeared 4–6 years and 7–11 years after harvesting. 
\hl{This phenomenon is even more likely to occur in mixed stands, such as those in our study, where the crowns of tolerant hardwoods left on the ground after harvesting typically have large branches, unlike coniferous forests where branches are generally thinner. }
In the short term, the presence of CWD promotes soil fauna conservation by maintaining a microclimate and nutrient source favorable to amphibians and arthropods \citep{spotilaRoleTemperatureWater1972,Huhta1976Effectsclearcutting,Seibold2021contributioninsects,Ochs2022Responseterrestrial}. 
Retaining deadwood is often recommended as a forest management strategy to mitigate the negative impacts of logging on soil fauna \citep{McKenny2006Effectsstructural,Raymond-Leonard2020Deadwood}. 
However, the benefits of CWD can decrease over time as \hl{the pulse of CWD generated by harvesting} decays, leading to population declines \hl{for sensitive species} over the medium and long term \citep{Ochs2022Responseterrestrial}. 
The quality of CWD for soil fauna also depends on factors such as decomposition stage, tree species, and size \citep{Bunnell2010woodbiodiversity}. 
Freshly felled wood often shows low decomposition levels, providing few moist refuges suitable for some species, particularly amphibians \citep{Petranka1994Effectstimber,Morneault2004effectshelterwood,Owens2008Amphibianreptile,Otto2013Amphibianresponse}. 
\cite{Petranka1994Effectstimber} suggest that retaining dead wood can be more beneficial for salamanders in dry environments than in moist ones. 
The authors found that salamander abundance in dry plots tended to increase with the availability of CWD, whereas in wet plots, salamander capture rates remained unaffected by the amount of CWD. 
This assumes that CWD may have a greater impact in clear-cuts, where conditions are more challenging for species sensitive to temperature and humidity fluctuations.
 
Regarding canopy openness, we expected that the most intensive treatments would lead to greater canopy opening. 
Our results confirmed this, as both harvesting treatments significantly increased canopy openness compared to control sites. 
Our results indicate a substantial reduction in forest cover in clear-cut treatments compared to partial cuts, consistent with observations in similar studies \citep{Nolet2018Comparingeffects,Mazerolle2021Woodlandsalamander}. 
Despite the removal of 50\% of \hl{merchantable basal area}, partial cuts preserve a large proportion of the canopy and therefore can potentially maintain environmental conditions similar to those in mature forests. 
The substantial canopy opening in clear-cut treatments leads to major changes at the ground level, such as increased solar radiation and wind exposure, leading to higher temperatures, reduced humidity, and litter desiccation \citep{Keenan1993ecologicaleffects,Chen1999MicroclimateForest,Lindo2003Microbialbiomass,Brooks2008Forestfloor}. 
These changes cause significant shifts in soil fauna, affecting their composition, abundance, and species richness \citep{Staab2023Insectdecline}. 
Species adapted to open and dry habitats tend to thrive, often at the cost of those requiring cool and moist conditions \citep{Niemela2007effectsforestry,Ochs2022Responseterrestrial,Staab2023Insectdecline}.

We also predicted a decrease in litter depth with increasing harvest intensity. 
Our results confirmed this prediction, showing that logging treatments negatively impact litter depth, with more intensive treatments leading to greater reductions. 
These findings are consistent with observations reported in several other studies \citep{Marshall2000Impactsforest,Mazerolle2021Woodlandsalamander}. 
The decrease in litter depth may result from reduced leaf input following stem removal or from an accelerated decomposition rate caused by canopy opening, 
which leads to warmer conditions and increased precipitation interception \citep{Fierer2005LitterQuality,Butenschoen2011Interactiveeffects,Ameray2021Forestcarbon}. 
Site preparation in clear-cut treatments may also contribute to a reduction in organic matter compared to other methods \citep{Prevost1992Effetsscarifiage}. 
The loss of litter can have significant consequences for soil fauna by eliminating a cool, moist microhabitat \citep{spotilaRoleTemperatureWater1972,groverInfluenceCoverMoisture1998a,Niemela2007effectsforestry}. 
This change increases the risk of desiccation and predation on amphibians and arthropods, reducing the time they can spend on the surface for feeding and reproduction \citep{deMaynadier1995relationshipforest,koivula.LeafLitterSmallscale1999,Walton2013Topdownregulation}. 
The loss of litter also represents a reduction in food resources for decomposer communities, which has long-term impacts on carbon and nutrient cycling in forest ecosystems \citep{Handa2014Consequencesbiodiversity}. 


\subsection*{Soil fauna}
\label{disc:soil_fauna}
\phantomsection\addcontentsline{toc}{subsection}{\nameref{disc:soil_fauna}} 

\subsubsection*{Logging effects on the different species groups}
\label{disc:logging_effects}

The hypothesis related to our second objective states that harvest treatments could alter directly habitat use by soil fauna. 
We suggested that logging would negatively affect the occupancy probability of red-backed salamanders. 
\hl{In our study, salamander occupancy probability was marginally higher in partial cuts compared to clear-cut sites, although this effect was weak (at 90\% confidence level). 
Additionally, salamander occupancy in either partial or clear-cut treatments did not differ from control sites.}
Clear-cutting is often perceived as detrimental to salamander populations, but its impact varies greatly across studies \citep{Hocking2013Effectsexperimental,Chaudhary2016Impactforest}. 
Some studies have reported declines in salamander abundance 4-6 years after clear-cutting \citep{Petranka1993Effectstimber,Herbeck1999PlethodontidSalamander,Grialou2000effectsforest,Macneil2014Effectstimber}  
while others, including ours, found no significant difference between clear-cuts and control sites \citep{Renken2004EffectsForest,Mazerolle2021Woodlandsalamander}. 
Our findings also align with the literature regarding the absence of effects of partial harvesting on salamanders five years after logging \citep{McKenny2006Effectsstructural,Mazerolle2021Woodlandsalamander,Ochs2022Responseterrestrial}. 
However, other authors have observed decreases in salamander abundance in partial cuts during the early years following logging, although these effects tend to diminish after five years \citep{Harpole1999Effectsseven,Knapp2003Initialeffects,Morneault2004effectshelterwood}.  
We believe the marginal difference observed between partial and clear-cuts may be explained by the ability of partial cuts to preserve a more suitable habitat for salamanders. 
This could be attributed to the retention of a significant amount of CWD in partial cuts, as this is an essential resource for salamanders during dry conditions \citep{Nolet2018Comparingeffects,Peterman2014Spatialvariation,Achat2015Quantifyingconsequences,Peele2017Effectswoody}.  
CWD and other microstructures allow salamanders to remain on the surface for feeding and reproduction. 
According to our results, the harvesting treatments had similar effects on salamander occupancy probability and the volume of CWD. 
\hl{Previous studies have found that} the effects of logging on salamanders vary over time. 
For example, \cite{Ochs2022Responseterrestrial} found no significant effects of clear-cutting on salamander abundance during the first 3 years post-harvest but reported declines between 4 and 6 years, with no recovery observed up to 11 years later. 
However, other studies have observed declines immediately following harvesting \citep{deMaynadier1995relationshipforest,Macneil2014Effectstimber}. 
Short-term responses may depend on the availability of CWD, which decomposes over time and affects the ability of salamanders to persist in clear-cut areas \citep{Ochs2022Responseterrestrial}. 
The retention of forest canopy in partial cuts and the rapid regrowth of understory vegetation associated with this treatment type may also explain the differences in impacts between the two harvesting methods \citep{Raybuck2015silviculturalpractices}. 
A closed canopy helps to maintain cool and moist microclimates. 
These conditions are critical for salamander cutaneous respiration and reduce desiccation risks \citep{Homyack2011Energeticssurface}. 
Previous studies have shown that retaining forest cover supports higher salamander densities \citep{Hocking2013Effectsexperimental,Harper2015Impactforestry,Mahoney2016Woodlandsalamander}. 
The presence of forest litter may contribute to the differences in occupancy probability between partial and clear-cuts \citep{tilghmanMetaanalysisEffectsCanopy2012}. 
For instance, \cite{Ash1997DisappearanceReturn} reported that red-backed salamander abundance returned to pre-harvest levels 4-6 years after clear-cutting, attributing this recovery to the replenishment of forest litter. 
However, salamander recovery may take decades in some cases \citep{Homyack2013Effectsrepeatedstand,Ochs2022Responseterrestrial}. 
Litter depth might explain the observed differences in occupancy probability, as our study found significantly deeper litter in partial cuts compared to clear-cuts. 

\hl{We also hypothesized that} logging would directly reduce the occupancy probability of ground beetles. 
Our findings suggest that forest harvesting treatments do not directly affect the occupancy probability of large ground beetles and small ground beetles. 
Canopy openness has a positive effect on ground beetle species richness and taxonomic composition \citep{Halme1993Carabidbeetles,Heliola2001Distributioncarabid,Koivula2002Alternativeharvesting}. 
This effect can be attributed to the creation of new habitats through clear-cutting and the ability of ground beetles to move quickly across landscapes \citep{Niemela2007effectsforestry}. 
Open, warmer, and drier habitats, such as those created by clear-cuts, are favored by many ground beetle species \citep{Niemela2007effectsforestry}. 
In contrast, closed-canopy forests provide less favorable conditions for most ground beetles, with only a few species adapted to cool and shaded sites \citep{Niemela1993Effectsclearcut,koivulaBorealCarabidbeetleColeoptera2002a}. 
Ground beetles are often categorized based on their habitat preferences, typically as forest specialists, open-habitat specialists, or generalist species \citep{Niemela2007effectsforestry}. 
Species associated with open habitats are intolerant of canopy closure and tend to disappear 20 to 30 years after harvesting, as the canopy regenerates \citep{Niemela1996importancesmallscale,Koivula2002Alternativeharvesting}. 
Conversely, forest specialists depend on specific forest attributes, such as microclimatic conditions or the presence of structural elements like CWD \citep{Niemela1996importancesmallscale,Heliola2001Distributioncarabid,Koivula2002Alternativeharvesting,Work2004Standcomposition}. 
The availability of forest litter is another factor influencing the abundance and distribution of ground beetles \citep{koivula.LeafLitterSmallscale1999,Heliola2001Distributioncarabid,Magura2005ImpactsLeaflitter}. 
Forest species may also colonize open habitats near old stands, as is often observed in clear-cut areas \citep{Spence1996Northernforestry,Koivula2002Alternativeharvesting}. 
Moreover, certain forest species adapted to natural disturbances may persist in open environments \citep{Niemela2007effectsforestry}. 
Although the occupancy probability of ground beetles remains similar after forest harvesting, the species composition and richness in harvested areas may change. 
Some studies suggest that species responses to forest harvesting vary depending on their size. 
For instance, \cite{Nolte2019Habitatspecialization} found that larger ground beetles are more likely to decline after forest harvesting compared to smaller species. 
However, in our study, we did not observe significant differences in occupancy probability between large and small ground beetles. 
\hl{It might be interesting in future studies to investigate if occupancy probability of ground beetles differs between logging treatments at the species or genus level.}

Regarding springtails, we supposed that logging treatments would directly lead to a reduction in springtail biomass. 
Our results \hl{indicate} that springtail biomass was not significantly affected by harvesting. 
This result diverges from previous observations that increased disturbance intensity leads to a decline in springtail abundance \citep{Lindo2004Forestfloor,Laigle2021Directindirect,Kudrin2023metaanalysiseffects}. 
Nonetheless, the absence of significant differences between partial cuts and control sites concurs with findings reported in the literature \citep{Kudrin2023metaanalysiseffects}. 
Declines in springtail populations following logging are often attributed to soil compaction, loss of microhabitats such as CWD, 
disappearance of organic layers, as well as reduced moisture and nutrient availability \citep{Bird1986Effectwholetree,Baath1995Microbialcommunity,Lindo2004Forestfloor,rousseauForestFloorMesofauna2018}. 
Studies reporting a decline in springtails following forest harvesting were conducted in forests dominated by conifers, where soils are sometimes more limited in nutrients. 
However, \cite{Kudrin2023metaanalysiseffects} observed a decrease in springtail abundance in coniferous forests but found no significant effect in mixed or deciduous forests. 
\hl{Therefore, our results could be due to our study being conducted in a temperate mixed forest}, where springtails living in more nutrient-rich soils could be less affected by environmental disturbances \citep{chauvatChangesSoilFaunal2011a}. 
Springtail responses to disturbances can also vary depending on species \citep{raymond-leonardSpringtailCommunityStructure2018a}. 
\hl{For example, hemiedaphic springtails, which are closely tied to mature forest soils, tend to be the springtail group most impacted by logging activities} \citep{Laigle2021Directindirect}. 
In contrast, epiedaphic species are better adapted to xeric habitats and higher disturbance levels \citep{Makkonen2011Traitsexplain,rousseauWoodyBiomassRemoval2019}. 
However, even these species are sensitive to reductions in litter, an important food resource for them \citep{rousseauForestFloorMesofauna2018}. 
\hl{It would have been relevant to explore this aspect of the research further, but the limited availability of data was a constraint. }
In our study, all species from the litter and the soil were pooled to measure the biomass per sampling unit. 
Thus, logging may affect certain species or guilds without this effect being detectable among springtails \citep{Addison1998Responsesoil}. 


\subsubsection*{Relations between taxa}
\label{disc:relations_between_taxa}

Our second hypothesis suggests that harvesting treatments indirectly influence soil fauna through the trophic network, with effects propagating from predators to prey. 
We predicted that logging would affect the occupancy of salamanders and large ground beetles, which subsequently alters small ground beetles occupancy and finally the springtail biomass. 
\hl{In theory,} the impact of forest disturbances on population trends can be closely linked to trophic level \citep{Gotelli2006FoodWebModels,Nolte2019Habitatspecialization}. 
Additionally, salamander occupancy seems to be associated with complex dynamics within forest soil food webs \citep{baileyEstimatingSiteOccupancy2004,Walton2006Salamandersforestfloor,Rooney2000impactsalamander}. 
Declines in salamander populations can lead to shifts in soil community composition and alter ecosystem functions \citep{Hairston1987evolutioncompeting,Wyman1998Experimentalassessment,Rooney2000impactsalamander,Walton2005Contrastingeffects}. 
However, in our study, the effects of logging do not appear to extend through the trophic network, contrary to our initial hypothesis, as no indirect impacts were detected. 
Specifically, the occupancy of small ground beetles did not vary with the presence of red-backed salamanders. 
Similarly, springtail biomass did not vary with the occupancy probabilities of salamanders and ground beetles. 

Several studies suggested a complex relationship between red-backed salamanders and arthropod predators \citep{Gall2003BehavioralInteractions,Walton2006Salamandersforestfloor,Hickerson2018Behavioralinteractions}. 
The interaction between salamanders and ground beetles appears to vary across studies.  
For example, \cite{Gall2003BehavioralInteractions} and \cite{Ovaska1988Predatorybehavior} observed aggressive behaviors between the two taxa, whereas \cite{Hickerson2012Interactionsforestfloor} reported a positive effect of red-backed salamanders on carabid abundance. 
Other research found no influence of salamanders on the abundance of predators such as ground beetles, centipedes, and spiders \citep{Hocking2013Effectsexperimental}. 
However, the fact that salamanders and ground beetles share the same habitats, rely on similar resources, and that ground beetles may be part of the salamanders diet, strongly suggests a close ecological relationship between these two taxa \citep{Jaeger1980MicrohabitatsTerrestrial,loveiEcologyBehaviorGround1996}.

Regarding the relationship between salamanders and springtails, salamander presence can positively affect the density of detritivore mesofauna by reducing the number of springtail competitors or predators in their habitat \citep{Wyman1998Experimentalassessment,Rooney2000impactsalamander,Walton2005Contrastingeffects,Walton2006Salamandersforestfloor}. 
Salamanders defend their habitat to preserve food resources and favorable attributes for their survival \citep{Gall2003BehavioralInteractions,Hickerson2004Behavioralinteractions,Hickerson2012Interactionsforestfloor}. 
They also prioritize the most nutritious prey and become more selective as the density of their preferred prey increases \citep{Jaeger1981Foragingtactics,Jaeger1982ForagingTactics}. 
Consequently, salamander presence reduces macrofaunal detritivores and intermediate predators, favoring an increase in microbivorous and detritivorous mesofauna \citep{Rooney2000impactsalamander,Walton2005Contrastingeffects,Walton2006Salamandersforestfloor}. 
Conversely, in environments with lower macrofauna densities, salamanders tend to have a negative impact on springtails due to increased predation on the mesofauna. 
For instance, \cite{Walton2006Salamandersforestfloor} reported that salamander presence was associated with an increase in springtails, but with a decrease in isopods, millipedes, and pseudoscorpions. 
However, salamanders had a negative effect on springtails in environments with initially low macroinvertebrate density. 
This outcome aligns with other studies that have also reported a negative impact of salamanders on springtail density \citep{Hickerson2017Easternredbacked}. 
Lastly, some studies found no effect of salamanders on mesofauna or on organic matter decomposition, which is consistent with our observations \citep{Hocking2013Effectsexperimental,Hocking2014Effectsredbacked}.

Salamander effects on invertebrate communities appear to partly depend on community composition and habitat heterogeneity \citep{MichaelWalton2005Salamandersforestfloor,Walton2006Salamandersforestfloor,Walton2013Topdownregulation,Best2014trophicrole}. 
For example, \cite{Walton2013Topdownregulation} observed that thick litter increases the number of invertebrate predators and reduces salamander ability to exclude competitors, leading to higher overall predation on mesofauna. 
However, our results do not support these findings. 
Despite significant differences in litter depth across our different logging treatments, we observed no effects of salamanders on ground beetle occupancy or springtail biomass.


\section*{Conclusion}
\label{sec:conclu1}
\phantomsection\addcontentsline{toc}{section}{\nameref{sec:conclu1}}

Our study confirmed that environmental variables favorable to habitat use by our species fluctuate based on the intensity of forest harvesting, five years after the interventions. 
Overall, changes in environmental variables increased with the level of disturbance associated with forest harvesting. 
Clear-cuts generally showed the largest canopy openings and the shallowest litter depths, followed by partial harvesting treatments. 
Additionally, the volume of CWD was lower in clear-cuts compared to partial cuts. 
Therefore, forest harvesting treatments reflected changes in environmental conditions. 
Our findings generally do not support the hypothesis that forest harvesting treatments influence the occupancy probabilities or biomass of the studied taxa, nor that these effects propagate through the trophic network from predators to prey. 
Our results did not show significant effects of harvesting treatments on the three species groups studied, whether directly or indirectly, 
except salamanders, where a weak direct effect on occupancy probability was detected (significant at 90\% credible intervals). 

\clearpage

\section*{Acknowledgements}
\label{sec:acknowl1}
\phantomsection\addcontentsline{toc}{section}{\nameref{sec:acknowl1}}

This project received financial support from the Ministère des Ressources naturelles et des Forêts and Conseil de recherches en sciences naturelles et en génie du Canada (CRSNG Alliance). 
We thank P. Raymond, and E. Champagne for their support through the DREAM project and meteorological data. 
We are also grateful to R. Dubé-Messier for her helpful contribution to the fieldwork and arthropod identification, and to K. Theriault, M.C. Martin and K. Després, who provided logistic support. 
We also appreciate the contributions of T.T. Work and E. Champagne for their constructive comments to improve this paper.

\section*{Conflict of interest}
\label{sec:conflict1}
\phantomsection\addcontentsline{toc}{section}{\nameref{sec:conflict1}}

None declared

% \section*{Author contributions}
% \label{sec:author1}
% \phantomsection\addcontentsline{toc}{section}{\nameref{sec:author1}}

\cleardoublepage

\begin{otherlanguage}{english}
\bibliographystyle{ecologyNewEN} % Style de citation en anglais
\bibliography{References}
\addcontentsline{toc}{section}{References}
\end{otherlanguage}
