\chapter{Direct and indirect effects of forest harvest on soil fauna cooccurence in assisted tree migration context}     % numéroté
\label{chapitre1-articles}    

William Devos$^1$, Mathieu Bouchard$^1$, Marc Mazerolle$^1$

%href{mailto:william.devos.1@ulaval.ca}
$^1$ Centre d'étude de la forêt, Département des sciences du bois \\ 
et de la forêt, Université Laval, Québec, QC G1V 0A6, Canada. \\ 

\clearpage

\section*{Résumé}
\label{sec:resume1}
\phantomsection\addcontentsline{toc}{section}{\nameref{sec:resume1}}

\begin{otherlanguage*}{french}
  <Résumé de l'article en français. Obligatoire.>

  \textbf{Mots-clés} : <ajouter des mots clés>
\end{otherlanguage*}

\clearpage

\section*{Abstract}
\label{sec:abstract1}
\phantomsection\addcontentsline{toc}{section}{\nameref{sec:abstract1}}

\begin{otherlanguage*}{english}
  <English abstract of the paper. Optional, but recommended.>

\textbf{Keywords}: <add some keywords> 
\end{otherlanguage*}

\cleardoublepage

\section*{Introduction}
\label{sec:intro1}
\phantomsection\addcontentsline{toc}{section}{\nameref{sec:intro1}}

%\defcitealias{keylist}{alias}


We integrate occupancy and structural equation models to assess the direct and indirect impacts of site preparation, in a assisted tree migration context, 
on soil fauna cooccurence, while accounting for environmental disturbance in the mixedwood forest.
We first hypothesized that forest harvesting associated with site preparation influences directly environmental variables such as litter depth, canopy openness, and coarse woody debris (CWD). 
We predict that larger forest cuts will result in decreased litter depth and CWD as a consequence of reduced accumulation of leaves and wood on the forest floor.
Conversely, more intensive forest cutting will increase canopy openness. 
Our second hypothesis suggested that forest harvesting treatments altered habitat use of soil fauna by a top-down effect (Figure \ref*{fig:SEM}). 
Thereby forest harvesting treatments modify habitat use at the top of the trophic chain first (salamanders, carabids of the salamanders competitor group) and lead to 
a cascading effect that subsequently influences lower trophic levels such as carabids of the salamander prey group, and finally, springtails.


\section*{Material and methods}
\label{sec:matmet1}
\phantomsection\addcontentsline{toc}{section}{\nameref{sec:matmet1}}

\subsection*{Study area}
\label{subsec:area}
\phantomsection\addcontentsline{toc}{subsection}{\nameref{subsec:area}}

\begin{otherlanguage*}{english}
  Our study was conduct within the Portneuf Wildlife Reserve in the Captial-Nationale administrative region (47°07'N, 72°24'W, Figure \ref{fig:area}) near Rivière-à-Pierre and Lac Amanites. 
  This area is located within the balsam fir (\textit{Abies balsamea})-yellow birch (\textit{Betula alleghaniensis}) bioclimatic domain \citep{saucierChapitreEcologieForestiere2009}.
  Tree cover also contain sugar maple (\textit{Acer saccharum}), red maple (\textit{Acer rubrum}), white spruce (\textit{Picea glauca}), black spruce (\textit{Picea mariana}), red spruce (\textit{Picea rubens}), white birch (\textit{Betula papyrifera}) and quaking aspen (\textit{Populus tremuloides}) \citep{olaBelowgroundCarbonStocks2024}. 
  The site lie on a deep glacial till as surface deposit with a moderately well-drained sandy loams soil \citep{CanadianSystemSoil1998}.
  The mean daily temperature is 4$^{o}$C for the 1981-2010 period of the nearest weather station (Lac aux sables) \citep{environmentcanadaCanadianClimateNormals2019}. 
  Based on the same report, the mean annual precipitation and snowfall are respectively 1133.2 mm and 230.3 cm.
  We used the assisted migration experimental system establish in 2018 by the Minister of Natural Resources and Forests to collect our data (\citealp{royoDesiredREgenerationAssisted2023}).
  This system is a factorial experimental design with a split-split-split-plots using 4 replications (complete random blocks). 
  Each whole block (200 m x 140 m) is occupied by an overstory treatment (clear-cut and partial-cut). 
  A cervid exclusion treatments (excluded and non-excluded) is used as a split plot and a competing vegetation treatments as a split-split plot. 
  Climatic analogs associated with three climate projections (current climate, mid-century 2050, and end of century 2080) 
  are used as a split-split-split plots with the seedlings of 9 species in mixed planting: black cherry (\textit{Prunus serotina}), northern red oak (\textit{Quercus rubra}), 
  northern white-cedar (\textit{Thuja occidentalis}), shagbark hickory, (\textit{Carya ovata}), sugar maple (\textit{Acer saccharum}), red pine (\textit{Pinus resinosa}), 
  red spruce (\textit{Picea rubens}), white spruce (\textit{Picea glauca}), and white pine (\textit{Pinus strobus}) \citep{royoDesiredREgenerationAssisted2023}.

\end{otherlanguage*}

\begin{figure}[ht!]
	\centering
	\includegraphics[scale=0.30]{fig_area.png}
	\caption[Study area in Portneuf Wildlife Reserve, Québec, Canada]{Study area in Portneuf Wildlife Reserve where the assisted migration experimental system is implemented (47°07'N, 72°24'W)}
	\label{fig:area}
	\end{figure}  



\subsection*{Sampling design}
\label{subsec:sampling}
\phantomsection\addcontentsline{toc}{subsection}{\nameref{subsec:sampling}}





We selected 60 sampling units (10 m $\times$ 7,5 m) to collect our data : six units per overstory treatment (clear-cut and partial-cut) with three control outside the blocks for four blocks (Figure \ref*{fig:blockSU}).
Controls were separated from the blocks per at least 10 meters to remove treatments effects.
Each sampling unit contained three methods to collect species data. 
Artificial cover boards were used to count red-backed salamanders \citep{hesedUncoveringSalamanderEcology2012,mazerolleWoodlandSalamanderPopulation2021a,mooreComparisonPopulationEastern2009c}, 
pitfall traps for the carabid monitoring \citep{spenceSamplingCarabidAssemblages1994a}, and soil cores with litter collection for springtails assessment \citep{rousseauForestFloorMesofauna2018}.
We conducted four sampling visits of five consecutive days each, namely in mid-May, mid-June, mid-July, and mid-August, during the summer 2022, to assesse the temporal effect. 
Blocks were randomly visited to reduce time of day and sampler's fatigue level effects.

\begin{figure}[ht!]
	\centering
	\includegraphics[scale=0.50]{fig_blockSU.png}
	\caption[Block and a sampling unit design]{
  Design of a block (left) and a sampling unit (right). 
  The block contain two overstory treatments : clear cut (square background), partial cut (dotted background). 
  Fifteen sampling units were used per block : six per overstory treatment and three controls (\textbf{c}) outside the block.
  Each sampling unit contained six artificial coverboards (stars) and one pitfall trap (crossed circle). Two soil cores (rhombus) were harvest per visits
  }
	\label{fig:blockSU}
	\end{figure}  

Artificial coverboards is commonly employed for salamander sampling and yield similar or superior results compared to other methods such as active searching \citep{hydeSamplingPlethodontidSalamanders2001,mooreComparisonPopulationEastern2009c}. 
This method helps reduce variability in the number and size of sampled ground objects \citep{hydeSamplingPlethodontidSalamanders2001}. 
Coverboards were made of spruce wood, measured 25 cm x 30 cm x 5 cm and were placed directly on the ground without litter underneath to maintain higher humidity level under the boards \citep{mazerolleWoodlandSalamanderPopulation2021a}. 
Six boards were set per sample and control units, resulting in a total of 360 coverboards.
They were positioned in two rows of three boards each, centered and oriented along the length of the sampling units.
Boards were spaced 2.5 m apart along the length of the sampling unit and 5 m apart along the width (Figure\ref{fig:blockSU}).
All boards were put outside during march 2022 to provide aging of the boards under natural conditions, thereby increasing their likelihood of being used by salamanders \citep{hedrickEffectsCoverboardAge2021,smithEffectsCoverBoard2014a}.
Throughout the sampling visits, cover boards were inspected once on the same day, and salamanders were counted without any manipulation.

Pitfall trapping is a passive sampling method used to assess the species richness of ground-dwelling invertebrates \citep{knappEffectPitfallTrap2012,kotzeFortyYearsCarabid2011a,loveiEcologyBehaviorGround1996}.
This capture method works particularly well for active arthropods moving on the ground, such as carabids \citep{baarsCatchesPitfallTraps1979,loveiEcologyBehaviorGround1996,spenceSamplingCarabidAssemblages1994a}. 
Pitfall traps were Multipher\up{\textregistered{}} traps and included a container with a diameter of 12.5 cm, a depth of 25 cm and a cover raised 4.5 cm above the trap 
to prevent debris and rain from filling the container \citep{bouchardBeetleCommunityResponse2016b,mooreEffectsTwoSilvicultural2004}.
They were equipped with a protective grid with a mesh size of 15 mm, limiting trap access to carabid-sized individuals and reducing the chances of predation by small mammals. 
Typically, a preserving liquid (such as propylene glycol or alcohol) is placed in the bottom of the container to preserve captured individuals. 
However, salamanders can have a width comparable to that of certain carabids, and therefore get captured and drown in the traps. 
For ethical reasons and considering the difficulties to limit access to salamanders without influencing the sampling of carabids, we decided to use dry pitfall traps without any preserving liquid \citep{luffFeaturesInfluencingEfficiency1975}. 
We also add wet sponges to the bottom of each container to maintain a suitable level of humidity for salamanders.
We centered one pitfall in each sampling and control units, resulting in a total of 60 pitfall traps (Figure\ref{fig:blockSU}). 
Traps were inserted in the soil at a depth allowing the container's opening to be juxtaposed with the soil surface. 
Outside the sampling visits, all pitfall traps were closed with adhesive tape around the opening to prevent individuals capture. 
On the first day of each visit, traps were opened and carabids captured were collected for each remaining day. 
Individuals were placed in a container with 20\% alcohol and identified at the species level afterwards.
Identification was conduct with a ZEISS SteREO Discovery.V12 binocular using \cite{larochelleManuelIdentificationCarabidae1976} identification keys.
Carabids were categorized in two groups as salamander prey or competitors based on the salamander gape size \citep{jaegerFoodLimitedResource1972,magliaModulationPreycaptureBehavior1995,magliaOntogenyFeedingEcology1996}.

Soil cores is commonly used to sample mesofauna in litter and different soil horizons \citep{chauvatChangesSoilFaunal2011a,farskaManagementIntensityAffects2014,pongeVerticalDistributionCollembola2000,salamonEffectsPlantDiversity2004,wuCompositionSpatiotemporalVariation2014}. 
Two soil cores with litter collection were harvested per sampling unit per visit using a soil sampling pedological probe. 
Cores were harvested inside the coverboards area to associate the presence of springtails with salamanders and carabids detection (Figure\ref{fig:blockSU}). 
Cores had a diameter of 5 cm, a depth of 5 cm and a 15 cm x 15 cm litter quadrat was collected above each soil sample \citep{rousseauForestFloorMesofauna2018}.
Both substrates were used to target mesofauna and obtain springtail communities directly related to the ecology of salamanders and carabids \citep{chauvatChangesSoilFaunal2011a,edwardsAssessmentPopulationsSoilinhabiting1991,raymond-leonardSpringtailCommunityStructure2018a,rousseauForestFloorMesofauna2018}.
Soil and litter from the same unit were pooled in Ziploc\up{\texttrademark{}} bags and stored in a cooler at $\pm$ 4°C \citep{chauvatChangesSoilFaunal2011a,rousseauForestFloorMesofauna2018}, providing 60 unit samples per visit.
Samples were placed in a Tullgren dry-funnel for springtails extraction within a maximum of 48 hours after the harvest \citealp{rousseauForestFloorMesofauna2018,rusekBiodiversityCollembolaTheir1998,wuCompositionSpatiotemporalVariation2014}). 
The extraction process lasted six days with a gradual temperature escalation (25°C to 50°C) \citep{raymond-leonardSpringtailCommunityStructure2018a}.
Springtails were preserved in 75\% alcohol \citep{wuCompositionSpatiotemporalVariation2014} prior to isolation from surrounding organisms, with subsequent identification at the family level.
Identification was done with a ZEISS SteREO Discovery.V12 binocular and a Leitz orthoplan phase-contrast fluorescent trinocular microscope by using \cite{bellingerChecklistCollembolaWorld1996} identification keys.
Springtails dry biomass per samples units was quantified with a Sartorius Cubis\up{\texttrademark{}} MSA3.6P-000-DM micro balance after being lyophilized with a Labconco FreeZone Bulk tray dryers 78060 series.




\subsection*{Environmental variables}
\label{subsec:EnvVar}
\phantomsection\addcontentsline{toc}{subsection}{\nameref{subsec:EnvVar}}

In each sample units, we measured several environmental variables that could affect occupancy probability, as it has an impact on habitat use by species.
Coarse woody debris (CWD) and litter depth play a crucial role in habitat use for salamanders, carabids and springtails as
they serve for protection, feeding and to maintain suitable temperature and humidity \citep{birdChangesSoilLitter2004,groverInfluenceCoverMoisture1998a,harmonEcologyCoarseWoody1986,koivula.LeafLitterSmallscale1999,mckennyEffectsStructuralComplexity2006,patrickEffectsExperimentalForestry2006a}. 
We used a 400 m\up2 plots centered inside every sampling units to estimate CWD per units (20 m $\times$  20 m)(\citealp{methotGuideInventaireEchantillonnage2014}). 
We only selected CWD with a basal diameter greater than or equal to nine centimeters and a length greater than or equal to one meter.
Subsequently, we measured basal diameters, apical diameters and CWD's length with a tree caliper. 
Segments of CWD outside the plots boundaries wasn't considered.
We employed \cite{fraverRefiningVolumeEstimates2007} conic–paraboloid formula to estimate the volume of each CWD :

\begin{equation}
  \text{Volume} = L/12 \times (5A_b + 5A_u + 2\sqrt{A_b \times A_u})
\end{equation}


Where $L$ is the length of logs, $A_b$ the basal area, and $A_u$ the apical area.
We measured litter depth next to each coverboards and estimated the mean depth per sampling units \citep{mazerolleWoodlandSalamanderPopulation2021a}. 
We assessed canopy openness, as it may influence species occupancy \citep{henneronForestPlantCommunity2017,koivulaBorealCarabidbeetleColeoptera2002a,kotzeFortyYearsCarabid2011a,messereForestFloorDistribution1998,tilghmanMetaanalysisEffectsCanopy2012}.
Measurements were conducted at the center of all sampling units using a spherical densiometer \citep{lemmonSphericalDensiometerEstimating1956} and were taken at 130 cm above the ground. 
We took four measurements per units, oriented toward each of the four cardinal points, and used the mean results to estimate the canopy openness per sampling units.

We used air temperature, air humidity and precipitation levels as variables who may affects species detection.
These variables fluctuate on a daily basis, affecting species activity and, consequently, the probability of detecting individuals. 
\citep{butterfieldCarabidLifeCycle1996,kotzeFortyYearsCarabid2011a,loveiEcologyBehaviorGround1996,odonnellPredictingVariationMicrohabitat2014a,spotilaRoleTemperatureWater1972}.
Two compact weather stations (Em50 Digital Decagon Data Logger, Part \#40800, Meter Group Inc., USA), were used during the summer 2022 inside both overstory treatments.
Those station were equipped with a probe measuring temperature, air humidity, and atmospheric pressure (1.30 m above the ground)(VP-4 Sensor (Temp/RH/Barometer), Part \#40023). 
Clear-cut treatments were also equipped with a a rain gauge to measure precipitation level. 
The temperature and humidity sensors were programmed for readings every 15 minutes. 
We used the means of both weather stations to get daily average measurements.

\subsection*{Statistical analyses}
\label{subsec:analyses}
\phantomsection\addcontentsline{toc}{subsection}{\nameref{subsec:analyses}} 

To assess the effects of overstory treatments on the site occupancy and environmental variables, we used a structural equation models (SEM) integrating occupancy and linear mixed models \citep{graceSpecificationStructuralEquation2010,josephIntegratingOccupancyModels2016,mackenzieOccupancyEstimationModeling2006a}.
SEM are used to analysed complex relationships among observed and latent variables and allow to investigate on an ecological process \citep{graceStructuralEquationModeling2008}.
Based on a theoretical framework, all studied variables are interconnected in a network where the effect of a disturbance on each variable can be estimated by taking into account 
the disturbance effects on other related variables \citep{graceStructuralEquationModeling2008}.
SEM use multiples equations to describe the relationships between variables and estimate parameters.
This method allow us to measure direct and indirect processes that influence species occurrence by considering interrelationships between treatments, taxa and environmental variables (Figure \ref*{fig:SEM}, \citealp{graceSpecificationStructuralEquation2010}).
\\

\begin{figure}[ht!]
	\centering
	\includegraphics[scale=0.55]{fig_sem3.png}
	\caption[Theoretical model illustrating the relationships between overstory treatments, environmental variables and species groups.]{Theoretical model illustrating the relationships between overstory treatments, environnementale variables, salamanders, carabids, and springtails. 
  Each arrow indicates the direction of a potential effect, from an explanatory variable to a response variable.}
	\label{fig:SEM}
	\end{figure}  

We integrated occupancy models to SEM to measured the effects of overstory treatments on salamanders and carabids occurrences. 
Occupancy models allow us to account for imperfect detection of species since salamanders and carabids are known to be cryptic species and therefore may be present but unobserved \citep{baileyEstimatingSiteOccupancy2004,spiersEstimatingSpeciesMisclassification2022}
Occupancy modeling uses repeated surveys to compute detection probabilities and estimate the species true occupancy \citep{mackenzieEstimatingSiteOccupancy2002,mazerolleMakingGreatLeaps2007}.
We estimated occupancy probabilities for salamanders and carabids form the salamanders competitor group at a site $i$ at a time $j$ with a Bernoulli distribution where the probability of success were estimated from the latent occupancy state $z$ and the detection $p$.
Latent occupancy states were estimated on a probability scale with a Bernoulli distribution considering species occupancy $\psi$ within each overstory treatments groups.
We included precipitation levels, CWD and a block random effect to estimate detection probability with $\beta$s and $\alpha$s specific to each taxa.
Air temperature and relative humidity were excluded from detection variables due to a high correlation with the precipitation level.

\begin{align}
z_{\text{Salamander}_i} &\sim 
\text{Bernoulli}(\psi_{\text{Salamander}_{\text{Group}_i}}) \nonumber \\
\text{logit}(p_{\text{Salamander}_{ij}}) &= 
\alpha_{0[\text{Salamander}]} + \alpha_{\text{CWD}[\text{Salamander}]} \times \text{CWD}_i + \\
&\alpha_{\text{Precipitation}[\text{Salamander}]} \times \text{Precipitation}_{ij} + \alpha_{\text{Block}[\text{Salamander}]_{\text{Block}_i}} \nonumber \\
\text{Salamander}_{ij} &\sim 
\text{Bernoulli}(z_{\text{Salamander}_i} \times p_{\text{Salamander}_{ij}}) \nonumber
\end{align} 

\begin{align}
  z_{\text{Carabid.comp}_i} &\sim 
  \text{Bernoulli}(\psi_{\text{Carabid.comp}_{\text{Group}_i}}) \nonumber \\
  \text{logit}(p_{\text{Carabid.comp}_{ij}}) &= 
  \alpha_{0[\text{Carabid.comp}]} + \alpha_{\text{CWD}[\text{Carabid.comp}]} \times \text{CWD}_i + \\
  &\alpha_{\text{Precipitation}[\text{Carabid.comp}]} \times \text{Precipitation}_{ij} + \alpha_{\text{Block}[\text{Carabid.comp}]_{\text{Block}_i}} \nonumber \\
  \text{Carabid.comp}_{ij} &\sim 
  \text{Bernoulli}(z_{\text{Carabid.comp}_i} \times p_{\text{Carabid.comp}_{ij}}) \nonumber
  \end{align}

We applied the same process to assess occupancy probabilities of carabids from the salamanders prey group except that we applied a logit scale to estimate $\psi$ 
and included overstory treatments and salamanders latent occupancy state $z_{Salamander}$ in the linear predictor of $\psi$.

\begin{align}
  \text{logit}(\psi_{\text{Carabid.prey}_i}) &= 
  \beta_{0[\text{Carabid.prey}]} + \beta_{z_{\text{Salamander}}[\text{Carabid.prey}]} \times z_{\text{Salamander}_i} + \nonumber \\
  &\beta_{\text{Cutpartial}[\text{Carabid.prey}]} \times \text{Cutpartial}_i + \beta_{\text{Cutclear}[\text{Carabid.prey}]} \times \text{Cutclear}_i \nonumber\\
  z_{\text{Carabid.prey}_i} &\sim 
  \text{Bernoulli}(\psi_{\text{Carabid.prey}_i}) \nonumber \\
  \text{logit}(p_{\text{Carabid.prey}_{ij}}) &= 
  \alpha_{0[\text{Carabid.prey}]} + \alpha_{\text{CWD}[\text{Carabid.prey}]} \times \text{CWD}_i +  \\
  &\alpha_{\text{Precipitation}[\text{Carabid.prey}]} \times \text{Precipitation}_{ij} + \alpha_{\text{Block}[\text{Carabid.prey}]_{\text{Block}_i}} \nonumber \\
  \text{Carabid.prey}_{ij} &\sim 
  \text{Bernoulli}(z_{\text{Carabid.prey}_i} \times p_{\text{Carabid.prey}_{ij}}) \nonumber
\end{align}

We used linear mixed models to assess how overstory treatments can affect environmental variables (i.e. CWD, Canopy, Litter) and springtails biomass. 
We used a Normal distribution to estimate posteriors and we included overstory treatments (i.e. Cutpartial, Cutclear) and a block random effect in linear predictors 
to estimate variables means. Linear predictor for springtails also accounted for latent occupancy states of salamanders and both carabids groups (i.e. $z_{Salamander}$, $z_{SCarabid.comp}$, $z_{Carabid.prey}$ ).
% ajouter les elements d/equation des collemboles.

\begin{align}
  \mu_{\text{Springtail}_i} &=
  \beta_{0[\text{Springtail}]} + \beta_{\text{Cutpartial}[\text{Springtail}]} \times \text{Cutpartial}_i + \nonumber\\
  &\beta_{\text{Cutclear}[Springtail]} \times \text{Cutclear}_i + \beta_{z_{\text{Salamander}}[\text{Springtail}]} \times z_{Salamander} + \\
  &\beta_{z_{\text{Carabid.prey}}[\text{Springtail}]} \times z_{Carabid.prey} + \beta_{z_{\text{Carabid.comp}}[\text{Springtail}]} \times z_{Carabid.comp} \nonumber\\
  &\alpha_{\text{Block}[\text{Springtail}]_{\text{Block}_i}} \nonumber
\end{align}

\begin{align}
  \mu_{\text{CWD}_i} &= 
  \beta_{0[\text{CWD}]} + \beta_{\text{Cutpartial}[\text{CWD}]} \times \text{Cutpartial}_{i} + \nonumber\\
  & \beta_{\text{Cutclear}[\text{CWD}]} \times \text{Cutclear}_{i} + \alpha_{\text{Block}[\text{CWD}]_{\text{block}_i}} \\
  \text{CWD}_{i} &\sim 
  \text{Normal} (\mu_{\text{CWD}_i}, \tau_{\text{CWD}_{\text{Group}_i}}) \nonumber 
\end{align}


\begin{align}
  \mu_{\text{Canopy}_i} &= 
  \beta_{0[\text{Canopy}]} + \beta_{\text{Cutpartial}[\text{Canopy}]} \times \text{Cutpartial}_{i} + \nonumber \\
  & \beta_{\text{Cutclear}[\text{Canopy}]} \times \text{Cutclear}_{i} + \alpha_{\text{Block}[\text{Canopy}]_{\text{block}_i}} \\
  \text{Canopy}_{i} &\sim 
  \text{Normal} (\mu_{\text{Canopy}_i}, \tau_{\text{Canopy}_{\text{Group}_i}}) \nonumber 
\end{align}

\begin{align}
  \mu_{\text{Litter}_i} &= 
  \beta_{0[\text{Litter}]} + \beta_{\text{Cutpartial}[\text{Litter}]} \times \text{Cutpartial}_{i} + \nonumber\\
  & \beta_{\text{Cutclear}[\text{Litter}]} \times \text{Cutclear}_{i} + \alpha_{\text{Block}[\text{Litter}]_{\text{block}_i}} \\
  \text{Litter}_{i} &\sim 
  \text{Normal} (\mu_{\text{Litter}_i}, \tau_{\text{Litter}_{\text{Group}_i}}) \nonumber 
\end{align}

For CWD, canopy and springtails estimations, we also allowed each overstory groups $j$ to have there own variances due to heteroscedasticity.

\begin{align}
  \tau_j &= \sigma^{-2} \\
  \sigma_j &\sim \text{Uniform}(0,150) \nonumber
\end{align}


Analysis was performed with a Bayesian approche and parameters were estimated using Markov chain Monte Carlo method (MCMC) with JAGS 4.3.0 include in the jagsUI package in R 4.3.1 \citep{lunnBUGSProjectEvolution2009,kellnerJagsUIWrapperRjags2024,rcoreteamLanguageEnvironmentStatistical2020}.
We applied non-informative priors to all parameters by using normal distribution $\text{Normal}(\mu = 0, \sigma^2 = 100$) and uniform distribution $\text{Uniform}(0,1)$ as $\beta$ estimates. 
We executed the model with five chains, 200,000 iterations each \citep{gelmanUnderstandingPredictiveInformation2014}. A burn-in period of 75,000 iterations was used and a thinning rate of 5 was applied. 
We verified the convergence of MCMC chains by examining trace plots, posterior density plots, and applying the Brooks-Gelman-Rubin statistic.

\clearpage



\section*{Results}
\label{sec:results1}
\phantomsection\addcontentsline{toc}{section}{\nameref{sec:results1}}

\subsection*{Environmental variables in overstory treatments}
\label{subsec:ResEnv}
\phantomsection\addcontentsline{toc}{subsection}{\nameref{subsec:ResEnv}} 

Environmental variables usually differ between forest cutting treatments and control conditions. We found that total cutting treatments have significantly less CWD compared to partial cutting (Figure \ref{fig:envar} A, Table\ref{tab:overstory}). 
Results indicate that canopy openness is significantly higher in total cutting treatments, followed by partial cutting treatments, when compared to control sites (Figure \ref{fig:envar} B, Table\ref{tab:overstory}). 
Conversely, sites with total cutting show the lowest litter depths, followed by partial cutting treatments, in contrast to control sites (Figure \ref{fig:envar} C, Table\ref{tab:overstory}).

\begin{figure}[ht]
  \centering
  \includegraphics[scale=0.50]{fig_envar.png}
  \caption[Overstory treatments effects on environmental variables]{ Environmental variables estimations within two different overstory treatments and control during the summer 2021 in the Portneuf Wildlife Reserve device, Quebec, Canada. 
  Error bars denote 95\% credible intervals around estimates.}
  \label{fig:envar}
\end{figure}


\begin{table}[ht]
  \centering
  \caption[Direct and indirect estimated effects of overstory treatments and species occupancy on environnementale variables and species groups]
  {Direct and indirect estimated effects of overstory treatments and species occupancy on environnementale variables and species groups.}
  \label{tab:overstory}
  \begin{tabular}{lllll} 
      \hline
      &&&95\% Bayesian \\
      Variable & Comparison & Estimate &  credible interval \\ [0.5ex] 
      \hline
      Coarse woody debris & Partial vs control & \hspace{1mm}0.02 & [-1.01, 1.06] \\ 
                          & Clear vs control  & -0.77 & [-1.79, 0.23] \\ 
                          & Clear vs partial  & -0.79 & [-1.15, -0.43] \\
      Canopy openness     & Partial vs control & \hspace{1mm}6,49 & [1.97, 11.02] \\ 
                          & Clear vs control  & \hspace{1mm}64.19 & [51.39, 77.06] \\ 
                          & Clear vs partial  & \hspace{1mm}57.69 & [44.61, 70.76] \\ 
      Litter depth        & Partial vs control & -1.54 & [-2.44, -0.65] \\ 
                          & Clear vs control  & -3.39 & [-4.28, -2.50] \\ 
                          & Clear vs partial  & -1.85 & [-2.57, -1.12] \\       
      Salamander          & Partial vs control & \hspace{1mm}0.07 & [-0.29, 0.45] \\ 
                          & Clear vs control  & -0.38 & [-0.75, 0.11] \\ 
                          & Clear vs partial  & -0.45 & [-0.74, -0.07]$^{a}$ \\       
      Carabid$_{competitor}$ & Partial vs control & -0.12 & [-0.35, 0.15] \\
                          & Clear vs control  & -0.06 & [-0.29, 0.20] \\ 
                          & Clear vs partial  & \hspace{1mm}0.06 & [-0.19, 0.30] \\ 
      Carabid$_{prey}$    & Partial vs control & \hspace{1mm}3.31 & [-10.12, 17.72] \\
                          & Clear vs control  & \hspace{1mm}10.19 & [-4.15, 24.45] \\ 
                          & Clear vs partial  & \hspace{1mm}6.88 & [-12.81, 23.42] \\  
                          & Salamander        & -2.20 & [-17.15, 16.59] \\  
      Springtail         & Partial vs control & \hspace{1mm}8.11 & [-9.38, 25.40] \\
                          & Clear vs control  & \hspace{1mm}2.11 & [-13.98, 18.11] \\ 
                          & Clear vs partial  & -6.00 & [-29.09, 17.26] \\  
                          & Salamander        & \hspace{1mm}6.80 & [-10.43, 23.26] \\ 
                          & Carabid$_{competitor}$      & \hspace{1mm}0.56 & [-16.75, 17.66] \\ 
                          & Carabid$_{prey}$      & \hspace{1mm}7.62 & [-8.93, 24.09] \\ 
      \hline
      \multicolumn{4}{l}{\textbf{Note:} Estimates from Bayesian SEM are presented in terms of mean} \\
      \multicolumn{4}{l}{number with 95\% credible intervals, where an interval excluding 0 indicates} \\
      \multicolumn{4}{l}{a difference between groups.} \\
      \multicolumn{4}{l}{$^{a}$Marginal difference based on 90\% Bayesian credible interval excluding 0}
  \end{tabular}
\end{table}


\subsection*{Soil fauna}
\label{subsec:taxa}
\phantomsection\addcontentsline{toc}{subsection}{\nameref{subsec:taxa}} 

During the summer of 2021, we observed a total of 31 salamanders: 17 individuals in the partial cut treatments, 6 in the clearcuts, and 8 in the control sites. We captured 189 carabids belonging to 30 species, with 49 carabids found in the partial cut treatments, 105 in the clearcuts, and 35 in the control sites. We collected 468 springtails representing 12 families, with 219 springtails collected from partial cut treatments, 131 from clearcuts, and 118 from the control areas. The biomass of springtails collected amounted to 1364 $\mu$g, 747 $\mu$g, and 292 $\mu$g in the partial cut treatments, clearcuts, and control sites, respectively.

\begin{figure}[ht]
  \centering
  \includegraphics[scale=0.60]{fig_pcin.png}
  \caption[Occupancy probability of salamander under differentes overstory treatments]{Occupancy probability of salamander within control sites, partial cut and clearcut treatments during the summer 2021 in the Portneuf Wildlife Reserve device, Quebec, Canada. 
  Error bars denote 95\% credible intervals around estimates.}
  \label{fig:pcin}
\end{figure}

The effects of overstory treatments on habitat selection by species groups were generally limitied.
The probability of salamander occupancy was marginally lower in sites subjected to clear-cutting compared to those with partial cutting (Figure \ref{fig:pcin}, Table \ref{tab:overstory}). However, no significant difference was observed between these two groups and the control sites. 
The probability of occupancy for both groups of carabids, as well as the biomass of springtails, did not vary significantly between the overstory treatments and control sites. 
Overall, the presence of salamanders did not significantly affect the probability of occupancy of carabids representing prey for salamanders and springtail biomass was not significantly affected by the presence of salamanders or carabids.
Precipitatation had a positive effects on both carabid groups detection (Table \ref{tab:detection}). 

\begin{table}[ht]
  \centering
  \caption[Estimated effects of coarse woody debris and precipitation on the detection of salamanders and both carabids groups]
  {Estimated effects of coarse woody debris and precipitation on the detection of salamanders and both carabids groups.}
  \label{tab:detection}
  \begin{tabular}{lllll} 
      \hline
      &&&95\% Bayesian \\
      Variable & Taxa & Estimate &  credible interval \\ [0.5ex] 
      \hline      
      Precipitation       & Salamander              & \hspace{1mm}0.11 & [-0.83, 1.06] \\ 
                          & Carabid$_{competitor}$  & \hspace{1mm}1.17 & [0.59, 1.77] \\ 
                          & Carabid$_{prey}$        & \hspace{1mm}1.87 & [0.70, 3.23] \\       
      Coarse woody debris & Salamander              & -0.59 & [-1.39, 0.12] \\ 
                          & Carabid$_{competitor}$  & \hspace{1mm}0.06 & [-0.26, 0.38] \\ 
                          & Carabid$_{prey}$        & \hspace{1mm}0.27 & [-0.74, 1.37] \\   

      \hline
      \multicolumn{4}{l}{\textbf{Note:} Estimates from Bayesian SEM are presented in terms of mean} \\
      \multicolumn{4}{l}{number with 95\% credible intervals, where an interval excluding 0 indicates} \\
      \multicolumn{4}{l}{a difference between groups.} \\
  \end{tabular}
\end{table}

The model diagnostics indicate that the chain lengths are sufficient, as the Brooks-Gelman-Rubin statistic is below 1.1.
Trace plot analysis reveals that all chains have converged towards similar values, and none of the ratios of MCMC error to posterior standard deviation exceed 5\%.

\clearpage

\section*{Discussion}
\label{sec:discu1}
\phantomsection\addcontentsline{toc}{section}{\nameref{sec:discu1}}

\section*{Conclusion}
\label{sec:conclu1}
\phantomsection\addcontentsline{toc}{section}{\nameref{sec:conclu1}}

\section*{Acknowledgements}
\label{sec:acknowl1}
\phantomsection\addcontentsline{toc}{section}{\nameref{sec:acknowl1}}

\section*{Conflict of interest}
\label{sec:conflict1}
\phantomsection\addcontentsline{toc}{section}{\nameref{sec:conflict1}}

None declared
spotilaRoleTemperatureWater1972
\section*{Author contributions}
\label{sec:author1}
\phantomsection\addcontentsline{toc}{section}{\nameref{sec:author1}}

\cleardoublepage

\begin{otherlanguage}{english}
\bibliographystyle{ecologyNewEN} % Style de citation en français
\bibliography{References}
\addcontentsline{toc}{section}{References}
\end{otherlanguage}
