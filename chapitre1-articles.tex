\chapter{Soil fauna occupancy responses to cutting treatments in assisted tree migration context}     % numéroté
\label{chap:SEM}    

William Devos$^1$, Mathieu Bouchard$^1$, Marc Mazerolle$^1$

%href{mailto:william.devos.1@ulaval.ca}
$^1$ Centre d'étude de la forêt, Département des sciences du bois \\ 
et de la forêt, Université Laval, Québec, QC G1V 0A6, Canada. \\ 

\clearpage



\section*{Résumé}
\label{sec:resume1}
\phantomsection\addcontentsline{toc}{section}{\nameref{sec:resume1}}

\begin{otherlanguage*}{french}
  <Résumé de l'article en français. Obligatoire.>

  \textbf{Mots-clés} : <ajouter des mots clés>
\end{otherlanguage*}

\clearpage

\section*{Abstract}
\label{sec:abstract1}
\phantomsection\addcontentsline{toc}{section}{\nameref{sec:abstract1}}

\begin{otherlanguage*}{english}
  <English abstract of the paper. Optional, but recommended.>

\textbf{Keywords}: <add some keywords> 
\end{otherlanguage*}

\cleardoublepage

\section*{Introduction}
\label{sec:intro1}
\phantomsection\addcontentsline{toc}{section}{\nameref{sec:intro1}}

%\defcitealias{keylist}{alias}

\section*{Material and methods}
\label{sec:matmet1}
\phantomsection\addcontentsline{toc}{section}{\nameref{sec:matmet1}}

\subsection*{Study area}
\label{subsec:area}
\phantomsection\addcontentsline{toc}{subsection}{\nameref{subsec:area}}

\begin{otherlanguage*}{english}
  Our study was conduct within the Portneuf Wildlife Reserve in the Captial-Nationale administrative region (72°24'W, 47°07'N) near Rivière-à-Pierre and Lac Amanites. 
  This area is located within the balsam fir-yellow birch bioclimatic domain (Saucier et al., 2009) and lie on a deep glacial till as surface deposit (Gosselin, 1998).
  The mean daily temperature is 4 ◦C for the 1981-2010 period of the nearest weather station (Lac aux sables)(Environment Canada, 2019). 
  Based on the same report, the mean annual precipitation and snowfall are respectively 1133.2 mm and 230.3 cm.
  We used the assisted migration experimental system establish in 2018 by the Minister of Natural Resources and Forests to collect our data (MNRF)(Royo, 2023).
  This system is a factorial experimental design with a split-split-split-plots using 4 replications (complete random blocks). 
  Each whole block (200 m x 140 m) is occupied by an overstory treatment (clearcut and 50\% shelterwood cut). 
  A cervid exclusion treatments (excluded and non-excluded) is used as a split plot and a competing vegetation treatments as a split-split plot. 
  Climatic analogs associated with three climate projections (current climate, mid-century 2050, and end of century 2080) 
  are used as a split-split-split plots with the seedlings of 9 species in mixed planting: black cherry (\textit{Prunus serotina}), northern red oak (\textit{Quercus rubra}), 
  northern white-cedar (\textit{Thuja occidentalis}), shagbark hickory, (\textit{Carya ovata}), sugar maple (\textit{Acer saccharum}), red pine (\textit{Pinus resinosa}), 
  red spruce (\textit{Picea rubens}), white spruce (\textit{Picea glauca}), and white pine (\textit{Pinus strobus}).

\end{otherlanguage*}


\subsection*{Sampling design}
\label{subsec:sampling}
\phantomsection\addcontentsline{toc}{subsection}{\nameref{subsec:sampling}}

We selected six samples units (10 m x 7,5 m each) in both overstory treatment in four blocks and added three controls units outside every blocks for a total of 60 units in our system. 
Controls were separated from the blocks per at least 10 meters to remove treatments effects.
Each sample unit contained three sampling methods to collect our data. 
Artificial cover boards were used to count red-backed salamanders (Hesed, 2012; Mazerolle et al., 2021; Moore, 2009), 
pitfall traps for the carabid monitoring (sources), and soil cores for springtails assessment (sources) (figure).

Artificial cover boards is commonly employed for salamander sampling and yield similar or superior results compared to other methods such as active searching (Hyde, Simons, 2001; Moore, 2009). 
The use of coverboards helps reduce variability in the number and size of sampled ground objects (Hyde and Simons 2001). 
Cover boards were made of spruce wood, measured 25 cm x 30 cm x 5 cm and were placed directly on the ground without litter underneath 
to maintain higher humidity level under the boards (Mazerolle et al., 2021). 
Six boards were centered in each sample and control units and separate between each other by a minimum distance of 1.5 m (figure), resulting in a total of 360 cover boards. 
All 360 boards were placed outside duing march 2022 to provide aging of the boards under natural conditions, thereby increasing their likelihood of being used by salamanders (article).  

Pitfall trapping is a passive sampling method used to assess the species richness of ground-dwelling invertebrates (Knapp, Růžička, 2012; Kotze et al., 2011; Lövei,  Sunderland, 1996). 
This capture method works particularly well for active arthropods moving on the ground, such as ground beetles (Baars, 1979; Lövei, Sunderland, 1996). 
One pitfall was positioned at the center of each sample and control units (figure), resulting in a total of 60 pitfall traps.

Pitfall traps included a container with a diameter of 12.5 cm, a depth of 25 cm and a cover with an opening below the cover (figure). 
Each pitfall traps were placed in the soil at a depth allowing the container's opening to be juxtaposed with 
the soil surface (Figure 4). These traps are produced by Bio.Contrôle services and are equipped with a cover raised 4.5 cm above the trap to prevent debris and rain from filling 
the container, as well as a protective grid with a mesh size of 15 mm, limiting trap access to carabid-sized individuals and reducing the chances of predation by small mammals (Figure 5). 
Typically, a preserving liquid (such as propylene glycol) is placed in the bottom of the container to preserve captured individuals. 
For our sampling, we will recommend using traps without preserving liquid (Luff, 1975). 
This choice was made considering the probabilities of salamander capture. 
Salamanders can have a width similar to that of some ground beetles, making it difficult to limit access to one without influencing the sampling of the other. 
For ethical reasons, we prefer not to use liquid and will add a wet sponge to the bottom of each container to maintain a suitable level of humidity for salamanders (Figure 4).
During the trapping period, the pitfall traps will be visited daily to collect their contents, release accidentally captured salamanders, and limit predation in the traps (Luff, 1975). 
We will conduct four trapping periods of five consecutive days each, namely in mid-May, mid-June, mid-July, and mid-August. 
Outside the sampling periods, all traps will be closed with their cover and adhesive tape around the mesh of the grid to prevent capturing individuals. 
During site visits, arthropods will be collected using a net and then transferred to an individual and labeled container containing 75\% alcohol. 
The containers will be brought back to the laboratory at the Forest Research Directorate for sorting and identification of ground beetles.

Soil cores

limitation des capture (detection vs abondances du au retrait des individues)
However, it would be possible to account for imperfect and variable species detection in our models to enable comparison (Mazerolle et al., 2007).


The site will be visited four times during the summer of 2022, namely in mid-May, mid-June, mid-July, and mid-August. 

This number of repetitions will allow for assessing the temporal effect during analyses. The visit to all sites will last for one week. 
The order in which blocks will be visited will be determined randomly so that the time of day or the sampler's fatigue level does not impact the results. 
During each visit, all boards will be checked to count the number of salamanders present per sampling unit. 
No individual manipulation is deemed necessary in our study. Each lifted board will be replaced, minimizing disturbance and environmental impact as much as possible.






 

\subsection*{Sites habitat characteristics}
\label{subsec:habitat}
\phantomsection\addcontentsline{toc}{subsection}{\nameref{subsec:habitat}}

\subsection*{Statistical analyses}
\label{subsec:analyses}
\phantomsection\addcontentsline{toc}{subsection}{\nameref{subsec:analyses}}

\clearpage

\section*{Results}
\label{sec:results1}
\phantomsection\addcontentsline{toc}{section}{\nameref{sec:results1}}

\clearpage

\section*{Discussion}
\label{sec:discu1}
\phantomsection\addcontentsline{toc}{section}{\nameref{sec:discu1}}

\section*{Conclusion}
\label{sec:conclu1}
\phantomsection\addcontentsline{toc}{section}{\nameref{sec:conclu1}}

\section*{Acknowledgements}
\label{sec:acknowl1}
\phantomsection\addcontentsline{toc}{section}{\nameref{sec:acknowl1}}

\section*{Conflict of interest}
\label{sec:conflict1}
\phantomsection\addcontentsline{toc}{section}{\nameref{sec:conflict1}}

None declared

\section*{Author contributions}
\label{sec:author1}
\phantomsection\addcontentsline{toc}{section}{\nameref{sec:author1}}

\cleardoublepage


\begin{otherlanguage}{english}
\bibliography{references.bib}
\bibliographystyle{ecologyNewEN.bst}
\addcontentsline{toc}{section}{References}
\end{otherlanguage}
