\chapter{Direct and indirect effects on soil fauna of silvicultural treatments in the context of assisted forest migration}     % numéroté
\label{chapitre1-articles}    

William Devos$^1$, Mathieu Bouchard$^1$, Marc J. Mazerolle$^1$

%href{mailto:william.devos.1@ulaval.ca}
$^1$ Centre d'étude de la forêt, Département des sciences du bois \\ 
et de la forêt, Université Laval, Québec, QC G1V 0A6, Canada. \\ 

\clearpage

\section*{Résumé}
\label{sec:resume1}
\phantomsection\addcontentsline{toc}{section}{\nameref{sec:resume1}}

\begin{otherlanguage*}{french}
  <Résumé de l'article en français. Obligatoire.>

  \textbf{Mots-clés} : <ajouter des mots clés>
\end{otherlanguage*}

\clearpage

\section*{Abstract}
\label{sec:abstract1}
\phantomsection\addcontentsline{toc}{section}{\nameref{sec:abstract1}}

\begin{otherlanguage*}{english}
  <English abstract of the paper. Optional, but recommended.>

\textbf{Keywords}: <add some keywords> 
\end{otherlanguage*}

\cleardoublepage

\section*{Introduction}
\label{sec:intro1}
\phantomsection\addcontentsline{toc}{section}{\nameref{sec:intro1}}

%\defcitealias{keylist}{alias}

Forest ecosystems play a vital role in the biosphere, both economically and ecologically, by providing essential services such as carbon sequestration, climate regulation, water retention, and biodiversity conservation \citep{Balvanera2006Quantifyingevidence,Diaz2006BiodiversityLoss,Canadell2008Managingforests,Pawson2013Plantationforests}. 
However, the growing demand for forest products and other services has intensified harvesting practices, leading to ecosystem degradation and loss of biodiversity\citep{Bengtsson2000Biodiversitydisturbances,Sala2000Globalbiodiversity,Foley2005GlobalConsequences,Naeem2012functionsbiological}. 

Numerous studies have highlighted the effects of forest management on various species, including birds, bats, butterflies, turtles, small mammals, and insects \citep{Summerville2011Managingforest,Currylow2012ShortTermForest,Kaminski2013EffectsForest,Kellner2013Shorttermresponses,Caldwell2019ComparisonBat}. .
Logging activities cause habitat loss and fragmentation, reducing access to food, shelter, and breeding sites for certain species \citep{Bouderbala2023Longtermeffect}. 
It also reduces connectivity between habitats, restricting individual movement, which reduces genetic flow between populations and increases the risk of local extinction \citep{Saccheri1998Inbreedingextinction}. 
At the stand level, logging often results in an overrepresentation of early-successional forests, reducing the presence of later successional stages \citep{Cyr2009Forestmanagement,Boucher2017Cumulativepatterns}. 
This shift reduces the structural complexity of stands, such as tree species composition, vertical stratification, age structure, successional dynamics, and disturbance frequency \citep{Commarmot2005Structurevirgin}. 
Moreover, stands with more homogeneous structures are less resilient to disturbances, as it slows down the regeneration of existing tree species and the return of those that have disappeared \citep{Kuuluvainen2009Forestmanagement}. 

Overall, modifications of forest attributes contribute to a loss of species and functional diversity. 
In the long term, this erodes the resilience of forests at a local scale and can lead to a decline in ecosystem services \citep{Hooper2012globalsynthesis,Edwards2014Maintainingecosystem}.

%

Among the silvicultural treatments that significantly affect natural environments, logging plays a major role. 
However, the level of disturbance caused by these operations depends on the type of treatment employed \citep{Ameray2021Forestcarbon}. 

Clear-cutting is one of the most widely used practices in temperate and boreal forests \citep{Fedrowitz2014Canretention,Chaudhary2016Impactforest}. 
It is part of intensive forest management used to increase short-term wood productivity and quality \citep{Irland2011TimberProductivitya}.
This method involves the complete removal of commercial trees and results in a single-species, even-aged structure, leading to a drastic homogenization of stands \citep{Rosenvald2008whatwhen}. 
Additionally, shorter rotation periods increase the frequency of disturbances. 
These practices substantially alter ecosystem structures compared to natural conditions, increasing the risk of biodiversity loss and local extinctions \citep{Hanski2000Extinctiondebt}. 
However, some authors suggest that clear-cutting can mimic large-scale natural disturbances, such as wildfires or storms \citep{Greenberg1995comparisonbird}. 

Over the past few decades, ecosystem-based management has been proposed as a more sustainable approach to forest ecosystem \citep{Perry1998scientificbasis,Kuuluvainen2002Naturalvariabilitya}. 
This strategy aims to emulate natural disturbances and the resulting stand structures and successional processes.
The goal is to preserve biodiversity and maintain the resilience of forest ecosystems while ensuring the availability of a variety of ecosystem services \citep{Szaro1998emergenceecosystem,MacDicken2015Globalprogress}. 

Partial cuts are thus integrated into extensive management plans that prioritize regeneration and mimic natural disturbances \citep{Irland2011Timberproductivity}. 
These treatments rely on multi-aged structures, potentially including species mixes, and are characterized by longer rotations \citep{Kuuluvainen2009Forestmanagement}. 
They are used to stimulate the growth of the most vigorous trees, promote species diversity, or maintain an open canopy \citep{Irland2011Timberproductivity}.
Tree retention associated with these types of cuts promotes late-successional structures, which support richer biodiversity \citep{Ameray2021Forestcarbon}. 
Partial cuts also also help carbon sequestration and the supply of organic matter, while maintaining a heterogeneous structure and diverse ecological niches for wildlife \citep{Dahlgren1994effectswholetree,Barg1999Influencepartial,Tong2020Forestmanagement,Ameray2021Forestcarbon}. 

In addition to the challenge of conserving biodiversity and forest ecosystems while meeting economic demands, climate change adds further complexity to forest management. 
Rising global temperatures pose an additional threat to the sustainability of flora and fauna by significantly altering environmental conditions \citep{McKenney2009Climatechange,Trumbore2015Foresthealth,Seidl2017Forestdisturbances,Messier2022Warningnatural}.  
Climate stresses often act additively or synergistically with forestry activities, amplifying their impact on biodiversity and ecosystems \citep{Brook2008Synergiesextinction,Tremblay2018Harvestinginteracts,Ochs2022Responseterrestrial,Bouderbala2023Longtermeffect}. 
Forest composition could thus be altered, requiring adjustments in management practices and conservation strategies \citep{McKenney2009Climatechange,Chmura2011Forestresponses,Lo2011Linkingclimate}. 

For example, assisted tree migration, which involves relocating species or genetic material from their original climatic region to areas better suited to future conditions, 
has been proposed as a potential mitigation strategy to maintain ecosystem services and the economic value of forests \citep{Vitt2010Assistedmigration,Pedlar2011implementationassisted,Ste-Marie2011Assistedmigration,Winder2011Ecologicalimplications}. 
However, there remains a lack of knowledge and uncertainties surrounding these new adaptation methods \citep{Klenk2015assistedmigration,Park2018Informationunderload}. 
Therefore, it is essential to better understand the impact of silvicultural practices on biodiversity, especially in a context where forest management must adapt to climate change, biodiversity loss, and economic pressures.

%

Soil fauna plays a major role in forest ecosystems by facilitating the flow of matter and energy. 
However, this community is among the most impacted by disturbances due to their sensitivity to environmental changes and limited dispersal abilities.

Canopy removal significantly alters soil conditions by increasing its exposure to sunlight, leading to increased temperatures, changes in humidity, higher wind speeds, and intensified precipitation. 
The movement of machinery also increases soil compaction, reducing porosity and consequently affecting the organisms that live there.
These changes at the surface level affect nutrient availability by altering litter, root secretions, leaching, and soil chemical properties. 
Logging also reduces microhabitats, such as deadwood, cavities in mature trees, and root plates, which provide shelter for soil fauna. 
These modifications create unfavorable conditions for species that rely on cool, moist environments, such as amphibians and arthropods, leading to population declines or local extinctions.

Due to their sensitivity to environmental shifts and limited dispersal capacity, amphibians and arthropods serve as important indicators to understand the effects of forestry practices on biodiversity and the ecological integrity of forests. 
Moreover, both groups have experienced significant declines in recent decades, primarily due to forest management practices and climate change. 
Species frequently studied in this context include the Eastern Red-backed Salamander (\textit{Plethodon cinereus} (Green, 1818)), ground beetles (Carabidae), and springtails (Collembola).

The Eastern Red backed Salamander represents one of the largest vertebrate biomasses in North American forests \citep{Burton1975Salamanderpopulations,Petranka1993Effectstimber,semlitschAbundanceBiomassProduction2014a}. 
As a generalist predator, this plethodontid species plays a key role in regulating detritivorous invertebrates, influencing decomposition processes, nutrient cycling, and carbon dynamics \citep{Burton1975Energyflow,Wyman1998Experimentalassessment,Walton2013Topdownregulation,Hickerson2017Easternredbacked}. 
Moreover, it serves as a high nutritional prey for many predators, including birds, mammals, and reptiles \citep{Burton1975Energyflow,Pough1987abundancesalamanders}. 
Without lungs, the Eastern Red backed Salamander relies on cutaneous respiration for gas exchange \citep{Heatwole1961Relationsubstrate}. 
This type of respiration requires salamanders to occupy specific microhabitats, staying on the surface when temperature and humidity are favorable, or retreating into the soil during less suitable conditions \citep{Grizzell1949HibernationSite,FraserEmpiricalEvaluation1976,Jaeger1980MicrohabitatsTerrestrial}. 
The loss of refuges due to logging decreases habitat quality, limiting the time it spends on the forest floor to feed and reproduce \citep{Achat2015Quantifyingconsequences,Peele2017EffectsWoody}. 
Consequently, poor surface conditions, soil compaction, and low levels of coarse woody debris can negatively impact population dynamics \citep{Peterman2014Spatialvariation}. 

On the other hand, ground beetles are a well documented taxon both in terms of taxonomy and ecology \citep{loveiEcologyBehaviorGround1996}. 
These insects have short lifespans, occupy a high position in the soil food web, and respond rapidly and in complex ways to environmental changes \citep{loveiEcologyBehaviorGround1996}. 
They play an ecological role by regulating invertebrate populations, while also serving as prey for various amphibians, reptiles, birds, and mammals \citep{loveiEcologyBehaviorGround1996}. 
With around 40,000 known species, ground beetles are one of the most diverse beetle families and among the most abundant soil dwelling arthropods \citep{Erwin1985taxonpulse,loveiEcologyBehaviorGround1996,Rochefort2006GroundBeetle}. 
They are widely distributed across nearly all terrestrial ecosystems, although species vary in their habitat preferences  \citep{loveiEcologyBehaviorGround1996,kotzeFortyYearsCarabid2011a,Larochelle2003naturalhistory}. 
Environmental changes can benefit some species at the expense of others, making their diversity and sensitivity valuable for studying the effects of environmental disturbances \citep{Rainio2003Groundbeetles}. 

Springtails represent one of the most abundant and diverse groups within the mesofauna \citep{rusekBiodiversityCollembolaTheir1998}. 
Different springtail communities occupy a variety of ecological niches within the soil \citep{pongeVerticalDistributionCollembola2000}.
Their vertical distribution is largely influenced by abiotic factors such as light, humidity levels, and soil porosity. 
Primarily fungivores and detritivores, springtails play a key role in the decomposition of wood. 
They also impact the physical structure and mineralization rates of litter, influence the nutrient absorption, 
regulate microbial communities, and contribute to soil microstructure formation \citep{Petersen1982comparativeanalysis,Neher2012Linkinginvertebrate,Maass2015Functionalrole,Potapov2016Connectingtaxonomy}. 
Additionally, springtails serve as an important food source for a range of organisms, including amphibians, beetles, arachnids, birds, and reptiles.

Through their trophic relationships, their sensitivity to environmental condition changes, and their dependence on forest features such as woody debris and litter, 
these three species groups are relevant models for studying the impact of forestry practices on soil fauna.

Most studies addressing the impacts of forestry treatments on fauna generally discuss the direct effects of disturbances on one or more species groups, 
without considering the relationships between environmental variables and different species groups, thereby neglecting the indirect effects of logging on soil fauna. 
My project aimed to fill this gap by trying to understand how the effects of forestry treatments propagate through the forest ecological network and influence soil fauna dynamics. 
Ultimately, this knowledge gain will provide valuable tools to facilitate sustainable forest management.

%%

These three taxa are relevant to quantifying the impact of silvicultural treatments as their sensitivity to environmental changes and trophic relationships enable the analysis of disturbance effects on soil fauna dynamics \citep{Salmon2008Relationshipssoil}.
Several papers have already studied the effects of overstory treatments on soil fauna.
However, most of those researches have only focused on the direct impacts of disturbances for one or more species groups, neglecting the potential relationships between environmental variables and species groups \citep{josephIntegratingOccupancyModels2016,Pollierer2021Diversityfunctional,Kudrin2023metaanalysiseffects}. 
Quantifying treatments effects and their propagation within the ecological network, namely on soil fauna, will provide useful tools to improve sustainable forest management.

Our study aimed to understand how silvicultural practices, conducted in an assisted tree migration context, affect the dynamics of forest soil ecosystems. 
The specific objectives were to quantify the effect of overstory treatments on environmental variables that influence habitat use by soil fauna
and to evaluate the impact of overstory treatments on habitat use by soil fauna.
We hypothesized that environmental variables, favorable to habitat use by taxa, fluctuate according to the intensity of forest harvests. 
We predicted that more intense treatments would decrease litter depth and CWD, as a consequence of reduced accumulation of leaves and woody debris on the forest floor and increase canopy openness. 
We also hypothesized that overstory treatments modify habitat use by soil fauna and propagate through the trophic network. 
We predicted that tree harvest impacts occupancy of salamanders and large ground beetles, 
followed by small ground beetles and ultimately affecting springtail biomass. 


\section*{Material and methods}
\label{sec:matmet1}
\phantomsection\addcontentsline{toc}{section}{\nameref{sec:matmet1}}

\subsection*{Study area}
\label{subsec:area}
\phantomsection\addcontentsline{toc}{subsection}{\nameref{subsec:area}}

\begin{otherlanguage*}{english}

  Our study was conducted within the Portneuf Wildlife Reserve in the Capitale-Nationale administrative region near Lac des Amanites (47°07’N, 72°24’W, Figure \ref{fig:area}). 
  This area is located within the balsam fir (\textit{Abies balsamea})-yellow birch (\textit{Betula alleghaniensis}) bioclimatic domain, according to the ecological classification used in Québec \citep{saucierChapitreEcologieForestiere2009}. 
  Other tree species in this bioclimatic domain included sugar maple (\textit{Acer saccharum}), red maple (\textit{Acer rubrum}), white spruce (\textit{Picea glauca}), black spruce (\textit{Picea mariana}), red spruce (\textit{Picea rubens}), white birch (\textit{Betula papyrifera}), and quaking aspen (\textit{Populus tremuloides})\citep{olaBelowgroundCarbonStocks2024}. 
  The experimental sites are located on deep glacial tills with moderately well-drained sandy loams soil \citep{CanadianSystemSoil1998}. 
  The mean daily temperature is 4\up{o}C based on the 1981-2010 period at the nearest weather station (Lac aux sables, \citealp{environmentcanadaCanadianClimateNormals2019}). 
  Based on the same report, the average annual precipitation, including snow-covered months, is 1133.2 mm, with snowfall averaging 230.3 cm.

  We conducted our study within the assisted migration experimental system established in 2018 by the Ministère des Ressources naturelles et des Forêts (\citealp{royoDesiredREgenerationAssisted2023}). 
  This experimental system uses a factorial experimental design with split-plots replicated in four blocks. 
  Each whole block (200 m x 140 m) is split in two overstory treatment : clear-cut and regular shelterwood cut at 50\% of the merchantable basal area (partial-cut). 
  The harvest took place during the summer 2017, followed by trenching in June 2018 within the clear-cut areas to create more uniform site conditions and facilitate planting. 
  No site preparation was conducted in the shelterwood cuts.

\end{otherlanguage*}

\begin{figure}[ht!]
	\centering
	\includegraphics[scale=0.31]{fig_area6.png}
	\caption[Localization of the Capitale-Nationale administrative region in Quebec, Canada and position of the study area near Lac des Amanites in Portneuf Wildlife Reserve, Quebec, Canada.]
  {Localization of the Captial-Nationale administrative region in Quebec, Canada (A) and position of the study area in Portneuf Wildlife Reserve, Quebec, Canada (B) where the clear-cut (orange) and the partial-cut (green) associated with the assisted migration experimental system were applied in 2017 (47°07'N, 72°24'W)(C).}
	\label{fig:area}
	\end{figure}  


\subsection*{Environmental variables}
\label{subsec:EnvVar}
\phantomsection\addcontentsline{toc}{subsection}{\nameref{subsec:EnvVar}}

We selected a total of 60 sampling units measuring 10 m by 7.5 m to collect our data: we used four blocks serving as replicates, 
with each block containing six sampling units for both the clear-cut and partial-cut overstory treatments, 
whereas three additional sampling units were positioned outside the block, serving as uncut controls (Figure \ref*{fig:blockSU}). 
The uncut controls were separated from blocks by at least 10 m. 

In each sampling unit, we measured several environmental variables that could affect occupancy probability of red-backed salamanders, ground beetles, and springtails.
CWD and litter depth play a crucial role in habitat use for salamanders, ground beetles and springtails as
they serve for feeding and protection  \citep{harmonEcologyCoarseWoody1986,koivula.LeafLitterSmallscale1999,birdChangesSoilLitter2004,mckennyEffectsStructuralComplexity2006}. 
Salamanders also utilize CWD as shelter to maintain suitable temperature and moisture levels during dry periods \citep{Jaeger1980MicrohabitatsTerrestrial,groverInfluenceCoverMoisture1998a,patrickEffectsExperimentalForestry2006a}.
We used 400 m\up2 plots centered on every sampling unit to estimate CWD (20 m $\times$  20 m) (\citealp{methotGuideInventaireEchantillonnage2014}). 
We only considered CWD with a basal diameter $\geq$ 9 cm and a length $\geq$ 1 m.
For each CWD, we measured the basal diameter, the apical diameter and length with a tree caliper.
Segments of CWD outside the plot boundaries were not included.
We employed the conic–paraboloid formula to estimate the volume of each CWD \citep{fraverRefiningVolumeEstimates2007} :

\begin{equation}
  \text{Volume} = L/12 \times (5A_b + 5A_u + 2\sqrt{A_b \times A_u})
\end{equation}

\vspace{0.5cm}

Where $L$ is the length of log (cm), $A_b$ the basal area (cm\up{2}) and $A_u$ the apical area (cm\up{2}).
We measured litter depth next to each coverboard and computed the average litter depth for each sampling unit \citep{Mazerolle2021Woodlandsalamander}. \\

We assessed canopy openness, as it may influence species occupancy \citep{messereForestFloorDistribution1998,koivulaBorealCarabidbeetleColeoptera2002a,tilghmanMetaanalysisEffectsCanopy2012,henneronForestPlantCommunity2017}.
Canopy openness was measured 130 cm above the ground at the center of each sampling unit using a spherical densiometer \citep{lemmonSphericalDensiometerEstimating1956}. 
We averaged four measurements per sampling unit, oriented toward each of the four cardinal points as an estimate of canopy openness.

We collected data locally for air temperature, air humidity and precipitation levels during the summer 2022. 
Two weather stations (Em50 Digital Decagon Data Logger, Part \#40800, Meter Group Inc., USA) were installed inside both overstory treatments. 
Each weather station measured temperature, air humidity, and atmospheric pressure, 130 cm above the ground (VP-4 Sensor (Temp/RH/Barometer), Part \#40023). 
Rain gauges were installed in the clear-cut treatments to monitor precipitation levels. 
The temperature and humidity sensors were programmed to record data every 15 minutes. 
We averaged the measurements across both weather stations to get daily average measurements. 
These variables fluctuate on a daily basis, affecting species activity and, consequently, the probability of detecting individuals \citep{spotilaRoleTemperatureWater1972,butterfieldCarabidLifeCycle1996,loveiEcologyBehaviorGround1996,odonnellPredictingVariationMicrohabitat2014a}.


\subsection*{Sampling design}
\label{subsec:sampling}
\phantomsection\addcontentsline{toc}{subsection}{\nameref{subsec:sampling}}

In each sampling unit, we used three sampling methods to collect species data : artificial coverboards, pitfall traps, and soil cores.
We used artificial coverboards to sample eastern red-backed salamanders \textit{Plethodon cinereus} at our study sites \citep{hydeSamplingPlethodontidSalamanders2001,mooreComparisonPopulationEastern2009c,hesedUncoveringSalamanderEcology2012,Mazerolle2021Woodlandsalamander}. 
Coverboards consisted of 25 cm x 30 cm x 5 cm pieces of untreated spruce wood. Each board was in direct contact with the soil after we had cleared the litter underneath \citep{Mazerolle2021Woodlandsalamander}. 
Six coverboards spaces by at least 2.5 m were arranged in a rectangular array in each sampling unit, resulting in a total of 360 coverboards across our 60 sampling stations (Figure \ref{fig:blockSU}). 
All coverboards were placed outdoors in March 2022 to allow for natural aging \citep{hedrickEffectsCoverboardAge2021,Grasser2014Effectscover}. 
We conducted four visits to coverboards during the summer 2022, with each visit spaced one month apart, namely in mid-May, mid-June, mid-July and mid-August, during the summer 2022. 
Blocks were visited in a random sequence to reduce the potential effects of time of day and observer fatigue. 
During a given visit, the 360 coverboards were inspected on the same day and we counted the number of red-backed salamanders underneath.  

We used pitfall traps were used to capture ground beetles \citep{baarsCatchesPitfallTraps1979,spenceSamplingCarabidAssemblages1994a,loveiEcologyBehaviorGround1996,kotzeFortyYearsCarabid2011a,knappEffectPitfallTrap2012}. 
Trap design was based on Multipher\up{\textregistered{}} traps and included a container with a diameter of 12.5 cm, a depth of 25 cm and a cover raised 4.5 cm above the trap 
to prevent debris and rain from filling the container \citep{Jobin1988MultiPherinsect,mooreEffectsTwoSilvicultural2004,bouchardBeetleCommunityResponse2016b}. 
We covered each trap with a protective stainless steel mesh size of 15 mm, allowing trap access to carabid-sized individuals and limiting access by predators.  
We did not add preserving liquid in traps, but we added wet sponges to avoid harming vertebrates small enough to pass through the mesh. 
We centered a pitfall in each sampling unit (Figure \ref{fig:blockSU}). 
Traps were inserted in the soil at a depth allowing the container’s opening to be level with the soil surface. 
Trapping occurred during four sampling periods during 2022 (mid-May, mid-June, mid-July and mid-August). 
Traps were opened on the first day of each survey, and we collected captured ground beetles daily during five days. 
Outside of the trapping periods, pitfall traps were sealed with adhesive tape to prevent captures. 
Individuals were preserved in 70\% alcohol and identified at the species level afterwards. 
Identification was conducted with a ZEISS SteREO Discovery.V12 bionocular microscope using the \cite{larochelleManuelIdentificationCarabidae1976} taxonomic keys. 
We categorized ground beetles into two groups, either as small carabids (salamander prey )or large carabids (salamander competitors), based on the red-backed salamander gape size (Table \ref{tab:carabid}, \citealp{jaegerFoodLimitedResource1972,magliaModulationPreycaptureBehavior1995,magliaOntogenyFeedingEcology1996}).

We extracted soil cores to sample springtails \citep{pongeVerticalDistributionCollembola2000,salamonEffectsPlantDiversity2004,chauvatChangesSoilFaunal2011a,farskaManagementIntensityAffects2014}. 
Core sampling occurred during the same four sampling periods as for the salamanders and ground beetles (mid-May, mid-June, mid-July and mid-August). 
During a sampling period, we collected two cores in each sampling unit using a pedological probe (5 cm diameter x 5 cm depth). 
We also collected a 15 cm x 15 cm litter quadrat each soil sample \citep{raymond-leonardSpringtailCommunityStructure2018a,rousseauForestFloorMesofauna2018}.
We extracted springtail communities directly related to the ecology of salamanders and ground beetles \citep{edwardsAssessmentPopulationsSoilinhabiting1991,chauvatChangesSoilFaunal2011a,raymond-leonardSpringtailCommunityStructure2018a,rousseauForestFloorMesofauna2018}.
Soil and litter from the same unit were pooled in Ziploc\up{\texttrademark{}} bags and stored in a cooler at ca. 4 °C \citep{chauvatChangesSoilFaunal2011a,rousseauForestFloorMesofauna2018}, providing 60 samples during each of the four sampling periods.
We placed each sample in an individual Tullgren dry-funnel for springtail extraction within 48 h after collection \citep{rusekBiodiversityCollembolaTheir1998,wuCompositionSpatiotemporalVariation2014,rousseauForestFloorMesofauna2018}. 
The extraction process lasted six days with a gradual temperature increase (25 °C to 50 °C) \citep{raymond-leonardSpringtailCommunityStructure2018a}.
Springtails were separated from other invertebrates, pooled by sampling unit and preserved in 75\% alcohol \citep{wuCompositionSpatiotemporalVariation2014}.
Identification at the family level for all individuals was done with a ZEISS SteREO Discovery.V12 binocular microscope and a Leitz orthoplan phase-contrast fluorescent trinocular microscope using \cite{bellingerChecklistCollembolaWorld1996} identification keys. 
Following identification, springtails within each sample were dried in a freeze dryer (Labconco FreeZone Bulk tray dryer 78060 series) for 24 hours. 
We determined the total dry biomass of springtails in each sample with a micro balance (Sartorius Cubis\up{\texttrademark{}} MSA3.6P-000-DM).

\pagebreak

\begin{figure}[ht]
	\centering
	\includegraphics[scale=0.50]{fig_blockSU.png}
	\caption[Design of one block and one sampling unit with three sampling methods.]{
  Design of a block (left) and a sampling unit (right). 
  The block contains two overstory treatments : clear-cut (grey background), partial-cut (white background). 
  Fifteen sampling units were used per block : six per overstory treatment and three controls (\textbf{c}) outside each block.
  Each sampling unit contained six artificial coverboards (squares) and one pitfall trap (circle). Two soil cores (stars) were collected per survey.
  }
	\label{fig:blockSU}
	\end{figure}  

\vspace{0.5cm}

\subsection*{Statistical analyses}
\label{subsec:analyses}
\phantomsection\addcontentsline{toc}{subsection}{\nameref{subsec:analyses}} 


\subsubsection{Structural equations models} 

To assess the effects of overstory treatments on habitat use by soil fauna (hypothesis 1.1) and the relationship between overstory treatments 
and environmental variables (hypothesis 2.1), we employed a structural equation model (SEM) combining occupancy models and linear mixed models \citep{mackenzieOccupancyEstimationModeling2006a,graceSpecificationStructuralEquation2010,josephIntegratingOccupancyModels2016}.
This approach enabled us to test both hypotheses within one analysis. 

One part of the SEM focused on the variations in environmental variables across different overstory treatments. 
Specifically, we estimated the effects of partial-cut and clear-cut on coarse woody debris volume, canopy openness and litter depth. 
The other part of the SEM was designed to evaluate the direct and indirect effects of overstory treatments on taxa (Figure \ref{fig:SEM}). 
We presumed that partial-cut and clear-cut will directly impact each taxon, while salamander occupancy will influence small ground beetle occupancy and springtail biomass. 
We aslo suggested that both ground beetle groups will affect springtails biomass.

Some components of the SEM consisted of linear mixed models to estimate the effect overstory treatments on environmental variables (Table \ref{ann:SEM_Env_eq}) and springtail biomass. 
The component of the model predicting springtail biomass also included the latent occupancy state of salamanders and ground beetles to measure the impact of these groups on springtails. 
Other components of our SEM consisted of occupancy models to quantify the impact of overstory treatments on the occupancy (presence) probabilities of salamanders and ground beetles (Table \ref{ann:SEM_Sp_eq}). 
Occupancy models estimate the presence of species difficult to detect after accounting for imperfect detection probability \citep{mackenzieEstimatingSiteOccupancy2002,baileyEstimatingSiteOccupancy2004,mazerolleMakingGreatLeaps2007,spiersEstimatingSpeciesMisclassification2022}. 
The data type required to distinguish between occupancy and detection probabilities consists of repeated visits at a series of sites 
(here, 4 visits at the sampling units). Specifically, a detection history is constructed for each site, 
using 1 to denote the detection of the species on a given visit and 0 to non-detection. 
For example, the detection history 0010 at a site would indicate that the species was detected at the site on the third visit, but not detected during the first, 
second, and fourth visits. 

We formulated the SEM using a Bayesian framework, which we describe in Table \ref{ann:SEM_script}. 
Parameters were estimated using Markov chain Monte Carlo (MCMC) with JAGS 4.3.0 included in the jagsUI package in R 4.3.1 \citep{lunnBUGSProjectEvolution2009,rcoreteamLanguageEnvironmentStatistical2020,kellnerJagsUIWrapperRjags2024}. 
We ran the model with five chains and 200,000 iterations each \citep{gelmanUnderstandingPredictiveInformation2014}. 
The first 75,000 iterations were used as burn-in and we used a thinning rate of 5. 
We assessed convergence of MCMC chains by examining trace plots, posterior density plots, and using the Brooks-Gelman-Rubin statistic. 
The JAGS model code is available in Table \ref{ann:SEM_script}.

\vspace{10pt}

\begin{figure}[h!]
	\centering
	\includegraphics[scale=0.55]{fig_sem.png}
	\caption[Theoretical model illustrating the anticipated relationships between overstory treatments, environmental variables and species groups.]
  {Theoretical model illustrating the anticipated relationships between overstory treatments, coarse woody debris volume, canopy openness, litter depth,
   salamander occupancy, ground beetle occupancy and springtail biomass in the Portneuf Wildlife Reserve, Quebec, Canada. 
   Each arrow indicates the direction of a potential effect, from an explanatory variable to a response variable. 
   Note that the large and small carabid categories are based on salamander gape size.}
	\label{fig:SEM}
\end{figure} 

\clearpage

\section*{Results}
\label{sec:results1}
\phantomsection\addcontentsline{toc}{section}{\nameref{sec:results1}}


\subsection*{Environmental variables}
\label{subsec:ResEnv}
\phantomsection\addcontentsline{toc}{subsection}{\nameref{subsec:ResEnv}} 

Model diagnostics indicated that the chains were of sufficient length, as the Brooks-Gelman-Rubin statistic was below 1.04. 
Trace plot analysis revealed that all chains had converged towards similar values, and none of the ratios of MCMC error to posterior standard deviation exceeded 5\%.

Environmental variables usually differed between overstory treatments and control conditions.
We found that clear-cutting treatments had significantly less CWD compared to partial-cutting
(95\% CI : [-1.15, -0.43]) (Figure \ref{fig:envar} A, Table \ref{tab:overstoryenvar}). 
However, these treatments did not differ from the control sites. 
Canopy openness was significantly higher in both the partial-cut (95\% CI : [1.97, 11.02]) and clear-cut treatments (95\% CI : [51.39, 77.06]) when compared 
to the control sites, with the clear-cuts being more open than the partial-cuts (95\% CI : [44.61, 70.76], Figure \ref{fig:envar} B, Table \ref{tab:overstoryenvar}). 
In contrast, litter depth was lower in both the partial-cut (95\% CI : [-2.44, -0.65]) and clear-cut treatments (95\% CI : [-4.28, -2.50]) compared to the controls, 
with the litter being shallower in the clear-cuts than in the partial-cuts (95\% CI : [-2.57, -1.12], Figure \ref{fig:envar} C, Table \ref{tab:overstoryenvar}).


\vspace{10pt}

\begin{figure}[ht]
  \centering
  \includegraphics[scale=0.23]{fig_envar2.png}
  \caption[Environmental variables with a potential effect on soil species within two different overstory treatments and control.]
  {Environmental variables with a potential effect on soil species estimations within two different overstory treatments and control 
  during the summer 2022 in the Portneuf Wildlife Reserve, Quebec, Canada. Error bars denote 95\% credible intervals around estimates.}
  \label{fig:envar}
\end{figure}

\begin{table}[ht]
  \centering
  \caption[Contrasts between overstory treatments for environmental variables that could affect habitat selection of fauna on the forest soil.]
  {Contrasts between overstory treatments for environmental variables that could affect habitat use of fauna on the forest soil during the summer 2022 in the Portneuf Wildlife Reserve,
  Quebec, Canada.}
  \label{tab:overstoryenvar}
  \begin{tabular}{lllll} 
      \hline
      &&&&95\% Bayesian \\
      Variable&Unit& Comparison & Estimate &  credible interval \\ [0.5ex] 
      \hline
      Coarse woody debris &m\up{3}& Partial vs control & \hspace{1mm}0.02 & [-1.01, 1.06] \\ 
                 && Clear vs control  & -0.77 & [-1.79, 0.23] \\ 
                          && Clear vs partial  & -0.79 & [-1.15, -0.43] \\
      Canopy openness     &\%& Partial vs control & \hspace{1mm}6.49 & [1.97, 11.02] \\ 
                      && Clear vs control  & \hspace{1mm}64.19 & [51.39, 77.06] \\ 
                          && Clear vs partial  & \hspace{1mm}57.69 & [44.61, 70.76] \\ 
      Litter depth        &cm& Partial vs control & -1.54 & [-2.44, -0.65] \\ 
                      && Clear vs control  & -3.39 & [-4.28, -2.50] \\ 
                          && Clear vs partial  & -1.85 & [-2.57, -1.12] \\       
      \hline
      \multicolumn{5}{l}{\textbf{Note:} Estimates from Bayesian SEM are presented in terms of posterior mean with 95\%} \\
      \multicolumn{5}{l}{credible intervals, where an interval excluding 0 indicates a difference between groups.} \\
  \end{tabular}
\end{table}

\clearpage


\subsection*{Soil fauna}
\label{subsec:taxa}
\phantomsection\addcontentsline{toc}{subsection}{\nameref{subsec:taxa}} 

\vspace{10pt}

Across the sixty sampling units, Red-backed salamanders were detected in 0, 1, 11, and 11 sites during the May, June, July, and August 2022 surveys, respectively. 
Over the same periods, small ground beetles were detected in 2, 5, 12, and 2 sites, while large ground beetles were detected in 16, 30, 35, and 21 sites.
A total of 30 ground beetle species were identified from harvest in the pitfall traps and under the coverboards (Table \ref{tab:carabid}). 
We collected 468 springtails representing 12 families, with 219 springtails collected from partial-cut treatments, 131 from clear-cuts, and 118 from the control areas (Table \ref{tab:springtail}). 
The average springtail biomass collected per overstory treatments was 24.3 $\mu$g (SD = 18.2 $\mu$g), 56.8 $\mu$g (SD = 78.0 $\mu$g), and 31.1 $\mu$g (SD = 52.8 $\mu$g) in the partial-cut treatments, clear-cuts and control sites, respectively.

Occupancy and biomass generally did not vary significantly across the cutting treatments. 
Salamander occupancy probability was marginally lower in sites subjected to clear-cutting compared to those with partial-cutting (90\% CI : [-0.74, -0.07], Figure \ref{fig:pcin}, Table \ref{tab:overstorysp}). 
However, these two groups did not differ from the control sites. 
Occupancy probability for each carabid group and the springtail biomass did not vary between the overstory treatments and control sites (Table \ref{tab:overstorysp}). 
Furthermore, the occupancy probabilities of small ground beetles (salamander prey) did not vary with the presence of salamanders. 
Similarly, the biomass of springtails did not vary with the presence of salamanders, large ground beetles or small ground beetles (Table \ref{tab:overstorysp}).

\begin{figure}[h!]
	\centering
	\includegraphics[scale=0.60]{fig_sem_res.png}
	\caption[Results from structural equation modeling analysis revealing effects of overstory treatments on coarse woody debris volume,
  canopy openness, litter depth, salamander occupancy, ground beetle occupancy, and springtail biomass.]
  {Results from SEM analysis showing effects of overstory treatments on CWD, 
  canopy openness, litter depth, salamander occupancy, ground beetle occupancy, and springtail biomass in the Portneuf Wildlife Reserve, 
  Quebec, Canada. Bold arrows represent significant effects, while dotted line indicate no discernible effects. 
  Estimates marked with one asterisk (*) indicate a 90\% credible interval (CI) excluding 0, while estimates marked with two asterisks (**) indicate a 95\% CI excluding 0. 
  Note that the large and small carabid categories are based on salamander gape size.}
	\label{fig:SEMres}
\end{figure}  

\vspace{10pt}

\begin{table}[h!]
  \centering
  \caption[Contrasts between overstory treatments for salamander occupancy, ground beetle occupancy, and springtail biomass.]
  {Contrasts between overstory treatments for salamander occupancy, ground beetle occupancy, and springtail biomass, during the summer 2022 in the Portneuf Wildlife Reserve, Quebec, Canada. 
  This table also shows the estimated effect of interactions between different groups: salamander presence on small and large ground beetles and the effects of the presence of salamanders and both ground beetle groups on springtail biomass.}
  \label{tab:overstorysp}
  \begin{tabular}{lllll} 
      \hline
      &&&&95\% Bayesian \\
      Variable&Unit& Comparison & Estimate &  credible interval \\ [0.5ex] 
      \hline     
      Salamander           &probability& Partial vs control & \hspace{1mm}0.07 & [-0.29, 0.45] \\ 
      occupancy       && Clear vs control  & -0.38 & [-0.75, 0.11] \\ 
                          && Clear vs partial  & -0.45 & [-0.74, -0.07]$^{a}$ \\       
      Carabid$_{large}$ &probability& Partial vs control & -0.12 & [-0.35, 0.15] \\
      occupancy       && Clear vs control  & -0.06 & [-0.29, 0.20] \\ 
                          && Clear vs partial  & \hspace{1mm}0.06 & [-0.19, 0.30] \\ 
      Carabid$_{small}$    &logit& Partial vs control & \hspace{1mm}3.31 & [-10.12, 17.72] \\
      occupancy             && Clear vs control  & \hspace{1mm}10.19 & [-4.15, 24.45] \\ 
                          && Clear vs partial  & \hspace{1mm}6.88 & [-12.81, 23.42] \\  
                          && Salamander        & -2.20 & [-17.15, 16.59] \\  
      Springtail          &$\mu$g& Partial vs control & \hspace{1mm}8.11 & [-9.38, 25.40] \\
      biomass             && Clear vs control  & \hspace{1mm}2.11 & [-13.98, 18.11] \\ 
                          && Clear vs partial  & -6.00 & [-29.09, 17.26] \\  
                          && Salamander        & \hspace{1mm}6.80 & [-10.43, 23.26] \\ 
                          && Carabid$_{large}$      & \hspace{1mm}0.56 & [-16.75, 17.66] \\ 
                          && Carabid$_{small}$      & \hspace{1mm}7.62 & [-8.93, 24.09] \\ 
      \hline
      \multicolumn{5}{l}{\textbf{Note:} Estimates from Bayesian SEM are presented in terms of posterior mean with 95\%} \\
      \multicolumn{5}{l}{credible intervals, where an interval excluding 0 indicates a difference between groups.} \\
      \multicolumn{5}{l}{$^{a}$Marginal difference based on 90\% Bayesian credible interval excluding 0}
  \end{tabular}
\end{table}

\clearpage

\begin{figure}[h!]
  \centering
  \includegraphics[scale=0.55]{fig_pcin.png}
  \caption[Occupancy probability of salamanders under overstory treatments]
  {Occupancy probability of salamanders within two overstory treatments and controls during the summer 2022 in the Portneuf Wildlife Reserve device, Quebec, Canada. 
  Error bars denote 95\% credible intervals around estimates.}
  \label{fig:pcin}
\end{figure}

\vspace{10pt}

\clearpage

We did not observe significant impacts of CWD volume and precipitation level on salamander detection probabilities. 
However, the precipitation level had a positive effect on detection probability for small ground beetles (95\% CI : [0.59, 1.77]) and large ground beetles (95\% CI : [0.70, 3.23]) (Table \ref{tab:detection}). 
Detection probabilities of ground beetles did not vary with the volume of CWD.

\begin{table}[ht]
  \centering
  \caption[Estimated effects of coarse woody debris and precipitation level on detection probabilities of salamanders and both ground beetles.]
  {Estimated effects of coarse woody debris and precipitation level on detection probabilities of salamanders and both ground beetles, during the summer 2022 in the Portneuf Wildlife Reserve,  Quebec, Canada.}
  \label{tab:detection}
  \begin{tabular}{lllll} 
      \hline
      &&&95\% Bayesian \\
      Variable & Taxa & Estimate &  credible interval \\ [0.5ex] 
      \hline      
      Precipitation       & Salamander              & \hspace{1mm}0.11 & [-0.83, 1.06] \\ 
                          & Carabid$_{large}$  & \hspace{1mm}1.17 & [0.59, 1.77] \\ 
                          & Carabid$_{small}$        & \hspace{1mm}1.87 & [0.70, 3.23] \\  
      \hline      
      Coarse woody debris & Salamander              & -0.59 & [-1.39, 0.12] \\ 
                          & Carabid$_{large}$  & \hspace{1mm}0.06 & [-0.26, 0.38] \\ 
                          & Carabid$_{small}$        & \hspace{1mm}0.27 & [-0.74, 1.37] \\   

      \hline
      \multicolumn{4}{l}{\textbf{Note:} Estimates from Bayesian SEM are presented in terms of posterior mean} \\
      \multicolumn{4}{l}{with 95\% credible intervals, where an interval excluding 0 indicates} \\
      \multicolumn{4}{l}{a difference between groups.} \\
  \end{tabular}
\end{table}

\clearpage

\section*{Discussion}
\label{sec:discu1}
\phantomsection\addcontentsline{toc}{section}{\nameref{sec:discu1}}

% petit paragraphe sur ce qu'a montrer  note étude
% Revenir sur les objectifs et hypothèeses
% dire ce qu'a amené le projet
% revenir sur les hypothèese, sont elles confirmé ou non
% ce que suggèere les analyses, qeuls sont les effets
% ce que ca suggèere

% transfer de perturbation a travers le réseau trophique 


\subsection*{Environnementala variables}
\label{disc:env_var}
\phantomsection\addcontentsline{toc}{subsection}{\nameref{disc:env_var}} 

We found that clear-cutting treatments had significantly less CWD compared to partial-cutting. However, these treatments did not differ from the control sites. 
Canopy openness was significantly higher in both the partial-cut and clear-cut treatments when compared to the control sites, with the clear-cuts being more open than the partial-cuts. 
In contrast, litter depth was lower in both the partial-cut and clear-cut treatments compared to the controls, with the litter being shallower in the clear-cuts than in the partial-cuts.

% Effet des coupes sur CWD, impact sur le milieu et comment cela peu affecter la faune
disponibilité en refuge et en nutriment
Nolet et al.2018 Although irregular shelterwood cutting often maintains higher CWD volumes than other silvicultural treatments 
Otto et al. (2014) reported that apparent survival of P. cinereus 1–5 years following harvesting increased with canopy cover and the number of CWD objects in experimental enclosures of 9 m2, whereas abundance in the same enclosures was neither related to canopy cover nor CWD. 
Previous research has focused on certain forest characteristics (e.g., overstory cover, coarse woody debris, leaf litter depth) and their influence on salamanders post-harvest in order to guide forest management (Semlitsch, 2002; McKenny et al., 2006).
Further, the removal of surface refugia such as tree limbs, treetops, and other coarse woody debris (CWD) following harvesting can reduce habitat quality for salamanders and reduce the amount of time they spend at the forest floor surface (Achat et al., 2015; Peele et al., 2017)
Consequently, because woodland salamanders forage and breed at the forest floor surface, harsh surface conditions, soil compaction, and low levels of CWD can affect population dynamics (Peterman and Semlitsch, 2014).
Greater amounts of CWD can also mitigate the negative effects of even-aged forest management by providing surface refugia that maintains suitable environmental conditions underneath logs and rocks and allows individuals to remain at the surface for longer periods of time (Grover, 1998; Moseley et al., 2009; Strojny and Hunter, 2009; Carusco, 2016).
Mazerolle et al. (2021)  et margenau 2023 also found that BCI was higher in strip-cutting treatments, where there were greater amounts of early-stage decay, indicating that woodland salamander density and BCI responses can vary in relation to the amount of downed CWD present.
Ecological forestry practices include retaining some preharvest downed CWD levels post-harvest because it provides important structure and habitat for an array of taxa, including woodland salamanders (Franklin et al., 2007).

% Effet des coupes sur canope, impact sur le milieu et comment cela peu affecter la faune
changement des condition en forêt
Mazerolle 2021 : Overall, control treatments had greater basal area and litter depth as well as a lower degree of canopy openness and of soil compaction than irregular shelterwoods.
Mazerolle 2021 : The control sites in our study had greater canopy cover, whereas shrub and herbaceous cover were highest in gap and strip treatments, consistent with previous studies (Nolet et al. 2018; Raymond and Bédard 2017).
(Otto et al. 2014). For sites of 3600 m2,the authors observed a positive relationship between P. cinereus abundance and canopy cover 
Previous research has focused on certain forest characteristics (e.g., overstory cover, coarse woody debris, leaf litter depth) and their influence on salamanders post-harvest in order to guide forest management (Semlitsch, 2002; McKenny et al., 2006).

% Effet des coupes sur litter, impact sur le milieu et comment cela peu affecter la faune
effet sur la disponibilité en nutriment
Mazerolle 2021 : Overall, control treatments had greater basal area and litter depth as well as a lower degree of canopy openness and of soil compaction than irregular shelterwoods.
Mazerolle 2021 : Clearcuts are typically conducted at large scales but are similar in nature to strip cuttings of a 10 m width in our study, which still had a lower litter depth than control sites 5–6 years after treatment.
Previous research has focused on certain forest characteristics (e.g., overstory cover, coarse woody debris, leaf litter depth) and their influence on salamanders post-harvest in order to guide forest management (Semlitsch, 2002; McKenny et al., 2006).

voit t-on des différence entre les coupe partielle et les coupe total , une coupe est t'elle meilleur que l'autre pour les gourpe taxonomique ou les variable environnementale?


\subsection*{Soil fauna}
\label{disc:soil_fauna}
\phantomsection\addcontentsline{toc}{subsection}{\nameref{disc:soil_fauna}} 

% transfer de perturbation a travers le réseau trophique, ce que dit notre modèle


Occupancy and biomass generally did not vary significantly across the cutting treatments. 
Salamander occupancy probability was marginally lower in sites subjected to clear-cutting compared to those with partial-cutting. However, these two groups did not differ from the control sites. 
Occupancy probability for each carabid group and the springtail biomass did not vary between the overstory treatments and control sites. 
Furthermore, the occupancy probabilities of small ground beetles (salamander prey) did not vary with the presence of salamanders. 
Similarly, the biomass of springtails did not vary with the presence of salamanders, large ground beetles or small ground beetles.



% effet des coupes sur salamandre
  échantillonnage entre 1 et 3 ans par rapport a 5-7 ans  voir ochs2022 
  Cantrell 2013, found no negative short-term effects of shelterwood treatments on amphibians and reptiles.
  Messere and ducey 1998 observed no differences in P. cinereus density among one-year gap treatments, gap edges, and control forest.
  Ash 1997 also showed that salamanders returned to pre-disturbance levels 4–6 years following clearcut logging and their reappearance corresponded to the timing needed for the replenishment of the litter layer
  (Morneault et al. 2004).A potential explanation for the use of irregular shelterwood by salamanders of each size class is rapid regrowth of the understory vegetation after harvesting 
  (Raybuck et al. 2015) This rapid growth maintains humidity and low temperatures on the soil surface 
  Similarly, McKenny et al. (2006) found no impact of uneven-aged harvesting on P. cinereus. They attributed their results to a high well-decayed CWD volume on the soil of their treatment. 
  Mazerolle 2021 In our own study, we suggest that the high CWD volume and percent shrub cover maintain salamanders in irregular shelterwood sites.
  mazerolle 2021 Our results suggest that eastern red-backed salamanders are resilient to irregular shelterwood treatments 5–6 years after harvesting.
  Previous research has focused on certain forest characteristics (e.g., overstory cover, coarse woody debris, leaf litter depth) and their influence on salamanders post-harvest in order to guide forest management (Semlitsch, 2002; McKenny et al., 2006).
  Even-aged forest management (e.g., clearcutting and shelterwood) is often thought to be the most detrimental to woodland salamanders in the short-term (Duguay and Wood, 2002; Hocking et al., 2013a) 
  because these techniques greatly reduce overstory cover which increases the amount of solar radiation reaching the forest floor and subsequently dries out leaf litter and increases soil temperature (Zheng et al., 2000; Brooks and Kyker-Snowman, 2008).
  Due to physiological constraints (e.g., thin, vascular skin; Hillman et al. 2009), woodland salamanders are at increased risk of desiccation when exposed to drier environments at the forest floor.
  To cope with the environmental stress, individuals will often remain belowground where soils are cooler and moisture is not as limiting. 
  However, heavy machinery used during operational silviculture compacts soils creating physical barriers for fossorial salamanders (Steinbrenner, 1995).
  Further, the removal of surface refugia such as tree limbs, treetops, and other coarse woody debris (CWD) following harvesting can reduce habitat quality for salamanders and reduce the amount of time they spend at the forest floor surface (Achat et al., 2015; Peele et al., 2017)
Consequently, because woodland salamanders forage and breed at the forest floor surface, harsh surface conditions, soil compaction, and low levels of CWD can affect population dynamics (Peterman and Semlitsch, 2014).
Previous work in hardwood forests of eastern North America and the Appalachian region has shown that increased retention of overstory cover through partial harvesting approaches, such as uneven-aged management (e.g., group selection, single-tree selection), 
can result in higher relative densities of salamanders compared to even-aged treatments that remove greater amounts of overstory (Hocking et al., 2013a; Harper et al., 2015; Mahoney et al., 2016).
Greater retention of the overstory limits the amount of solar radiation reaching the forest floor which retains microclimates generally compatible with the physiology of salamanders (e.g., wetter, cooler; Homyack et al., 2011).
A common outcome of all forms of forest management is the reduction of the overstory layer due to the removal of standing timber, and percent overstory cover and woodland salamander density are often strongly positively correlated (Tilghman et al., 2012).
The effect of overstory cover reduction on woodland salamanders is often immediate (Knapp et al., 2003; Hocking et al., 2013b; Mossman et al., 2019).
Tree cutting results in drier and hotter conditions at the soil surface that are often not compatible with salamander physiology (Hillman et al., 2009; Homyack et al., 2011).
Margenau 2023 : However, overstory cover did not have a strong influence on RBS counts in cut-back borders or regeneration stands in our study. This could indicate that both overstory reduction mitigation techniques that we assessed may have counteracted the negative effects of overstory removal, either by providing downed CWD or eliminating soil compaction.
However, it is unlikely that overstory cover alone drives woodland salamander populations following tree cutting (Otto et al., 2014).
Eastern red-backed salamander surface counts were not significantly reduced in cut-back border treatments following cutting relative to the control. The lack of responses we observed to treatments may be due to the retention of downed CWD following cutting as we detected a positive response to number of natural cover objects. Downed CWD can increase habitat quality for woodland salamanders by creating surface refugia that moderates microclimates at the forest floor (Otto et al., 2013; Peele et al., 2017).
These conditions may allow individuals to remain at the surface for extended periods of time.

CHaudhary : impact negative des coupes total sur les amphibians

Des études antérieures dans les forêts de feuillus de l'est de l'Amérique du Nord et de la région des Appalaches ont montré qu'une plus grande rétention de la couverture de la canopée grâce à des approches de 
récolte partielle, telles que la gestion inéquienne, pouvait entraîner des densités relatives plus élevées de salamandres par rapport aux traitements équiens qui enlèvent une plus grande partie de la canopée \citep{Hocking2013Effectsexperimental,Harper2015Impactforestry,Mahoney2016Woodlandsalamander}. 
Une plus grande rétention de la canopée limite la quantité de rayonnement solaire atteignant le sol forestier, ce qui conserve des microclimats plus humides et plus frais favorable pour la physiologie des salamandres \citep{Homyack2011Energeticssurfaceactive}


% effet des coupes sur carabes
However, invertebrate populations, the primary food source for woodland salamanders, generally decrease immediately following overstory removal (Perry and Herms, 2016) due to changes in forest floor temperature and moisture (Gray et al., 2002).
esting. Coleoptera do not respond ( abundance and speacies richness) kudrin

Nous n'avons pas observé de réduction significative de la richesse de la faune du sol. 
Cependant, l'absence de changements dans la richesse ne signifie pas nécessairement qu'il n'y a pas de modifications dans la composition taxonomique. 
Par exemple, les espèces de coléoptères Carabidae associées aux forêts à couvert fermé peuvent diminuer ou même disparaître des parcelles coupées à blanc, 
tandis que les espèces associées aux milieux ouverts les colonisent rapidement \citep{Kudrin2023metaanalysiseffects}.
The meta-regression analysis shows that the types of harvesting and forest can modify the soil fauna responses to forest harvesting. Among the groups included in the analysis, Collembola, Oribatida, Coleoptera, and Araneae changed their response to harvesting (Table 1). The abundance of Collembola and Coleoptera was negatively affected by clear-cutting but not by partial cutting (Figure 4). The richness of these groups positively responded to clear-cutting, but in the case of Collembola, such an effect was not statistically significant. Partial cutting leads to an increase in Collembola richness but not in Coleoptera (Figure 4). Collembola and Oribatida decreased their abundance after harvesting in coniferous forests, while in deciduous and mixed forests, they did not change (Figure 5) kudrin

We did not observe a significant reduction in soil fauna richness, which suggests the possibility of resistance [27] and/or the activation of mechanisms that promote the restoration and compensation of lost taxa by colonizing species [28,29]. This finding is consistent with the results of a meta-analysis of forest management impacts on species richness, which revealed a weak effect of clear-cutting on arthropods [11] kudrin

While the abundance of springtails and beetles decreases after clear-cutting, it does not change in response to partial cutting, which is consistent with the conventional idea of the less dramatic effect of this harvesting practice. The observed increase in Coleoptera richness in response to clear-cutting, as opposed to partial cutting (Figure 4), essentially also reflects perturbative impact. The formation of new habitats after clear-cutting leads to the colonization of new Coleoptera species, which creates this effect [30,31]. kudrin

% différence entre les groupe de carabes pour l'effet des coupes
  importance de la taille des indivius face au déclin

  The effect of harvesting is dependent on body size (abundance Qm = 22.5; p < 0.001 and richness Qm = 7.4; p = 0.006); the negative effect is increased with a decrease in the size group. Feeding type also impacts on the effects of harvesting (abundance Qm = 0.1; p = 0.892 and richness Qm = 6.3; p = 0.012) but only in the case of richness (Figure 3). kudrin

  Body size is a fundamental trait that determines numerous physiological and life history parameters of an organism [39–41]. The conducted meta-analysis shows that as the size of invertebrates decreases, the negative impact of harvesting on soil fauna becomes more pronounced (Figure 3). In our opinion, one of the reasons for this dependence may be the limited dispersal ability of small-bodied soil fauna [42–44]. Small soil invertebrates have a reduced ability to actively avoid unfavorable points and/or colonize disturbed plots from adjacent undisturbed areas or refuges compared to actively moving groups of macrofauna [45–47]. For instance, oribatid mites are highly limited in their dispersal ability, even over distances of only several centimeters [48], whereas ground beetles are able to cover hundreds of meters [49]. kudrin

% effet des coupes sur les collemboles
However, invertebrate populations, the primary food source for woodland salamanders, generally decrease immediately following overstory removal (Perry and Herms, 2016) due to changes in forest floor temperature and moisture (Gray et al., 2002).

the effect of harvesting on abundance and richness differs among faunal groups. According to the overall effect, the number of Nematoda -42\% Collembola -24\%, kudrin

The meta-regression analysis shows that the types of harvesting and forest can modify the soil fauna responses to forest harvesting. Among the groups included in the analysis, Collembola, Oribatida, Coleoptera, and Araneae changed their response to harvesting (Table 1). The abundance of Collembola and Coleoptera was negatively affected by clear-cutting but not by partial cutting (Figure 4). The richness of these groups positively responded to clear-cutting, but in the case of Collembola, such an effect was not statistically significant. Partial cutting leads to an increase in Collembola richness but not in Coleoptera (Figure 4). Collembola and Oribatida decreased their abundance after harvesting in coniferous forests, while in deciduous and mixed forests, they did not change (Figure 5) kudrin

Another fundamental trait is trophic specialization. The conducted meta-analysis did not show differences in the response of the abundance of predators and decomposers (Figure 3). However, the differences may lie at lower levels of trophic specialization (e.g., lichen feeders, fungal feeders, bacterial feeders, etc.). It is worth noting that the reduction of available food resources may be one of the reasons for the decrease in the abundance of certain groups of soil fauna in clear-cut areas. The significant reduction of fungal biomass [50], and especially mycorrhizal mycelium [51,52], is likely the reason for the decrease in the number of oribatids and fungivorous nematodes in harvesting plots [12,53], which are trophically associated with this resource [54,55]. The conducted meta-analysis indicates that the richness of predatory invertebrates responds positively to harvesting. This trend is mainly due to actively moving Coleoptera and Araneae, which can increase their richness in harvesting sites through active colonization of newly formed habitats [30,31]. kudrin 

While the abundance of springtails and beetles decreases after clear-cutting, it does not change in response to partial cutting, which is consistent with the conventional idea of the less dramatic effect of this harvesting practice. The observed increase in Coleoptera richness in response to clear-cutting, as opposed to partial cutting (Figure 4), essentially also reflects perturbative impact. The formation of new habitats after clear-cutting leads to the colonization of new Coleoptera species, which creates this effect [30,31]. kudrin


% effet des salamandres sur les petits carabes
% effet des salamandres sur les collemboles

% effet des carabes sur les collemboles

Overall le transfer de perturbation a travers la chaine trophique

The various reactions of soil invertebrates to disturbances may be attributed to their functional traits [34,35]. Recent works suggest that focusing on functional traits can provide greater insights into the mechanisms driving ecosystem change and recovery [36–38]. kudrin

Slight changes in the abundance and richness of some soil invertebrate groups in response to partial cutting suggest the benefits of this harvesting practice for the conservation of soil animal communities. However, even partial cutting can lead to disturbances and changes in soil fauna.kudrin

% Quels sont les defaults ou amelioration du projet, ainsi que ses limitations.
  % une plus longue prériode d'échantillonnage
  limited number of sampling periods
  limitation dans la complexité du des analyses SEM
  % limitation dans le nombre d'individus pour les salamandres
    intervalle de confiance
  % essaie d,autre methodes
  reproduire l,étude avec de l'abondance fournira une compréhention approfondit mais ne pouvez pas etre utilisé ici en raison de la recolte des individus
  contrainte limitant de population fermé
  abondance au lieu d'occupation nécéssite ( carabes et collemboles), CMR pour les salamandres
  s'interresser au trait fonctionnel 

  % choix de classification des groupes
  trait fonctionnel a la place de taxonomique


% ouverture
  % tester un effet bottom up a la place de top down
  % Comparer avec d'autres types de foret (spatiale)
  répéter le type d'étude dans d'autre type de foret, conifère, feuillu. (kudrin)
  Our analysis revealed a significantly weaker effect of harvesting on the abundance of Collembola and Oribatida in deciduous and mixed forests compared to coniferous forests (Figure 5). The negative response of soil fauna to harvesting may be due to significant changes in abiotic conditions [12]. Several studies have documented significant alterations in temperature, soil moisture, soil compaction, and the quality and thickness of forest litter resulting from harvesting coniferous forests [59–61]. kudrin.

  % Faire de étude à plus court terme mais surtout a plus long terme (temporel)
  we have limited knowledge of how quickly soil animals react to harvesting and what the rate of their recovery is.
  Despite the relatively large datasets on the abundance of Collembola and Coleoptera, no temporal dependence of the effect size over twenty years was found. kudrin


  Several mechanisms may explain the effect of management on forest biodiversity: changes in tree age structure, vertical stratification, and composition of tree species, which affect light, temperature, moisture, litter, and topsoil conditions (Sebastia et al. 2005; Standovar et al. 2006); presence of microhabitats (e.g., dead wood, veteran trees, cavities, root plates) specific to unmanaged (Berg et al. 1994; Bouget 2005a; Christensen et al. 2005; Gibb et al. 2005) or managed forests (e.g., skid trails and haul roads) (Hansen et al. 1991; Gosselin 2004); and forest cover continuity and features resulting from extensive management in the past (Hjalten et al. 2007). The pattern of response may therefore depend on which of the above mechanisms, or which combinations of them, have the strongest effects on different taxonomic or functional groups. paillet 

  kudrin :
Les forêts de conifères et les forêts de feuillus, par exemple, sont très différentes en termes de conditions pédologiques et microclimatiques, ce qui se reflète dans la dissimilarité de la composition et de la structure de la faune du sol [14,19,20].


\section*{Conclusion}
\label{sec:conclu1}
\phantomsection\addcontentsline{toc}{section}{\nameref{sec:conclu1}}

peu d'étude se sont interréssé au interaction entre plusieurs groupe tel que les amphibien et les arthropodes, il sont souvent étudié séparément.
importance des effet cumulé
importance de relation trophique lors d'étude de perturbation

\section*{Acknowledgements}
\label{sec:acknowl1}
\phantomsection\addcontentsline{toc}{section}{\nameref{sec:acknowl1}}

\section*{Conflict of interest}
\label{sec:conflict1}
\phantomsection\addcontentsline{toc}{section}{\nameref{sec:conflict1}}

None declared
\section*{Author contributions}
\label{sec:author1}
\phantomsection\addcontentsline{toc}{section}{\nameref{sec:author1}}

\cleardoublepage

\begin{otherlanguage}{english}
\bibliographystyle{ecologyNewEN} % Style de citation en français
\bibliography{References}
\addcontentsline{toc}{section}{References}
\end{otherlanguage}
