\chapter{Direct and indirect effects on soil fauna of site preparation in the context of forest assisted migration}     % numéroté
\label{chapitre1-articles}    

William Devos$^1$, Mathieu Bouchard$^1$, Marc Mazerolle$^1$

%href{mailto:william.devos.1@ulaval.ca}
$^1$ Centre d'étude de la forêt, Département des sciences du bois \\ 
et de la forêt, Université Laval, Québec, QC G1V 0A6, Canada. \\ 

\clearpage

\section*{Résumé}
\label{sec:resume1}
\phantomsection\addcontentsline{toc}{section}{\nameref{sec:resume1}}

\begin{otherlanguage*}{french}
  <Résumé de l'article en français. Obligatoire.>

  \textbf{Mots-clés} : <ajouter des mots clés>
\end{otherlanguage*}

\clearpage

\section*{Abstract}
\label{sec:abstract1}
\phantomsection\addcontentsline{toc}{section}{\nameref{sec:abstract1}}

\begin{otherlanguage*}{english}
  <English abstract of the paper. Optional, but recommended.>

\textbf{Keywords}: <add some keywords> 
\end{otherlanguage*}

\cleardoublepage

\section*{Introduction}
\label{sec:intro1}
\phantomsection\addcontentsline{toc}{section}{\nameref{sec:intro1}}

%\defcitealias{keylist}{alias}


We combined occupancy and structural equation models to assess the direct and indirect impacts of site preparation, in a assisted tree migration context, 
on soil fauna cooccurence, while accounting for environmental disturbance in the mixedwood forest.
We first hypothesized that forest harvesting treatments alter habitat use of soil fauna propagating through the trophic network (Figure \ref*{fig:SEM}). 
Thus, the harvests initially affect habitat use by salamanders and large carabids and subsequently alter habitat selection for small carabids and, finally, springtails.
Our second hypothesis suggested that forest harvesting associated with site preparation influences directly environmental variables such as litter depth, canopy openness, and coarse woody debris (CWD). 
We predicted that larger forest cuts result in decreased litter depth and CWD as a consequence of reduced accumulation of leaves and woody debris on the forest floor.
Conversely, we predicted that site preparation following clearcut reduces canopy cover.

\section*{Material and methods}
\label{sec:matmet1}
\phantomsection\addcontentsline{toc}{section}{\nameref{sec:matmet1}}

\subsection*{Study area}
\label{subsec:area}
\phantomsection\addcontentsline{toc}{subsection}{\nameref{subsec:area}}

\begin{otherlanguage*}{english}
  Our study was conducted within the Portneuf Wildlife Reserve in the Captiale-Nationale administrative region near Lac des Amanites and Rivière-à-Pierre (47°07'N, 72°24'W, Figure \ref{fig:area}). 
  This area is located within the balsam fir (\textit{Abies balsamea})-yellow birch (\textit{Betula alleghaniensis}) bioclimatic domain \citep{saucierChapitreEcologieForestiere2009}.
  Other tree species include sugar maple (\textit{Acer saccharum}), red maple (\textit{Acer rubrum}), white spruce (\textit{Picea glauca}), black spruce (\textit{Picea mariana}), red spruce (\textit{Picea rubens}), white birch (\textit{Betula papyrifera}) and quaking aspen (\textit{Populus tremuloides}) \citep{olaBelowgroundCarbonStocks2024}. 
  Study sites rest on a deep glacial till as surface deposit with a moderately well-drained sandy loams soil \citep{CanadianSystemSoil1998}.
  The mean daily temperature is 4\up{o}C based on the 1981-2010 period at the nearest weather station (Lac aux sables, \citealp{environmentcanadaCanadianClimateNormals2019}). 
  Based on the same report, the mean annual precipitation and snowfall are 1133.2 mm and 230.3 cm, respectively.
  We used the assisted migration experimental system established in 2018 by the Ministère des Ressources naturelles et des Forêts to collect our data (\citealp{royoDesiredREgenerationAssisted2023}).
  This system is a factorial experimental design with a split-plots replicated in four blocks. 
  Each whole block (200 m x 140 m) is split in two overstory treatment : clear-cut and partial-cut. 
  See \cite{royoDesiredREgenerationAssisted2023} for more details about the assisted migration experiment.

\end{otherlanguage*}

\begin{figure}[ht!]
	\centering
	\includegraphics[scale=0.60]{fig_area4.png}
	\caption[Localisation of the Captiale-Nationale administrative region in Quebec, Canada and position of the study area near Lac des Amanites in Portneuf Wildlife Reserve, Quebec, Canada.]
  {Localisation of the Captial-Nationale administrative region in Quebec, Canada (A) and position of the study area near Lac des Amanites in Portneuf Wildlife Reserve, Quebec, Canada (B) where the assisted migration experimental system was implemented in 2018 (47°07'N, 72°24'W).}
	\label{fig:area}
	\end{figure}  



\subsection*{Sampling design}
\label{subsec:sampling}
\phantomsection\addcontentsline{toc}{subsection}{\nameref{subsec:sampling}}


We selected a total of 60 sampling units measuring 10 meters by 7.5 meters each to collect our data: we used four blocks serving as replicates, 
with each block containing six sampling units for both the clear-cut and partial-cut overstory treatments, 
while three additional sampling units were positioned outside the block, serving as controls (Figure \ref*{fig:blockSU}).
Sampling units serving as controls were separated from blocks by at least 10 meters to remove treatments effects.
In each sampling unit, we used three sampling methods to collect species data: artificial coverboards, pitfall traps, and soil cores. 

Artificial coverboards is commonly employed to count salamanders \citep{hesedUncoveringSalamanderEcology2012,mazerolleWoodlandSalamanderPopulation2021a,mooreComparisonPopulationEastern2009c}. 
This method helps to standardize the number and size of ground objects sampled, thereby reducing variability \citep{hydeSamplingPlethodontidSalamanders2001}. 
Coverboards were made of spruce wood, measured 25 cm x 30 cm x 5 cm and were placed directly on the ground without litter underneath \citep{mazerolleWoodlandSalamanderPopulation2021a}. 
Six coverboards were set per sample unit and spaced by at least 2.5 m, resulting in a total of 360 coverboards.
Two rows of three boards each were aligned along the length of the sampling units and centrally positioned (Figure \ref{fig:blockSU}).
All coverboards were placed outdoors in March 2022 to allow for natural aging, increasing the likelihood of salamander usage \citep{hedrickEffectsCoverboardAge2021,smithEffectsCoverBoard2014a}.
Throughout the surveys, coverboards were inspected once on the same day, and salamanders were counted without any manipulation.

Pitfall traps were used to capture carabids.
This method is commonly employed to captured ground-dwelling invertebrates, such as carabids \citep{baarsCatchesPitfallTraps1979,knappEffectPitfallTrap2012,kotzeFortyYearsCarabid2011a,loveiEcologyBehaviorGround1996,spenceSamplingCarabidAssemblages1994a}. 
Pitfall traps were Multipher\up{\textregistered{}} traps and included a container with a diameter of 12.5 cm, a depth of 25 cm and a cover raised 4.5 cm above the trap 
to prevent debris and rain from filling the container \citep{bouchardBeetleCommunityResponse2016b,mooreEffectsTwoSilvicultural2004}.
They were equipped with a protective grid with a mesh size of 15 mm, limiting trap access to carabid-sized individuals and reducing the chances of predation by small mammals. 
Typically, a preserving liquid (such as propylene glycol or alcohol) is placed in the bottom of the container to preserve captured individuals. 
However, salamanders can have a width comparable to that of certain carabids, and therefore get captured and drown in the traps. 
For ethical reasons and considering the difficulties to limit access to salamanders without influencing the sampling of carabids, we decided to use dry pitfall traps without any preserving liquid \citep{luffFeaturesInfluencingEfficiency1975}. 
We also add wet sponges to the bottom of each container to maintain a suitable level of humidity for salamanders.
We centered one pitfall in each sampling and control units, resulting in a total of 60 pitfall traps (Figure \ref{fig:blockSU}). 
Traps were inserted in the soil at a depth allowing the container's opening to be juxtaposed with the soil surface. 
Outside the surveys, all pitfall traps were closed with adhesive tape around the opening to prevent individuals capture. 
On the first day of each survey, traps were opened and carabids captured were collected for each remaining day. 
Individuals were preserved in 70\% alcohol and identified at the species level afterwards.
Identification was conducted with a ZEISS SteREO Discovery.V12 bionocular microscope using \cite{larochelleManuelIdentificationCarabidae1976} identification keys.
Carabids were categorized in two groups as salamander prey or competitors based on the salamander gape size (Table \ref{tab:carabid}, \citealp{jaegerFoodLimitedResource1972,magliaModulationPreycaptureBehavior1995,magliaOntogenyFeedingEcology1996}).

Soil cores was used to sample mesofauna in litter and different soil horizons \citep{chauvatChangesSoilFaunal2011a,farskaManagementIntensityAffects2014,pongeVerticalDistributionCollembola2000,salamonEffectsPlantDiversity2004,wuCompositionSpatiotemporalVariation2014}. 
Two soil cores with litter collection were harvested per sampling unit per survey using a soil sampling pedological probe. 
Cores had a diameter of 5 cm, a depth of 5 cm and a 15 cm x 15 cm litter quadrat was collected above each soil sample \citep{raymond-leonardSpringtailCommunityStructure2018a,rousseauForestFloorMesofauna2018}.
Both substrates were used to target mesofauna and obtain springtail communities directly related to the ecology of salamanders and carabids \citep{chauvatChangesSoilFaunal2011a,edwardsAssessmentPopulationsSoilinhabiting1991,raymond-leonardSpringtailCommunityStructure2018a,rousseauForestFloorMesofauna2018}.
Soil and litter from the same unit were pooled in Ziploc\up{\texttrademark{}} bags and stored in a cooler at $\pm$ 4°C \citep{chauvatChangesSoilFaunal2011a,rousseauForestFloorMesofauna2018}, providing 60 samples per survey.
Each sample was placed in an individual Tullgren dry-funnel for springtail extraction within 48 h of harvest \citep{rousseauForestFloorMesofauna2018,rusekBiodiversityCollembolaTheir1998,wuCompositionSpatiotemporalVariation2014}. 
The extraction process lasted six days with a gradual temperature escalation (25°C to 50°C) \citep{raymond-leonardSpringtailCommunityStructure2018a}.
Springtails were preserved in 75\% alcohol \citep{wuCompositionSpatiotemporalVariation2014} prior to isolation from surrounding organisms, with subsequent identification at the family level.
Identification was done with a ZEISS SteREO Discovery.V12 binocular microscope and a Leitz orthoplan phase-contrast fluorescent trinocular microscope using \cite{bellingerChecklistCollembolaWorld1996} identification keys.
Springtail dry biomass of each sampled was measured with micro balance (Sartorius Cubis\up{\texttrademark{}} MSA3.6P-000-DM, city, state, country) after being lyophilized (Labconco FreeZone Bulk tray dryer 78060 series, city, state, country).

We conducted four survey of five consecutive days each, namely in mid-May, mid-June, mid-July, and mid-August, during the summer 2022.
Blocks were visited in a random sequence to reduce effects of time of day and observer fatigue.

\pagebreak

\begin{figure}[ht]
	\centering
	\includegraphics[scale=0.50]{fig_blockSU2.png}
	\caption[Design of one block and one sampling unit with three sampling methods.]{
  Design of a block (left) and a sampling unit (right). 
  The block contain two overstory treatments : clear cut (grey background), partial cut (white background). 
  Fifteen sampling units were used per block : six per overstory treatment and three controls (\textbf{c}) outside each block.
  Each sampling unit contained six artificial coverboards (rhombus) and one pitfall trap (triangles). Two soil cores (circles) were collected per survey.
  }
	\label{fig:blockSU}
	\end{figure}  

  \vspace{0.5cm}


\subsection*{Environmental variables}
\label{subsec:EnvVar}
\phantomsection\addcontentsline{toc}{subsection}{\nameref{subsec:EnvVar}}

In each sampling unit, we measured several environmental variables that could affect occupancy probability.
CWD and litter depth play a crucial role in habitat use for salamanders, carabids and springtails as
they serve for protection, feeding and maintain suitable temperature and humidity \citep{birdChangesSoilLitter2004,groverInfluenceCoverMoisture1998a,harmonEcologyCoarseWoody1986,koivula.LeafLitterSmallscale1999,mckennyEffectsStructuralComplexity2006,patrickEffectsExperimentalForestry2006a}. \\
We used 400 m\up2 plots centered inside every sampling unit to estimate CWD (20 m $\times$  20 m)(\citealp{methotGuideInventaireEchantillonnage2014}). 
We only selected CWD with a basal diameter greater than or equal to 9 cm and a length greater than or equal to 1 m.
Subsequently, we measured the basal diameter, the apical diameter, and length of CWD with a tree caliper.
Segments of CWD outside the plot boundaries were not considered.
We employed the conic–paraboloid formula to estimate the volume of each CWD \citep{fraverRefiningVolumeEstimates2007} :

\begin{equation}
  \text{Volume} = L/12 \times (5A_b + 5A_u + 2\sqrt{A_b \times A_u})
\end{equation}

\vspace{0.5cm}

Where $L$ is the length of logs (cm), $A_b$ the basal area (cm\up{2}), and $A_u$ the apical area (cm\up{2}).
We measured litter depth next to each coverboard and estimated the mean depth per sampling unit \citep{mazerolleWoodlandSalamanderPopulation2021a}. \\
We also assessed canopy openness, as it may influence species occupancy \citep{henneronForestPlantCommunity2017,koivulaBorealCarabidbeetleColeoptera2002a,kotzeFortyYearsCarabid2011a,messereForestFloorDistribution1998,tilghmanMetaanalysisEffectsCanopy2012}.
Measurements were conducted at the center of all sampling units using a spherical densiometer \citep{lemmonSphericalDensiometerEstimating1956} and were taken 130 cm above the ground. 
We took four measurements per sampling unit, oriented toward each of the four cardinal points, and computed the mean as an estimate of canopy openness in each sampling unit.

We collected data for air temperature, air humidity and precipitation levels during the summer 2022.

(sources + décrire)

Two compact weather stations (Em50 Digital Decagon Data Logger, Part \#40800, Meter Group Inc., USA), were used inside both overstory treatments.
Each weather station was equipped with a probe measuring temperature, air humidity, and atmospheric pressure, 1.30 m above the ground (VP-4 Sensor (Temp/RH/Barometer), Part \#40023). 
Rain gauges were installed in the clear-cut treatments to monitor precipitation levels.
The temperature and humidity sensors were programmed for record data every 15 minutes. 
We used the means of both weather stations to get daily average measurements.
These variables fluctuate on a daily basis, affecting species activity and, consequently, the probability of detecting individuals. 
\citep{butterfieldCarabidLifeCycle1996,kotzeFortyYearsCarabid2011a,loveiEcologyBehaviorGround1996,odonnellPredictingVariationMicrohabitat2014a,spotilaRoleTemperatureWater1972}.

\subsection*{Statistical analyses}
\label{subsec:analyses}
\phantomsection\addcontentsline{toc}{subsection}{\nameref{subsec:analyses}} 

\subsubsection{Structural equations models} 

To assess the effects of overstory treatments on the site occupancy (hypothesis 1.1) and environmental variables (hypothesis 1.2), we used a structural equation models (SEM) 
combining occupancy and linear mixed models \citep{graceSpecificationStructuralEquation2010,josephIntegratingOccupancyModels2016,mackenzieOccupancyEstimationModeling2006a}.
SEM are employed to analyse complex relationships among observed and latent variables, allowing to investigate ecological processes \citep{graceStructuralEquationModeling2008}.
Structural equation models use multiples equations to describe the relationships between variables and estimate parameters.
We used this framework to measure direct and indirect processes that influence species occurrence by considering interrelationships between treatments, taxa and environmental variables (Figure \ref*{fig:SEM}).

\begin{figure}[ht]
	\centering
	\includegraphics[scale=0.55]{fig_sem.png}
	\caption[Theoretical model illustrating the anticipated relationships between overstory treatments, environmental variables and species groups.]
  {Theoretical model illustrating the anticipated relationships between overstory treatments, coarse woody debris volume, canopy opensess, litter depth,
   salamanders occupancy, carabids occupancy, and springtail biomass in the Portneuf Wildlife Reserve, Quebec, Canada. 
   Each arrow indicates the direction of a potential effect, from an explanatory variable to a response variable.}
	\label{fig:SEM}
\end{figure}  

\subsubsection{Occupancy models} 

We combined occupancy models to SEM to assess the impact of overstory treatments on the occupancy probabilities of salamanders and carabids.
Occupancy models allow us to account for imperfect detection of species since salamanders and carabids are cryptic species \citep{baileyEstimatingSiteOccupancy2004,spiersEstimatingSpeciesMisclassification2022}.
Occupancy modeling uses repeated surveys to estimate detection probabilities and estimate the species true occupancy \citep{mackenzieEstimatingSiteOccupancy2002,mazerolleMakingGreatLeaps2007}.

To estimate the occupancy probabilities of salamanders and carabids, we used the observed detection at site $i$ during survey $j$, 
represented by $\text{Salamander}_{ij}$, $\text{Carabid.comp}_{j}$ and $\text{Carabid.prey}_{ij}$. Observed detection followed a Bernoulli distribution with $z_{i} \times p_{ij}$ as parameter, 
where $z_{i}$ represents the latent occupancy state of species groups at a site $i$ and $p_{ij}$ represents the probability of detecting species groups at a site $i$ during a survey $j$ :


\begin{align}
  \text{Salamander}_{ij} &\sim \text{Bernoulli}(z_{\text{Salamander}_i} \times p_{\text{Salamander}_{ij}}) \nonumber \\
  \text{Carabid.comp}_{ij} &\sim \text{Bernoulli}(z_{\text{Carabid.comp}_i} \times p_{\text{Carabid.comp}_{ij}})  \\
  \text{Carabid.prey}_{ij} &\sim \text{Bernoulli}(z_{\text{Carabid.prey}_i} \times p_{\text{Carabid.prey}_{ij}}) \nonumber
\end{align}


$z_{i}$ follows a Bernoulli distribution with $\psi$ as parameter, 
which represent the occupancy probability of each species group. 
The latent occupancy state $z$ indicates whether a site is occupied by a species group ($z = 1$) or not ($z = 0$) :


\begin{align}
  z_{\text{Salamander}_i} &\sim \text{Bernoulli}(\psi_{\text{Salamander}_{\text{Group}_i}}) \nonumber \\
  z_{\text{Carabid.comp}_i} &\sim \text{Bernoulli}(\psi_{\text{Carabid.comp}_{\text{Group}_i}}) \\
  z_{\text{Carabid.prey}_i} &\sim \text{Bernoulli}(\psi_{\text{Carabid.prey}_{i}}) \nonumber
\end{align}


Salamanders and large carabids occupancy probabilities for each overstory groups $i$ ($\psi_{\text{Salamander}_{\text{Group}_i}}$, $\psi_{\text{Carabid.comp}_{\text{Group}_i}}$ ) follow a uniform distribution $U(0,1)$ 
whereas small carabids occupancy probabilities ($\psi_{\text{Carabid.prey}_{i}}$) were estimated from a linear predictor with salamander latent occupancy state ($z_{\text{Salamander}_i}$) and overstory treatments 
($\text{Cutpartial}_i$, $\text{Cutclear}_i$) as parameters on a logit scale. 
$\beta$s represent the coefficients associated with each explanatory variables affecting occupancy in the linear predictor :


\begin{align}
  \text{logit}(\psi_{\text{Carabid.prey}_i}) &= \beta_{0[\text{Carabid.prey}]} + \beta_{z_{\text{Salamander}}[\text{Carabid.prey}]} \times z_{\text{Salamander}_i} + \nonumber \\
  &\beta_{\text{Cutpartial}[\text{Carabid.prey}]} \times \text{Cutpartial}_i + \\
  &\beta_{\text{Cutclear}[\text{Carabid.prey}]} \times \text{Cutclear}_i \nonumber
\end{align}

We applied a logit scale on parameters influencing the probability of detecting species groups ($\text{logit}(p_{ij})$), allowing us to use 
the coarse woody debris ($\text{CWD}_i$), the precipitation levels ($\text{Precipitation}_{ij}$) and a block random effect ($\alpha_{Block}$) as explanatory variables. 
Air temperature and relative humidity were excluded from detection variables due to a high correlation with the precipitation level.
$\alpha$s represent the coefficients associated with each explanatory variables affecting detection in the linear predictor : 


\begin{align}
  \text{logit}(p_{\text{Salamander}_{ij}}) &= \alpha_{0[\text{Salamander}]} + \alpha_{\text{CWD}[\text{Salamander}]} \times \text{CWD}_i + \nonumber \\
  &\alpha_{\text{Precipitation}[\text{Salamander}]} \times \text{Precipitation}_{ij} + \alpha_{\text{Block}[\text{Salamander}]_{\text{Block}_i}} \nonumber
\end{align}

\begin{align}
  \text{logit}(p_{\text{Carabid.comp}_{ij}}) &= \alpha_{0[\text{Carabid.comp}]} + \alpha_{\text{CWD}[\text{Carabid.comp}]} \times \text{CWD}_i + \\
  &\alpha_{\text{Precipitation}[\text{Carabid.comp}]} \times \text{Precipitation}_{ij} + \alpha_{\text{Block}[\text{Carabid.comp}]_{\text{Block}_i}} \nonumber 
\end{align}

\begin{align}
  \text{logit}(p_{\text{Carabid.prey}_{ij}}) &= \alpha_{0[\text{Carabid.prey}]} + \alpha_{\text{CWD}[\text{Carabid.prey}]} \times \text{CWD}_i + \nonumber \\
  &\alpha_{\text{Precipitation}[\text{Carabid.prey}]} \times \text{Precipitation}_{ij} + \alpha_{\text{Block}[\text{Carabid.prey}]_{\text{Block}_i}} \nonumber 
\end{align}

%%%%%%%%%%%%%%%%%%%
%%%%%%%%%%%%%%%%%%% fin


% \begin{align}
%   z_{\text{Salamander}_i} &\sim \text{Bernoulli}(\psi_{\text{Salamander}_{\text{Group}_i}}) \nonumber \\
%   \text{logit}(p_{\text{Salamander}_{ij}}) &= 
%   \alpha_{0[\text{Salamander}]} + \alpha_{\text{CWD}[\text{Salamander}]} \times \text{CWD}_i + \\
%   &\alpha_{\text{Precipitation}[\text{Salamander}]} \times \text{Precipitation}_{ij} + \alpha_{\text{Block}[\text{Salamander}]_{\text{Block}_i}} \nonumber \\
%   \text{Salamander}_{ij} &\sim \text{Bernoulli}(z_{\text{Salamander}_i} \times p_{\text{Salamander}_{ij}}) \nonumber
% \end{align} 

% \begin{align}
% z_{\text{Salamander}_i} &\sim 
% \text{Bernoulli}(\psi_{\text{Salamander}_{\text{Group}_i}}) \nonumber \\
% \text{logit}(p_{\text{Salamander}_{ij}}) &= 
% \alpha_{0[\text{Salamander}]} + \alpha_{\text{CWD}[\text{Salamander}]} \times \text{CWD}_i + \\
% &\alpha_{\text{Precipitation}[\text{Salamander}]} \times \text{Precipitation}_{ij} + \alpha_{\text{Block}[\text{Salamander}]_{\text{Block}_i}} \nonumber \\
% \text{Salamander}_{ij} &\sim 
% \text{Bernoulli}(z_{\text{Salamander}_i} \times p_{\text{Salamander}_{ij}}) \nonumber
% \end{align} 

% \begin{align}
%   z_{\text{Carabid.comp}_i} &\sim 
%   \text{Bernoulli}(\psi_{\text{Carabid.comp}_{\text{Group}_i}}) \nonumber \\
%   \text{logit}(p_{\text{Carabid.comp}_{ij}}) &= 
%   \alpha_{0[\text{Carabid.comp}]} + \alpha_{\text{CWD}[\text{Carabid.comp}]} \times \text{CWD}_i + \\
%   &\alpha_{\text{Precipitation}[\text{Carabid.comp}]} \times \text{Precipitation}_{ij} + \alpha_{\text{Block}[\text{Carabid.comp}]_{\text{Block}_i}} \nonumber \\
%   \text{Carabid.comp}_{ij} &\sim \text{Bernoulli}(z_{\text{Carabid.comp}_i} \times p_{\text{Carabid.comp}_{ij}}) \nonumber
%   \end{align}

% \begin{align}
%   \text{logit}(\psi_{\text{Carabid.prey}_i}) &= 
%   \beta_{0[\text{Carabid.prey}]} + \beta_{z_{\text{Salamander}}[\text{Carabid.prey}]} \times z_{\text{Salamander}_i} + \nonumber \\
%   &\beta_{\text{Cutpartial}[\text{Carabid.prey}]} \times \text{Cutpartial}_i + \beta_{\text{Cutclear}[\text{Carabid.prey}]} \times \text{Cutclear}_i \nonumber\\
%   z_{\text{Carabid.prey}_i} &\sim 
%   \text{Bernoulli}(\psi_{\text{Carabid.prey}_i}) \nonumber \\
%   \text{logit}(p_{\text{Carabid.prey}_{ij}}) &= \alpha_{0[\text{Carabid.prey}]} + \alpha_{\text{CWD}[\text{Carabid.prey}]} \times \text{CWD}_i +  \\
%   &\alpha_{\text{Precipitation}[\text{Carabid.prey}]} \times \text{Precipitation}_{ij} + \alpha_{\text{Block}[\text{Carabid.prey}]_{\text{Block}_i}} \nonumber \\
%   \text{Carabid.prey}_{ij} &\sim \text{Bernoulli}(z_{\text{Carabid.prey}_i} \times p_{\text{Carabid.prey}_{ij}}) \nonumber
% \end{align}

\subsubsection{Linear mixed models} 

We used linear mixed models to assess how overstory treatments affect springtail biomass ($\text{Springtail}_{i}$) and 
environmental variables ($\text{CWD}_{i}$, $\text{Canopy}_{i}$, $\text{Litter}_{i}$) at site $i$. 
We employed Normal distributions to estimate posteriors :

\begin{align}
  \text{Springtail}_{i} &\sim \text{N} (\mu_{\text{Springtail}_i}, \sigma_{\text{Springtail}_{\text{Group}_i}}) \nonumber \\
  \text{CWD}_{i} &\sim \text{N} (\mu_{\text{CWD}_i}, \sigma_{\text{CWD}_{\text{Group}_i}}) \\
  \text{Canopy}_{i} &\sim \text{N} (\mu_{\text{Canopy}_i}, \sigma_{\text{Canopy}_{\text{Group}_i}}) \nonumber \\
  \text{Litter}_{i} &\sim \text{N} (\mu_{\text{Litter}_i}, \sigma_{\text{Litter}_{i}}) \nonumber 
\end{align}

Variables means estimations at site $i$ ($\mu_{i}$) included overstory treatments ($\text{Cutpartial}_i$, $\text{Cutclear}_i$) and block random effect ($\alpha_{\text{Block}}$) as linear predictors. 
Estimations of $\mu_{\text{Springtail}_i}$ also accounted for latent occupancy state of salamanders and both carabids groups ($z_{Salamander}$, $z_{SCarabid.comp}$, $z_{Carabid.prey}$ ). 
The $\beta$ coefficients represent the weights assigned to each explanatory in the linear predictor.
Due to heteroscedasticity with $\text{Springtail}$, $\text{CWD}$ and $\text{Canopy}$, we allowed each overstory groups $j$ to have their own variances for those variables ($\sigma_j \sim \text{U}(0,150)$) :

\begin{align}
  \mu_{\text{Springtail}_i} &= \beta_{0[\text{Springtail}]} + \beta_{\text{Cutpartial}[\text{Springtail}]} \times \text{Cutpartial}_i + \nonumber\\
  &\beta_{\text{Cutclear}[Springtail]} \times \text{Cutclear}_i + \beta_{z_{\text{Salamander}}[\text{Springtail}]} \times z_{Salamander} +  \nonumber\\
  &\beta_{z_{\text{Carabid.prey}}[\text{Springtail}]} \times z_{Carabid.prey} + \beta_{z_{\text{Carabid.comp}}[\text{Springtail}]} \times z_{Carabid.comp} + \nonumber\\
  &\alpha_{\text{Block}[\text{Springtail}]_{\text{Block}_i}} \nonumber
\end{align}

\begin{align}
  \mu_{\text{CWD}_i} &= \beta_{0[\text{CWD}]} + \beta_{\text{Cutpartial}[\text{CWD}]} \times \text{Cutpartial}_{i} + \nonumber\\
  & \beta_{\text{Cutclear}[\text{CWD}]} \times \text{Cutclear}_{i} + \alpha_{\text{Block}[\text{CWD}]_{\text{block}_i}} 
\end{align}


\begin{align}
  \mu_{\text{Canopy}_i} &= \beta_{0[\text{Canopy}]} + \beta_{\text{Cutpartial}[\text{Canopy}]} \times \text{Cutpartial}_{i} + \nonumber \\
  & \beta_{\text{Cutclear}[\text{Canopy}]} \times \text{Cutclear}_{i} + \alpha_{\text{Block}[\text{Canopy}]_{\text{block}_i}} \nonumber
\end{align}

\begin{align}
  \mu_{\text{Litter}_i} &= \beta_{0[\text{Litter}]} + \beta_{\text{Cutpartial}[\text{Litter}]} \times \text{Cutpartial}_{i} + \nonumber\\
  & \beta_{\text{Cutclear}[\text{Litter}]} \times \text{Cutclear}_{i} + \alpha_{\text{Block}[\text{Litter}]_{\text{block}_i}} \nonumber
\end{align}


Analysis was performed with a Bayesian approach and parameters were estimated using Markov chain Monte Carlo (MCMC) with JAGS 4.3.0 include in the jagsUI package in R 4.3.1 \citep{lunnBUGSProjectEvolution2009,kellnerJagsUIWrapperRjags2024,rcoreteamLanguageEnvironmentStatistical2020}.
We applied non-informative priors to all parameters by using normal distribution $\text{Normal}(\mu = 0, \sigma^2 = 100$) and uniform distribution $\text{U}(0,1)$ as $\beta$ estimates. 
We ran the model with five chains, 200,000 iterations each \citep{gelmanUnderstandingPredictiveInformation2014}. A burn-in period of 75,000 iterations was used and a thinning rate of 5 was applied. 
We verified the convergence of MCMC chains by examining trace plots, posterior density plots, and applying the Brooks-Gelman-Rubin statistic.

\clearpage



\section*{Results}
\label{sec:results1}
\phantomsection\addcontentsline{toc}{section}{\nameref{sec:results1}}

\subsection*{Environmental variables}
\label{subsec:ResEnv}
\phantomsection\addcontentsline{toc}{subsection}{\nameref{subsec:ResEnv}} 

Environmental variables usually differ between forest cutting treatments and control conditions. We found that total cutting treatments have significantly less CWD compared to partial cutting (Figure \ref{fig:envar} A, Table \ref{tab:overstoryenvar}). 
Canopy openness in the total cut treatment was higher than in partial cuts and treatments (Figure \ref{fig:envar} B, Table \ref{tab:overstoryenvar}). 
Conversely, sites with total cutting show the lowest litter depths, followed by partial cutting treatments, in contrast to control sites (Figure \ref{fig:envar} C, Table \ref{tab:overstoryenvar}).

\vspace{0.5cm}

\begin{figure}[ht]
  \centering
  \includegraphics[scale=0.50]{fig_envar.png}
  \caption[Environmental variables estimations with a potential effect soil species within two different overstory treatments and control.]
  {Environmental variables estimations with a potential effect soil species within two different overstory treatments and control 
  during the summer 2022 in the Portneuf Wildlife Reserve, Quebec, Canada. Error bars denote 95\% credible intervals around estimates.}
  \label{fig:envar}
\end{figure}

\vspace{0.5cm}

\begin{table}[ht]
  \centering
  \caption[Differences between overstory treatments on environmental variables that could effect occupancy of fauna on the forest soil.]
  {Differences between overstory treatments on environmental variables that could effect occupancy of fauna on the forest soil during the summer 2022 in the Portneuf Wildlife Reserve,
  Quebec, Canada.}
  \label{tab:overstoryenvar}
  \begin{tabular}{lllll} 
      \hline
      &&&&95\% Bayesian \\
      Variable&Unit& Comparison & Estimate &  credible interval \\ [0.5ex] 
      \hline
      Coarse woody debris &m\up{3}& Partial vs control & \hspace{1mm}0.02 & [-1.01, 1.06] \\ 
                 && Clear vs control  & -0.77 & [-1.79, 0.23] \\ 
                          && Clear vs partial  & -0.79 & [-1.15, -0.43] \\
      Canopy openness     &\%& Partial vs control & \hspace{1mm}6.49 & [1.97, 11.02] \\ 
                      && Clear vs control  & \hspace{1mm}64.19 & [51.39, 77.06] \\ 
                          && Clear vs partial  & \hspace{1mm}57.69 & [44.61, 70.76] \\ 
      Litter depth        &cm& Partial vs control & -1.54 & [-2.44, -0.65] \\ 
                      && Clear vs control  & -3.39 & [-4.28, -2.50] \\ 
                          && Clear vs partial  & -1.85 & [-2.57, -1.12] \\       
      \hline
      \multicolumn{5}{l}{\textbf{Note:} Estimates from Bayesian SEM are presented in terms of mean number with 95\%} \\
      \multicolumn{5}{l}{credible intervals, where an interval excluding 0 indicates a difference between groups.} \\
  \end{tabular}
\end{table}

\vspace{0.5cm}

\begin{table}[ht]
  \centering
  \caption[Differences between overstory treatments on the forest soil fauna.]
  {Differences between overstory treatments on salamanders occupancy, both carabids groups occupancy and springtail biomass. 
  This table also show estimated effect of salamanders presence on carabids form the salamanders prey group occupancy 
  and the effects of salamanders and both carabids groups presence on springtail biomass, during the summer 2022 in the Portneuf Wildlife Reserve, Quebec, Canada.}
  \label{tab:overstorysp}
  \begin{tabular}{lllll} 
      \hline
      &&&&95\% Bayesian \\
      Variable&Unit& Comparison & Estimate &  credible interval \\ [0.5ex] 
      \hline     
      Salamander occupancy          &probability& Partial vs control & \hspace{1mm}0.07 & [-0.29, 0.45] \\ 
             && Clear vs control  & -0.38 & [-0.75, 0.11] \\ 
                          && Clear vs partial  & -0.45 & [-0.74, -0.07]$^{a}$ \\       
      Carabid$_{competitor}$ occupancy &probability& Partial vs control & -0.12 & [-0.35, 0.15] \\
             && Clear vs control  & -0.06 & [-0.29, 0.20] \\ 
                          && Clear vs partial  & \hspace{1mm}0.06 & [-0.19, 0.30] \\ 
      Carabid$_{prey}$ occupancy    &logit& Partial vs control & \hspace{1mm}3.31 & [-10.12, 17.72] \\
                   && Clear vs control  & \hspace{1mm}10.19 & [-4.15, 24.45] \\ 
                          && Clear vs partial  & \hspace{1mm}6.88 & [-12.81, 23.42] \\  
                          && Salamander        & -2.20 & [-17.15, 16.59] \\  
      Springtail biomass          &logit& Partial vs control & \hspace{1mm}8.11 & [-9.38, 25.40] \\
                   && Clear vs control  & \hspace{1mm}2.11 & [-13.98, 18.11] \\ 
                          && Clear vs partial  & -6.00 & [-29.09, 17.26] \\  
                          && Salamander        & \hspace{1mm}6.80 & [-10.43, 23.26] \\ 
                          && Carabid$_{competitor}$      & \hspace{1mm}0.56 & [-16.75, 17.66] \\ 
                          && Carabid$_{prey}$      & \hspace{1mm}7.62 & [-8.93, 24.09] \\ 
      \hline
      \multicolumn{5}{l}{\textbf{Note:} Estimates from Bayesian SEM are presented in terms of mean number with 95\%} \\
      \multicolumn{5}{l}{credible intervals, where an interval excluding 0 indicates a difference between groups.} \\
      \multicolumn{5}{l}{$^{a}$Marginal difference based on 90\% Bayesian credible interval excluding 0}
  \end{tabular}
\end{table}

\vspace{2ex}

\begin{figure}[ht]
	\centering
	\includegraphics[scale=0.60]{fig_sem_res2.png}
	\caption[Theoretical model illustrating the anticipated relationships between overstory treatments, environmental variables and species groups.]
  {Theoretical model illustrating the anticipated relationships between overstory treatments, coarse woody debris volume, canopy opensess, litter depth,
   salamanders occupancy, carabids occupancy, and springtail biomass in the Portneuf Wildlife Reserve, Quebec, Canada. 
   Each arrow indicates the direction of a potential effect, from an explanatory variable to a response variable.}
	\label{fig:SEMres}
	\end{figure}  

\vspace{0.5cm}



\subsection*{Soil fauna}
\label{subsec:taxa}
\phantomsection\addcontentsline{toc}{subsection}{\nameref{subsec:taxa}} 

During the summer of 2021, we observed a total of 31 salamanders: 17 individuals in the partial cut treatments, 6 in the clearcuts, and 8 in the control sites. 
We captured 189 carabids belonging to 30 species, with 49 carabids found in the partial cut treatments, 105 in the clearcuts, and 35 in the control sites (Table \ref{tab:carabid}). 
We collected 468 springtails representing 12 families, with 219 springtails collected from partial cut treatments, 131 from clearcuts, and 118 from the control areas (Table \ref{tab:springtail}). 
The biomass of springtails collected amounted to 1364 $\mu$g, 747 $\mu$g, and 292 $\mu$g in the partial cut treatments, clearcuts, and control sites, respectively.

\begin{figure}[ht]
  \centering
  \includegraphics[scale=0.55]{fig_pcin.png}
  \caption[Occupancy probability of salamanders under overstory treatments]
  {Occupancy probability of salamanders within two overstory treatmens and controls during the summer 2022 in the Portneuf Wildlife Reserve device, Quebec, Canada. 
  Error bars denote 95\% credible intervals around estimates.}
  \label{fig:pcin}
\end{figure}

\vspace{0.5cm}

The effects of overstory treatments on habitat selection by species groups were generally limitied.
The probability of salamander occupancy was marginally lower in sites subjected to clear-cutting compared to those with partial cutting (Figure \ref{fig:pcin}, Table \ref{tab:overstory}). 
However, these two groups did not differ from the control sites.
The probability of occupancy for both groups of carabids, as well as the springtail biomass, did not vary significantly between the overstory treatments and control sites. 
Overall, the presence of salamanders did not significantly affect the probability of occupancy of carabids representing prey for salamanders and springtail biomass was not significantly affected by the presence of salamanders or carabids. \\
Precipitation had a positive effects on both carabid groups detection (Table \ref{tab:detection}). 
The model diagnostics indicate that the chain lengths are sufficient, as the Brooks-Gelman-Rubin statistic is below 1.1.
Trace plot analysis reveals that all chains have converged towards similar values, and none of the ratios of MCMC error to posterior standard deviation exceed 5\%.

\vspace{0.5cm}

\begin{table}[ht]
  \centering
  \caption[Estimated effects of coarse woody debris and precipitation level on detection probabilities of salamanders and both carabids groups.]
  {Estimated effects of coarse woody debris and precipitation level on detection probabilities of salamanders and both carabids groups, during the summer 2022 in the Portneuf Wildlife Reserve,  Quebec, Canada.}
  \label{tab:detection}
  \begin{tabular}{lllll} 
      \hline
      &&&95\% Bayesian \\
      Variable & Taxa & Estimate &  credible interval \\ [0.5ex] 
      \hline      
      Precipitation       & Salamander              & \hspace{1mm}0.11 & [-0.83, 1.06] \\ 
                          & Carabid$_{competitor}$  & \hspace{1mm}1.17 & [0.59, 1.77] \\ 
                          & Carabid$_{prey}$        & \hspace{1mm}1.87 & [0.70, 3.23] \\  
      \hline      
      Coarse woody debris & Salamander              & -0.59 & [-1.39, 0.12] \\ 
                          & Carabid$_{competitor}$  & \hspace{1mm}0.06 & [-0.26, 0.38] \\ 
                          & Carabid$_{prey}$        & \hspace{1mm}0.27 & [-0.74, 1.37] \\   

      \hline
      \multicolumn{4}{l}{\textbf{Note:} Estimates from Bayesian SEM are presented in terms of mean} \\
      \multicolumn{4}{l}{number with 95\% credible intervals, where an interval excluding 0 indicates} \\
      \multicolumn{4}{l}{a difference between groups.} \\
  \end{tabular}
\end{table}

\clearpage

\section*{Discussion}
\label{sec:discu1}
\phantomsection\addcontentsline{toc}{section}{\nameref{sec:discu1}}

\section*{Conclusion}
\label{sec:conclu1}
\phantomsection\addcontentsline{toc}{section}{\nameref{sec:conclu1}}

\section*{Acknowledgements}
\label{sec:acknowl1}
\phantomsection\addcontentsline{toc}{section}{\nameref{sec:acknowl1}}

\section*{Conflict of interest}
\label{sec:conflict1}
\phantomsection\addcontentsline{toc}{section}{\nameref{sec:conflict1}}

None declared
\section*{Author contributions}
\label{sec:author1}
\phantomsection\addcontentsline{toc}{section}{\nameref{sec:author1}}

\cleardoublepage

\begin{otherlanguage}{english}
\bibliographystyle{ecologyNewEN} % Style de citation en français
\bibliography{References}
\addcontentsline{toc}{section}{References}
\end{otherlanguage}
