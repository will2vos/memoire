\chapter{Direct and indirect effects on soil fauna of silvicultural treatments in the context of assisted forest migration}     % numéroté
\label{chapitre1-articles}    

William Devos$^1$, Mathieu Bouchard$^1$, Marc J. Mazerolle$^1$

%href{mailto:william.devos.1@ulaval.ca}
$^1$ Centre d'étude de la forêt, Département des sciences du bois \\ 
et de la forêt, Université Laval, Québec, QC G1V 0A6, Canada. \\ 

\clearpage

\section*{Résumé}
\label{sec:resume1}
\phantomsection\addcontentsline{toc}{section}{\nameref{sec:resume1}}

\begin{otherlanguage*}{french}
  <Résumé de l'article en français. Obligatoire.>

  \textbf{Mots-clés} : <ajouter des mots clés>
\end{otherlanguage*}

\clearpage

\section*{Abstract}
\label{sec:abstract1}
\phantomsection\addcontentsline{toc}{section}{\nameref{sec:abstract1}}

\begin{otherlanguage*}{english}
  <English abstract of the paper. Optional, but recommended.>

\textbf{Keywords}: <add some keywords> 
\end{otherlanguage*}

\cleardoublepage

\section*{Introduction}
\label{sec:intro1}
\phantomsection\addcontentsline{toc}{section}{\nameref{sec:intro1}}

%\defcitealias{keylist}{alias}

Forest ecosystems play a vital role in the biosphere, both economically and ecologically, by providing essential services such as carbon sequestration, climate regulation, water retention, and biodiversity conservation \citep{Balvanera2006Quantifyingevidence,Diaz2006BiodiversityLoss,Canadell2008Managingforests,Pawson2013Plantationforests}. 
However, the growing demand for forest products and other services has intensified harvesting practices, leading to ecosystem degradation and loss of biodiversity\citep{Bengtsson2000Biodiversitydisturbances,Sala2000Globalbiodiversity,Foley2005GlobalConsequences,Naeem2012functionsbiological}. 

Numerous studies have highlighted the effects of forest management on various species, including birds, bats, butterflies, turtles, small mammals, and insects \citep{Summerville2011Managingforest,Currylow2012ShortTermForest,Kaminski2013EffectsForest,Kellner2013Shorttermresponses,Caldwell2019ComparisonBat}. .
Logging activities cause habitat loss and fragmentation, reducing access to food, shelter, and breeding sites for certain species \citep{Bouderbala2023Longtermeffect}. 
It also reduces connectivity between habitats, restricting individual movement, which reduces genetic flow between populations and increases the risk of local extinction \citep{Saccheri1998Inbreedingextinction}. 
At the stand level, logging often results in an overrepresentation of early-successional forests, reducing the presence of later successional stages \citep{Cyr2009Forestmanagement,Boucher2017Cumulativepatterns}. 
This shift reduces the structural complexity of stands, such as tree species composition, vertical stratification, age structure, successional dynamics, and disturbance frequency \citep{Commarmot2005Structurevirgin}. 
Moreover, stands with more homogeneous structures are less resilient to disturbances, as it slows down the regeneration of existing tree species and the return of those that have disappeared \citep{Kuuluvainen2009Forestmanagement}. 

Overall, modifications of forest attributes contribute to a loss of species and functional diversity. 
In the long term, this erodes the resilience of forests at a local scale and can lead to a decline in ecosystem services \citep{Hooper2012globalsynthesis,Edwards2014Maintainingecosystem}.

%

Among the silvicultural treatments that significantly affect natural environments, logging plays a major role. 
However, the level of disturbance caused by these operations depends on the type of treatment employed \citep{Ameray2021Forestcarbon}. 

Clear-cutting is one of the most widely used practices in temperate and boreal forests \citep{Fedrowitz2014Canretention,Chaudhary2016Impactforest}. 
It is part of intensive forest management used to increase short-term wood productivity and quality \citep{Irland2011TimberProductivitya}.
This method involves the complete removal of commercial trees and results in a single-species, even-aged structure, leading to a drastic homogenization of stands \citep{Rosenvald2008whatwhen}. 
Additionally, shorter rotation periods increase the frequency of disturbances. 
These practices substantially alter ecosystem structures compared to natural conditions, increasing the risk of biodiversity loss and local extinctions \citep{Hanski2000Extinctiondebt}. 
However, some authors suggest that clear-cutting can mimic large-scale natural disturbances, such as wildfires or storms \citep{Greenberg1995comparisonbird}. 

Over the past few decades, ecosystem-based management has been proposed as a more sustainable approach to forest ecosystem \citep{Perry1998scientificbasis,Kuuluvainen2002Naturalvariabilitya}. 
This strategy aims to emulate natural disturbances and the resulting stand structures and successional processes.
The goal is to preserve biodiversity and maintain the resilience of forest ecosystems while ensuring the availability of a variety of ecosystem services \citep{Szaro1998emergenceecosystem,MacDicken2015Globalprogress}. 

Partial cuts are thus integrated into extensive management plans that prioritize regeneration and mimic natural disturbances \citep{Irland2011Timberproductivity}. 
These treatments rely on multi-aged structures, potentially including species mixes, and are characterized by longer rotations \citep{Kuuluvainen2009Forestmanagement}. 
They are used to stimulate the growth of the most vigorous trees, promote species diversity, or maintain an open canopy \citep{Irland2011Timberproductivity}.
Tree retention associated with these types of cuts promotes late-successional structures, which support richer biodiversity \citep{Ameray2021Forestcarbon}. 
Partial cuts also also help carbon sequestration and the supply of organic matter, while maintaining a heterogeneous structure and diverse ecological niches for wildlife \citep{Dahlgren1994effectswholetree,Barg1999Influencepartial,Tong2020Forestmanagement,Ameray2021Forestcarbon}. 

In addition to the challenge of conserving biodiversity and forest ecosystems while meeting economic demands, climate change adds further complexity to forest management. 
Rising global temperatures pose an additional threat to the sustainability of flora and fauna by significantly altering environmental conditions \citep{McKenney2009Climatechange,Trumbore2015Foresthealth,Seidl2017Forestdisturbances,Messier2022Warningnatural}.  
Climate stresses often act additively or synergistically with forestry activities, amplifying their impact on biodiversity and ecosystems \citep{Brook2008Synergiesextinction,Tremblay2018Harvestinginteracts,Ochs2022Responseterrestrial,Bouderbala2023Longtermeffect}. 
Forest composition could thus be altered, requiring adjustments in management practices and conservation strategies \citep{McKenney2009Climatechange,Chmura2011Forestresponses,Lo2011Linkingclimate}. 

For example, assisted tree migration, which involves relocating species or genetic material from their original climatic region to areas better suited to future conditions, 
has been proposed as a potential mitigation strategy to maintain ecosystem services and the economic value of forests \citep{Vitt2010Assistedmigration,Pedlar2011implementationassisted,Ste-Marie2011Assistedmigration,Winder2011Ecologicalimplications}. 
However, there remains a lack of knowledge and uncertainties surrounding these new adaptation methods \citep{Klenk2015assistedmigration,Park2018Informationunderload}. 
Therefore, it is essential to better understand the impact of silvicultural practices on biodiversity, especially in a context where forest management must adapt to climate change, biodiversity loss, and economic pressures.

%

Soil fauna plays a major role in forest ecosystems by facilitating the flow of matter and energy. 
However, this community is among the most impacted by disturbances due to their sensitivity to environmental changes and limited dispersal abilities.

Canopy removal significantly alters soil conditions by increasing its exposure to sunlight, leading to increased temperatures, changes in humidity, higher wind speeds, and intensified precipitation. 
The movement of machinery also increases soil compaction, reducing porosity and consequently affecting the organisms that live there.
These changes at the surface level affect nutrient availability by altering litter, root secretions, leaching, and soil chemical properties. 
Logging also reduces microhabitats, such as deadwood, cavities in mature trees, and root plates, which provide shelter for soil fauna. 
These modifications create unfavorable conditions for species that rely on cool, moist environments, such as amphibians and arthropods, leading to population declines or local extinctions.

Due to their sensitivity to environmental shifts and limited dispersal capacity, amphibians and arthropods serve as important indicators to understand the effects of forestry practices on biodiversity and the ecological integrity of forests. 
Moreover, both groups have experienced significant declines in recent decades, primarily due to forest management practices and climate change. 
Species frequently studied in this context include the Eastern Red-backed Salamander (\textit{Plethodon cinereus} (Green, 1818)), ground beetles (Carabidae), and springtails (Collembola).

The Eastern Red backed Salamander represents one of the largest vertebrate biomasses in North American forests \citep{Burton1975Salamanderpopulations,Petranka1993Effectstimber,semlitschAbundanceBiomassProduction2014a}. 
As a generalist predator, this plethodontid species plays a key role in regulating detritivorous invertebrates, influencing decomposition processes, nutrient cycling, and carbon dynamics \citep{Burton1975Energyflow,Wyman1998Experimentalassessment,Walton2013Topdownregulation,Hickerson2017Easternredbacked}. 
Moreover, it serves as a high nutritional prey for many predators, including birds, mammals, and reptiles \citep{Burton1975Energyflow,Pough1987abundancesalamanders}. 
Without lungs, the Eastern Red backed Salamander relies on cutaneous respiration for gas exchange \citep{Heatwole1961Relationsubstrate}. 
This type of respiration requires salamanders to occupy specific microhabitats, staying on the surface when temperature and humidity are favorable, or retreating into the soil during less suitable conditions \citep{Grizzell1949HibernationSite,FraserEmpiricalEvaluation1976,Jaeger1980MicrohabitatsTerrestrial}. 
The loss of refuges due to logging decreases habitat quality, limiting the time it spends on the forest floor to feed and reproduce \citep{Achat2015Quantifyingconsequences,Peele2017Effectswoody}. 
Consequently, poor surface conditions, soil compaction, and low levels of coarse woody debris can negatively impact population dynamics \citep{Peterman2014Spatialvariation}. 

On the other hand, ground beetles are a well documented taxon both in terms of taxonomy and ecology \citep{loveiEcologyBehaviorGround1996}. 
These insects have short lifespans, occupy a high position in the soil food web, and respond rapidly and in complex ways to environmental changes \citep{loveiEcologyBehaviorGround1996}. 
They play an ecological role by regulating invertebrate populations, while also serving as prey for various amphibians, reptiles, birds, and mammals \citep{loveiEcologyBehaviorGround1996}. 
With around 40,000 known species, ground beetles are one of the most diverse beetle families and among the most abundant soil dwelling arthropods \citep{Erwin1985taxonpulse,loveiEcologyBehaviorGround1996,Rochefort2006GroundBeetle}. 
They are widely distributed across nearly all terrestrial ecosystems, although species vary in their habitat preferences  \citep{loveiEcologyBehaviorGround1996,kotzeFortyYearsCarabid2011a,Larochelle2003naturalhistory}. 
Environmental changes can benefit some species at the expense of others, making their diversity and sensitivity valuable for studying the effects of environmental disturbances \citep{Rainio2003Groundbeetles}. 

Springtails represent one of the most abundant and diverse groups within the mesofauna \citep{rusekBiodiversityCollembolaTheir1998}. 
Different springtail communities occupy a variety of ecological niches within the soil \citep{pongeVerticalDistributionCollembola2000}.
Their vertical distribution is largely influenced by abiotic factors such as light, humidity levels, and soil porosity. 
Primarily fungivores and detritivores, springtails play a key role in the decomposition of wood. 
They also impact the physical structure and mineralization rates of litter, influence the nutrient absorption, 
regulate microbial communities, and contribute to soil microstructure formation \citep{Petersen1982comparativeanalysis,Neher2012Linkinginvertebrate,Maass2015Functionalrole,Potapov2016Connectingtaxonomy}. 
Additionally, springtails serve as an important food source for a range of organisms, including amphibians, beetles, arachnids, birds, and reptiles.

Through their trophic relationships, their sensitivity to environmental condition changes, and their dependence on forest features such as woody debris and litter, 
these three species groups are relevant models for studying the impact of forestry practices on soil fauna.

Most studies addressing the impacts of forestry treatments on fauna generally discuss the direct effects of disturbances on one or more species groups, 
without considering the relationships between environmental variables and different species groups, thereby neglecting the indirect effects of logging on soil fauna. 
My project aimed to fill this gap by trying to understand how the effects of forestry treatments propagate through the forest ecological network and influence soil fauna dynamics. 
Ultimately, this knowledge gain will provide valuable tools to facilitate sustainable forest management.

%% re voire la fin et regarder Laigle si on peux améliorer

These three taxa are relevant to quantifying the impact of silvicultural treatments as their sensitivity to environmental changes and trophic relationships enable the analysis of disturbance effects on soil fauna dynamics \citep{Salmon2008Relationshipssoil}.
Several papers have already studied the effects of overstory treatments on soil fauna.
However, most of those researches have only focused on the direct impacts of disturbances for one or more species groups, neglecting the potential relationships between environmental variables and species groups \citep{josephIntegratingOccupancyModels2016,Pollierer2021Diversityfunctional,Kudrin2023metaanalysiseffects}. 
Quantifying treatments effects and their propagation within the ecological network, namely on soil fauna, will provide useful tools to improve sustainable forest management.

Our study aimed to understand how silvicultural practices, conducted in an assisted tree migration context, affect the dynamics of forest soil ecosystems. 
The specific objectives were to quantify the effect of overstory treatments on environmental variables that influence habitat use by soil fauna
and to evaluate the impact of overstory treatments on habitat use by soil fauna.
We hypothesized that environmental variables, favorable to habitat use by taxa, fluctuate according to the intensity of forest harvests. 
We predicted that more intense treatments would decrease litter depth and CWD, as a consequence of reduced accumulation of leaves and woody debris on the forest floor and increase canopy openness. 
We also hypothesized that overstory treatments modify habitat use by soil fauna and propagate through the trophic network. 
We predicted that tree harvest impacts occupancy of salamanders and large ground beetles, 
followed by small ground beetles and ultimately affecting springtail biomass. 


\section*{Material and methods}
\label{sec:matmet1}
\phantomsection\addcontentsline{toc}{section}{\nameref{sec:matmet1}}

\subsection*{Study area}
\label{subsec:area}
\phantomsection\addcontentsline{toc}{subsection}{\nameref{subsec:area}}

\begin{otherlanguage*}{english}

  Our study was conducted within the Portneuf Wildlife Reserve in the Capitale-Nationale administrative region near Lac des Amanites (47°07’N, 72°24’W, Figure \ref{fig:area}). 
  This area is located within the balsam fir (\textit{Abies balsamea})-yellow birch (\textit{Betula alleghaniensis}) bioclimatic domain, according to the ecological classification used in Québec \citep{saucierChapitreEcologieForestiere2009}. 
  Other tree species in this bioclimatic domain included sugar maple (\textit{Acer saccharum}), red maple (\textit{Acer rubrum}), white spruce (\textit{Picea glauca}), black spruce (\textit{Picea mariana}), red spruce (\textit{Picea rubens}), white birch (\textit{Betula papyrifera}), and quaking aspen (\textit{Populus tremuloides})\citep{olaBelowgroundCarbonStocks2024}. 
  The experimental sites are located on deep glacial tills with moderately well-drained sandy loams soil \citep{CanadianSystemSoil1998}. 
  The mean daily temperature is 4\up{o}C based on the 1981-2010 period at the nearest weather station (Lac aux sables, \citealp{environmentcanadaCanadianClimateNormals2019}). 
  Based on the same report, the average annual precipitation, including snow-covered months, is 1133.2 mm, with snowfall averaging 230.3 cm.

  We conducted our study within the assisted migration experimental system established in 2018 by the Ministère des Ressources naturelles et des Forêts (\citealp{royoDesiredREgenerationAssisted2023}). 
  This experimental system uses a factorial experimental design with split-plots replicated in four blocks. 
  Each whole block (200 m x 140 m) is split in two overstory treatment : clear-cut and regular shelterwood cut at 50\% of the merchantable basal area (partial-cut). 
  The harvest took place during the summer 2017, followed by trenching in June 2018 within the clear-cut areas to create more uniform site conditions and facilitate planting. 
  No site preparation was conducted in the shelterwood cuts.

\end{otherlanguage*}

\begin{figure}[ht!]
	\centering
	\includegraphics[scale=0.31]{fig_area6.png}
	\caption[Localization of the Capitale-Nationale administrative region in Quebec, Canada and position of the study area near Lac des Amanites in Portneuf Wildlife Reserve, Quebec, Canada.]
  {Localization of the Captial-Nationale administrative region in Quebec, Canada (A) and position of the study area in Portneuf Wildlife Reserve, Quebec, Canada (B) where the clear-cut (orange) and the partial-cut (green) associated with the assisted migration experimental system were applied in 2017 (47°07'N, 72°24'W)(C).}
	\label{fig:area}
	\end{figure}  


\subsection*{Environmental variables}
\label{subsec:EnvVar}
\phantomsection\addcontentsline{toc}{subsection}{\nameref{subsec:EnvVar}}

We selected a total of 60 sampling units measuring 10 m by 7.5 m to collect our data: we used four blocks serving as replicates, 
with each block containing six sampling units for both the clear-cut and partial-cut overstory treatments, 
whereas three additional sampling units were positioned outside the block, serving as uncut controls (Figure \ref*{fig:blockSU}). 
The uncut controls were separated from blocks by at least 10 m. 

In each sampling unit, we measured several environmental variables that could affect occupancy probability of red-backed salamanders, ground beetles, and springtails.
CWD and litter depth play a crucial role in habitat use for salamanders, ground beetles and springtails as
they serve for feeding and protection  \citep{harmonEcologyCoarseWoody1986,koivula.LeafLitterSmallscale1999,birdChangesSoilLitter2004,mckennyEffectsStructuralComplexity2006}. 
Salamanders also utilize CWD as shelter to maintain suitable temperature and moisture levels during dry periods \citep{Jaeger1980MicrohabitatsTerrestrial,groverInfluenceCoverMoisture1998a,patrickEffectsExperimentalForestry2006a}.
We used 400 m\up2 plots centered on every sampling unit to estimate CWD (20 m $\times$  20 m) (\citealp{methotGuideInventaireEchantillonnage2014}). 
We only considered CWD with a basal diameter $\geq$ 9 cm and a length $\geq$ 1 m.
For each CWD, we measured the basal diameter, the apical diameter and length with a tree caliper.
Segments of CWD outside the plot boundaries were not included.
We employed the conic–paraboloid formula to estimate the volume of each CWD \citep{fraverRefiningVolumeEstimates2007} :

\begin{equation}
  \text{Volume} = L/12 \times (5A_b + 5A_u + 2\sqrt{A_b \times A_u})
\end{equation}

\vspace{0.5cm}

Where $L$ is the length of log (cm), $A_b$ the basal area (cm\up{2}) and $A_u$ the apical area (cm\up{2}).
We measured litter depth next to each coverboard and computed the average litter depth for each sampling unit \citep{Mazerolle2021Woodlandsalamander}. \\

We assessed canopy openness, as it may influence species occupancy \citep{messereForestFloorDistribution1998,koivulaBorealCarabidbeetleColeoptera2002a,tilghmanMetaanalysisEffectsCanopy2012,henneronForestPlantCommunity2017}.
Canopy openness was measured 130 cm above the ground at the center of each sampling unit using a spherical densiometer \citep{lemmonSphericalDensiometerEstimating1956}. 
We averaged four measurements per sampling unit, oriented toward each of the four cardinal points as an estimate of canopy openness.

We collected data locally for air temperature, air humidity and precipitation levels during the summer 2022. 
Two weather stations (Em50 Digital Decagon Data Logger, Part \#40800, Meter Group Inc., USA) were installed inside both overstory treatments. 
Each weather station measured temperature, air humidity, and atmospheric pressure, 130 cm above the ground (VP-4 Sensor (Temp/RH/Barometer), Part \#40023). 
Rain gauges were installed in the clear-cut treatments to monitor precipitation levels. 
The temperature and humidity sensors were programmed to record data every 15 minutes. 
We averaged the measurements across both weather stations to get daily average measurements. 
These variables fluctuate on a daily basis, affecting species activity and, consequently, the probability of detecting individuals \citep{spotilaRoleTemperatureWater1972,butterfieldCarabidLifeCycle1996,loveiEcologyBehaviorGround1996,odonnellPredictingVariationMicrohabitat2014a}.


\subsection*{Sampling design}
\label{subsec:sampling}
\phantomsection\addcontentsline{toc}{subsection}{\nameref{subsec:sampling}}

In each sampling unit, we used three sampling methods to collect species data : artificial coverboards, pitfall traps, and soil cores.
We used artificial coverboards to sample eastern red-backed salamanders \textit{Plethodon cinereus} at our study sites \citep{hydeSamplingPlethodontidSalamanders2001,mooreComparisonPopulationEastern2009c,hesedUncoveringSalamanderEcology2012,Mazerolle2021Woodlandsalamander}. 
Coverboards consisted of 25 cm x 30 cm x 5 cm pieces of untreated spruce wood. Each board was in direct contact with the soil after we had cleared the litter underneath \citep{Mazerolle2021Woodlandsalamander}. 
Six coverboards spaces by at least 2.5 m were arranged in a rectangular array in each sampling unit, resulting in a total of 360 coverboards across our 60 sampling stations (Figure \ref{fig:blockSU}). 
All coverboards were placed outdoors in March 2022 to allow for natural aging \citep{hedrickEffectsCoverboardAge2021,Grasser2014Effectscover}. 
We conducted four visits to coverboards during the summer 2022, with each visit spaced one month apart, namely in mid-May, mid-June, mid-July and mid-August, during the summer 2022. 
Blocks were visited in a random sequence to reduce the potential effects of time of day and observer fatigue. 
During a given visit, the 360 coverboards were inspected on the same day and we counted the number of red-backed salamanders underneath.  

We used pitfall traps were used to capture ground beetles \citep{baarsCatchesPitfallTraps1979,spenceSamplingCarabidAssemblages1994a,loveiEcologyBehaviorGround1996,kotzeFortyYearsCarabid2011a,knappEffectPitfallTrap2012}. 
Trap design was based on Multipher\up{\textregistered{}} traps and included a container with a diameter of 12.5 cm, a depth of 25 cm and a cover raised 4.5 cm above the trap 
to prevent debris and rain from filling the container \citep{Jobin1988MultiPherinsect,mooreEffectsTwoSilvicultural2004,bouchardBeetleCommunityResponse2016b}. 
We covered each trap with a protective stainless steel mesh size of 15 mm, allowing trap access to carabid-sized individuals and limiting access by predators.  
We did not add preserving liquid in traps, but we added wet sponges to avoid harming vertebrates small enough to pass through the mesh. 
We centered a pitfall in each sampling unit (Figure \ref{fig:blockSU}). 
Traps were inserted in the soil at a depth allowing the container’s opening to be level with the soil surface. 
Trapping occurred during four sampling periods during 2022 (mid-May, mid-June, mid-July and mid-August). 
Traps were opened on the first day of each survey, and we collected captured ground beetles daily during five days. 
Outside of the trapping periods, pitfall traps were sealed with adhesive tape to prevent captures. 
Individuals were preserved in 70\% alcohol and identified at the species level afterwards. 
Identification was conducted with a ZEISS SteREO Discovery.V12 bionocular microscope using the \cite{larochelleManuelIdentificationCarabidae1976} taxonomic keys. 
We categorized ground beetles into two groups, either as small ground beetles (salamander prey )or large ground beetles (salamander competitors), based on the red-backed salamander gape size (Table \ref{tab:carabid}, \citealp{jaegerFoodLimitedResource1972,magliaModulationPreycaptureBehavior1995,magliaOntogenyFeedingEcology1996}).

We extracted soil cores to sample springtails \citep{pongeVerticalDistributionCollembola2000,salamonEffectsPlantDiversity2004,chauvatChangesSoilFaunal2011a,farskaManagementIntensityAffects2014}. 
Core sampling occurred during the same four sampling periods as for the salamanders and ground beetles (mid-May, mid-June, mid-July and mid-August). 
During a sampling period, we collected two cores in each sampling unit using a pedological probe (5 cm diameter x 5 cm depth). 
We also collected a 15 cm x 15 cm litter quadrat each soil sample \citep{raymond-leonardSpringtailCommunityStructure2018a,rousseauForestFloorMesofauna2018}.
We extracted springtail communities directly related to the ecology of salamanders and ground beetles \citep{edwardsAssessmentPopulationsSoilinhabiting1991,chauvatChangesSoilFaunal2011a,raymond-leonardSpringtailCommunityStructure2018a,rousseauForestFloorMesofauna2018}.
Soil and litter from the same unit were pooled in Ziploc\up{\texttrademark{}} bags and stored in a cooler at ca. 4 °C \citep{chauvatChangesSoilFaunal2011a,rousseauForestFloorMesofauna2018}, providing 60 samples during each of the four sampling periods.
We placed each sample in an individual Tullgren dry-funnel for springtail extraction within 48 h after collection \citep{rusekBiodiversityCollembolaTheir1998,wuCompositionSpatiotemporalVariation2014,rousseauForestFloorMesofauna2018}. 
The extraction process lasted six days with a gradual temperature increase (25 °C to 50 °C) \citep{raymond-leonardSpringtailCommunityStructure2018a}.
Springtails were separated from other invertebrates, pooled by sampling unit and preserved in 75\% alcohol \citep{wuCompositionSpatiotemporalVariation2014}.
Identification at the family level for all individuals was done with a ZEISS SteREO Discovery.V12 binocular microscope and a Leitz orthoplan phase-contrast fluorescent trinocular microscope using \cite{bellingerChecklistCollembolaWorld1996} identification keys. 
Following identification, springtails within each sample were dried in a freeze dryer (Labconco FreeZone Bulk tray dryer 78060 series) for 24 hours. 
We determined the total dry biomass of springtails in each sample with a micro balance (Sartorius Cubis\up{\texttrademark{}} MSA3.6P-000-DM).

\pagebreak

\begin{figure}[ht]
	\centering
	\includegraphics[scale=0.50]{fig_blockSU.png}
	\caption[Design of one block and one sampling unit with three sampling methods.]{
  Design of a block (left) and a sampling unit (right). 
  The block contains two overstory treatments : clear-cut (grey background), partial-cut (white background). 
  Fifteen sampling units were used per block : six per overstory treatment and three controls (\textbf{c}) outside each block.
  Each sampling unit contained six artificial coverboards (squares) and one pitfall trap (circle). Two soil cores (stars) were collected per survey.
  }
	\label{fig:blockSU}
	\end{figure}  

\vspace{0.5cm}

\subsection*{Statistical analyses}
\label{subsec:analyses}
\phantomsection\addcontentsline{toc}{subsection}{\nameref{subsec:analyses}} 


\subsubsection{Structural equations models} 

To assess the effects of overstory treatments on habitat use by soil fauna (hypothesis 1.1) and the relationship between overstory treatments 
and environmental variables (hypothesis 2.1), we employed a structural equation model (SEM) combining occupancy models and linear mixed models \citep{mackenzieOccupancyEstimationModeling2006a,graceSpecificationStructuralEquation2010,josephIntegratingOccupancyModels2016}.
This approach enabled us to test both hypotheses within one analysis. 

One part of the SEM focused on the variations in environmental variables across different overstory treatments. 
Specifically, we estimated the effects of partial-cut and clear-cut on coarse woody debris volume, canopy openness and litter depth. 
The other part of the SEM was designed to evaluate the direct and indirect effects of overstory treatments on taxa (Figure \ref{fig:SEM}). 
We presumed that partial-cut and clear-cut will directly impact each taxon, while salamander occupancy will influence small ground beetle occupancy and springtail biomass. 
We aslo suggested that both ground beetle groups will affect springtails biomass.

Some components of the SEM consisted of linear mixed models to estimate the effect overstory treatments on environmental variables (Table \ref{ann:SEM_Env_eq}) and springtail biomass. 
The component of the model predicting springtail biomass also included the latent occupancy state of salamanders and ground beetles to measure the impact of these groups on springtails. 
Other components of our SEM consisted of occupancy models to quantify the impact of overstory treatments on the occupancy (presence) probabilities of salamanders and ground beetles (Table \ref{ann:SEM_Sp_eq}). 
Occupancy models estimate the presence of species difficult to detect after accounting for imperfect detection probability \citep{mackenzieEstimatingSiteOccupancy2002,baileyEstimatingSiteOccupancy2004,mazerolleMakingGreatLeaps2007,spiersEstimatingSpeciesMisclassification2022}. 
The data type required to distinguish between occupancy and detection probabilities consists of repeated visits at a series of sites 
(here, 4 visits at the sampling units). Specifically, a detection history is constructed for each site, 
using 1 to denote the detection of the species on a given visit and 0 to non-detection. 
For example, the detection history 0010 at a site would indicate that the species was detected at the site on the third visit, but not detected during the first, 
second, and fourth visits. 

We formulated the SEM using a Bayesian framework, which we describe in Table \ref{ann:SEM_script}. 
Parameters were estimated using Markov chain Monte Carlo (MCMC) with JAGS 4.3.0 included in the jagsUI package in R 4.3.1 \citep{lunnBUGSProjectEvolution2009,rcoreteamLanguageEnvironmentStatistical2020,kellnerJagsUIWrapperRjags2024}. 
We ran the model with five chains and 200,000 iterations each \citep{gelmanUnderstandingPredictiveInformation2014}. 
The first 75,000 iterations were used as burn-in and we used a thinning rate of 5. 
We assessed convergence of MCMC chains by examining trace plots, posterior density plots, and using the Brooks-Gelman-Rubin statistic. 
The JAGS model code is available in Table \ref{ann:SEM_script}.

\vspace{10pt}

\begin{figure}[h!]
	\centering
	\includegraphics[scale=0.55]{fig_sem.png}
	\caption[Theoretical model illustrating the anticipated relationships between overstory treatments, environmental variables and species groups.]
  {Theoretical model illustrating the anticipated relationships between overstory treatments, coarse woody debris volume, canopy openness, litter depth,
   salamander occupancy, ground beetle occupancy and springtail biomass in the Portneuf Wildlife Reserve, Quebec, Canada. 
   Each arrow indicates the direction of a potential effect, from an explanatory variable to a response variable. 
   Note that the large and small carabid categories are based on salamander gape size.}
	\label{fig:SEM}
\end{figure} 

\clearpage

\section*{Results}
\label{sec:results1}
\phantomsection\addcontentsline{toc}{section}{\nameref{sec:results1}}


\subsection*{Environmental variables}
\label{subsec:ResEnv}
\phantomsection\addcontentsline{toc}{subsection}{\nameref{subsec:ResEnv}} 

Model diagnostics indicated that the chains were of sufficient length, as the Brooks-Gelman-Rubin statistic was below 1.04. 
Trace plot analysis revealed that all chains had converged towards similar values, and none of the ratios of MCMC error to posterior standard deviation exceeded 5\%.

Environmental variables usually differed between overstory treatments and control conditions.
We found that clear-cutting treatments had significantly less CWD compared to partial-cutting
(95\% CI : [-1.15, -0.43]) (Figure \ref{fig:envar} A, Table \ref{tab:overstoryenvar}). 
However, these treatments did not differ from the control sites. 
Canopy openness was significantly higher in both the partial-cut (95\% CI : [1.97, 11.02]) and clear-cut treatments (95\% CI : [51.39, 77.06]) when compared 
to the control sites, with the clear-cuts being more open than the partial-cuts (95\% CI : [44.61, 70.76], Figure \ref{fig:envar} B, Table \ref{tab:overstoryenvar}). 
In contrast, litter depth was lower in both the partial-cut (95\% CI : [-2.44, -0.65]) and clear-cut treatments (95\% CI : [-4.28, -2.50]) compared to the controls, 
with the litter being shallower in the clear-cuts than in the partial-cuts (95\% CI : [-2.57, -1.12], Figure \ref{fig:envar} C, Table \ref{tab:overstoryenvar}).


\vspace{10pt}

\begin{figure}[ht]
  \centering
  \includegraphics[scale=0.23]{fig_envar2.png}
  \caption[Environmental variables with a potential effect on soil species within two different overstory treatments and control.]
  {Environmental variables with a potential effect on soil species estimations within two different overstory treatments and control 
  during the summer 2022 in the Portneuf Wildlife Reserve, Quebec, Canada. Error bars denote 95\% credible intervals around estimates.}
  \label{fig:envar}
\end{figure}

\begin{table}[ht]
  \centering
  \caption[Contrasts between overstory treatments for environmental variables that could affect habitat selection of fauna on the forest soil.]
  {Contrasts between overstory treatments for environmental variables that could affect habitat use of fauna on the forest soil during the summer 2022 in the Portneuf Wildlife Reserve,
  Quebec, Canada.}
  \label{tab:overstoryenvar}
  \begin{tabular}{lllll} 
      \hline
      &&&&95\% Bayesian \\
      Variable&Unit& Comparison & Estimate &  credible interval \\ [0.5ex] 
      \hline
      Coarse woody debris &m\up{3}& Partial vs control & \hspace{1mm}0.02 & [-1.01, 1.06] \\ 
                 && Clear vs control  & -0.77 & [-1.79, 0.23] \\ 
                          && Clear vs partial  & -0.79 & [-1.15, -0.43] \\
      Canopy openness     &\%& Partial vs control & \hspace{1mm}6.49 & [1.97, 11.02] \\ 
                      && Clear vs control  & \hspace{1mm}64.19 & [51.39, 77.06] \\ 
                          && Clear vs partial  & \hspace{1mm}57.69 & [44.61, 70.76] \\ 
      Litter depth        &cm& Partial vs control & -1.54 & [-2.44, -0.65] \\ 
                      && Clear vs control  & -3.39 & [-4.28, -2.50] \\ 
                          && Clear vs partial  & -1.85 & [-2.57, -1.12] \\       
      \hline
      \multicolumn{5}{l}{\textbf{Note:} Estimates from Bayesian SEM are presented in terms of posterior mean with 95\%} \\
      \multicolumn{5}{l}{credible intervals, where an interval excluding 0 indicates a difference between groups.} \\
  \end{tabular}
\end{table}

\clearpage


\subsection*{Soil fauna}
\label{subsec:taxa}
\phantomsection\addcontentsline{toc}{subsection}{\nameref{subsec:taxa}} 

\vspace{10pt}

Across the sixty sampling units, Red-backed salamanders were detected in 0, 1, 11, and 11 sites during the May, June, July, and August 2022 surveys, respectively. 
Over the same periods, small ground beetles were detected in 2, 5, 12, and 2 sites, while large ground beetles were detected in 16, 30, 35, and 21 sites.
A total of 30 ground beetle species were identified from harvest in the pitfall traps and under the coverboards (Table \ref{tab:carabid}). 
We collected 468 springtails representing 12 families, with 219 springtails collected from partial-cut treatments, 131 from clear-cuts, and 118 from the control areas (Table \ref{tab:springtail}). 
The average springtail biomass collected per overstory treatments was 24.3 $\mu$g (SD = 18.2 $\mu$g), 56.8 $\mu$g (SD = 78.0 $\mu$g), and 31.1 $\mu$g (SD = 52.8 $\mu$g) in the partial-cut treatments, clear-cuts and control sites, respectively.

Occupancy and biomass generally did not vary significantly across the cutting treatments. 
Salamander occupancy probability was marginally lower in sites subjected to clear-cutting compared to those with partial-cutting (90\% CI : [-0.74, -0.07], Figure \ref{fig:pcin}, Table \ref{tab:overstorysp}). 
However, these two groups did not differ from the control sites. 
Occupancy probability for each carabid group and the springtail biomass did not vary between the overstory treatments and control sites (Table \ref{tab:overstorysp}). 
Furthermore, the occupancy probabilities of small ground beetles (salamander prey) did not vary with the presence of salamanders. 
Similarly, the biomass of springtails did not vary with the presence of salamanders, large ground beetles or small ground beetles (Table \ref{tab:overstorysp}).

\begin{figure}[h!]
	\centering
	\includegraphics[scale=0.60]{fig_sem_res.png}
	\caption[Results from structural equation modeling analysis revealing effects of overstory treatments on coarse woody debris volume,
  canopy openness, litter depth, salamander occupancy, ground beetle occupancy, and springtail biomass.]
  {Results from SEM analysis showing effects of overstory treatments on CWD, 
  canopy openness, litter depth, salamander occupancy, ground beetle occupancy, and springtail biomass in the Portneuf Wildlife Reserve, 
  Quebec, Canada. Bold arrows represent significant effects, while dotted line indicate no discernible effects. 
  Estimates marked with one asterisk (*) indicate a 90\% credible interval (CI) excluding 0, while estimates marked with two asterisks (**) indicate a 95\% CI excluding 0. 
  Note that the large and small carabid categories are based on salamander gape size.}
	\label{fig:SEMres}
\end{figure}  

\vspace{10pt}

\begin{table}[h!]
  \centering
  \caption[Contrasts between overstory treatments for salamander occupancy, ground beetle occupancy, and springtail biomass.]
  {Contrasts between overstory treatments for salamander occupancy, ground beetle occupancy, and springtail biomass, during the summer 2022 in the Portneuf Wildlife Reserve, Quebec, Canada. 
  This table also shows the estimated effect of interactions between different groups: salamander presence on small and large ground beetles and the effects of the presence of salamanders and both ground beetle groups on springtail biomass.}
  \label{tab:overstorysp}
  \begin{tabular}{lllll} 
      \hline
      &&&&95\% Bayesian \\
      Variable&Unit& Comparison & Estimate &  credible interval \\ [0.5ex] 
      \hline     
      Salamander           &probability& Partial vs control & \hspace{1mm}0.07 & [-0.29, 0.45] \\ 
      occupancy       && Clear vs control  & -0.38 & [-0.75, 0.11] \\ 
                          && Clear vs partial  & -0.45 & [-0.74, -0.07]$^{a}$ \\       
      Carabid$_{large}$ &probability& Partial vs control & -0.12 & [-0.35, 0.15] \\
      occupancy       && Clear vs control  & -0.06 & [-0.29, 0.20] \\ 
                          && Clear vs partial  & \hspace{1mm}0.06 & [-0.19, 0.30] \\ 
      Carabid$_{small}$    &logit& Partial vs control & \hspace{1mm}3.31 & [-10.12, 17.72] \\
      occupancy             && Clear vs control  & \hspace{1mm}10.19 & [-4.15, 24.45] \\ 
                          && Clear vs partial  & \hspace{1mm}6.88 & [-12.81, 23.42] \\  
                          && Salamander        & -2.20 & [-17.15, 16.59] \\  
      Springtail          &$\mu$g& Partial vs control & \hspace{1mm}8.11 & [-9.38, 25.40] \\
      biomass             && Clear vs control  & \hspace{1mm}2.11 & [-13.98, 18.11] \\ 
                          && Clear vs partial  & -6.00 & [-29.09, 17.26] \\  
                          && Salamander        & \hspace{1mm}6.80 & [-10.43, 23.26] \\ 
                          && Carabid$_{large}$      & \hspace{1mm}0.56 & [-16.75, 17.66] \\ 
                          && Carabid$_{small}$      & \hspace{1mm}7.62 & [-8.93, 24.09] \\ 
      \hline
      \multicolumn{5}{l}{\textbf{Note:} Estimates from Bayesian SEM are presented in terms of posterior mean with 95\%} \\
      \multicolumn{5}{l}{credible intervals, where an interval excluding 0 indicates a difference between groups.} \\
      \multicolumn{5}{l}{$^{a}$Marginal difference based on 90\% Bayesian credible interval excluding 0}
  \end{tabular}
\end{table}

\clearpage

\begin{figure}[h!]
  \centering
  \includegraphics[scale=0.55]{fig_pcin.png}
  \caption[Occupancy probability of salamanders under overstory treatments]
  {Occupancy probability of salamanders within two overstory treatments and controls during the summer 2022 in the Portneuf Wildlife Reserve device, Quebec, Canada. 
  Error bars denote 95\% credible intervals around estimates.}
  \label{fig:pcin}
\end{figure}

\vspace{10pt}

\clearpage

We did not observe significant impacts of CWD volume and precipitation level on salamander detection probabilities. 
However, the precipitation level had a positive effect on detection probability for small ground beetles (95\% CI : [0.59, 1.77]) and large ground beetles (95\% CI : [0.70, 3.23]) (Table \ref{tab:detection}). 
Detection probabilities of ground beetles did not vary with the volume of CWD.

\begin{table}[ht]
  \centering
  \caption[Estimated effects of coarse woody debris and precipitation level on detection probabilities of salamanders and both ground beetles.]
  {Estimated effects of coarse woody debris and precipitation level on detection probabilities of salamanders and both ground beetles, during the summer 2022 in the Portneuf Wildlife Reserve,  Quebec, Canada.}
  \label{tab:detection}
  \begin{tabular}{lllll} 
      \hline
      &&&95\% Bayesian \\
      Variable & Taxa & Estimate &  credible interval \\ [0.5ex] 
      \hline      
      Precipitation       & Salamander              & \hspace{1mm}0.11 & [-0.83, 1.06] \\ 
                          & Carabid$_{large}$  & \hspace{1mm}1.17 & [0.59, 1.77] \\ 
                          & Carabid$_{small}$        & \hspace{1mm}1.87 & [0.70, 3.23] \\  
      \hline      
      Coarse woody debris & Salamander              & -0.59 & [-1.39, 0.12] \\ 
                          & Carabid$_{large}$  & \hspace{1mm}0.06 & [-0.26, 0.38] \\ 
                          & Carabid$_{small}$        & \hspace{1mm}0.27 & [-0.74, 1.37] \\   

      \hline
      \multicolumn{4}{l}{\textbf{Note:} Estimates from Bayesian SEM are presented in terms of posterior mean} \\
      \multicolumn{4}{l}{with 95\% credible intervals, where an interval excluding 0 indicates} \\
      \multicolumn{4}{l}{a difference between groups.} \\
  \end{tabular}
\end{table}

\clearpage

\section*{Discussion}
\label{sec:discu1}
\phantomsection\addcontentsline{toc}{section}{\nameref{sec:discu1}}

Our study confirmed that environmental variables favorable to habitat use by species fluctuate based on the intensity of forest harvesting, five years after the interventions.  
Overall, changes in environmental variables intensify with the level of disturbance associated with forest harvesting. 
Clear-cuts generally show the largest canopy openings and the shallowest litter depths, followed by partial harvesting treatments. 
Additionally, the volume of woody debris is lower in clear-cuts compared to partial cuts.  
However, no significant differences were detected between control sites and the various harvesting treatments for this variable.
Therefore, forest harvesting treatments act as a comprehensive variable encompassing changes in environmental conditions.  
However, our findings do not support the hypothesis that forest harvesting treatments alter habitat use by soil fauna and that these effects propagate through the trophic network.  
Our results did not show significant effects of harvesting treatments on the three species groups studied, whether directly or indirectly,  
with the exception of salamanders, where a marginal direct effect on occupancy probability was detected.

\subsection*{Environnemental variables}
\label{disc:env_var}
\phantomsection\addcontentsline{toc}{subsection}{\nameref{disc:env_var}} 

% résultats : CWD

Our results indicate that partial cuts maintain a higher volume of woody debris compared to clear-cuts, aligning with similar studies \citep{Nolet2018Comparingeffects,Ochs2022Responseterrestrial}.  
However, the lack of a significant difference between control sites and clear-cuts contrasts with other research that reports a reduction in woody debris volume following clear-cutting \citep{Mazerolle2021Woodlandsalamander}.  
This difference may stem from the variable amount of wood debris left on-site after harvesting, which could temporarily increase the availability of dead wood in the habitat during the initial years post-harvest \citep{McCarthy1994Distributionabundance}.  
For example, \cite{Ochs2022Responseterrestrial} found that the volume of woody debris was higher in clear-cuts than in control sites within the first 0–3 years following harvesting. 
However, this difference disappeared 4–6 years and 7–11 years after harvesting.  
In the short term, the presence of woody debris promotes soil fauna conservation by providing shelters, a favorable microclimate, and a nutrient source that supports the survival of amphibians and arthropods \citep{spotilaRoleTemperatureWater1972,Huhta1976Effectsclearcutting,Seibold2021contributioninsects,Ochs2022Responseterrestrial}.  
Retaining dead wood is often recommended as a forest management strategy to mitigate the negative impacts of logging \citep{McKenny2006Effectsstructural}.  
However, the benefits of woody debris decrease over time as the wood decays, leading to population declines in less favorable habitats over the medium and long term.  
The utility of woody debris for soil fauna also depends on factors such as decomposition stage, tree species, and debris size \citep{Bunnell2010woodbiodiversity}.  
Freshly felled wood is often small in size and shows low decomposition levels, providing limited moist refuges suitable for some species, particularly amphibians \citep{Petranka1994Effectstimber,Morneault2004effectshelterwood,Owens2008Amphibianreptile,Otto2013Amphibianresponse}.  
Moreover, \cite{Petranka1994Effectstimber} found that retaining dead wood is more beneficial for salamanders in dry environments than in moist ones.  
This suggests that woody debris may have a greater impact in clear-cuts, where conditions are more challenging for species sensitive to temperature and humidity fluctuations.

% Jaeger1980MicrohabitatsTerrestrial
% deMaynadier1995relationshipforest
% Bunnell2010woodbiodiversitya
% Raymond2017irregularshelterwood
% Otto2014ComparingPopulation
% Ameray2021Forestcarbon
% Raymond-Leonard2020Deadwood
% rousseauWoodyBiomassRemoval2019
% Nolet2018Comparingeffects
% James2016effectharvest
% Chaudhary2016Impactforest
% Jaeger1980MicrohabitatsTerrestrial
% deMaynadier1995relationshipforest


% résultats : Canopy

Concernant l'ouverture de la canopé, les deux traitement de coupes indique une augmentation significative de l'ouverture de la canopé comparativement au témoins. 
Toutefois, nos résultats indiquent une diminution majeure du couvert forestier dans les traitements de coupes totales comparativement aux coupes partielles, ce qui correspond aux résultats observé dans des études similaires \citep{Nolet2018Comparingeffects,Mazerolle2021Woodlandsalamander}. 
Malgré le retrait de 50\% des tiges commerciales, les coupes partielle permettent de conserver une forte proportion de la canopé et pourraient ainsi maintenir des conditions environnementales proche de celles associés aux forêt mature. 
L'ouverture importante de la canopé dans les coupes totale entraine des changements majeur au niveau du sol, comme l'augmentation des rayonnement solaires et la vitesse des vents, ce qui amène une hausse de température, une baisse de l'humidité ainsi qu'un assèchement de la litière \citep{Keenan1993ecologicaleffects,Chen1999MicroclimateForest,Lindo2003Microbialbiomass,Brooks2008Forestfloor}. 
Ces changements amène une modification significative de la faune du sol, notamment en termes de composition, d'abondance et de richesse spécifique \citep{Staab2023Insectdecline}. 
Cela favorise les espèces associées aux milieux ouverts et secs au détriment de celles préférant des conditions fraîches et humides \citep{Niemela2007effectsforestry,Ochs2022Responseterrestrial,Staab2023Insectdecline}.
Regarding canopy openness, both harvesting treatments showed a significant increase in canopy opening compared to control sites.  
However, our results indicate a substantial reduction in forest cover in clear-cut treatments compared to partial cuts, consistent with observations in similar studies \citep{Nolet2018Comparingeffects,Mazerolle2021Woodlandsalamander}.  
Despite the removal of 50\% of commercial stems, partial cuts preserve a large proportion of the canopy, potentially maintaining environmental conditions similar to those in mature forests.  
The significant canopy opening in clear-cut treatments leads to major changes at the soil level, such as increased solar radiation and wind exposure, leading to higher temperatures, reduced humidity, and litter desiccation \citep{Keenan1993ecologicaleffects,Chen1999MicroclimateForest,Lindo2003Microbialbiomass,Brooks2008Forestfloor}.  
These changes cause significant shifts in soil fauna, affecting their composition, abundance, and species richness \citep{Staab2023Insectdecline}.  
Species adapted to open, dry habitats tend to thrive, often at the expense of those requiring cool and moist conditions \citep{Niemela2007effectsforestry,Ochs2022Responseterrestrial,Staab2023Insectdecline}.

% McKenny2006Effectsstructural
% Keenan1993ecologicaleffects
% Chen1999MicroclimateForest
% Marshall2000Impactsforest
% Salmon2008Relationshipssoil
% Huhta2001Effectstemperature


% résultats : Litter

Our results also indicate that forest harvesting treatments negatively affect litter depth, with the reduction becoming more pronounced as treatment intensity increases.  
These findings are consistent with observations reported in several other studies \citep{Marshall2000Impactsforest,Mazerolle2021Woodlandsalamander}.  
The decrease in litter depth may result from reduced leaf input following stem removal or from an accelerated decomposition rate caused by canopy opening, 
which leads to warmer conditions and increased precipitation interception \citep{Fierer2005LitterQuality,Butenschoen2011Interactiveeffects,Ameray2021Forestcarbon}.  
The loss of litter can have significant consequences for soil fauna by eliminating a cool, moist microhabitat \citep{spotilaRoleTemperatureWater1972,groverInfluenceCoverMoisture1998a,Niemela2007effectsforestry}.  
This change increases the risk of desiccation and predation for amphibians and arthropods, reducing the time they can spend on the surface for feeding and reproduction \citep{deMaynadier1995relationshipforest,koivula.LeafLitterSmallscale1999,Walton2013Topdownregulation}.  
Additionally, the loss of litter represents a significant reduction in food resources for decomposer communities, which has long-term impacts on carbon and nutrient cycling in forest ecosystems \citep{Handa2014Consequencesbiodiversity}.  


% Handa2014Consequencesbiodiversity
% Hartshorn2021reviewforest
% Smith2017Investigatingeffect
% Haggerty2019Effectsforestrydriven
% raymond-leonardSpringtailCommunityStructure2018a
% Pollierer2017Drivingfactors
% Magura2005ImpactsLeaflitter
% Hattenschwiler2005Biodiversitylitter
% Eaton2004Effectsorganic
% birdChangesSoilLitter2004
% Lindo2003Microbialbiomass
% Blair1988LitterDecomposition
% Covington1981Changesforest
% Zhang2023Plantlitter
% Petersen1982ComparativeAnalysisa
% Bardgett2014Belowgroundbiodiversitya
% Fierer2005LitterQuality
% Temperature is often the primary factor determining rates of litter decomposition (Meentemeyer 1978, Anderson 1991, Hobbie 1996)
% Gessner2010Diversitymeets
% Butenschoen2011Interactiveeffects



\subsection*{Soil fauna}
\label{disc:soil_fauna}
\phantomsection\addcontentsline{toc}{subsection}{\nameref{disc:soil_fauna}} 


\subsubsection*{Logging effects}
\label{disc:logging_effects}

% Otto2013Amphibianresponse

In our study, salamander occupancy probability was higher in partia-cuts compared to clear-cut sites, although this effect was marginal. Additionally, neither partial nor clear-cut treatments showed significant differences from control sites. 
Clear-cutting is often perceived as detrimental to salamander populations, but its impact varies greatly across studies \citep{Hocking2013Effectsexperimental,Chaudhary2016Impactforest}. 
Some studies have reported declines in salamander abundance 4 to 6 years after clear-cutting \citep{Petranka1993Effectstimber,Herbeck1999PlethodontidSalamander,Grialou2000effectsforest,Macneil2014Effectstimber},  
while others, including ours, found no significant difference between clear-cuts and control sites \citep{Renken2004EffectsForest,Mazerolle2021Woodlandsalamander}. 
Our findings also align with the literature regarding the absence of effects of partial harvesting on salamanders five years after logging \citep{McKenny2006Effectsstructural,Mazerolle2021Woodlandsalamander,Ochs2022Responseterrestrial}. 
However, other authors have observed decreases in salamander abundance in partial-cuts during the early years following logging, although these effects tend to diminish after five years \citep{Harpole1999Effectsseven,Knapp2003Initialeffects,Morneault2004effectshelterwood}.  
We believe the marginal effect observed between partial and clear-cuts may be explained by the ability of partial-cuts to preserve a more suitable habitat for salamanders. 
This could be attributed to the retention of a significant amount of woody debris in partial-cuts, as this is an essential resource for salamanders that provides shelter during unfavorable environmental conditions \citep{Nolet2018Comparingeffects,Peterman2014Spatialvariation,Achat2015Quantifyingconsequences,Peele2017Effectswoody}.  
Woody debris and other microstructures allow salamanders to remain on the surface for feeding and reproduction. 
According to our results, the harvesting treatments had similar effects on salamander occupancy probability and the volume of woody debris. 
The retention of forest canopy in partial-cuts and the rapid regrowth of understory vegetation associated with this treatment type may also explain the differences in impacts between the two harvesting methods \citep{Raybuck2015silviculturalpractices}. 
A preserved canopy reduces solar radiation reaching the ground and mitigates wind exposure, helping maintain cool and moist microclimates. 
These conditions are critical for salamanders cutaneous respiration and reduce desiccation risks \citep{Homyack2011Energeticssurface}. 
Previous studies have shown that retaining forest cover supports higher salamander densities \citep{Hocking2013Effectsexperimental,Harper2015Impactforestry,Mahoney2016Woodlandsalamander}. 
The presence of forest litter may also contribute to the differences in occupancy probability between partial and clear-cuts \citep{tilghmanMetaanalysisEffectsCanopy2012}. 
For instance, \cite{Ash1997DisappearanceReturn} reported that red-backed salamander abundance returned to pre-harvest levels 4-6 years after clear-cutting, attributing this recovery to the reappearance of forest litter. 
However, salamander recovery may take decades in some cases \citep{Homyack2013Effectsrepeatedstand,Ochs2022Responseterrestrial}. 
Litter depth might explain the observed differences in occupancy probability, as our study found significantly deeper litter in partial-cuts compared to clear-cuts. 
It is also worth noting that the effects of logging on salamanders vary over time. 
For example, \cite{Ochs2022Responseterrestrial} found no significant effects of clear-cutting on salamander abundance during the first 3 years post-harvest but reported declines between 4 and 6 years, with no recovery observed up to 11 years later. 
However, other studies have observed declines immediately following harvesting \citep{deMaynadier1995relationshipforest,Macneil2014Effectstimber}. 
Short-term responses may depend on the availability of woody debris, which decomposes over time and affects salamanders ability to persist in clear-cut areas \citep{Ochs2022Responseterrestrial}. 

% carabes et collemboles

Our findings suggest that forest harvesting treatments do not directly affect the occupancy probability of both ground beetle groups. 
However, canopy openness is known to have a positive effect on species richness and taxonomic composition \citep{Halme1993Carabidbeetles,Heliola2001Distributioncarabid,Koivula2002Alternativeharvesting}. 
This effect can be attributed to the creation of new habitats through clear-cutting and the ability of ground beetles to move quickly across landscapes \citep{Niemela2007effectsforestry}. 
Open, warmer, and drier habitats, such as those created by clear-cuts, are favored by many species commonly found in grasslands and similar environments. 
In contrast, closed forests provide less favorable conditions for most ground beetles, with only a few species adapted to the cool, dark, and enclosed environment \citep{Niemela1993Effectsclearcut,koivulaBorealCarabidbeetleColeoptera2002a}. 
Ground beetles are often categorized based on their habitat preferences, typically as forest specialists, open-habitat specialists, or generalist species \citep{Niemela2007effectsforestry}. 
Species associated with open habitats are intolerant of canopy closure and tend to disappear 20 to 30 years after harvesting, as the canopy regenerates \citep{Niemela1996importancesmallscale,Koivula2002Alternativeharvesting}. 
Conversely, forest specialists depend on specific forest attributes, such as microclimatic conditions or the presence of structural elements like coarse woody debris \citep{Niemela1996importancesmallscale,Heliola2001Distributioncarabid,Koivula2002Alternativeharvesting,Work2004Standcomposition}. 
The availability of forest litter is another factor influencing the abundance and distribution of ground beetles \citep{koivula.LeafLitterSmallscale1999,Heliola2001Distributioncarabid,Magura2005ImpactsLeaflitter}. 
Forest species may also colonize open habitats near old stands, as is often observed in clear-cut areas \citep{Spence1996Northernforestry,Koivula2002Alternativeharvesting}. 
Moreover, certain forest species adapted to natural disturbances may persist in open environments \citep{Niemela2007effectsforestry}. 
Therefore, while the likelihood of ground beetle occupancy remains similar after forest harvesting, the species composition and richness in harvested areas may change. 
Some studies also suggest that species responses to forest harvesting vary depending on their size. 
For instance, \cite{Nolte2019Habitatspecialization} found that larger ground beetles are more likely to decline after forest harvesting compared to smaller species. 
However, in our study, we did not observe significant differences in occupancy probability between large and small ground beetles. 



% effet des coupes sur les collemboles

Notre étude montre que la biomasse des collemboles ne semble pas être affectée par les différents traitements de coupe. 
Ce constat diffère de ce qui est généralement observé, où une diminution de l’abondance des collemboles accompagne généralement une intensification des perturbations \citep{Lindo2004Forestfloor,Laigle2021Directindirect,Kudrin2023metaanalysiseffects}. 
En revanche, l’absence d’effet entre les coupes partielles et les témoins semble refléter les résultats mesurés dans la littérature \citep{Kudrin2023metaanalysiseffects}. 

La diminution habituellement observée dans les populations de collemboles après une récolte forestière est souvent associée à la compaction des sols, à la perte de micro-habitats comme les débris ligneux, 
à la disparition de la couverture organique, ainsi qu’à une baisse de l'humidité et des nutriments \citep{Bird1986Effectwholetree,Baath1995Microbialcommunity,Lindo2004Forestfloor,rousseauForestFloorMesofauna2018}. 
La réponse des collemboles aux traitements varie selon le type de communauté \citep{raymond-leonardSpringtailCommunityStructure2018a}. 
Par exemple, les collemboles hémiedaphiques sont plus associés aux sols des forêts matures et sont davantage affectés par les coupes forestières \citep{Laigle2021Directindirect}. 
À l'inverse, les espèces épiédaphiques tolèrent mieux les habitats xériques et les niveaux de perturbation élevés \citep{Makkonen2011Traitsexplain,rousseauWoodyBiomassRemoval2019}. 
Cependant, ce dernier groupe reste sensible à la diminution de la litière, une ressource alimentaire importante pour ces espèces \citep{rousseauForestFloorMesofauna2018}. 

Malgré leur sensibilité aux perturbations du sol, l’augmentation de débris ligneux et de racines mortes après une coupe forestière peut aussi stimuler la densité des populations de collemboles 
au cours de la première année suivant la coupe \citep{Bird1986Effectwholetree,Marshall2000Impactsforest}. 
Par exemple, \cite{Huhta1976Effectsclearcutting} ont observé une hausse de la densité et de la biomasse des collemboles dans les premières années suivant une coupe totale, 
avec un retour à une biomasse normale six ans après. 
Par ailleurs, la rétention d'une partie des débris ligneux dans les coupes totales pourrait contribuer au maintien des communautés de collemboles, 
car ces débris constituent un habitat et une source de nourriture, même pour les espèces non saproxyliques \citep{rousseauForestFloorMesofauna2018,Raymond-Leonard2020Deadwood}.


% Henneron2017Forestplant
% Marshall2000Impactsforest
% Venier2017Grounddwellingarthropod
% Salmon2008Relationshipssoil
% Salmon2014Linkingspecies
% Huhta2001Effectstemperature
% Bird1986Effectwholetree
% Potapov2016Connectingtaxonomy
% Pollierer2017Drivingfactors
% pongeVerticalDistributionCollembola2000
% Rusek1998BiodiversityCollembola


\subsubsection*{Relations between taxa}
\label{disc:relations_between_taxa}


% qU,est qu'on penser observé selon les différent type de coupes

Un traitement de coupe forestière plus intense aurait un impcate plus importants sur les salamandres. 
une diminution de la probabilité d'occupation des salamandre pourrait ainsi avoir un effet négatif sur la biomasse des collemboles dans les milieu avec une forte densité en macroinvertébrés, comme les milieux ouvert ou les possédant une litière épaisse.
En revanche la diminution de salamandres dans les milieu a faible densité de macroinvertébré pourrait avoir un effet positifs sur la biomasse des collemboles puisqu'il y aurait une plus faible pression de prédations sur la mesofaune. 

Baisse de probabilité d'occupation des salamandres dans les milieu possédant un 

% Davic2004ecologicalroles
% Hairston1987evolutioncompeting
% Soliveres2016Biodiversitymultiple
% Walton2006Salamandersforestfloor
% MichaelWalton2005Salamandersforestfloor
% Homyack2010Doesplethodon
% Otto2013Amphibianresponse
% baileyEstimatingSiteOccupancy2004
% Rooney2000impactsalamander
% Hocking2014Effectsredbacked
% Ovaska1988Predatorybehavior


The impact of forest disturbances on population trends can be closely linked to trophic level \citep{Gotelli2006FoodWebModels,Nolte2019Habitatspecialization}. 
Species most sensitive to disturbances are typically those with small population sizes, large body sizes, and high trophic levels \citep{Seibold2015Associationextinction,Nolte2019Habitatspecialization,Hagge2021Whatdoes}. 
As top consumers, salamanders are particularly vulnerable to environmental changes. 
There seems to be a complex dynamic within forest soil food webs related to salamander occupancy \citep{baileyEstimatingSiteOccupancy2004,Walton2006Salamandersforestfloor,Rooney2000impactsalamander}. 
Declines in salamander populations can lead to shifts in soil community composition and alter ecosystem functions \citep{Hairston1987evolutioncompeting,Wyman1998Experimentalassessment,Rooney2000impactsalamander,Walton2005Contrastingeffects}.

However, in our study, the effects of logging disturbances do not appear to extend through the trophic network, contrary to our initial hypothesis, as no indirect impacts were measured.
Specifically, small ground beetles did not seem influenced by salamander occupancy probabilities, and springtail biomass remained unaffected by the occupancy probabilities of salamanders and ground beetles.

Several studies have suggested a complex relationship between red-backed salamanders and arthropod predators \citep{Gall2003BehavioralInteractions,Walton2006Salamandersforestfloor,Hickerson2018Behavioralinteractions}. 
The interaction between salamanders and ground beetles appears to vary across studies.  
For example, \cite{Gall2003BehavioralInteractions} and \cite{Ovaska1988Predatorybehavior} observed aggressive behaviors between the two taxa, whereas \cite{Hickerson2012Interactionsforestfloor} reported a positive effect of red-backed salamanders on carabid abundance. 
Other research found no influence of salamanders on the abundance of predators such as ground beetles, centipedes, and spiders \citep{Hocking2013Effectsexperimental}. 
However, since salamanders and ground beetles use the same habitats, consume similar resources, depend on common microhabitats and that ground beetles may be part of the salamanders diet still suggests a close relationship between these two taxa \citep{Jaeger1980MicrohabitatsTerrestrial,loveiEcologyBehaviorGround1996}.

Regarding the relationship between salamanders and springtails, salamander presence can positively affect the density of detritivore mesofauna by reducing the number of springtail competitors or predators in their habitat \citep{Wyman1998Experimentalassessment,Rooney2000impactsalamander,Walton2005Contrastingeffects,Walton2006Salamandersforestfloor}. 
Salamanders are territorial amphibians that defend their habitat to preserve food resources and favorable attributes for their survival \citep{Gall2003BehavioralInteractions,Hickerson2004Behavioralinteractions,Hickerson2012Interactionsforestfloor}. 
They also prioritize the most nutritious prey and become more selective as the density of their preferred prey increases \citep{Jaeger1981Foragingtactics,Jaeger1982ForagingTactics}. 
Consequently, salamander presence reduces macrofaunal detritivores and intermediate predators, favoring an increase in microbivorous and detritivorous mesofauna \citep{Rooney2000impactsalamander,Walton2005Contrastingeffects,Walton2006Salamandersforestfloor}. 
Conversely, in environments with lower macrofauna densities, salamanders tend to have a negative impact on springtails due to increased predation on the mesofauna. 
For instance, \cite{Walton2006Salamandersforestfloor} found that salamander presence was associated with an increase in springtails, correlated with a decrease in isopods, millipedes, and pseudoscorpions.  
However, salamanders had a negative effect on springtails in environments with initially low macroinvertebrate density. 
This outcome aligns with other studies that have also reported a negative impact of salamanders on springtail density \citep{Hickerson2017Easternredbacked}. 
Lastly, some studies found no effect of salamanders on mesofauna or on organic matter decomposition, which is consistent with our observations \citep{Hocking2013Effectsexperimental,Hocking2014Effectsredbacked}.

Salamander effects on invertebrate communities appear to partly depend on community composition and habitat heterogeneity \citep{MichaelWalton2005Salamandersforestfloor,Walton2006Salamandersforestfloor,Walton2013Topdownregulation,Best2014trophicrole}. 
For example, \cite{Walton2013Topdownregulation} observed that thick litter increases the number of invertebrate predators and reduces salamanders ability to exclude competitors, leading to higher overall predation on mesofauna. 
However, our results do not support these findings, despite significant differences in litter depth across our different logging treatments, we observed no variation within the trophic network.


\section*{Conclusion}
\label{sec:conclu1}
\phantomsection\addcontentsline{toc}{section}{\nameref{sec:conclu1}}

\section*{Acknowledgements}
\label{sec:acknowl1}
\phantomsection\addcontentsline{toc}{section}{\nameref{sec:acknowl1}}

\section*{Conflict of interest}
\label{sec:conflict1}
\phantomsection\addcontentsline{toc}{section}{\nameref{sec:conflict1}}

None declared
\section*{Author contributions}
\label{sec:author1}
\phantomsection\addcontentsline{toc}{section}{\nameref{sec:author1}}

\cleardoublepage

\begin{otherlanguage}{english}
\bibliographystyle{ecologyNewEN} % Style de citation en anglais
\bibliography{References}
\addcontentsline{toc}{section}{References}
\end{otherlanguage}
