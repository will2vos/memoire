\chapter*{Avant-propos}         % ne pas numéroter
\label{chap-avantpropos}        % étiquette pour renvois
\phantomsection\addcontentsline{toc}{chapter}{\nameref{chap-avantpropos}} % inclure dans TdM

%\usepackage[french]{babel}

Ce mémoire « par article » en Sciences forestières présenté à l'Université Laval est divisé en plusieurs parties : 
l'introduction, écrite en français, aborde en premier lieu la problématique liée à l'impact potentiel des traitements sylvicoles sur 
l'utilisation de l'habitat par la petite faune du sol dans le cadre d'un projet de migration assistée des arbres.
Le corps de ce document comprend le chapitre principal, intitulé 
"Direct and indirect effects on soil fauna of silvicultural treatments in the context of forest assisted migration", rédigé en anglais sous la forme d'un article scientifique. 
Suite au dépôt de ce mémoire, ce chapitre sera soumis à la revue scientifique \textit{Forest Ecology and Management}. 
La conclusion, rédigée en français, constitue la dernière partie de ce mémoire.
William Devos est l'auteur principal de cet article, suivi de Mathieu Bouchard, co-directeur de recherche, et Marc J. Mazerolle, directeur de recherche et dernier auteur. 
Les trois auteurs ont participé à l'élaboration du projet et à la rédaction. 
William Devos a pris en charge la planification du terrain, ainsi que la collecte et le traitement des données.
Marc J. Mazerolle et William Devos ont élaboré l'intégralité des modèles statistiques.
Marc J. Mazerolle et Mathieu Bouchard ont supervisé l'ensemble du projet de recherche.
