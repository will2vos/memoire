\chapter*{Avant-propos}         % ne pas numéroter
\label{chap-avantpropos}        % étiquette pour renvois
\phantomsection\addcontentsline{toc}{chapter}{\nameref{chap-avantpropos}} % inclure dans TdM

Ce mémoire en Sciences forestières à l'université Laval s'inscrit dans le cadre du projet Desired Regeneration Through Assisted Migration (DREAM) organisé en partenariat par l'Université Laval et 
le Ministère des Ressources naturelles et des Forêts pour le volet québécois, ainsi que le USDA Forest Service pour le volet Wisconsin.
Ce grand projet a comme objectif d'étudier divers scénarios opérationnels de migration assistée, en réponse au changement climatique et ainsi d'acquérir les connaissances
suffisantes pour une saine mise en place de cette mesure d'adaptation. 

Ce mémoire "par article" est divisé en plusieurs parties : 
L'introduction générale, écrite en français, aborde en premier lieu la problématique liée à l'impact potentiel de la préparation des sites forestiers sur 
l'utilisation de l'habitat par la petite faune du sol dans le cadre de la migration assistée des arbres.
Le corps de ce document comprend le chapitre principal, intitulé 
"Direct and indirect effects of forest harvest on soil fauna cooccurence in assisted tree migration context", rédigé en anglais sous la forme d'un article scientifique. 
Suite au dépôt de ce mémoire, ce chapitre sera soumis à la revue scientifique \textit{Forest Ecology and Management}. 
William Devos est l'auteur principal de cet article, suivi de Mathieu Bouchard, co-directeur de recherche et Marc J. Mazerolle, directeur de recherche et dernier auteur.
La conclusion générale rédigée en français constitue la dernière partie de ce mémoire.
